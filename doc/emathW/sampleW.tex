\documentclass[fleqn]{jarticle}
  \usepackage{emathW}
  \usepackage{emathAe}
  \usepackage{showexample}

\begin{document}
\openKaiFile
%\kaikaigyousuizyun{0}
このパッケージには
\begin{jquote}
  正整数の乗除算\\
  整数係数整式の乗除算
\end{jquote}
を筆算形式で表すためのコマンドが4種類定義されています.
\begin{verbatim}
    \kakezan    正整数のかけざん
    \warizan    正整数のわりざん
    \izyouhou   整数係数整式の乗法
    \izyohou    整数係数整式の除法
\end{verbatim}

この他
\begin{verbatim}
    \soinsuubunkai  素因数分解
    \gozyohou       互除法
\end{verbatim}
があります。

\section{基本的使用法}
まず,基本的な使用法を紹介します.
\begin{Enumerate}[\bfseries 例 1.]
\item $正整数 \times 正整数$
\begin{jquote}(1\zw)
\begin{minipage}{12\zw}
\begin{itembox}{整数の乗法}
\verb/\kakezan{17}{12}/
\end{itembox}
\end{minipage}
\quad$\longrightarrow$\quad
\begin{minipage}{7\zw}
\begin{shadebox}
$\kakezan{17}{12}$
\end{shadebox}
\end{minipage}
\end{jquote}
%\[\verb/\kakezan{17}{12}/\quad\longrightarrow\quad\kakezan{17}{12} \]

\item $正整数 \div 正整数$
\begin{jquote}(1\zw)
\begin{minipage}{12\zw}
\begin{itembox}{整数の除法}
\verb/\warizan{1234}{98}/
\end{itembox}
\end{minipage}
\quad$\longrightarrow$\quad
\begin{minipage}{9\zw}
\begin{shadebox}
\warizan{1234}{98}
\end{shadebox}
\end{minipage}
\end{jquote}
%\[\verb/\warizan{1234}{98}/\quad\longrightarrow\quad\warizan{1234}{98}\]
\clearpage

\item 整数係数整式の乗法

整式の表しかたは,係数を降べき順にコンマで区切って並べます.
係数が0で飛んでいるところは,0を明記する必要があります.
(下の例では,0を省略していますが,
この場合はコンマが複数続くことになります.)
\begin{jquote}
\begin{minipage}{16\zw}
\begin{itembox}{整式の乗法}
\verb|\izyouhou{1,,-3,4}{9,8}|
\end{itembox}
\end{minipage}\vspace{1ex}\par
\qquad$\longrightarrow$\quad
\begin{minipage}{18\zw}
\begin{shadebox}
$\izyouhou{1,,-3,4}{9,8}$
\end{shadebox}
\end{minipage}
\end{jquote}
整式の文字は,デフォルトでは $x$ ですが,これを変更することができます.
その方法は次節で紹介します.

\item 整数係数整式の除法
\begin{jquote}
\begin{minipage}{16\zw}
\begin{itembox}{整式の除法}
\verb|\izyohou{9,-8,,6}{1,-2}|
\end{itembox}
\end{minipage}\vspace{1ex}\par
\qquad$\longrightarrow$\quad
\begin{minipage}{16\zw}
\begin{shadebox}
$\izyohou{9,-8,,6}{1,-2}$
\end{shadebox}
\end{minipage}
\end{jquote}
\clearpage

\item 素因数分解

\begin{showEx}{\cmd{soinsuubunkai}}
\soinsuubunkai{108}
\end{showEx}

(注)まぎらわしいですが,\textsf{emath.sty} には
\cmd{Soinsuubunkai}があります。

\begin{showEx}(.5,.44){\cmd{Soinsuubunkai}}
\Soinsuubunkai{108}\kotae
$108=\kotae$
\end{showEx}

\item 二つの自然数の最大公約数を求める互除法

\begin{showEx}{\cmd{gozyohou}}
\gozyohou{108}{60}
\end{showEx}
\end{Enumerate}
\clearpage

\section{整式の文字変更}
整式の文字はデフォルトでは $x$ となっています.
これを変更するには,\texttt{[.]} オプションを用います.

\begin{Enumerate*}
\item 文字を $y$ に変更
\begin{jquote}
\begin{minipage}{17\zw}
\begin{itembox}{文字の変更}
\verb|\izyouhou[y]{1,,-3,4}{9,8}|
\end{itembox}
\end{minipage}\vspace{1ex}\par
\qquad$\longrightarrow$\quad
\begin{minipage}{17\zw}
\begin{shadebox}
$\izyouhou[y]{1,,-3,4}{9,8}$
\end{shadebox}
\end{minipage}
\end{jquote}

\item 除法でも同様
\begin{jquote}
\begin{minipage}{17\zw}
\begin{itembox}{\texttt{[a]} オプション}
\verb|\izyohou[a]{1,,,,-1}{1,-1}|
\end{itembox}
\end{minipage}\vspace{1ex}\par
\qquad$\longrightarrow$\quad
\begin{minipage}{16\zw}
\begin{shadebox}
$\izyohou[a]{1,,,,-1}{1,-1}$
\end{shadebox}
\end{minipage}
\end{jquote}
\end{Enumerate*}
\clearpage

\section{問題部分のみ表示}
問題部分のみを表示させるには,\verb+<M>+オプションを付加します。

\begin{Enumerate}[\bfseries 例 1.]
\item $正整数 \times 正整数$

\begin{showEx}{\cmd{kakezan}}
$\kakezan<M>{17}{12}$
\end{showEx}

\item $正整数 \div 正整数$

\begin{showEx}{\cmd{warizan}}
$\warizan<M>{17}{12}$
\end{showEx}
\end{Enumerate}

\section{\texttt{emathA.sty} との併用}
解答を巻末にまとめるためのスタイルファイル \texttt{emathA.sty}
を使用するとき,
\begin{verbatim}
  \item $123 \times 45$
    \kaitou{$\kakezan{123}{45}}
\end{verbatim}
と書くのは二度手間ですね.\verb/\kakezan/ 等に \texttt{<A>} オプションを
つけてこの無駄を排することができます.

\begin{itembox}{\texttt{<A>} オプション}
\begin{verbatim}
\openKaiFile
次の計算をせよ.
\begin{enumerate}[(1)]
  \item $\kakezan<A>{12}{13}$
  \item $\warizan<A>{987}{12}$
  \item $\izyouhou<A>{1,2,3}{9,8}$
  \item $\izyohou<A>{9,-8,7,-6}{1,-2,3}$
\end{enumerate}
\closeKaiFile
\clearpage
\centerline{【解答】}
\inputKaiFile
\end{verbatim}
\end{itembox}
と記述することで,
\begin{jquote}
  問題部分は次ページのように通常の表現で問題が記述され,\\
  解答部分は次々ページのように筆算形式で解答が記述されます.
\end{jquote}
\clearpage

\begin{itemshadebox}{問題部分}
次の計算をせよ.
\begin{enumerate}[(1)]
  \item $\kakezan<A>{12}{13}$
  \item $\warizan<A>{987}{12}$
  \item $\izyouhou[t]<A>{1,2,3}{9,8}$
  \item $\izyohou<A>{9,-8,7,-6}{1,-2,3}$
\end{enumerate}
\end{itemshadebox}
\closeKaiFile
\clearpage
\begin{itemshadebox}{解答部分}
\centerline{【解答】}
\inputKaiFile
\end{itemshadebox}
\end{document}

