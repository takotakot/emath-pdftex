\documentclass[a4j]{jarticle}
\usepackage{emPoly}
\usepackage{showexample}

\begin{document}
\section{\textsf{emPoly.sty}}
正多角形を描画するためのスタイルファイルです。
\subsection{一辺を与えて正多角形}
一辺を与えて正多角形を描画するコマンドが\cmd{Rpolygon}です。
一例として,O\retu(0,0), A\retu(1,0)を結ぶ線分OAを一辺とする
正六角形を描画してみましょう。

\begin{showEx}(.6,.34){\cmd{Rpolygon}}
\begin{zahyou}[ul=10mm](-1.5,2)(-.5,2.5)
  \tenretu{[]O(0,0);A(1,0)s}
  \Rpolygon\O\A{6}{C}
\end{zahyou}
\end{showEx}

\subsection{\cmd{Rpolygon}の書式}

\cmd{Rpolygon}の書式は次のとおりです。

\begin{boxnote}
\begin{verbatim}
\Rpolygon[#1]#2#3#4#5
 線分#2#3を1辺とする正#4角形
   戻り値
     中心 \#5
     頂点 \#5i(=#2), \#5ii, ......
 オプション引数
   #1 = L (default) : 2つの中心のうち,x座標の大きいほう
      = l           :                         小さいほう
      = O           :                  大きいほうで正多角形を描画しない
      = o           :                  小さいほうで正多角形を描画しない
\end{verbatim}
\end{boxnote}

先の例では,
\begin{verbatim}
  #2 = \O
  #3 = \A
  #4 = 6
  #5 = C
\end{verbatim}
としていますから,OAを一辺とする正6角形が描画され,
\begin{jquote}
\begin{verbatim}
中心が \C
頂点が \Ci, \Cii, \Ciii, \Civ, \Cv, \Cvi
\end{verbatim}
\end{jquote}
と定義されます。そのうち\verb+\Ci=\O+, \verb+\Cii=\A+ です。

\begin{showEx}(.6,.34){\cmd{Rpolygon}の戻り値}
\begin{zahyou}[ul=10mm](-1.5,2)(-.5,2.5)
  \tenretu{[]O(0,0);A(1,0)s}
  \Rpolygon\O\A{6}{C}
  \Kuromaru\C
  \Put\C[n]{C}
  \Put\Ciii[e]{Ciii}
  \Put\Civ[ne]{Civ}
  \Put\Cv[nw]{Cv}
  \Put\Cvi[w]{Cvi}
\end{zahyou}
\end{showEx}

OAを一辺とする正多角形は2個存在しますが,
オプション引数\verb+#1+を指定しないときは,
中心の$x$座標の大きい方(等しいときは,$y$座標)が指定されたものとみなします。
逆の方を描画したければ,オプション引数を\verb+[l]+と指定することとなります。

オプション引数を\verb+[O]+ または \verb+[o]+ としたときは,
正多角形は描画されず,中心,頂点の位置のみが返ってきます。

\begin{showEx}(.6,.34){\cmd{Rpolygon}の\texttt{[l]}オプション}
\begin{zahyou}[ul=10mm](-1.5,2)(-2.5,.5)
  \tenretu{[]O(0,0);A(1,0)s}
  \Rpolygon[l]\O\A{6}{C}
\end{zahyou}
\end{showEx}


\subsection{中心と一頂点を与えて正多角形}

中心と円周上の一頂点を指定して,円に内接する正多角形を描画するマクロが
\cmd{CRpolygon}です。

\begin{showEx}(.6,.34){\cmd{CRpolygon}}
\unitlength10mm\footnotesize
\begin{zahyou}(-1,3)(-1,3)
\def\C{(1,1)}
\def\A{(2,2)}
\CRpolygon\C\A{5}{F}
\En\C{1.4142}\Put\C[e]{C}\Kuromaru\C
\Put\Fi[e]{A=Fi}\Kuromaru\Fi
\Put\Fii[n]{Fii}\Kuromaru\Fii
\Put\Fiii[w]{Fiii}\Kuromaru\Fiii
\Put\Fiv[s]{Fiv}\Kuromaru\Fiv
\Put\Fv[e]{Fv}\Kuromaru\Fv
\end{zahyou}
\end{showEx}

正多角形を描画せず,頂点のみを得るには\texttt{[O]}オプションを用います。

\begin{showEx}(.6,.34){\texttt{[O]}オプション}
\unitlength10mm\footnotesize
\begin{zahyou}(-1,3)(-1,3)
\def\C{(1,1)}
\def\A{(2,2)}
\CRpolygon[O]\C\A{5}{Q}
\En\C{1.4142}\Put\C[e]{C}\Kuromaru\C
\Put\Qi[e]{A=Pi}\Kuromaru\Qi
\Put\Qii[n]{Pii}\Kuromaru\Qii
\Put\Qiii[w]{Piii}\Kuromaru\Qiii
\Put\Qiv[s]{Piv}\Kuromaru\Qiv
\Put\Qv[e]{Pv}\Kuromaru\Qv
\end{zahyou}
\end{showEx}

\cmd{Rpolygon}と違って,正多角形は一意に決まりますから,\texttt{[l], [o]}
オプションは意味をもちません。

\cmd{Rpolygon}のときにつけ忘れた〔注〕ですが,正多角形の頂点記号に`P'
は使用できません。
これは,\cmd{Pi} が \LaTeX で定義済みであるものを破壊することになるからです。
また,\cmd{Pii}も\textsf{eclarith.sty}で$2\pi$を表す数値として
定義されています。

頂点記号として`P'を用いたいときは,\verb/\CRpolygon\C\A{5}{PP}/ などと
別の記号を用いて,表示ラベルをPとするなどの二重帳簿的な処理が必要となります。

最後に書式です。

\begin{boxnote}
\begin{verbatim}
\CRpolygon[#1]#2#3#4#5{%
  中心#2の円周上の点#3を一つの頂点とする,円に内接する正#4角形
  戻り値
    頂点 #5i(=#3), #5ii, #5iii, ...
  #1 = L (default) : 正多角形を描画する。
     = O           : 正多角形を描画しない
\end{verbatim}
\end{boxnote}
\clearpage

\subsection{円に外接する正多角形}
円の中心と,円周上の一点を与えて,円に外接する正多角形を描画するコマンドが
\cmd{CCRpolygon}です。

\begin{showEx}(.6,.34){\cmd{CCRpolygon}}
\begin{zahyou}[ul=10mm](-.5,2.5)(-.5,2.5)
  \tenretu{C(1,1)s;T(1,2)n}
  \kuromaru{\C;\T}
  \CCRpolygon\C\T{5}{E}
  \En\C{1}
\end{zahyou}
\end{showEx}

その書式です:

\begin{boxnote}
\begin{verbatim}
\CCRpolygon[#1]#2#3#4#5{%
  中心#2の円周上の点#3を一つの接点とする,円に外接する正#4角形
  戻り値
    頂点 #5i, #5ii, #5iii, ...
  #1 = L (default) : 正多角形を描画する。
     = O           : 正多角形を描画しない
\end{verbatim}
\end{boxnote}
\end{document}
