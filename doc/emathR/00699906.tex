% 年度 1999
% 出題 0069 九州大学
% 検索キーワード  プログラミング
% 科目 数B

\documentclass[fleqn]{jarticle}
\usepackage{emath}
%\usepackage[debug]{emathA}
\pagestyle{empty}

\begin{document}
次の(1),(2)では,それぞれ,その目的を実行するためのBASICによる
プログラムの始めの部分が与えられている.方針を記述してから,
プログラムの残りの部分を完成せよ.ただし,変数A(1),A(2)等には座標
$a_1$,$a_2$等が入力されるものとする.

\begin{description}
\item[注意:] (1)のブログラムでは配列を表すためにDIM文を使っているが,
  DIM文を使わないブログラムを作成してもよい.
  そのときは,行番号10の文は消去し,行番号20,30の文は
    \begin{jquote}
    \begin{verbatim}
20 INPUT Al,A2
30 INPUT Pl,P2
    \end{verbatim}
    \end{jquote}
で置き換えるものとする.(2)についても,同様である.
\end{description}
\begin{enumerate}[(1)]
  \item 座標平面上の原点Oと異なる点A$(a_1,a_2)$について,
    任意の点P$(p_1,p_2)$から直線OAへの距離を表示すること.
    \begin{jquote}
    \begin{verbatim}
10 DIM A(2),P(2)
2O INPUT A(1),A(2)
30 INPUT P(1),P(2)
    \end{verbatim}
    \end{jquote}
  \item 点A$(a_1,a_2)$,B$(b_l,b_2)$を座標平面上の相異なる点とし,
    直線ABで平面を二分する.点P$(p_1,p_2)$,Q$(q_1,q_2)$が
    この直線の同じ側にあるときは1を,異なる側にあるときは$-1$を,
    P,Qの少なくとも一方がこの直線上にあるときは0を表示すること.
    ただし,ある点と直線との距離が,与えられた正数0.00001より
    小さいときはその点は直線上にあるとみなすことにする.
    \begin{jquote}
    \begin{verbatim}
10 DIM A(2),B(2),P(2),Q(2)
20 EPS=0.00001
3O INPUT A(1),A(2)
40 INPUT B(1),B(2)
5O INPUT P(1),P(2)
60 INPUT Q(1),Q(2)
    \end{verbatim}
    \end{jquote}
\end{enumerate}
\end{document}
