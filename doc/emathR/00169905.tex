% 年度 1999
% 出題 0016 筑波大学
% 検索キーワード  極座標 極方程式
% 科目 数C

\documentclass[fleqn]{jarticle}
\usepackage{emath}
\usepackage{emathP}
%\usepackage[debug]{emathA}
\pagestyle{empty}

\begin{document}
\begin{mawarikomi}{50mm}{\vspace{.5ex}\par
\unitlength1cm\small%
\begin{zahyou}(-.5,4)(-1.5,2)%
\def\O{(0,0)}%
\def\A{(2,0)}%
\def\kaku{36}%
\kyokuTyoku(1,\kaku)\Q%
\Addvec\A\Q\Q%
\Kaiten\Q\A{90}\R%
\Suisen\O\Q\R\P%
\Put\A(0,2pt)[rb]{A}%
\Put\A(0,-2pt)[t]{2}%
\Put{(1,0)}(0,-2pt)[rt]{1 }%
\Put\Q(0,0)[lb]{Q}%
\Put\P(0,0)[l]{ P}%
\Put{(2,-1)}(0,2pt)[b]{$C$}%
\Tyokkaku\P\O
\Tyokkaku\Q\A
\KAKUkigou[a]{(3,0)}\A\Q{\makebox(0,0)[l]{$\theta$}}%
\Drawline{\A\Q}%
\Tyokusen\O\P\xmin[y]\ymax
\Tyokusen\P\Q[y]\ymax\xmax
\En\A{1}%
\end{zahyou}}%
$xy$平面上において,点A(2,0)を中心とする半径1の円を$C$とする.
$C$上の点Qにおける$C$の接線に原点O(0,0)から下した垂線の足をPとする.
図のように$x$軸と線分AQのなす角を$\theta$とする.
ただし,$\theta$は$-\pi<\theta\leqq \pi$を動くものとする.
\begin{enumerate}[(1)]
  \item 点P$(x,y)$の座標$(x,y)$を$\theta$を用いて表せ.
  \item 点P$(x,y)$の$x$座標が最小となるとき,Pの座標$(x,y)$を求めよ.
  \item 直線$x=k$が点Pの軌跡と相異なる4点で交わるとき,
    $k$のとりうる値の範囲を求めよ.
\end{enumerate}
\end{mawarikomi}

【解】\par
{\unitlength1cm\small%
\begin{zahyou}(-.5,4)(-2,2)%
\def\O{(0,0)}%
\def\A{(2,0)}%
\def\kaku{36}%
\kyokuTyoku(1,\kaku)\Q%
\Addvec\A\Q\Q%
\Kaiten\Q\A{90}\R%
\Suisen\O\Q\R\P%
\Put\A(0,2pt)[rb]{A}%
\Put\A(0,-2pt)[t]{2}%
\Put{(1,0)}(0,-2pt)[rt]{1 }%
\Put\Q(0,0)[lb]{Q}%
\Put\P(0,0)[l]{ P}%
\Put{(2,-1)}(0,2pt)[b]{$C$}%
\Tyokkaku\P\O
\Tyokkaku\Q\A
\KAKUkigou[a]{(3,0)}\A\Q{\makebox(0,0)[l]{$\theta$}}%
\Drawline{\A\Q}%
\Tyokusen\O\P\xmin[y]\ymax
\Tyokusen\P\Q[y]\ymax\xmax
\En\A{1}%
\def\Fnr#1#2{\Cos{#1}\r\Mul{2}\r\r\Add{1}\r\r\edef#2{\r}}%
{\thicklines\rGurafu\Fnr0\Pii}%
\end{zahyou}}%
\end{document}
