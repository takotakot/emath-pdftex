\begin{itembox}{問題}
問題問題問題問題問題問題問題問題問題問題問題問題問題問題問題
問題問題問題問題問題問題問題問題問題問題問題問題問題問題問題
\end{itembox}
\begin{tyuukai}
解答文,解答文,解答文,解答文,解答文,解答文,解答文,解答文

\tyuu{注釈1注釈1注釈1注釈1注釈1注釈1注釈1注釈1注釈1注釈1注釈1}
ここで注釈をつけます。
解答文,解答文,解答文,解答文,解答文,解答文,解答文,解答文,
解答文,解答文,解答文

\begin{mawarikomi}{}{%
  \begin{zahyou*}[ul=1cm](-1,1)(-1,1)
    \Drawline{\LT\LB\RB\RT\LT}
    \En\O1
  \end{zahyou*}}
\indent ここから\textsf{mawarikomi}環境です。
ここで注釈をつけます。
\tyuu{\textsf{mawarikomi}環境内の注釈です}
まわりこみまわりこみまわりこみまわりこみまわりこみまわりこみ
まわりこみまわりこみまわりこみまわりこみまわりこみまわりこみ
まわりこみまわりこみまわりこみまわりこみまわりこみまわりこみ
まわりこみまわりこみまわりこみまわりこみまわりこみまわりこみ
\end{mawarikomi}
\end{tyuukai}
