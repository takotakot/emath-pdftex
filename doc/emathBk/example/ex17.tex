{\tyuuhaba=5zw
\begin{itembox}{問題}
おおおおおおおおおおおおおおおおおおおおおおおおおおおおおお
おおおおおおおおおおおおおおおおおおおおおおおおおおおおおお
\end{itembox}
\begin{tyuukai}
ああああああああああああああああああああああああああああああああ
ああああああああああああああああああああああああああああああああ

\begin{mawarikomi}{}{%
  \begin{zahyou*}[ul=20mm](-1,1)(-1,1)
    \En\O{1}
  \end{zahyou*}}
  いいいいいいい
\begin{align}
  y&=x \tyuu*{一次}\\
  y&=x^2 \tyuu*{二次}\\
  y&=x^3 \tyuu*{三次}
\end{align}
上の\textsf{align}環境内の右注は,回り込みの最中ですから,
\verb+\tyuu*+コマンドを用いています。

ああああああああああああああああああああああああああああああああ
ああああああああああああああああああああああああああああああああ
ああああああああああああああああああああああああああああああああ
ああああああああああああああああああああああああああああああああ
\begin{align}
  y&=x \tyuu{一次}\\
  y&=x^2 \tyuu{二次}\\
  y&=x^3 \tyuu{三次}
\end{align}
ここの右注は,\textsf{mawarikomi}環境内ではありますが,
回り込みは終了していますから,\verb+\tyuu+を用います。
ここで\textsf{mawarikomi}環境が終了します。
\end{mawarikomi}
ええええええええええ
\begin{align}
  y&=x \tyuu{一次}\\
  y&=x^2 \tyuu{二次}\\
  y&=x^3 \tyuu{三次}
\end{align}
ここは,\textsf{mawarikomi}環境の外側ですから,
\verb+\tyuu+コマンドを用います。
\end{tyuukai}}
