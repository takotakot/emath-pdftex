\documentclass[a4j,fleqn]{jarticle}
\usepackage{emathBk}
\usepackage{itembbox}

\begin{document}
\begin{itembox}{重複順列}
次の4種類の数字を用いて,3桁以下の正の整数はいくつ作れるか。
ただし,同じ数字を繰り返し用いてもよいものとする。
\begin{edaenumerate}[(1)]
  \item 1, 2, 3, 4
  \item 0, 1, 2, 3
\end{edaenumerate}
\end{itembox}

\begin{tyuukai}
\abovedisplayskip=2pt\relax
\belowdisplayskip=2pt\relax
【解答】
\begin{enumerate}[(1)]
  \item 3桁,2桁,1桁の整数は,それぞれ$4^3$個,$4^2$個,$4$個
    あるから,全部で
    \[ 4^3+4^2+4=\bm{84}(個)\]
  \item 3桁の整数は,\tyuu{数字に0を含むときは要注意!\\
    最高位には0を置けない}
    百の位には0以外の数字がくるから,百の位の数字の選びかたは3通り。
    十,一の位は4種類の数字のどれでもよいから,その選びかたは$4^2$通り。
    {\mathindent=0pt\relax
      \begin{alignat*}{2}
        &よって,3桁の整数は \qquad && 3\times4^2=48(個)\\
        &同様にして,2桁の整数は && 3\times4=12(個)
          \tyuu{十の位の数字の選びかたは0以外の3通りで,
        一の位は4種類のどれでもよい。}\\
        &1桁の正の整数は && 3(個)
      \end{alignat*}
    }%
    ゆえに,3桁以下の正の整数は
    \[ 48+12+3=\bm{63}(個)\]
\end{enumerate}
\end{tyuukai}

\tyuuhaba=0.31\textwidth
\begin{itembox}{三角方程式・不等式}
$0\Deg\leqq \theta<360\Deg$のとき,次の方程式,不等式を解け。
\vspace{2ex}
\begin{enumerate}[(1)]
  \item $\cos2\theta-3\cos\theta+2=0$
  \item $\sin2\theta>\cos\theta$
\end{enumerate}
\end{itembox}

\begin{tyuukai}
\abovedisplayskip=2pt\relax
\belowdisplayskip=2pt\relax
【解答】
\begin{enumerate}[(1)]
  \item 
    $\cos2\theta=2\cos^2\theta-1$
      \tyuu{$\cos\theta$だけの式に変形する。}
    を等式に代入して整理すると
    {\mathindent=0pt\relax
      \begin{alignat*}{2}
        &&& 2\cos^2\theta-3\cos\theta+1=0\\
        &ゆえに\qquad && (\cos\theta-1)(2\cos\theta-1)=0
          \tyuu{\def\tasukikata{2}$\tasuki12{-1}{-1}$}
        \\
        &よって && \cos\theta=1~または~\cos\theta=\bunsuu{1}{2}
      \end{alignat*}
    }\vspace{-\baselineskip}

    \begin{mawarikomi}(0,20pt){}{\small%
      \begin{zahyou}[ul=11mm](-1.5,1.5)(-1.5,1.5)
        \rtenretu*{A(1,60);B(1,-60)}
        \Kakukigou[a]\XMAX\O\A<hankei=.4>{60\Deg}
        \Kakukigou[a]\XMAX\O\B<hankei=.3>[nw]{300\Deg}
        \Put{(0,1)}[nw]{1}
          \Put{(1,0)}[se]{1}
        \Put{(.5,0)}[se]{$\frac12$}
        \En\O{1}
        \Drawline{\A\O\B}
        \Tyokusen\A\B{}{}
      \end{zahyou}}
      $0\Deg\leqq \theta<360\Deg$であるから\\
      $\cos\theta=1$のとき\quad $\theta=0\Deg$%
      \tyuu{$\cos\theta=\bunsuu{1}{2}$についての参考図}\\
      $\cos\theta=\bunsuu{1}{2}$のとき \quad
        $\theta=60\Deg$, $300\Deg$\\[1ex]
      ゆえに $\bm{\theta=0\Deg}$, $\bm{60\Deg}$, $\bm{300\Deg}$
    \end{mawarikomi}
  \item 
    \begin{mawarikomi}[-3]{}{\small%
      \begin{zahyou}[ul=11mm,Ueyohaku=5mm]%
            (-1.5,1.5)(-1.5,1.5)
        \rtenretu*{A(1,30);B(1,150)}
        \Put{(0,1)}[nw]{1}
        \Put{(0,.5)}[nw]{$\frac12$}
        \Put\O{\ougigata*{1}{30}{90}}
        \Put\O{\ougigata*{1}{150}{270}}
        \En\O{1}
        \Drawline{\A\O\B}
        \Tyokusen\A\B{}{}
      \end{zahyou}}
    $\sin2\theta>\cos\theta$から
    {\mathindent=0pt\relax
      \begin{alignat*}{2}
        &&& 2\sin\theta\cos\theta>\cos\theta \\
        &ゆえに \qquad && \cos\theta(2\sin\theta-1)>0 
          \tyuu*{$AB>0$\\
            \hspace*{2\zw}$\iff A>0,~B>0$\\
            または $A<0,~B<0$}%}
          \\
        &よって && \cos\theta>0,~\sin\theta>\bunsuu{1}{2} \\
        &または && \cos\theta<0,~\sin\theta<\bunsuu{1}{2}
      \end{alignat*}
    }%
    $0\Deg\leqq \theta < 360\Deg$であるから
    
    $\bm{30\Deg<\theta<90\Deg},~\bm{150\Deg<\theta<270\Deg}$
    \end{mawarikomi}
\end{enumerate}
\end{tyuukai}

\tyuumark{$\blacktriangleleft$~}
\begin{itembox}{角の二等分線}
\begin{caprm}
\sankaku{ABC}において,$AB=5$, $AC=3$, $\kaku A=120\Deg$とする。
\kaku Aの二等分線とBCとの交点をDとするとき,次の線分の長さを求めよ。
\end{caprm}
\begin{edaenumerate}<3>[(1)]
  \item BC
  \item BD
  \item AD
\end{edaenumerate}
\end{itembox}\bigskip

\noindent 【解法の手順】\vspace{1ex}

\begin{small}
\begin{breakDbox}
\begin{enumerate}[\protect\expandafter\fbox1]
  \item 余弦定理を用いて,BCを計算する。
  \item 角の二等分線の性質より,BDを求める。
  \item 面積を利用して,ADを求める。
\end{enumerate}
\end{breakDbox}
\end{small}\bigskip

\hidarityuukeisentrue
\tyuuhaba=0.25\textwidth
\begin{tyuukai}
\begin{caprm}
【解答】
\begin{enumerate}[(1)]
  \item 
    \begin{mawarikomi}[-1]{}{%
      \begin{zahyou*}[ul=5mm](0,7.5)(-.5,4)
        \footnotesize
        \tenretu{B(0,0)s;C(7,0)s}
        \CandC\B5\C3\AA\A\Put\A[n]{A}
        \Bunten\B\C53\D\Put\D[s]{D}
        \Hen_ko[25]\A\B{5}
        \Hen_ko[15]\C\A{3}
        \Kakukigou<0>\B\A\D(0,0){$\bullet$}
        \Kakukigou<0>\D\A\C(0,0){$\bullet$}
        \Drawline{\A\B\C\A\D}
      \end{zahyou*}}
    \gyoutou{\small\fboxsep=3pt\fbox1}
    余弦定理より
    \mathindent=0pt\relax
    \begin{align*}
      BC^2&=AB^2+AC^2-2AB\cdot AC\cos120\Deg
          \tyuu*{$\cos120\Deg=-\bunsuu12$}\\
        &=5^2+3^2-2\cdot5\cdot3\cdot\left(-\bunsuu12\right)\\
        &=49
    \end{align*}
    $BC>0$であるから $\bm{BC=7}$
    \end{mawarikomi}

  \item 
    \gyoutou{\small\fboxsep=3pt\fbox2}
    \tyuu{二等分線と比例の関係}
    $AB:AC=BD:DC$であるから
    \[ BD:DC=5:3 \]
    よって $BD=\bunsuu{5}{8}BC=\bm{\bunsuu{35}{8}}$
  \item 
    \gyoutou{\small\fboxsep=3pt\fbox3}%
    \tyuu{面積に関する等式}%
    $\sankaku{ABC}=\sankaku{ABD}+\sankaku{ADC}$であるから,
    $AD=x$とおくと
    \[ \bunsuu{1}{2}\cdot5\cdot3\sin120\Deg
      =\bunsuu{1}{2}\cdot5x\sin60\Deg+\bunsuu{1}{2}\cdot3x\sin60\Deg 
       \tyuu{$\sin120\Deg=\bunsuu{\sqrt3}{2}$,\\
        $\sin60\Deg=\bunsuu{\sqrt3}{2}$}
    \]
    よって $\bm{AD=\bunsuu{15}{8}}$
\end{enumerate}
\end{caprm}
\end{tyuukai}
\end{document}
