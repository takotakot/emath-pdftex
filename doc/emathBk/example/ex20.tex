\begin{itembox}{問題}
問題問題問題問題問題問題問題問題問題問題問題問題問題問題問題
問題問題問題問題問題問題問題問題問題問題問題問題問題問題問題
\end{itembox}
\begin{tyuukai}
【解答】\tyuu{注釈領域です。
注釈を付けるには,\cmd{tyuu}コマンドを用います。}
ここで注釈をつけてみます。
解答解答解答解答解答解答解答解答解答解答解答解答解答

\gyoutou{\P}
段落の先頭で,行頭にマークをつけてみます。この行の冒頭には
\begin{jquote}
\begin{verbatim}
\gyoutou{\P}
\end{verbatim}
\end{jquote}
と記述してあります。

\noindent\gyoutou{\fbox{1}}この行はインデントをつけずに,\fbox1を
欄外につけています。この場合は
\begin{jquote}
\begin{verbatim}
\gyoutou{\fbox{1}}
\end{verbatim}
\end{jquote}
としてあります。

\gyoutou{☆}
\tyuu{右注釈}
右欄外の注釈をつけるコマンド \cmd{tyuu} と \cmd{gyoutou} を
併用することも可能です。
\end{tyuukai}
