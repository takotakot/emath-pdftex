\tyuuhaba=8zw\relax
\begin{itembox}{問題}
問題部分および地の文は,通常の\verb+\textwidth+で組まれます。
問題問題問題問題問題問題問題問題問題問題問題問題問題問題問題
問題問題問題問題問題問題問題問題問題問題問題問題問題問題問題
\end{itembox}
\begin{tyuukai}
%【解答】解答部分は\verb+tyuukai+環境内に記述します。
そこでは,右側一部が注釈のための領域となります。
ここでは,\verb+\tyuuhaba=8zw+として注釈領域の
横幅を制限しています。

\tyuu{注釈領域です。
注釈を付けるには,\cmd{tyuu}コマンドを用います。}
ここで注釈をつけてみます。
解答解答解答解答解答解答解答解答解答解答解答解答解答
解答解答解答解答解答解答解答解答解答解答解答解答解答
解答解答解答解答解答解答解答解答解答解答解答解答解答
解答解答解答解答解答解答解答解答解答解答解答解答解答
\end{tyuukai}
