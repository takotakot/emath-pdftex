{\hidarityuukeisentrue
\begin{itembox}{問題}
問題問題問題問題問題問題問題問題問題問題問題問題問題問題問題
問題問題問題問題問題問題問題問題問題問題問題問題問題問題問題
\end{itembox}
\begin{tyuukai}
【解答】解答部分は\verb+tyuukai+環境内に記述します。
そこでは,右側一部が注釈のための領域となります。

\tyuu{注釈領域です。
注釈を付けるには,\cmd{tyuu}コマンドを用います。}
ここで注釈をつけてみます。
解答解答解答解答解答解答解答解答解答解答解答解答解答
解答解答解答解答解答解答解答解答解答解答解答解答解答

次に左欄外に注をつけてみましょう。
\hidarityuu{左欄外に注釈をつけるコマンドが \cmd{hidarityuu}です。}

ただし,右欄外の注が本文幅の中に入るのに対して,
左欄外の注は本文外に入る仕様としてあります。
\end{tyuukai}}
