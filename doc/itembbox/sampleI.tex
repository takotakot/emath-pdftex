\documentclass[a4j]{jarticle}
  \usepackage{okumacro}
  \let\okukeytop\keytop
  \usepackage{itembbox}
  \usepackage{showexample}
  \usepackage{emathPh}

\begin{document}
\section*{タイトル付き罫線ボックス}
このスタイルファイルでは,
\verb/ascmac.sty/ で定義されている \verb/itembox/ 環境の変形を
いくつか用意しました.その一覧です.
\begin{center}
\begin{tabular}{|l|c|l|l|l|}\hline
環境名 & コーナーの形状 & 見出し & デフォルト位置 & 備考 \\\hline
(\tt itembox  & 丸           & 上枠   & 中央       & ascmac.sty) \\\hline
\tt itembbox  & 丸            & 下枠   & 右 & \\\hline
\tt itemtbbox  & 丸         & 上枠と下枠 & 上枠は左,下枠は右 & \\\hline
\tt itemsquarebox  & 角     & 上枠   & 中央 & \\\hline
\tt itembsquarebox  & 角      & 下枠   & 右 & \\\hline
\tt itemtbsquarebox  & 角   & 上枠と下枠 & 上枠は左,下枠は右 & \\\hline
\tt itemshadebox  & 角      & 上枠  & 中央 & 影つき \\\hline
%\hline
%\tt  squarebox & 角  & なし  & −  &  \\\hline
%\tt dblbox   & 角  & なし  & −  & 二重枠 \\\hline
\end{tabular}
\end{center}
なお,これらの環境は ascmac.sty の機能を使用しますから,
ascmac.sty を使用することが前提となります.

では,それぞれの使用例をごらん頂きましょう.

\section{itembox family}
\subsection{itembbox}
見出しを下枠につけます.\verb/item/ と \verb/box/ の間の \verb/b/ は
\verb/bottom/ の頭文字の積りです.

\begin{itembbox}{1998 凸凹大学}
次の各問に答えよ.
\end{itembbox}

これは

\begin{quote}\begin{verbatim}
  \documentclass{jarticle}
  \usepackage{ascmac,itembbox}
  \begin{document}
  \begin{itembbox}{1998 凸凹大学}
    次の各問に答えよ.
  \end{itembbox}
  \end{document}
\end{verbatim}
\end{quote}

というリストをタイプセットして得られます.\verb/itembbox/ の書式は

\begin{quote}\begin{verbatim}
  \begin{itembbox}[配置オプション]{下枠の見出し}
で
  配置オプションは
      l : 左    c : 中央    r : 右(デフォルト)
のいずれかです.
\end{verbatim}
\end{quote}

\pagebreak
\subsection{itemtbbox}
次は,上枠にも下枠にも見出しを付ける環境 \verb/itemtbbox/ です.
\verb/item/ と \verb/box/ の間の \verb/tb/ は
\verb/top & bottom/ の頭文字の積りです.

例えば,

\begin{itemtbbox}{例題1}{1998 凹凸大学}
次の設問に答えよ.
\begin{enumerate}
  \item うんたら
  \item かんたら
\end{enumerate}
\end{itemtbbox}

これは
\begin{quote}\begin{verbatim}
  \documentclass{jarticle}
  \usepackage{ascmac,itembbox}
  \begin{document}
  \begin{itemtbbox}{例題1}{1998 凹凸大学}
    次の設問に答えよ.
    \begin{enumerate}
      \item うんたら
      \item かんたら
    \end{enumerate}
  \end{itemtbbox}
  \end{document}
\end{verbatim}
\end{quote}
というリストをタイプセットして得られます.\verb/itemtbbox/ の書式は

\begin{quote}\begin{verbatim}
  \begin{itemtbbox}<上枠の配置>[下枠の配置]{上枠の見出し}{下枠の見出し}
            <上枠の配置> は省略可能なオプション引数で, 
                 l(左), c(中央), r(右) のいずれかです.デフォルトは l
            [下枠の配置] も省略可能なオプション引数で, 
                 l(左), c(中央), r(右) のいずれかです.デフォルトは r
\end{verbatim}
\end{quote}
\pagebreak

\section{itemsquarebox family}
\setcounter{subsection}{-1}%
\verb/itembox family/ はコーナーが円いボックスでした.\par\noindent
これを直角にしたものが\verb/itemsquarebox family/ です.
\subsection{itemsquarebox}
まずは,上枠に見出しをつけます.

\begin{itemsquarebox}{例題}
次の設問に答えよ.
\end{itemsquarebox}

これは
\begin{quote}\begin{verbatim}
  \documentclass{jarticle}
  \usepackage{ascmac,itembbox}
  \begin{document}
  \begin{itemsquarebox}{例題}
    次の設問に答えよ.
  \end{itemsquarebox}
  \end{document}
\end{verbatim}
\end{quote}
というリストをタイプセットして得られます.\verb/itemsquarebox/ の書式は
\begin{quote}\begin{verbatim}
  \begin{itemsquarebox}[配置オプション]{上枠の見出し}
で
  配置オプションは
      l : 左    c : 中央(デフォルト)    r : 右
のいずれかです.
\end{verbatim}
\end{quote}
\pagebreak

\subsection{itembsquarebox}
次は見出しを下枠につけます.
\begin{itembsquarebox}{1998 凸凹大学}
次の設問に答えよ.
\end{itembsquarebox}

これは
\begin{quote}\begin{verbatim}
  \documentclass{jarticle}
  \usepackage{ascmac,itembbox}
  \begin{document}
  \begin{itembsquarebox}{1998 凸凹大学}
    次の設問に答えよ.
  \end{itembsquarebox}
  \end{document}
\end{verbatim}
\end{quote}
というリストをタイプセットして得られます.\verb/itembsquarebox/ の書式は
\begin{quote}\begin{verbatim}
  \begin{itembsquarebox}[配置オプション]{下枠の見出し}
で
  配置オプションは
      l : 左    c : 中央    r : 右(デフォルト)
のいずれかです.
\end{verbatim}
\end{quote}

\subsection{itemtbsquarebox}
三番目は,上枠・下枠両方に見出しをつけます.

\begin{itemtbsquarebox}{例題1}{1998 凹凸大学}
次の設問に答えよ.
\end{itemtbsquarebox}

これは
\begin{quote}\begin{verbatim}
  \documentclass{jarticle}
  \usepackage{ascmac,itembbox}
  \begin{document}
  \begin{itemtbsquarebox}{例題1}{1998 凹凸大学}
    次の設問に答えよ.
  \end{itemtbsquarebox}
  \end{document}
\end{verbatim}
\end{quote}
というリストをタイプセットして得られます.\verb/itemtbsquarebox/ の書式は

\begin{quote}\begin{verbatim}
  \begin{itemtbsquarebox}<上枠の配置>[下枠の配置]{上枠の見出し}{下枠の見出し}
            <上枠の配置> は省略可能なオプション引数で, 
                 l(左), c(中央), r(右) のいずれかです.デフォルトは l
            [下枠の配置] も省略可能なオプション引数で, 
                 l(左), c(中央), r(右) のいずれかです.デフォルトは r
\end{verbatim}
\end{quote}
\pagebreak

\section{itemshadebox}
影つきボックスの上枠に見出しをつけます.

  \begin{itemshadebox}[r]{\gt 重複順列}
    $n$個のものから$r$個をとった重複順列の総数は$n^r$である.
  \end{itemshadebox}

これは
\begin{quote}\begin{verbatim}
  \documentclass{jarticle}
  \usepackage{ascmac,itembbox}
  \begin{document}
  \begin{itemshadebox}[r]{\gt 重複順列}
    $n$個のものから$r$個をとった重複順列の総数は$n^r$である.
  \end{itemshadebox}
  \end{document}
\end{verbatim}
\end{quote}
というリストをタイプセットして得られます.\verb/itemshadebox/ の書式は
\begin{quote}\begin{verbatim}
  \begin{itemshadebox}[配置オプション]{上枠の見出し}
で
  配置オプションは
      l : 左    c : 中央(デフォルト)    r : 右
のいずれかです.
\end{verbatim}
\end{quote}

\section{\cmd{EMkeytop}}
\subsection{\textsf{ascmac.sty}の\cmd{keytop}}
\textsf{ascmac.sty}で定義されている\cmd{keytop}は,枠線蓋の部分と
左右の縦罫線が少し離れてしまいます。
この現象は,\textsf{emath.sty}を用いない場合でも発生します。

\begin{showEx}(.7,.24){\textsf{ascmac.sty}の\cmd{keytop}}
\asckeytop{A}
\end{showEx}

(注) \textsf{itembbox.sty}では,\textsf{ascmac.sty}の\cmd{keytop}を
\cmd{asckeytop}と改名しています。

\subsection{\textsf{okumacro.sty}の\cmd{keytop}}
\cmd{keytop}というコマンドは,奥村先生の\textsf{okumacro.sty}でも
定義されています。こちらは枠線は綺麗につながっていますが,
\textsf{emath}パッケージが殆ど必須としている\textsf{eepic.sty}とは相性が悪く,
枠線が乱れます。

\begin{showEx}(.7,.24){\textsf{okumacro.sty}の\cmd{keytop}}
\okukeytop{A}
\end{showEx}

\subsection{\textsf{itembbox.sty}の\cmd{keytop}}
前節で指摘した枠線の乱れは,\textsf{eepic.sty}と\textsf{screen}環境などとの
関係として知られているもので,\textsf{itembbox.sty}では,奥村先生の
\cmd{keytop}に対して修正を施しています。その結果は

\begin{showEx}(.7,.24){\textsf{itembbox.sty}の\cmd{keytop}}
\keytop{A}
\end{showEx}

\noindent
枠線も綺麗につながっており,目出度し目出度しと思ったのですが.....
\clearpage

\subsection{\cmd{keytop}のサイズ}
\cmd{keytop}の中身のサイズが変わりますと,横方向は大丈夫です。

\begin{showEx}(.7,.24){\textsf{itembbox.sty}の\cmd{keytop}}
\keytop{Enter}
\end{showEx}

縦方向が問題で

\begin{showEx}(.78,.16){\textsf{itembbox.sty}の\cmd{keytop}}
\keytop{$\stackrel{\textstyle\mathsf{ROLL}}{\mathsf{UP}}$}
\end{showEx}

文字サイズを小さくしてみましょう。

\begin{showEx}(.78,.16){\textsf{itembbox.sty}の\cmd{keytop}}
\keytop{$\stackrel{\scriptscriptstyle\mathsf{ROLL}}{%
  \scriptscriptstyle\mathsf{UP}}$}
\end{showEx}

\noindent
やっと収まりましたが,天がやや窮屈です。

\subsection{\cmd{EMkeytop}}
前節でみたように,\textsf{okumacro.sty}の\cmd{keytop}は,
縦方向のサイズは固定となっているようです。

そこで,縦方向も可変となるように修正したものを
\cmd{EMkeytop}と称することとしました。

\begin{showEx}(.78,.16){\textsf{itembbox.sty}の\cmd{keytop}}
\EMkeytop{$\stackrel{\scriptscriptstyle\mathsf{ROLL}}{%
  \scriptscriptstyle\mathsf{UP}}$}
\end{showEx}

通常の文字サイズに対しても

\begin{showEx}(.78,.16){\textsf{itembbox.sty}の\cmd{keytop}}
\EMkeytop{$\stackrel{\textstyle\mathsf{ROLL}}{\mathsf{UP}}$}
\end{showEx}

もっとも,縦サイズを可変とすることが善とは言えません。
下の2つを比較してください。

\begin{showEx}(.78,.16){縦サイズ可変}
\EMkeytop{\textsf{A}}
\EMkeytop{$\stackrel{\scriptscriptstyle\mathsf{ROLL}}{%
  \scriptscriptstyle\mathsf{UP}}$}
\end{showEx}

\begin{showEx}(.78,.16){縦サイズ固定}
\keytop{A}
\keytop{$\stackrel{\scriptscriptstyle\mathsf{ROLL}}{%
  \scriptscriptstyle\mathsf{UP}}$}
\end{showEx}

固定の方が望ましいでしょう。可変としたいときのために
\cmd{EMkeytop}を用意したということです。

\subsection{\cmd{keytop}の書体}
奥村先生の\cmd{keytop}の書体は\verb+\normalfont\sffamily+となっています。
一方,\cmd{EMkeytop}は\verb+\normalfont+です。

\begin{showEx}(.7,.24){書体の比較}
\keytop{A}\par
\EMkeytop{A}
\end{showEx}

\cmd{EMkeytop}の書体を変更するには
\cmd{EMkeytopfont}を用います。

\begin{showEx}(.7,.24){書体の変更}
\keytop{A}\par
\EMkeytopfont{\sffamily}
\EMkeytop{A}\par
\EMkeytopfont{\ttfamily}
\EMkeytop{A}
\end{showEx}

\subsection{コーナーサイズの変更}
\cmd{EMkeytop}では,コーナーサイズを\cmd{EMkeytopcornersize}で
指定可能としてあります。
引数は1〜20で,20とすると,\textsf{screen}環境のコーナーと同じ形となります。
デフォルト値は4です。

\begin{showEx}(.7,.24){\cmd{EMkeytopcornersize}}
\EMkeytop{A}\par
\EMkeytopcornersize{12}
\EMkeytop{A}
\end{showEx}

\subsection{\cmd{keytop}左右のアキ}
奥村先生の\cmd{keytop}は,左右それぞれ\texttt{1pt}のアキが入ります。
一方,\cmd{EMkeytop}はアキナシです。

\begin{showEx}(.7,.24){左右のアキ}
abc\keytop{A}xyz\par
abc\EMkeytop{A}xyz
\end{showEx}

\if0
アキを除去したのは,次のような使用を想定したからです。

\begin{showEx}(.9,.9){使用例}
\begin{zahyou}[ul=10mm,Sitayohaku=10mm](-1,7)(-.2,2.5)
  \footnotesize
  \def\Ft{T-sin(T)}
  \def\Gt{1-cos(T)}
  \PutStr{(.5,-.5)}(0,0)[l]{\EMkeytop{$t=0$}}to[-3]\O
  \BTen\Ft\Gt{$pi/2}\P
  \PutStr{(1.5,.5)}(0,0)[l]{\EMkeytop{$t=\bunsuu\pi2$}}to[-2]\P
  \BTen\Ft\Gt{$pi}\P
  \PutStr{(3.5,1.5)}(0,0)[t]{\EMkeytop{$t=\pi$}}to[-3]\P
  \BTen\Ft\Gt{2*$pi}\P
  \PutStr{(6,-.5)}(0,0)[r]{\EMkeytop{$t=2\pi$}}to[3]\P
  {\thicklines\BGurafu\Ft\Gt{-3}{9}}%
\end{zahyou}
\end{showEx}
\fi
\end{document}
