\documentclass[a4j]{jarticle}
\usepackage{hakoMS}
\usepackage{showexample}

\begin{document}
\title{マークシート作成マクロ\\
hakoMS.sty {\normalsize ver.0.03}\\使用例}%
\author{tDB}%
\date{2004/09/03}%

\maketitle\thispagestyle{empty}%
\begin{abstract}%
マークシートを作成するためのマクロです.
emathPh パッケージの使用を前提とします.

このマクロ集のマクロについてのご質問,バグ報告,修正・追加の提案等は
emath のサポートページ(現時点では
\begin{center}
    http://emath.s40.xrea.com/
\end{center}
)の掲示板でご発言ください.
\end{abstract}
\pagebreak
\pagenumbering{roman}%

\tableofcontents

\pagebreak

\pagenumbering{arabic}

\section{基本的用法}
\cmd{EMmarksheet}コマンドの必須の引数は
\begin{jquote}
\begin{verbatim}
#1 : 行数
#2 : 記号配列(コンマ区切り)
\end{verbatim}
\end{jquote}
の2つです。

\begin{showEx}(.9,.9){\cmd{EMmarksheet}}
\EMmarksheet{5}{$-$,0,1,2,3,4,5,6,7,8,9,s,t}
\end{showEx}

\section{配点表示}
配点を表示するときは,\verb+[...]+オプションで与えます。

\begin{showEx}(.9,.95){配点表示}
\EMmarksheet%
  [2,,4,,4]%
  {5}{$-$,0,1,2,3,4,5,6,7,8,9,a,b,c}
\end{showEx}
\newpage

\section{正答表示}
正答を黒く塗りつぶすには,\verb+'....'+オプションを用います。

\begin{showEx}(.9,.95){正答表示}
\def\arraystretch{1.2}%
\scalebox{.9}{%
  \EMmarksheet%
    [,6,,,2,,2]%
    '$-$,4,$\pm$,5,s,6,t'%
    {7}{$-$,$\pm$,0,1,2,3,4,5,6,7,8,9,s,t}
}%
\end{showEx}

ここでは
\begin{jquote}
\begin{verbatim}
\scalebox{.9}{%
\end{verbatim}
\end{jquote}
を用いて,縮小しています。
\textsf{hakoMS.sty}は,\textsf{emathPh.sty}を読み込みますから,
グラフィック関係のコマンドは自由に用いることができます。

また,この表は\textsf{tabular}環境を用いて実現していますから,
\begin{jquote}
\begin{verbatim}
\def\arraystretch{1.2}%
\end{verbatim}
\end{jquote}
などとして,行間を調整することも可能です。
\newpage

\section{欄記号の変更}
欄の名称は,デフォルトでは
\begin{jquote}
ア,イ,ウ,.....
\end{jquote}
と進行しますが,これを変更するには,
\textsf{hako.sty}で定義されたコマンド\cmd{hakomozisyu}を用います。

\begin{showEx}(.9,.95){欄記号の変更}
\hakomozisyu{\inhibitglue(あ)}
\EMmarksheet%
  [2,,4,,4]%
  {5}{$-$,0,1,2,3,4,5,6,7,8,9,a,b,c}
\end{showEx}

\section{欄記号書体の変更}
欄記号の書体を変更するには,
\textsf{hako.sty}で定義されたコマンド\cmd{hokosyotai}を用います。

\begin{showEx}(.9,.95){欄記号書体の変更}
\hakomozisyu{(1)}
\hakosyotai{\sffamily}
\EMmarksheet%
  [2,,4,,4]%
  {5}{$-$,0,1,2,3,4,5,6,7,8,9,a,b,c}
\end{showEx}
\newpage

\section{欄記号初期値の変更}
マークシートを作成する際,欄記号を`ア'からではなく,
途中`タ'から始める機能を附加しました。

\cmd{EMmarksheet}に,\verb+<syokiti=15>+などと,オプションで指定します。

\begin{showEx}(.9,.9){\texttt{<syokiti=15>}オプション}
\EMmarksheet<syokiti=15>{13}{$-$,0,1,2,3,4,5,6,7,8,9,s,t}
\end{showEx}
\clearpage

\section{書式}
\cmd{EMmarksheet}の書式です。

\begin{boxnote}
\begin{verbatim}
\EMmarksheet<#1>[#2]'#3'#4#5
  #1 : key=val の形式
         syokiti=.. : 初期値の変更
  #2 : 配点をコンマ区切りで与えます。
       いくつかの行をセットとする場合は,
           最終行に配点を記述し,
           他の行には空文字列を与えておきます。
  #3 : 正答をコンマ区切りで与えます。
       後述の #5 に与えるものと同一でなければなりません。
  #4 : 行数
  #5 : 記号配列をコンマ区切りで与えます。
       数式モードにするときは $....$ で挟みます。
次の場合は,エラーとなります。
  (1) #2 で与えられた個数と行数が一致しない場合。
  (2) #3 で与えられた個数と行数が一致しない場合。
\end{verbatim}
\end{boxnote}
\end{document}
