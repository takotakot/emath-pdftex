\documentclass[fleqn]{jarticle}
\usepackage{theorem,multicol,hako}
\usepackage{emathEy}

%-------------------------------------------------------------
% emathAs パッケージは,
%     練習と問題の番号を別立てにするか,
%     通し番号にするかで,
% ロード方法が異なります.
%
\usepackage{emathAs}%                 練習と問題を別番号
%\usepackage[toosibangou]{emathAs}%   練習と問題を通し番号
%
% 解答を置く位置の指定は emathA と同様です.
%\usepackage[debug]{emathAs}%
%\usepackage[debug,toosibangou]{emathAs}%
%
%-------------------------------------------------------------
% mondai環境のキャプションなどの修正
% emathAs.sty では,問題の環境が次の3種類定義されています.
%
%     環境名      キャプション
% 1:  rensyuu     練習
% 2:  mondai      問題
% 3:  hatten      発展問題
%
%   このうち,練習,問題は section 毎に番号がリセットされますが,
%   発展問題はファイルを通しての番号となります.
%
% この環境名とキャプションを修正する例です.
%
\labelmondai{問}%
\newenvironment{mon}{ \begin{mondai}}{%
  \end{mondai}}%
\labelhatten{}%
%
% この他,問題ではありませんが,
% 次の環境も \newtheorem で定義されています.
%
%     環境名      キャプション
% 4:  teigi       定義
% 5:  teiri       定理
% 6:  reidai      例題
% 7:  rei         例
%
%-------------------------------------------------------------
% 解答の部で,改行する問題レベルの設定
%\kaikaigyousuizyun{1}%   enumerate 第1層の問題で改行
%\kaikaigyousuizyun{0}%   デフォルト 練習(問題)ごとに改行
%\kaikaigyousuizyun{-1}%  改行一切なし.
%-------------------------------------------------------------

\renewcommand{\labelenumi}{(\arabic{enumi})}%

%\raggedcolumns
\columnseprule=0.4pt%
\theorempreskipamount=0pt%
\theorempostskipamount=0pt%
\preedaenumskip=-.25\baselineskip%
%\postedaenumskip=-2.5\baselineskip%

%\renewcommand{\kaitou}{\par 【解】}% デバッグ中は先頭の % を取る.

\begin{document}
\openrensyuuFile%
\openmondaiFile%
\openhattenFile%
\section{導関数}
\begin{rensyuu}
    定義により,次の関数の導関数を求めよ.%
    \kaitou{}%
    \begin{edaenumerate}
      \item $x^2$
        \kaitou{
          $2x~\left[\dlim{h \to 0}\bunsuu{(x+h)^2-x^2}{h}\right]$
        }
      \item $x^3$
        \kaitou{
        $3x^2$
        }
    \end{edaenumerate}
\end{rensyuu}
\begin{mon}
  次の関数を微分せよ.\kaitou{}%
  \begin{edaenumerate}<3>
    \item $y=x^2$
      \kaitou{$2x$}
    \item $y=x^3$
      \kaitou{$3x^2$}
    \item $y=x$
      \kaitou{$1$}
  \end{edaenumerate}
\end{mon}
\begin{rensyuu}
    次の関数を微分せよ.\kaitou{}
    \begin{edaenumerate}<1>
    \item $y=x^3-5x^2+6x-7$
      \kaitou{$3x^2-10x+6$}
    \item $y=(2x-1)(3x+1)$
      \kaitou{$12x-1$}
    \end{edaenumerate}
\end{rensyuu}
\section{接線}
\begin{mon}
  放物線 $y=x^2$ 上の点 $(-1,~1)$ における接線の方程式を求めよ.
\kaitou{$y=-2x-1$}
\end{mon}

\begin{rensyuu}
  放物線 $y=x^2$ に点 $(1,~0)$ から引いた接線の方程式を求めよ.
\kaitou{
  $y=x^2$ を微分して,$y'=2x$

  接点の $x$座標を $t$ とすれば接線の方程式は\
  \begin{equation}
    y-t^2=2t(x-t) \label{eq:setu1}
  \end{equation}
  これが点 $(1,~0)$ を通るから,
  \[0-t^2=2t(1-t)\]
  これを解いて, $t=0,~2$
  
  それぞれ \eqref{eq:setu1} に代入して,求める方程式は
  \[ \boldsymbol{y=0,~y=4x-4} \]
}
\end{rensyuu}
\vspace{\baselineskip}

\begin{center}
\textgt{\large 発展問題}
\end{center}

\begin{hatten}
\hakosyokika
\hangafter=1\hangindent2\zw%
$p$ を正の数とし,関数 $f(x)=\bunsuu{1}{3}x^3-px^2+4$ を考える.$f(x)$ は
\begin{quote}
  $x=\Hako[L0000199837a] $ で極大値 \Hako をとり,\\
  $x=\Hako[L0000199837u] p$ で極小値 $\bunsuu{\Hako<2>}{\Hako}p^3+\Hako $ 
  をとる.
\end{quote}
\hangafter=0\hangindent2\zw%
$a=\refHako*{L0000199837a}$, $b=\refHako*{L0000199837u} p$ とおく.

4点 A$(a,~f(a))$, B$(a,~f(b))$, C$(b,~f(b))$, D$(b,~f(a))$ を頂点とする
四角形ABCDが正方形となるのは
\[ p=\bunsuu{\sqrt{\Hako }}{\Hako } \]
のときである.
\syutten{1998 センター}
\kaitou{$ア=0$, $イ=4$, $ウ=2$, $エ=-$, $オ=4$, $カ=3$, $キ=4$, $ク=6$, $ケ=2$}
\end{hatten}
\closerensyuuFile
\closemondaiFile
\closehattenFile
\vspace{.5ex}

\vfill

%\hrule\vspace{.5ex}
\clearpage

\preEqlabel{$\cdots\cdots$}%
\small%
\abovedisplayskip=2pt%
\belowdisplayskip=2pt%
\topsep=0pt%
\begin{multicols}{2}
%\prekailabel{\bfseries}

\section*{問の解答}
\everypar{\hangafter1\hangindent1\zw\parindent2\zw}
\inputmondaiFile
\vspace{\baselineskip}
\section*{練習の解答}
\everypar{\hangafter1\hangindent1\zw\parindent2\zw}
\inputrensyuuFile
\vspace{\baselineskip}
\section*{発展問題の解答}
\everypar{\hangafter1\hangindent1\zw\parindent2\zw}
\inputhattenFile
\end{multicols}
\end{document}
