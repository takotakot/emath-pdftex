%%% 考査用紙スタイル B4横三段組 サンプルファイル
%%% 複数ページの例です.

\documentclass[b4paper,landscape,fleqn]{jarticle}
\usepackage[papersize]{emath}%
\usepackage{b4yoko3}%

\begin{document}
%%% step 0 ------------------------------------------------------------
%%% テストが複数ページのとき,ページ番号を印刷するか否かのフラッグ
%\prpagetrue% ページを印刷するときは行頭の % を削除する.
%
\testname{\ 1学年『数学I』二学期中間考査\ \ 1998.10.22(木)\ }%
\begin{sheet}
  \begin{column}
    \item 1段目の1問目です.\vspace{10cm}
    \item 1段目の2問目です.
  \end{column}%
  \begin{column}%
    \item 2段目のはじめの問題です.
        \begin{enumerate}
            \item 小問1 \vspace{4.5cm}
            \item 小問2 \vspace{4.5cm}
        \end{enumerate}
    \item 2段目の次の問題です.
  \end{column}%
  \begin{column}%
    \item 3段目の問題です.
  \end{column}%
\end{sheet}
\begin{sheet}%  二枚目です.
  \begin{column}
    \item 2ページ目の1段目の1問目です.\vspace{10cm}
    \item 2ページ目の1段目の2問目です.
  \end{column}%
  \begin{column}%
    \item 2ページ目の2段目の問題です.\vspace{5cm}
    \item 2ページ目の2段目の問題です.\vspace{5cm}
    \item ここで問題番号が2桁になります.
  \end{column}%
  \begin{column}% 2ページ目が2段で終りで,3段目が無いときは
                % 3段目の記述は省略することができます.
    \item 2ページ目の3段目の問題です.
  \end{column}%
\end{sheet}% 二枚目の終りです.
\end{document}
