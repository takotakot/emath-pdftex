%%% 考査用紙スタイル B4横三段組 サンプルファイル
%%% 
%   解答を裏面に印刷する実例です.
%

\documentclass[b4paper,landscape,fleqn]{jarticle}
%   両面印刷のときは twoside オプションをつけます.

\usepackage[papersize]{hako}%
\usepackage{b4yoko3}%
\usepackage{emathEy}%

\begin{document}
\hakosyokika
\hakomozisyu{ }%
\hakosyotai{\vrule width 0pt height .7\baselineskip depth .4\baselineskip}%
%\prpagetrue
\testname{\ 1学年『数学I』一学期期末考査\ \ 1998. 7. 9\ }%

\begin{sheet}
\begin{column}
\item 次の\Hako を補え.(ただし$i$は虚数単位とする.)[6点×9]
\vspace{-1.5ex}

  \begin{enumerate}
    \item $a(x-1)+b(x-2)=1$が$x$についての
      恒等式となるとき,定数$a,~b$の値は$a=\Hako,~b=\Hako$である.
    \item $(x+y\,i)(1+i)=3-2\,i$を満たす実数$x,~y$の値は
    
      $x=\Hako,~y=\Hako~である.$
    \item 次の計算をせよ.
      \begin{edaenumerate}[(i)]
        \edaitem<2>{$(1+i)^3=\Hako(4zw)$}
        \edaitem<2>{$\bunsuu{1-i}{1+2\,i}=\Hako(4zw)$}
        \edaitem<1>{ $\kongou{-3}\times\kongou{-27}=\Hako(4zw)$}
      \end{edaenumerate}
    \item 次の二次方程式の解を判別せよ.
      \begin{enumerate}[(i)]
        \item $x^2-3x+1=0$ の解は \Hako(11zw)
        \item $x^2-3x+4=0$ の解は \Hako(11zw)
        \item $x^2=2x-1$の解は \Hako(11zw)
      \end{enumerate}
    \item 二次方程式$3x^2-4x+5=0$の解は$x=\Hako(6zw)$であるから,
      $3x^2-4x+5$を一次式の積に因数分解すると\Hako(20zw)となる.
    \item $x$についての二次方程式$kx^2-9x+k=0$が重解をもつような
      正の定数$k$の値は\Hako であり,このときの重解は$x=\Hako$である.
    \item 二次方程式$3x^2-4x+2=0$の2つの解を$\alpha$, $\beta$とすると,
      $\alpha+\beta=\Hako(2zw)$,\quad $\alpha\beta=\Hako(2zw)$となるから,

      $\alpha^2+\beta^2=\Hako(2zw)$,\quad $\alpha-\beta=\Hako(2zw)$
      となる.
    \item 整式$P(x)=3x^4-4x+1$を
      \begin{enumerate}[(i)]
        \item $x-2$で割ったときの余りは\Hako, 
        \item $2x+1$で割ったときの余りは\Hako である.
      \end{enumerate}
    \item 
      3次方程式$6x^3-x^2-4x-1=0$の解を小さい方から並べると
      $\Hako{}<\Hako{}<\Hako$の3つである.
  \end{enumerate}
\end{column}%
\begin{column}%
  \item 二次方程式$x^2+3x+5=0$の2つの解を$\alpha$, $\beta$として,
    次の問いに答えよ.[12点]
    \begin{enumerate}[(1)]
      \item $(\alpha-1)(\beta-1)$の値を求めよ.\vspace{3cm}
      
      \item $\alpha-1$, $\beta-1$を解とする二次方程式を求めよ.
        \vspace{5cm}
    \end{enumerate}

  \item $p$を正の定数とし,2次方程式$px^2-3x-7=0$の
    2つの解を$\alpha$,$\beta$とするとき,次の問いに答えよ.[12点]
    \begin{enumerate}[(1)]
      \item $\alpha^3+\beta^3$を$p$の式で表せ.\vspace{2cm}
      
      \item $\alpha^3+\beta^3=36$が成り立つときの$p$の値を求めよ.
    \end{enumerate}
\end{column}%
\begin{column}%
  \item $x$の3次方程式$x^3+ax^2+bx-5=0$ ($a$,$b$は実数の定数)の
    解のひとつが$x=1+2i$(ただし$i^2=-1$)であるとき,次の問いに答えよ.[12点]
    \begin{enumerate}[(1)]
      \item 実数$a$, $b$の値を求めよ.\vspace{5cm}
      
      \item この3次方程式の他の2つの解を求めよ.\vspace{5cm}

    \end{enumerate}

  \item $(1+x)^{12}$を$1-x^2$で割つた余りを求めよ.[10点]
\end{column}%
\end{sheet}

\mondaibangou{1}%       裏面の問題番号を1から振り直すコマンドです.
\setlength{\abovedisplayskip}{0pt}
\setlength{\belowdisplayskip}{0pt}
\hakosyokika
\hakomozisyu{ }%
\hakosyotai{\vrule width 0pt height .6\baselineskip depth .2\baselineskip}%
\begin{sheet}
\begin{column}
\item 次の\Hako を補え.(ただし$i$は虚数単位とする.)[6点×9]
\vspace{-1.5ex}

  \begin{enumerate}[(1)]
    \item $a(x-1)+b(x-2)=1$が$x$についての
      恒等式となるとき,定数$a,~b$の値は$a=\hako{1},~b=\hako{$-1$}$である.
    \item $(x+y\,i)(1+i)=3-2\,i$を満たす実数$x,~y$の値は
    
      $x=\hako{$\bunsuu{1}{2}$},~y=\hako{$-\bunsuu{5}{2}$}~である.$
    \item 次の計算をせよ.
      \begin{edaenumerate}[(i)]
        \edaitem<2>{$(1+i)^3=\hako{$-2+2i$}$}
        \edaitem<2>{$\bunsuu{1-i}{1+2\,i}=\hako{$\bunsuu{-1-3i}{5}$}$}
        \edaitem<1>{ $\kongou{-3}\times\kongou{-27}=\hako{$-9$}$}
      \end{edaenumerate}
    \item 次の二次方程式の解を判別せよ.
      \begin{enumerate}[(i)]
        \item $x^2-3x+1=0$ の解は \hako{異なる2つの実数解である}
        \item $x^2-3x+4=0$ の解は \hako{異なる2つの虚数解である}
        \item $x^2=2x-1$の解は \hako{重解である}
      \end{enumerate}
    \item 二次方程式$3x^2-4x+5=0$の解は$x=\hako{$\bunsuu{2\pm\kongou{11}i}{3}$}$であるから,
      $3x^2-4x+5$を一次式の積に因数分解すると\hako{$3\left(x-\bunsuu{2+\kongou{11}i}{3}\right)\left(x-\bunsuu{2-\kongou{11}i}{3}\right)$}となる.
    \item $x$についての二次方程式$kx^2-9x+k=0$が重解をもつような
      正の定数$k$の値は\hako{$\bunsuu{9}{2}$}であり,
      このときの重解は$x=\hako{1}$である.
    \item 二次方程式$3x^2-4x+2=0$の2つの解を$\alpha$, $\beta$とすると,
      $\alpha+\beta=\hako{$\bunsuu{4}{3}$}$,
      \quad $\alpha\beta=\hako{$\bunsuu{2}{3}$}$となるから,

      $\alpha^2+\beta^2=\hako{$\bunsuu{4}{9}$}$,
      \quad $\alpha-\beta=\hako{$\pm\bunsuu{2\kongou{2}}{3}i$}$
      となる.
    \item 整式$P(x)=3x^4-4x+1$を
      \begin{enumerate}[(i)]
        \item $x-2$で割ったときの余りは\hako{41}, 
        \item $2x+1$で割ったときの余りは\hako{$\bunsuu{51}{16}$} である.
      \end{enumerate}
    \item 
      3次方程式$6x^3-x^2-4x-1=0$の解を小さい方から並べると
      $\hako{$-\bunsuu{1}{2}$}<\hako{$-\bunsuu{1}{3}$}<\hako{1}$の3つである.
  \end{enumerate}
\end{column}
\begin{column}
  \item 二次方程式$x^2+3x+5=0$の2つの解を$\alpha$, $\beta$として,
    次の問いに答えよ.[12点]
    \begin{enumerate}[(1)]
      \item $(\alpha-1)(\beta-1)$の値を求めよ.

      \noindent
      【解】解と係数の関係から,
        \[\alpha+\beta=-3,~\alpha\beta=5 \]
        ゆえに
        \begin{align*}
          (\alpha-1)(\beta-1)&=\alpha\beta-(\alpha+\beta)+1\\
            &=5-(-3)+1=\bm{9}\quad\cdots \cdots (答)
        \end{align*}\vspace{.5cm}
      
      \item $\alpha-1$, $\beta-1$を解とする二次方程式を求めよ.

      \noindent
      【解】$(\alpha-1)+(\beta-1)=(\alpha+\beta)-2=-5$
      
      (1)の結果とあわせて,求める方程式の一つは
      \[x^2-(-5)x+9=0~すなわち~x^2+5x+9=0 \]
      一般には,0以外の任意定数を$a$として,
      \[\bm{a(x^2+5x+9)=0},~a\neqq 0 \quad\cdots \cdots (答)\]
    \end{enumerate}\vspace{.75cm}

  \item $p$を正の定数とし,2次方程式$px^2-3x-7=0$の
    2つの解を$\alpha$,$\beta$とするとき,次の問いに答えよ.[12点]
    \begin{enumerate}[(1)]
      \item $\alpha^3+\beta^3$を$p$の式で表せ.

      \noindent
      【解】解と係数の関係から\vspace{1ex}
      \[ \alpha+\beta=\bunsuu{3}{p},~\alpha\beta=-\bunsuu{7}{p}     \]
      従って
      \begin{align*}
        \alpha^3+\beta^3&=(\alpha+\beta)^3-3\alpha\beta(\alpha+\beta)\\
          &=\bm{\bunsuu{27}{p^3}+\bunsuu{63}{p^2}}\quad\cdots \cdots (答)
      \end{align*}\vspace{.5cm}
      
      \item $\alpha^3+\beta^3=36$が成り立つときの$p$の値を求めよ.

      \noindent
      【解】$\alpha^3+\beta^3=36$は(1)の結果から
        \[ \bunsuu{27}{p^3}+\bunsuu{63}{p^2}=36\]
        両辺に$\bunsuu{p^3}{9}$をかけて,
        \[ 3+7p=4p^3~すなわち~4p^3-7p-3=0 \]
        因数定理を用いて左辺を因数分解して,
        \[(p+1)(2p+1)(2p-3)=0 \]\vspace{1ex}
        $p>0$だから $\bm{p=\bunsuu{3}{2}}\quad\cdots \cdots (答)$
    \end{enumerate}
\end{column}
\begin{column}
  \item $x$の3次方程式$x^3+ax^2+bx-5=0$ ($a$,$b$は実数の定数)の
    解のひとつが$x=1+2i$(ただし$i^2=-1$)であるとき,次の問いに答えよ.[12点]
    \begin{enumerate}[(1)]
      \item 実数$a$, $b$の値を求めよ.

      \noindent
      【解】$x=1+2i$が解だから
      \[(1+2i)^3+a(1+2i)^2+b(1+2i)-5=0 \]
      左辺を計算して,実部と虚部に分離すると
      \[(-3a+b-16)+(4a+2b-2)i=0 \]
      $a,~b$は実数だから,複素数の相等条件により
      \renritu{-3a+b&=16 \\ 4a+2b&=2}
      これを解いて,$\bm{a=-3,~b=7}\quad\cdots \cdots (答)$\vspace{1cm}
      
      \item この3次方程式の他の2つの解を求めよ.

      \noindent
      【解】(1)の結果から,与えられた3次方程式は
      \[x^3-3x^2+7x-5=0 \]
      左辺を因数分解して $(x-1)(x^2-2x+5)=0$
      
      これを解いて $x=1,~1\pm2i$
      
      ゆえに他の2つの解は $\bm{1,~1-2i}\quad\cdots \cdots (答)$
    \end{enumerate}\vspace{2cm}

  \item $(1+x)^{12}$を$1-x^2$で割つた余りを求めよ.[10点]

      \noindent
      【解】2次式で割った余りは高々1次であるから,
      求める余りを$ax+b$とおくことができる.
      また,商を$Q(x)$とおくと
      \[ (1+x)^{12}=(1-x^2)Q(x)+ax+b \]
      これは$x$についての恒等式であるから,$x=1,~-1$とおくと,
      \[\begin{cases}
        2^{12}=a+b \\
        0=-a+b
      \end{cases}
       \]
      連立させて解くと,$a=b=2048$
      
      従って求める余りは $\bm{2048(x+1)}\quad\cdots \cdots (答)$
\end{column}
\end{sheet}
\end{document}
