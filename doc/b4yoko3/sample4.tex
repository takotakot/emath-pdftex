%%% 考査用紙スタイル B4横三段組 サンプルファイル
%
% テストの実例です.
%
\documentclass[b4paper,landscape,fleqn]{jarticle}
\usepackage[papersize]{hako}%
\usepackage{b4yoko3}%
\usepackage{emathEy}%

\begin{document}
\hakosyokika
\hakomozisyu{ }%
\hakosyotai{\vrule width 0pt height .7\baselineskip depth .4\baselineskip}%
\testname{\ 1学年『数学I』一学期期末考査\ \ 1998. 7. 9\ }%
%
\begin{sheet}
\begin{column}
  \item 次の\Hako を補え.(ただし$i$は虚数単位とする.)[6点×9]
    \vspace{-1.5ex}

  \begin{enumerate}
    \item $a(x-1)+b(x-2)=1$が$x$についての
      恒等式となるとき,定数$a,~b$の値は$a=\Hako,~b=\Hako$である.
    \item $(x+y\,i)(1+i)=3-2\,i$を満たす実数$x,~y$の値は
    
      $x=\Hako,~y=\Hako~である.$
%   \item 複素数$\alpha=3+4i$について,
%     その共役複素数は$\kyouyaku{\alpha}=\Hako$,
%     絶対値は$\zettaiti{\alpha}=\Hako$である.
    \item 次の計算をせよ.
      \begin{edaenumerate}[(i)]
        \edaitem<2>{$(1+i)^3=\Hako(4\zw)$}
        \edaitem<2>{$\bunsuu{1-i}{1+2\,i}=\Hako(4\zw)$}
        \edaitem<1>{ $\kongou{-3}\times\kongou{-27}=\Hako(4\zw)$}
      \end{edaenumerate}
    \item 次の二次方程式の解を判別せよ.
      \begin{enumerate}[(i)]
        \item $x^2-3x+1=0$ の解は \Hako(11\zw)
        \item $x^2-3x+4=0$ の解は \Hako(11\zw)
        \item $x^2=2x-1$の解は \Hako(11\zw)
      \end{enumerate}
    \item 二次方程式$3x^2-4x+5=0$の解は$x=\Hako(6\zw)$であるから,
      $3x^2-4x+5$を一次式の積に因数分解すると\Hako(20\zw)となる.
    \item $x$についての二次方程式$kx^2-9x+k=0$が重解をもつような
      正の定数$k$の値は\Hako であり,このときの重解は$x=\Hako$である.
    \item 二次方程式$3x^2-4x+2=0$の2つの解を$\alpha$, $\beta$とすると,
      $\alpha+\beta=\Hako(2\zw)$,\quad $\alpha\beta=\Hako(2\zw)$となるから,

      $\alpha^2+\beta^2=\Hako(2\zw)$,\quad $\alpha-\beta=\Hako(2\zw)$
%     $\bunsuu{\alpha}{\beta}+\bunsuu{\beta}{\alpha}=\Hako$
      となる.
    \item 整式$P(x)=3x^4-4x+1$を
      \begin{enumerate}[(i)]
        \item $x-2$で割ったときの余りは\Hako, 
        \item $2x+1$で割ったときの余りは\Hako である.
      \end{enumerate}
    \item 
      3次方程式$6x^3-x^2-4x-1=0$の解を小さい方から並べると
      $\Hako{}<\Hako{}<\Hako$の3つである.
  \end{enumerate}
\end{column}%
\begin{column}%
  \item 二次方程式$x^2+3x+5=0$の2つの解を$\alpha$, $\beta$として,
    次の問いに答えよ.[12点]
    \begin{enumerate}[(1)]
      \item $(\alpha-1)(\beta-1)$の値を求めよ.\vspace{3cm}
      
      \item $\alpha-1$, $\beta-1$を解とする二次方程式を求めよ.
        \vspace{5cm}
    \end{enumerate}

  \item $p$を正の定数とし,2次方程式$px^2-3x-7=0$の
    2つの解を$\alpha$,$\beta$とするとき,次の問いに答えよ.[12点]
    \begin{enumerate}[(1)]
      \item $\alpha^3+\beta^3$を$p$の式で表せ.\vspace{2cm}
      
      \item $\alpha^3+\beta^3=36$が成り立つときの$p$の値を求めよ.
    \end{enumerate}
\end{column}%
\begin{column}%
  \item $x$の3次方程式$x^3+ax^2+bx-5=0$ ($a$,$b$は実数の定数)の
    解のひとつが$x=1+2i$(ただし$i^2=-1$)であるとき,次の問いに答えよ.[12点]
    \begin{enumerate}[(1)]
      \item 実数$a$, $b$の値を求めよ.\vspace{5cm}
      
      \item この3次方程式の他の2つの解を求めよ.\vspace{5cm}

    \end{enumerate}

  \item $(1+x)^{12}$を$1-x^2$で割つた余りを求めよ.[10点]
\end{column}%
\end{sheet}%
\end{document}
