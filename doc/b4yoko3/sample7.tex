%%% 考査用紙スタイル B4横三段組 サンプルファイル
%%% 縦の間隔を \vfill で均等割り(LaTeX2e のみ)

\documentclass[b4paper,landscape,fleqn]{jarticle}
\usepackage[papersize]{emath}%
\usepackage{b4yoko3}%

\begin{document}
%%% step 1 ------------------------------------------------------------
%%% 考査のタイトルを \testname の引数として記述します.
%%% この step は必須です.省略するとエラーになります.
%
\testname{\ 1学年『数学I』二学期中間考査\ \ 1998.10.22(木)\ }%
%
%%% step 2 ------------------------------------------------------------
%
\begin{sheet}%  一枚のシートを sheet 環境内に記述します.
  \begin{column}% 一つの段を column 環境内に記述します.
                %%% 1段目の問題を column 環境の中に記述します.
                %%% この部分は enumerate 環境の中に入るという前提ですから,
                %%% 問題のはじめに \item を必要とします.
    \item 1段目の1問目です.
    \vfill
    \item 1段目の2問目です.
    \vfill
  \end{column}%
  \begin{column}% 二段目の記述です.
    \item 2段目のはじめの問題です.
        \begin{enumerate}% 必要なら小問を enumerate 環境で記述します.
            \item 小問1
            \vfill
            \item 小問2
            \vfill
        \end{enumerate}
    \item 2段目の次の問題です.
    \vfill
  \end{column}%
  \begin{column}% % 3段目の記述です.
    \item 3段目の問題です.
  \end{column}%
\end{sheet}%  一枚のシートの終わりです.
\end{document}
