%%% 考査用紙スタイル B4横三段組 サンプルファイル
%%% 両面印刷の例です.

\documentclass[landscape,b4paper,fleqn,twoside]{jarticle}%twoside オプション
\usepackage[papersize]{emath}%
\usepackage{b4yoko3}%

\begin{document}
\testname{\ 1学年『数学I』二学期中間考査\ \ 1998.10.22(木)\ }%
\prpagetrue
\pageran{\thepage/\pageref{lastpage}}% ページ番号の出力様式を変更してみます.
                                     % 総ページ数を必要とするときは,
                                     % 最終ページで \label{lastpage} として
                                     % ラベル lastpage を定義します.
\begin{sheet}
  \begin{column}
    \item aaa
  \end{column}
  \begin{column}
    \item aaa\vspace{10cm}
    \item bbb
  \end{column}
  \begin{column}
    \item ccc
  \end{column}
\end{sheet}

% 裏面です.
%\mondaibangou{1}%    裏面が表面の解答である場合は,問題番号を1から振り直す
%                     必要があります.そのためのコマンドです.

\danhaba{0.275}{0.275}{0.45}% 裏面は段の横巾を不均等にしてみます.
                            % 三つの段の横巾比率を 和が1になるような数値で
                            % 与えます.
\begin{sheet}
  \begin{column}
    \item ddd
  \end{column}
  \begin{column}
    \item eee
  \end{column}
% \begin{column}%     裏面の3段目が不要のときは省略することができます.
%   \item fff
% \end{column}
\end{sheet}%
\label{lastpage}%     全ページ情報をヘッダなどで使用するときは
                %     lastpage という名前のラベルを利用します.
\end{document}
