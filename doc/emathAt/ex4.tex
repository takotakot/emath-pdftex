\documentclass[a4j,fleqn]{jarticle}
\usepackage{itembbox}
\usepackage{emathAt}

\hidarityuukeisentrue%    左中途解答本文の間の罫線を引くか否かのスイッチ(v0.01)

\begin{document}
\renewcommand{\tyuumark}{$\blacktriangleleft$}% 右注のマーク変更(v0.01)
\begin{itembox}{条件付き最大最小}
$x-2y=3$のとき,$x^2+y^2$の最小値を求めよ。
\end{itembox}
\begin{tyuukai}
【解答】\resetcounter{equation}
$x-2y=3$から \houteisiki{x=2y+3\label{Esyoukyo}}
\tyuu{$x$を消去する}
{\mathindent=0pt\relax
\vspace{-\abovedisplayskip}
\begin{align*}
よって\quad x^2+y^2&=(2x+3)^2+y^2=5y^2+12+9\\
  &=5\left\{y^2+\bunsuu{12}{5}y+\left(\bunsuu65\right)^2\right\}
    -5\cdot\left(\bunsuu65\right)^2+9\\
  &=5\left(y+\bunsuu{6}{5}\right)^2+\bunsuu{9}{5}
\end{align*}}%
ゆえに,
\hidarityuu{放物線の頂点\hfill $\blacktriangleright$}%    左注をつける(v0.01)
$y=-\bunsuu{6}{5}$のとき,最小値$\bunsuu{9}{5}$をとる。

このとき,\eqref{Esyoukyo}から $x=\bunsuu{3}{5}$

\tyuu{最大値・最小値を与える変数$x$, $y$の値も明示する}
よって $\bm{x=\bunsuu{3}{5}}$, $\bm{y=-\bunsuu{6}{5}}$のとき
\textbf{最小値} $\bm{\bunsuu{9}{5}}$
\end{tyuukai}
\end{document}
