\documentclass[a4j]{jarticle}
%
% 次の行を有効にすると,最後のページで図の位置がおかしくなります。
%\usepackage[old]{emathMw}% mawarikomi環境を旧形式とする
%
\usepackage{emathMw}
\usepackage{emathThm}
\usepackage{showexample}

\theorembodyfont{\normalfont\upshape}
\newtheorem<leftmargin=0pt>{reidai}{例題}
\newtheorem{rei}{例題}
\newtheorem<leftmargin=2\zw>{mondai}{練習問題}
\newtheorem<leftmargin=3\zw,Leftmargin=2\zw>{tyuu}{注}

\begin{document}
\section{\textsf{theorem}では}
\cmd{newtheorem}で定義される諸環境では,左インデントはつきません。

\begin{showEx}(.9,.9){\cmd{newtheorem}で定義された環境}
\begin{reidai}
いいいいいいいいいいいいいいいいいいいいいいいいいいいい
いいいいいいいいいいいいいいいいいいいいいいいいいいいい
いいいいいいいいいいいいいいいいいいいいいいいいいいいい
\end{reidai}
\end{showEx}

\section{\textsf{emathThm}では}
\subsection{左インデントのデフォルトは\texttt{1\zw}}
\textsf{emathThm}は,左インデントを附加しようという目的で開発されました。
デフォルト値は\verb+1\zw+です。

\begin{showEx}(.9,.9){\textsf{emathThm}で修正された\textsf{newtheorem}}
\begin{rei}
うううううううううううううううううううううううううううう
うううううううううううううううううううううううううううう
うううううううううううううううううううううううううううう
うううううううううううううううううううううううううううう
\end{rei}
\end{showEx}
\clearpage

\subsection{左インデント量の調整}
左インデントを調整するには\verb+<leftmargin=2\zw>+などとオプションを与えます。

\begin{showEx}(.9,.9){\texttt{leftmargin=..}オプション}
\begin{mondai}
うううううううううううううううううううううううううううう
うううううううううううううううううううううううううううう
うううううううううううううううううううううううううううう
うううううううううううううううううううううううううううう
\end{mondai}
\end{showEx}

オプションを\verb+<leftmargin=0pt>+とすれば,
従来の\cmd{newtheorem}による環境と同じとなります。

なお,この文書で使用した環境は,プリアンブルで次のように定義してあります。

\begin{jquote}
\begin{verbatim}
\newtheorem<leftmargin=0pt>{reidai}{例題}% 旧形式互換
\newtheorem{rei}{例題}                   % 新形式(左インデント=1\zw)
\newtheorem<leftmargin=2\zw>{mondai}{練習問題}
\end{verbatim}
\end{jquote}
\clearpage

\subsection{\texttt{<Leftmargin=..>}オプション}
\verb+<leftmargin=..>+オプションは,
2行目以降の左インデントを指定するオプションですが,
1行目の左インデントを指定するオプションが
\verb+<Leftmargin=..>+です。

\begin{showEx}(.9,.9){\texttt{Leftmargin=..}オプション}
本文本文本文本文本文本文本文本文本文本文本文本文本文本文
本文本文本文本文本文本文本文本文本文本文本文本文本文本文
本文本文本文本文本文本文本文本文本文本文本文本文本文本文

\begin{tyuu}
うううううううううううううううううううううううううううう
うううううううううううううううううううううううううううう
うううううううううううううううううううううううううううう
うううううううううううううううううううううううううううう
\end{tyuu}
\end{showEx}

ここで用いた\textsf{tyuu}環境は
\begin{jquote}
\begin{verbatim}
\newtheorem<leftmargin=3\zw,Leftmargin=2\zw>{tyuu}{注}
\end{verbatim}
\end{jquote}
として定義されており,
\begin{jquote}
  1行目は\verb+2\zw+,\\
  2行目以降は\verb+3\zw+
\end{jquote}
版面左からインデントされています。
\clearpage

\section{\textsf{mawarikomi}環境との併用}
\cmd{newtheorem}で定義される環境では,1行目と2行目以降では,
行頭の位置に違いがあります。1行目から\textsf{mawarikomi}環境を併用すると
図が右のほうに追いやられてしまいます。この点を\textsf{emathMw.sty v. 0.04}
で修正しました。その確認です。

\begin{showEx}(.9,.9){\textsf{mawarikomi}環境との併用}
\begin{mondai}
\begin{mawarikomi}{}{%
  \begin{picture}(50,50)
    \put(0,0){\framebox(50,50){\Huge 図}}
  \end{picture}}
うううううううううううううううううううううううううううう

ええええええええええええええええええええええええええええ
ええええええええええええええええええええええええええええ
ええええええええええええええええええええええええええええ
ええええええええええええええええええええええええええええ
\end{mawarikomi}
\end{mondai}
\end{showEx}

従来の\textsf{emathMw.sty}を用いてタイプセットすると,
図は\textsf{shadebox}環境の右外に一部がはみ出してしまいます。

これは,\cmd{newtheorem}で定義した環境では,
\begin{jquote}
1行目冒頭の`う'と\\
2行目先頭の`う'
\end{jquote}
が,全角約3.5文字分ずれており,
\begin{jquote}
従来の\textsf{emathMw.sty}は2行目先頭の`う'と図との距離を
1行目先頭の`う'からの距離として図を配置していた
\end{jquote}
ことに起因します。この点を修正したのが\textsf{emathMw.sty v0.04}です。

ただし,この変更は「仕様の変更」となっています。
従来,このズレを\textsf{mawarikomi}環境の\verb+(dx,dy)+オプションで修正
した文書に対しては\textsf{emathMw.sty v0.03}以前をご使用願います。

なお,\textsf{emathMw.sty}の\verb+v0.04+以降を
\begin{jquote}
\begin{verbatim}
\usepackage[old]{emathMw}
\end{verbatim}
\end{jquote}
と,\textsf{emathMw.sty}を\verb+[old]+オプション付きでロードしておけば
旧仕様となります。ただし,そのタイミングは\textsf{emath.sty}を
ロードするより前,すなわち,emath関連の先頭でなければなりません。
\end{document}
