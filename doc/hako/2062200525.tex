% 年度 2005
% 出題 2062 慶應義塾大学 (商)
% 検索キーワード  複素数 極形式
% 科目 数B

\documentclass[fleqn]{jarticle}
\usepackage[tpic]{arhako}
\usepackage[continue]{emathAe}

\begin{document}
\hakosyokika
\centermodetrue
\hakosenhaba{.4pt}
\hakomozisyu{(1)}
\hakosyotai{\sffamily}
\sikirisen{\hasen[L=1.5pt,G=1.5pt]}{.4pt}{(1)}
\openHakoKaiFile
正の実数$r$と角$\theta$は,
$\polar[r]{\theta}=5\sqrt{2+\sqrt3}+5\sqrt{2-\sqrt3}i$
を満たしているものとする。ただし,$i^2=-1$とする。このとき,
$\polar[3r][\{]{(-2\theta)}=a-bi$を満たす実数$a$, $b$は,それぞれ
$a=\renHako'15'\sqrt{\renHako'3'}$, $b=\renHako'15'$である。
\closeHakoKaiFile
\begin{Kaitou}
  \HakoKaiKata{t}
  \noindent
  \inputHakoKaiFile
\end{Kaitou}
\end{document}
