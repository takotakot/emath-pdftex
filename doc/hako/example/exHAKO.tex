\begin{verbatim}
\hakosyokika
点Aを中心とする半径2の円と,点Bを中心とする半径3の円が点Cで
外接している.点Dは半径2の円上に,また点Eは半径3の円上にあり,直線DEは二つの円の共通接線となっている.
点Cにおける二つの円の共通接線と直線DEとの交点をFとし,
直線DAと直線ECの交点をGとする.\par
このとき,$\mathrm{DE=\HAKO\sqrt{\HAKO}}$ である.\par
また,$\mathrm{AD \heikou BE}$ から,
$\triangle$\Hako<2>[Lue][Lu,Le]\text{C}
と \sankaku{AGC} は相似であるので,線分AGと
線分 \Hako<2>[Loka][Lo,Lka] の長さは等しくなる.
(\refHako*{Lue} の \refHako{Lu} と \refHako{Le}, 
\refHako*{Loka} の\refHako{Lo}と\refHako{Lka}については,解答の順序を問わない.)\par
したがって,点Gは点Aを中心とする半径2の円上にあり,
$\kaku{GCD}=\HAKO<2>\Deg$ となる.
\end{verbatim}
