\begin{verbatim}
\hakosyokika
\hakomozisyu{イ}
\hakosyotai{\textmc}
\sankaku{ABC}において,辺BC,CA,ABの長さを
それぞれ $a,~b,~c$ とし \kaku{A} の大きさを $A$ で表す.
CからABに垂線CHを引く.\sankaku{ACH} において,
$\text{CH}=\Hako,~\text{AH}=\Hako$ 
したがって,$\text{BH}=\Hako$
そこで,\sankaku{BCH} に三平方の定理を適用することにより,
余弦定理 $a^2=\Hako$ が導かれる.
\end{verbatim}
