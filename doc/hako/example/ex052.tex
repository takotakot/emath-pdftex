\begin{verbatim}
\hakosyokika
\begin{caprm}
次の文中の空欄 \refHako*{L06a}〜\refHako*{L06ku} にあてはまる
整数を 0〜9 から選んで該当する解答欄(省略)にマークし,
空欄\refHako*{L06A}と\refHako*{L06I}には,あてはまる $a$ の
式を該当する解答欄(省略)にマークせよ.\par
正三角形ABCと同一平面上の点Pが不等式 $AP^2>BP^2+CP^2$ を満た
すとき,辺BCの中点を原点,直線BCを $x$軸,辺BCの垂直2等分線
を$y$軸とする座標系を設定して点Pの存在する範囲を求めよう.
正三角形の1辺の長さを $2a$, 頂点Bの座標を $(-a,~0)$, 
頂点Aの $y$座標を正とすると,頂点A,Cの座標は 
A$(0,\Hako[L06A]/あ/)$, C$(\Hako[L06I]/あ/,~0)$となる.
点P$(x,~y)$ が上記の不等式を満たすとき,$x,~y$ は不等式
\[ \Hako[L06a]x^2+\Hako y^2+\Hako x
  +\Hako\sqrt{\Hako}ay-a^2<0 \]
を満たすことになる.
したがって不等式 $AP^2>BP^2+CP^2$ を満たす点Pは$\Hako a$を半
径とし,点 $(\Hako,~-\sqrt{\Hako[L06ku]}a)$ を中心とする円の内部の点である.
\end{caprm}
\end{verbatim}
