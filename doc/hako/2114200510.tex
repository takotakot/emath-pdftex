% 年度 2005
% 出題 2114 東洋大学
% 検索キーワード  一次不等式
% 科目 数Ⅰ

\documentclass[fleqn]{jarticle}
\usepackage{hako}
\usepackage[continue]{emathAe}

\begin{document}
\centermodetrue
\openHakoKaiFile
$\bigcirc$で囲んだ文字には符号($+$, $-$),不等号($<$, $\leqq$)があてはまる。
求めた数値が0の場合は$+0$とする。

\houteisiki{0<x<1\label{E2114200510i}}, \houteisiki{|x-a|<2\label{E2114200510ii}}とする。
ここで,\eqref{E2114200510i}を満たすどのような$x$についても\eqref{E2114200510ii}
が満たされるとき,この実数$a$の範囲は
\[ \tmHako'-1'\tmHako'\leqq'a\tmHako'\leqq'\tmHako'+2' \]
である。また,\eqref{E2114200510i}を満たすある$x$について\eqref{E2114200510ii}が
満たされるとき,実数$a$の範囲は
\[ \tmHako'-2'\tmHako'<'a\tmHako'<'\tmHako'+3' \]
である。
\closeHakoKaiFile
\begin{Kaitou}
      \HakoKaiKata{t}
      \inputHakoKaiFile
\end{Kaitou}
\end{document}
