\documentclass{jarticle}
    \usepackage{moreverb}
    \usepackage{emath}
    \usepackage{showexample}

\begin{document}
\textsf{showexample.sty} は \LaTeX のソースとタイプセットした結果を
対比して示すためのスタイルファイルです.
\section{\texttt{showEx}環境}
\texttt{showEx}環境の使用例です。
\subsection{基本例}
例えば,
\begin{dblbox}
\begin{verbatim}
\begin{showEx}{簡単な例}
    $\frac12$\vspace{1ex}
    
    $\dfrac12$
\end{showEx}
\end{verbatim}
\end{dblbox}
と記述すると
\begin{showEx}{簡単な例}
    $\frac12$\vspace{1ex}

    $\dfrac12$
\end{showEx}
\texttt{showEx}環境内に記述されたものが
\begin{jquote}
左側には,ソースリストが角の丸い囲みの中に表示され,\\
右側には,タイプセットした結果が影付きの囲みの中に表示されます.
\end{jquote}
今までは,対比すべき部分を別ファイルにして \cmd{showexample} コマンドで
読み取っていましたが,今回新設した \texttt{showEx}環境では,
環境内に記述した部分が対比の対象となります.
\clearpage

\subsection{\texttt{showEx}の書式}
\texttt{showEx}環境の書式です.
\begin{boxnote}
\begin{verbatim}
\begin{showEx}(#1,#2)#3
    (#1,#2) はオプション引数
        #1 : 左側ソース部分横幅
             \linewidth の #1倍 (0<#1<=1)デフォルト0.45
        #2 : 右側結果表示部横幅 
             \linewidth の #2倍 (0<#2<=1)デフォルト0.45
    #3 : 左側ソース部分表題は必須の引数
        (表題なしのときは
                \begin{showExample}{}
          と,空文字列を与える.)
\end{verbatim}
\end{boxnote}

\begin{description}
\item[(注)] \texttt{showEx}環境は,
一時ファイル \textsf{showex.tmp} を作ります.
終了後もこのファイルは削除されません.
\end{description}
\subsection{表題が不要の場合}
表題が不要の場合の例です.
\begin{dblbox}
\begin{verbatim}
\begin{showEx}{}
    $\dfrac12$
\end{showEx}
\end{verbatim}
\end{dblbox}
と記述すると
\begin{showEx}{}
    $\dfrac12$
\end{showEx}
\clearpage

\subsection{左右の囲み幅の変更}
左右の囲みの横幅を変更したいときは,(左幅,右幅) オプション引数を与えます.
\begin{dblbox}
\begin{verbatim}
\begin{showEx}(.6,.3){\texttt{(.,.)}オプション}
    $\bunsuu12$
\end{showEx}
\end{verbatim}
\end{dblbox}
と記述すると
\begin{showEx}(.6,.3){\texttt{(.,.)}オプション}
    $\bunsuu12$
\end{showEx}
\medskip

枠が大きくなり,$(左幅)+(右幅)> 0.95$ になると,
1行には出来ませんから,上下2段になります.
\begin{dblbox}
\begin{verbatim}
\begin{showEx}(.75,.75){$(左幅)+(右幅)> 0.95$}}
    \begin{equation}x+y=1\end{equation}
\end{showEx}
\end{verbatim}
\end{dblbox}
と記述すると
\begin{showEx}(.75,.75){$(左幅)+(右幅)> 0.95$}
    \begin{equation}x+y=1\end{equation}
\end{showEx}
\clearpage

(右幅)が 0.95 を超えると矢印の位置・向きが変わります.
\begin{dblbox}
\begin{verbatim}
\begin{showEx}(.75,1){$(右幅)> 0.95$}
    \begin{equation}x+y=1\end{equation}
\end{showEx}
\end{verbatim}
\end{dblbox}
と記述すると
\begin{showEx}(.75,1){$(右幅)> 0.95$}
    \begin{equation}x+y=1\end{equation}
\end{showEx}

\subsection{ソース部に行番号}
左側のリストに行番号を付与するには,
\textsf{showEx}に\verb+[listing]+オプションを付与します。
ただし,この機能は \textsf{moreverb.sty}を前提としますので,
\begin{jquote}
\begin{verbatim}
\usepackage{showexample}
\end{verbatim}
\end{jquote}
に先立って
\begin{jquote}
\begin{verbatim}
\usepackage{moreverb}
\end{verbatim}
\end{jquote}
としておかねばなりません。

\begin{dblbox}
\begin{verbatim}
\begin{showEx}[listing](.5,.44){行番号附加オプション}
あああ
\abovedisplayskip=0pt
\belowdisplayskip=0pt
\[ y=f(x) \]
いいい
\end{showEx}
\end{verbatim}
\end{dblbox}
と記述すると

\begin{showEx}[listing](.5,.44){行番号附加オプション}
あああ
\abovedisplayskip=0pt
\belowdisplayskip=0pt
\[ y=f(x) \]
いいい
\end{showEx}

\clearpage

\section{\cmd{showex}コマンド}
ファイルから読み取った部分を対訳形式にするコマンド \cmd{showexample} も
以前の通り使えますが,\textsf{showEx} 環境と書式がずれてしまいましたので,
\cmd{showex} コマンドを新設しました.

\subsection{\cmd{showex}コマンドの使用例}
ファイル \texttt{ex1.tex} の内容が
\begin{jquote}
\begin{verbatim}
\begin{enumerate}[(1)]
    \item $\frac12$
    \item $\dfrac12$
    \item $\bunsuu12$
\end{enumerate}
\end{verbatim}
\end{jquote}
のとき,
\begin{dblbox}
\begin{verbatim}
    \showex(.6,.3){\cmd{showex}コマンド}{ex1}
\end{verbatim}
\end{dblbox}
と記述すると

    \showex(.6,.3){\cmd{showex}コマンド}{ex1}

\subsection{\cmd{showex}コマンドの書式}
\begin{boxnote}
\begin{verbatim}
\showex(#1,#2)#3#4
    (#1,#2) はオプション引数
        #1 : 左側ソース部分横幅
             \linewidth の #1倍 (0<#1<=1)デフォルト0.45
        #2 : 右側結果表示部横幅 
             \linewidth の #2倍 (0<#2<=1)デフォルト0.45
    #3 : 左側ソース部分表題は必須の引数
    #4 : 読み込むファイル名
        (ただし,拡張子は .tex でなければならず,
          引数には拡張子をつけずに記述します.)
\end{verbatim}
\end{boxnote}
\end{document}
