\documentclass[a4j]{jarticle}
\usepackage{emathPp}
\usepackage{emathPxy}
\usepackage{showexample}

\checkPerl

\begin{document}
\section{\textsf{picture}環境との共存}
\textsf{emathPxy}を導入すると,\textsf{picture}環境に対して,
\cmd{unitlength}が無効となっていましたが,これを修正しました。

\begin{showEx}(0.54,0.4){\textsf{picture}環境との共存}
\unitlength10mm
\begin{picture}(2,2)
\put(1,1){\circle{2}}
\put(0,0){%
  \drawline(0,0)(2,0)(2,2)(0,2)(0,0)}
\end{picture}
\end{showEx}

ただし,\textsf{picture}環境に対して,\texttt{yscale}オプションなどが
有効になる,ということではありません。\textsf{emathPxy}を読み込んだだけで
\cmd{unitlength}が無効になる事態を修正したということです。
\cmd{yscale}などが必要な場合は,\textsf{picture}環境ではなく,
\textsf{zahyou*}環境をご利用ください。

\begin{showEx}(0.54,0.4){\textsf{zahyou*}環境で\texttt{yscale}オプション}
\begin{zahyou*}%
[ ul=10mm,
  yscale=.6667
](0,3)(0,4)%
  \Put\O{\kousi{3}{4}}%
\end{zahyou*}
\end{showEx}

\section{周囲に余白}
\textsf{emathPh v 0.93}で新設した「周囲に余白を付ける」機能を
\textsf{emathPxy}でも利用することができます。

以下の例では,\TeX が認識している\textsf{zahyou}環境のサイズを
示すために\cmd{framebox}で囲んでいます。


\begin{showEx}(0.5,0.44){余白なし}
\fboxsep=0pt\fbox{%
\begin{zahyou}[%
  ul=3mm,yscale=4,
  yokozikuhaiti={[e]},
  tatezikuhaiti={[n]}
](-1,15)(-1.5,1.5)
\zahyouMemori[z][n]<dx=\Pih>
\YGurafu*{exp(-X/6)}
\YGurafu*{-exp(-X/6)}
\YGurafu*{exp(-X/6)*sin(X)}
\ymemori{1}
\ymemori{-1}
\xMemori<\frac{\pi}{2}>{1}
\xMemori<\pi>{2}
\end{zahyou}}
\end{showEx}

座標軸名などの文字が枠外に飛び出しています。
余白を付加してみましょう。

\begin{showEx}(0.5,0.44){余白を付加}
\fboxsep=0pt\fbox{%
\begin{zahyou}[%
  ul=3mm,yscale=4,
  yokozikuhaiti={[e]},
  tatezikuhaiti={[n]},
  Migiyohaku=10pt,
  Ueyohaku=10pt,
  Hidariyohaku=12pt
](-1,15)(-1.5,1.5)
\zahyouMemori[z][n]<dx=\Pih>
\YGurafu*{exp(-X/6)}
\YGurafu*{-exp(-X/6)}
\YGurafu*{exp(-X/6)*sin(X)}
\ymemori{1}
\ymemori{-1}
\xMemori<\frac{\pi}{2}>{1}
\xMemori<\pi>{2}
\end{zahyou}}
\end{showEx}
文字も枠内に納まりました。

\section{\textsf{zahyou}環境の書式}
\begin{boxnote}
\begin{verbatim}
    \begin{zahyou}[#1](#2,#3)(#4,#5)
      #1 : key=val の形式で,key には次のものが使えます。
          yokozikukigou デフォルトは $x$
          tatezikukigou デフォルトは $y$
          gentenkigou   デフォルトは O
          yokozikuhaiti デフォルトは (0,-3pt)[rt]
          tatezikuhaiti デフォルトは (-3pt,0)[rt]
          gentenhaiti   デフォルトは [sw]
          ul            \unitlength を変更します。デフォルトは 1pt
          yscale        デフォルトは1
          xscale        デフォルトは1
          arrowheadsize 矢印のサイズ(その時点のものに対する比率)
          zikusensyu    座標軸の線種
                        デフォルトは \arrowdrawline
          ueyohaku      右辺値は無名数(単位は\unitlength)
          sitayohaku
          migiyohaku
          hidariyohaku
          Ueyohaku      右辺値は単位を伴う長さ
          Sitayohaku
          Migiyohaku
          Hidariyohaku
      (#2,#3) : xの範囲
      (#4,#5) : yの範囲
\end{verbatim}
\end{boxnote}
\end{document}
