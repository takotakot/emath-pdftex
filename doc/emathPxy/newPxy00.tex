\documentclass[a4j]{jarticle}
\usepackage{emathPp}
\usepackage{emathPxy}
\usepackage{showexample}

\checkPerl

\begin{document}
$xy$平面の$x$軸,$y$軸の単位の長さを不揃いにすることを許容する試みです。
関連して,座標目盛りについての機能を補強しました。

新たに \textsf{emathPxy.sty} と名付けることとしましたが,
まだアルファテスト段階です。
バグ報告をBBSでお待ちしています。

\tableofcontents
\clearpage

\section{座標目盛り}
現在の座標目盛りは0を中心として$x$, $y$軸共通の間隔で
目盛線を入れています。

\begin{showEx}(.5,.44){間隔2の目盛線}
\small
\begin{zahyou}[ul=4mm](-5,5)(-5,5)
\zahyouMemori[g]<2>
\end{zahyou}
\end{showEx}

\subsection{目盛間隔の指定オプション}
今回,目盛間隔を$x$軸,$y$軸で別々の値を指定できるようにしました。
\verb/<#3>/の部分に
\begin{jquote}
\begin{verbatim}
dx=2,dy=3
\end{verbatim}
\end{jquote}
などと記述します。デフォルト値は\verb/dx/, \verb/dy/ともに1です。

\begin{showEx}(.5,.44){目盛間隔の指定}
\small
\begin{zahyou}[ul=4mm](-5,5)(-5,5)
\zahyouMemori[g]<dx=2>
\end{zahyou}
\end{showEx}

間隔は整数でなくても受け付けますが,目盛の数値は別途\cmd{xmomori}などで
打った方がよいでしょう。

\begin{showEx}(.5,.44){間隔指定}
\footnotesize
\begin{zahyou}[ul=5mm](-5,5)(-1.5,2)
\zahyouMemori[g][n]<dx=1.57,dy=.5>
\ymemori{1}
\ymemori{-1}
\xmemori<\pi>{3.14}
\xmemori<-\pi>{-3.14}
\end{zahyou}
\end{showEx}

\cmd{xmemori}も\texttt{dx}で指定した間隔で目盛を打つために
機能を強化したものを\cmd{xMemori}と号します。

下のリストは,$x$軸方向$\mbox{\cmd{Pis}}=\pi/6$間隔で目盛を入れています。

\begin{showEx}(.5,.44){\cmd{xmemori}}
\footnotesize
\begin{zahyou}[ul=6mm](-1,7)(-1.5,2)
\zahyouMemori[g][n]<%
dx=\Pis,dy=.5,sensyu=\drawline>
\ymemori{1}
\ymemori{-1}
\xMemori<\frac{\pi}{6}>{1}
\xMemori<\frac{\pi}{3}>{2}
\xMemori<\frac{\pi}{2}>{3}
\xMemori<\pi>{6}
\xMemori<\frac{3\pi}{2}>{9}
\xMemori<2\pi>(\Pis){12}
\end{zahyou}
\end{showEx}

\subsection{\cmd{xMemori}の書式}
新設した\cmd{xMemori}, \cmd{yMemori}の書式です。

\begin{boxnote}
\begin{verbatim}
座標軸上任意位置に目盛
\xMemori[#1]<#2>(#3)#4
\yMemori[#1]<#2>(#3)#4
  #1 : t = 座標軸の下に目盛(デフォルト)
       b =         上
  #2 : 目盛の文字
       (省略すれば,#4 に指定したものが代用されます.)
  #3 : 単位の長さ(デフォルト値=\zahyouMemoriで指定したdx)
  #4 : 単位の長さの#4倍の位置に目盛を打つ
\end{verbatim}
\end{boxnote}

\subsection{目盛基準点の変更}
通常は原点を基準として目盛を打ちますが,これを変更するオプションも
つけました。すなわち
\begin{jquote}
\begin{verbatim}
<xo=-0.5236,yo=0.5>
\end{verbatim}
\end{jquote}
で基準点が\verb/(-0.5236,0.5)/となります。
\cmd{xMemori}, \cmd{yMemori}もこれに追随します。

\begin{showEx}(.5,.44){\cmd{xmemori}}
\footnotesize
\begin{zahyou}[%
ul=6mm,gentenhaiti={[se]}](%
-1,7)(-1.5,2)
\zahyouMemori[+][n]<xo=-\Pis,dx=\Pih>
\ymemori{1}
\ymemori{-1}
\xMemori<-\frac{\pi}{6}>{0}
\xMemori<\frac{\pi}{3}>{1}
\xMemori<\frac{5\pi}{6}>{2}
\xMemori<\frac{4\pi}{3}>{3}
\xMemori<\frac{11\pi}{6}>{4}
\end{zahyou}
\end{showEx}

\subsection{\cmd{zahyouMemori}の書式}
拡張された\cmd{zahyouMemori}の書式です。

\begin{boxnote}
\begin{verbatim}
座標軸に目盛を打つ.
\zahyouMemori[#1][#2]<#3>
  #1 : g : グリッド
       + : 格子点に +マーク
       o : 格子点に黒丸
       z : 座標軸上の格子点に +マーク
  #2 : n : 目盛り数値なし
  #3 : 刻み
        dx= xの刻み値
        dy= yの刻み値
        xo= xの基準値
        yo= yの基準値
\end{verbatim}
\end{boxnote}

\section{単位長の変更}
\subsection{xscale, yscale オプション}
\textsf{emathPh.sty}で用意している座標系は$x$, $y$軸の単位長は
同一となっています。これを別のものにすることを可能とする試みです。

次のリストは\verb/yscale=4/として,$y$軸方向を4倍引き伸ばしています。

\begin{showEx}{\texttt{<yscale=...>}オプション}
\begin{zahyou}[ul=3mm,yscale=4](%
  -1,15)(-1.5,1.5)
\zahyouMemori[z][n]<dx=\Pih>
\YGurafu*{exp(-X/6)}
\YGurafu*{-exp(-X/6)}
\YGurafu*{exp(-X/6)*sin(X)}
\ymemori{1}
\ymemori{-1}
\xMemori<\frac{\pi}{2}>{1}
\xMemori<\pi>{2}
\end{zahyou}
\end{showEx}

次は横長の格子を描く例です。
\begin{jquote}
\begin{verbatim}
ul=3mm,xscale=3,yscale=2
\end{verbatim}
\end{jquote}
というオプションで
\begin{jquote}
\begin{verbatim}
ul=3mm で \unitlength は 3mm
x軸方向の単位長は ul×xscale 9mm で \xunitlength が 9mm
y軸方向の単位長は ul×yscale 6mm で \yunitlength が 6mm
\end{verbatim}
\end{jquote}
に設定されます。

\begin{showEx}{横長の格子}
\begin{zahyou*}[ul=3mm,xscale=3,%
  yscale=2](0,4)(0,5)%
  \Put\O{\kousi{4}{5}}%
  \Kuromaru{(2,3)}%
\end{zahyou*}%
\end{showEx}

最後に$x$軸方向のスケールを変更する例です。

\begin{showEx}{$x$軸方向のスケール変更}
\calcval{$pi/180}\xs
\begin{zahyou}[%
  ul=6mm,xscale=\xs](%
  -15,400)(-1.5,2)%
\zahyouMemori[g]<%
  sensyu=\protect\drawline,%
  dx=90>
\YGurafu*(1){sin($pi*X/180)}%
\Put\migiT[e]{$y=\sin x\Deg$}%
\end{zahyou}
\end{showEx}

\subsection{\textsf{zahyou}環境の書式}
拡張された\textsf{zahyou}環境の書式です。

\begin{boxnote}
\begin{verbatim}
    \begin{zahyou}[#1](#2,#3)(#4,#5)
      #1 : key=val の形式で,key には次のものが使えます。
          yokozikukigou デフォルトは $x$
          tatezikukigou デフォルトは $y$
          gentenkigou   デフォルトは O
          yokozikuhaiti デフォルトは (0,-3pt)[rt]
          tatezikuhaiti デフォルトは (-3pt,0)[rt]
          gentenhaiti   デフォルトは [sw]
          ul            \unitlength を変更します。デフォルトは 1pt
          yscale        デフォルトは1
          xscale        デフォルトは1
      (#2,#3) : xの範囲
      (#4,#5) : yの範囲
\end{verbatim}
\end{boxnote}

\subsection{適用除外コマンド}
\texttt{xscale}, \texttt{yscale}を指定した座標系には3つの単位長があります。
\begin{jquote}
\begin{verbatim}
\unitlength  : picture 環境の単位長
\xunitlength : zahyou 環境における x軸方向単位長
\yunitlength : zahyou 環境における y軸方向単位長
\end{verbatim}
\end{jquote}

次のコマンドは \cmd{unitlength}にのみ従います。
すなわち,\verb/xscale, yscale/ は無視されます。
\begin{jquote}
\begin{verbatim}
\En, \Enko, \Daen
\yumigata, \oogigata, ...
\end{verbatim}
\end{jquote}

\cmd{En}は\verb/xscale, yscale/の影響を受けません。

\begin{showEx}{\cmd{En}}
\begin{zahyou}[%
ul=8mm,xscale=2,yscale=.5](%
-1.5,1.5)(-4,4)
\zahyouMemori[g]
\En\O{1}
\end{zahyou}
\end{showEx}

\verb/xscale, yscale/を反映させた円を描画したいときは,
媒介変数表示を用います。

\begin{showEx}{円}
\begin{zahyou}[%
ul=8mm,xscale=2,yscale=.5](%
-1.5,1.5)(-4,4)
\zahyouMemori[g]
\BGurafu{cos(T)}{sin(T)}{0}{2*$pi}
\end{zahyou}
\end{showEx}

2次以下の整関数を描画する\cmd{Gurafu}も適用除外です。
\end{document}
