\documentclass{jarticle}
\usepackage{emathMw}

\title{\textsf{emathMw.sty} \\{\normalsize v 0.13}\\
使用例}
\author{tDB}
\date{2005/09/03}

\begin{document}
\maketitle\thispagestyle{empty}
\begin{abstract}%
\parindent1zw%
テキストを図の周りに回りこませるには, \verb/wrapfigure/ 環境があります。
しかし,これは \verb/list/ 環境と相性が悪いのが難点です。

そこで,\verb/list/環境と共存できる\verb/mawarikomi/環境を
作ってみました。

このマクロ集のマクロについてのご質問,バグ報告,修正・追加の提案等は
\begin{center}
http://emath.s40.xrea.com/
\end{center}
の掲示板へどうぞ。
\end{abstract}
\pagebreak
\pagenumbering{roman}%

\tableofcontents

\listoffigures

\listoftables

\pagebreak
\pagenumbering{arabic}
\section{\texttt{mawarikomi} 環境}
\verb/enumerate/ 環境など \verb/list/ 環境と併用して,
テキストを図のまわりに回り込ませる
\verb/mawarikomi/ 環境の使用法は

\begin{verbatim}
    \begin{enumerate}
      \item
        \begin{mararikomi}{図の横幅}{図}
          テキスト
        \end{mawarikomi}
      \item .....
      \item .....
      ...........
    \end{enumerate}
\end{verbatim}
となります。

\medskip
\begin{enumerate}
\item
\begin{mawarikomi}{50pt}{%
    \begin{picture}(50,50)
      \put(0,0){\framebox(50,50){図}}
    \end{picture}}
  ああああああああああああああああああああああああああああ
  ああああああああああああああああああああああああああああ
  ああああああああああああああああああああああああああああ
  ああああああああああああああああああああああああああああ
  ああああああああああああああああああああああああああああ
  ああああああああああああああああああああああああああああ
  ああああああああああああああああああああああああああああ
  ああああああああああああああああああああああああああああ
  ああああああああああああああああああああああああああああ
  ああああああああああああああああああああああああああああ
\end{mawarikomi}
\item いいいいいいいいいいいいいいいいいいいいいいいいい
  いいいいいいいいいいいいいいいいいいいいいいいいいいいい
\end{enumerate}
\clearpage


図が大きくて複数の問題で回り込みを行わせるには,

\begin{verbatim}
    \begin{enumerate}
      \item
        \begin{mawarikomi}{図の横幅}{図}
          第1問
        \end{mawarikomi}
      \item 
        \begin{mawarikomi*}
          第2問
        \end{mawarikomi*}
      \item 
        \begin{mawarikomi*}
          第3問
        \end{mawarikomi*}
      \item 第4問
      ...........
    \end{enumerate}
\end{verbatim}
と回り込みが続く問題をを \verb/mawarikomi*/ 環境でくるみます。

\begin{enumerate}
  \item\label{hukusuumon}
    \begin{mawarikomi}{50pt}{%
      \begin{picture}(50,150)
        \put(0,0){\framebox(50,150){図}}
      \end{picture}}
      ああああああああああああああああああああああああああああ
      $\displaystyle\frac{\sqrt3}{2}$
    \end{mawarikomi}
  \item 
    \begin{mawarikomi*}
      いいいいいいいいいいいいいいいいいいいいいいいいい
      いいいいいいいいいいいいいいいいいいいいいいいいいいいい
    \end{mawarikomi*}
  \item 
    \begin{mawarikomi*}
      ううううううううううううううううううううううううう
      うううううううううううううううううううううううううううう
    \end{mawarikomi*}
  \item 
    \begin{mawarikomi*}
      えええええええええええええええええええええええええ
      ええええええええええええええええええええええええええええ
      ええええええええええええええええええええええええええええ
      ええええええええええええええええええええええええええええ
      ええええええええええええええええええええええええええええ
      ええええええええええええええええええええええええええええ
      ええええええええええええええええええええええええええええ
    \end{mawarikomi*}
\item おおおおおおおおおおおおおおおおおおおおおおおおお
  おおおおおおおおおおおおおおおおおおおおおおおおおおおお
  おおおおおおおおおおおおおおおおおおおおおおおおおおおお
\end{enumerate}

この場合,テキスト部に \verb/display/ 数式があったりする関係で
回り込みのタイミングがずれてきました。
この修正法については後述します。
(p.\pageref{gyousuusitei}, 第\ref{gyousuusitei}節)

\section{\texttt{mawarikomi}環境内に\texttt{list}環境}
\verb/mawarikomi/環境内に\verb/list/環境を入れることができます。

\begin{verbatim}
   \item
     \begin{mawarikomi}{図の横幅}{図}
       問題文
       \begin{enumerate}
          \item 小問1
          \item 小問2
          \item ....
          .....
       \end{enumerate}
     \end{mawarikomi}
\end{verbatim}
という書式になります。

\begin{enumerate}
\item
  \begin{mawarikomi}{50pt}{%
    \begin{picture}(50,100)
      \put(0,0){\framebox(50,100){図}}
    \end{picture}}
    次の問いに答えよ。おおおおおおおおおおおおおおおおおおお
    おおおおおおおおおおおおおおおおおおおおおおおおおおおお
    \begin{enumerate}
      \item 小問1 アアアアアアアアアアアアアアアア
        アアアアアアアアアアアアアアアアアアアアア
      \item 小問2 イイイイイイイイイイイイイイイイイイイイイ
        イイイイイイイイイイイイイイイイイイイイイイイイイイ
        イイイイイイイイイイイイイイイイイイイイイイイイイイ
        イイイイイイイイイイイイイイイイイイイイイイイイイイ
        イイイイイイイイイイイイイイイイイイイイイイイイイイ
      \item 小問3 ウウウウウウウウウウウウウウウウウウウウウ
        ウウウウウウウウウウウウウウウウウウウウウウウウウウ
        ウウウウウウウウウウウウウウウウウウウウウウウウウウ
    \end{enumerate}
  \end{mawarikomi}
\item 第2問 いいいいいいいいいいいいいいいいいいいいいいい
  いいいいいいいいいいいいいいいいいいいいいいいいいいいい
\end{enumerate}

\clearpage

さらに \verb/enumerate/環境を深くして,小問1が更なる小問---以下,枝問と
呼ぶことにします---をもたせることもできます。

\begin{verbatim}
   \item
     \begin{mawarikomi}{図の横幅}{図}
       問題文
       \begin{enumerate}
        \item 小問1
          \begin{enumerate}
            \item 枝問1
            \item 枝問2
            ...........
          \end{enumerate}
        \item 小問2
        \item ....
        .....
       \end{enumerate}
     \end{mawarikomi}
\end{verbatim}

\begin{enumerate}
\item
  \begin{mawarikomi}{150pt}{%
    \begin{picture}(150,90)
      \put(0,0){\framebox(150,90){図}}
    \end{picture}}
    次の問いに答えよ。
    \begin{enumerate}
      \item 
      小問1 アアアアアアアアアアアアアアアアアアアアア
      \begin{enumerate}
        \item 枝問1 かかかかかかかかかかかかかかかか
        \item 枝問2 きききききききききききききききき
                    きききききききききききききききき
                    きききききききききききききききき
                    きききききききききききききききき
                    きききききききききききききききき
      \end{enumerate}
      \item 小問2 イイイイイイイイイイイイイイイイイイイイイ
        イイイイイイイイイイイイイイイイイイイイイイイイイイ
        イイイイイイイイイイイイイイイイイイイイイイイイイイ
      \item 小問3 ウウウウウウウウウウウウウウウウウウウウウ
        ウウウウウウウウウウウウウウウウウウウウウウウウウウ
        ウウウウウウウウウウウウウウウウウウウウウウウウウウ
    \end{enumerate}
  \end{mawarikomi}
\item 第2問 いいいいいいいいいいいいいいいいいいいいいいい
    いいいいいいいいいいいいいいいいいいいいいいいいいいいい
\end{enumerate}
\clearpage

\section{回り込みの行数指定オプション}\label{gyousuusitei}
回り込みを行う行数は,図の高さから計算で求めています。
しかし,テキスト部分の状態によっては不適切な場合もでてきます。
そこで\verb/mawarikomi/ 環境に \verb/[...]/ オプションで
行数の指定ができるようにしてあります。
\pageref{hukusuumon}ページの例では,
計算では11行となっていますが,テキスト部分の状況から
10行とした方が良いようです。

下の例では,
\begin{verbatim}
  \begin{mawarikomi}[10]{50pt}{図}
\end{verbatim}
としています。

なお,この指定法は相対指定も可能で,数値に `+' `$-$' を付けることにより
計算で得られた行数を増減させます。下の例を
\begin{verbatim}
  \begin{mawarikomi}[-1]{50pt}{図}
\end{verbatim}
としても同じ結果が得られます。

\begin{enumerate}
  \item
    \begin{mawarikomi}[10]{50pt}{%
      \begin{picture}(50,150)
        \put(0,0){\framebox(50,150){図}}
      \end{picture}}
      ああああああああああああああああああああああああああああ
      $\displaystyle\frac{\sqrt3}{2}$
    \end{mawarikomi}
  \item 
    \begin{mawarikomi*}
      いいいいいいいいいいいいいいいいいいいいいいいいい
      いいいいいいいいいいいいいいいいいいいいいいいいいいいい
    \end{mawarikomi*}
  \item 
    \begin{mawarikomi*}
      ううううううううううううううううううううううううう
      うううううううううううううううううううううううううううう
    \end{mawarikomi*}
  \item 
    \begin{mawarikomi*}
      えええええええええええええええええええええええええ
      ええええええええええええええええええええええええええええ
      ええええええええええええええええええええええええええええ
      ええええええええええええええええええええええええええええ
      ええええええええええええええええええええええええええええ
      ええええええええええええええええええええええええええええ
      ええええええええええええええええええええええええええええ
    \end{mawarikomi*}
\item おおおおおおおおおおおおおおおおおおおおおおおおお
  おおおおおおおおおおおおおおおおおおおおおおおおおおおお
  おおおおおおおおおおおおおおおおおおおおおおおおおおおお
\end{enumerate}
\clearpage

\section{テキストと図の間隔}
テキスト部分と図の部分との横間隔はデフォルトで
\begin{verbatim}
  \newlength{\mawarikomisep}%
  \setlength{\mawarikomisep}{2pt}%
\end{verbatim}
としてあり,その2倍\verb/4pt/が空くようにしてあります。

次の例では,
\begin{verbatim}
  \setlength{\mawarikomisep}{1zw}%
\end{verbatim}
として,その2倍\verb/2zw/の空きを作ってみました。

\begin{enumerate}
\item
  \setlength{\mawarikomisep}{1zw}%
  \begin{mawarikomi}{50pt}{
    \begin{picture}(50,100)
      \put(0,0){\framebox(50,100){図}}
    \end{picture}}
  次の問いに答えよ。
  \begin{enumerate}
    \item 小問1 アアアアアアアアアアアアアアアアアアアアア
      アアアアアアアアアアアアアアアアアアアアアアアアアア
    \item 小問2 イイイイイイイイイイイイイイイイイイイイイ
      イイイイイイイイイイイイイイイイイイイイイイイイイイ
      イイイイイイイイイイイイイイイイイイイイイイイイイイ
      イイイイイイイイイイイイイイイイイイイイイイイイイイ
      イイイイイイイイイイイイイイイイイイイイイイイイイイ
    \item 小問3 ウウウウウウウウウウウウウウウウウウウウウ
      ウウウウウウウウウウウウウウウウウウウウウウウウウウ
      ウウウウウウウウウウウウウウウウウウウウウウウウウウ
  \end{enumerate}
  \end{mawarikomi}
\item 第2問 いいいいいいいいいいいいいいいいいいいいいいい
  いいいいいいいいいいいいいいいいいいいいいいいいいいいい
\end{enumerate}
\clearpage

\section{図の位置の微調整}
図の位置を細かく調整したいことがあります。そのために
\verb/mawarikomi/環境に\verb/(x,y)/オプションを用意しました。
一例です。

\begin{enumerate}
\item
  \begin{mawarikomi}{120pt}{%
    \begin{picture}(100,50)%
      \put(0,0){\fboxsep0pt\framebox(100,50){図}}%
    \end{picture}}
  第1問あああああああああああああああああああああああああ
    ああああああああああああああああああああああああああああ
    ああああああああああああああああああああああああああああ
    ああああああああああああああああああああああああああああ
    ああああああああああああああああああああああああああああ
  \end{mawarikomi}
\item 第2問 いいいいいいいいいいいいいいいいいいいいいいい
    いいいいいいいいいいいいいいいいいいいいいいいいいいいい
\end{enumerate}

図をもう少し右上に動かしたいですね。
\verb/mawarikomi/環境に\verb/(10pt,2pt)/ オプションを付加してみました。

\begin{enumerate}
\item
  \begin{mawarikomi}(10pt,2pt){120pt}{%
    \begin{picture}(100,50)%
      \put(0,0){\fboxsep0pt\framebox(100,50){図}}%
    \end{picture}}
  第1問あああああああああああああああああああああああああ
    ああああああああああああああああああああああああああああ
    ああああああああああああああああああああああああああああ
    ああああああああああああああああああああああああああああ
    ああああああああああああああああああああああああああああ
  \end{mawarikomi}
\item 第2問 いいいいいいいいいいいいいいいいいいいいいいい
    いいいいいいいいいいいいいいいいいいいいいいいいいいいい
\end{enumerate}
\clearpage

\section{段落途中からの回り込み}
\verb/<n>/ オプションで段落の始めからn行は回り込みをしないように
指定することができます。下の例は\verb/<1>/としたものです。

\begin{enumerate}
\item
  \begin{mawarikomi}<1>{100pt}{%
    \begin{picture}(100,50)%
      \put(0,0){\fboxsep0pt\framebox(100,50){図}}%
    \end{picture}}
  第1問あああああああああああああああああああああああああ
    ああああああああああああああああああああああああああああ
    ああああああああああああああああああああああああああああ
    ああああああああああああああああああああああああああああ
    ああああああああああああああああああああああああああああ
    ああああああああああああああああああああああああああああ
    ああああああああああああああああああああああああああああ
  \end{mawarikomi}
\item 第2問 いいいいいいいいいいいいいいいいいいいいいいい
    いいいいいいいいいいいいいいいいいいいいいいいいいいいい
\end{enumerate}

ただしテキスト部に数式があるなど行ピッチが変わると,
図の位置の調整が必要になります。下の例では,
2行目の数式を回り込みの対象外にするため,
\texttt{<2>}オプションをつけました。
テキスト部はお望み通り3行目から回り込みが始まっていますが,
図の位置が不適切です。

\begin{enumerate}
\item
  \begin{mawarikomi}<2>{100pt}{%
    \begin{picture}(100,50)%
      \put(0,0){\fboxsep0pt\framebox(100,50){図}}%
    \end{picture}}
  第1問ああああああああああああああああああ
    ああああああああああああああああああああああああああああ
    $f(x)=\displaystyle\frac{1}{
      1+\displaystyle\frac{1}{1+\displaystyle\frac{1}{x}}}~(x>0)$
    ああああああああああああああああああああああああああああ
    ああああああああああああああああああああああああああああ
    ああああああああああああああああああああああああああああ
    ああああああああああああああああああああああああああああ
    ああああああああああああああああああああああああああああ
    ああああああああああああああああああああああああああああ
  \end{mawarikomi}
\item 第2問 いいいいいいいいいいいいいいいいいいいいいいい
    いいいいいいいいいいいいいいいいいいいいいいいいいいいい
\end{enumerate}

では,\verb/(0,-24pt)/オプションで図を下に動かしてみます。

\clearpage

\begin{enumerate}
\item
  \begin{mawarikomi}<2>(0,-24pt){100pt}{%
    \begin{picture}(100,50)%
      \put(0,0){\fboxsep0pt\framebox(100,50){図}}%
    \end{picture}}
  第1問ああああああああああああああああああ
    ああああああああああああああああああああああああああああ
    $f(x)=\displaystyle\frac{1}{
      1+\displaystyle\frac{1}{1+\displaystyle\frac{1}{x}}}~(x>0)$
    ああああああああああああああああああああああああああああ
    ああああああああああああああああああああああああああああ
    ああああああああああああああああああああああああああああ
    ああああああああああああああああああああああああああああ
    ああああああああああああああああああああああああああああ
    ああああああああああああああああああああああああああああ
  \end{mawarikomi}
\item 第2問 いいいいいいいいいいいいいいいいいいいいいいい
    いいいいいいいいいいいいいいいいいいいいいいいいいいいい
\end{enumerate}

\section{caption の使用}
\verb/mawarikomi/環境内では,\verb/\caption/を使用できません。
そこで,
\begin{verbatim}
    Fmawarikomi 環境:図番号を使用
    Tmawarikomi 環境:表番号を使用
\end{verbatim}
なる2つの環境を用意してあります。
下の例では,\verb/Tmawarikomi/環境,\verb/Fmawarikomi/環境
内で\verb/\caption/を使用しています。\bigskip

\begin{enumerate}
\item
  \begin{Tmawarikomi}(0,5pt){9zw}{%
    \caption{表の例}\label{T1}
    \hfil
    \begin{tabular}{|c|c|c|}\hline
      aaa & bbb & ccc \\\hline
      1 & 2 & 3 \\\hline
    \end{tabular}
    }
  第1問 右の表\ref{T1}において
    ああああああああああああああああああああああああああああ
    ああああああああああああああああああああああああああああ
    ああああああああああああああああああああああああああああ
    ああああああああああああああああああああああああああああ
    ああああああああああああああああああああああああああああ
  \end{Tmawarikomi}
\item 
    \begin{Fmawarikomi}{100pt}{%
      \unitlength1pt%
      \begin{picture}(100,50)%
        \put(0,0){\line(2,1){100}}%
      \end{picture}%
      \caption{図の例}\label{Fig1}
      }
    第2問 右の図\ref{Fig1}において,
    いいいいいいいいいいいいいいいいいいいいいいいいいいいい
    いいいいいいいいいいいいいいいいいいいいいいいいいいいい
    いいいいいいいいいいいいいいいいいいいいいいいいいいいい
    いいいいいいいいいいいいいいいいいいいいいいいいいいいい
    いいいいいいいいいいいいいいいいいいいいいいいいいいいい
    いいいいいいいいいいいいいいいいいいいいいいいいいいいい
    いいいいいいいいいいいいいいいいいいいいいいいいいいいい
    \end{Fmawarikomi}
\end{enumerate}
\bigskip

なお別法として,\verb/mawarikomi/環境において,
\begin{verbatim}
   \fgcaption
   \tbcaption
\end{verbatim}
コマンドで \verb/\caption/ の代行をさせることも可能です。

\section{書式}
\verb/mawarikomi/環境の書式です。

\begin{verbatim}
\begin{mawarikomi}<#1>[#2](#3,#4)#5#6
   #1 : 段落当初の回り込みをしない行数
   #2 : 回り込み行数(相対指定可)
        先頭に`l'を付けたときは図を左に配置
   (#3,#4) : 図の位置修正ベクトル
                #3 : (>0) 右,(<0) 左
                #4 : (>0) 上,(<0) 下
            #3,#4 共に単位必須(0のみは単位不要)
   #5 : 図の横幅(単位必須)
   #6 : 図,表など

関連パラメータ
   \mawarikomisep テキストと図の間隔(デフォルト値 2pt)
     この2倍の空きができます。
\end{verbatim}
\clearpage

\section{図の左配置}
行数指定オプション\verb/[#2]/で,引数の先頭に`l' (\underline{l}eft)をつけると
図などが左に配置されます。

\medskip
\begin{enumerate}
\item
\begin{mawarikomi}[l]{50pt}{%
    \begin{picture}(50,50)
      \put(0,0){\framebox(50,50){図}}
    \end{picture}}
  ああああああああああああああああああああああああああああ
  ああああああああああああああああああああああああああああ
  ああああああああああああああああああああああああああああ
  ああああああああああああああああああああああああああああ
  ああああああああああああああああああああああああああああ
  ああああああああああああああああああああああああああああ
  ああああああああああああああああああああああああああああ
  ああああああああああああああああああああああああああああ
  ああああああああああああああああああああああああああああ
  ああああああああああああああああああああああああああああ
\end{mawarikomi}
\item いいいいいいいいいいいいいいいいいいいいいいいいい
  いいいいいいいいいいいいいいいいいいいいいいいいいいいい
\end{enumerate}
\clearpage

図が大きくて複数の問題で回り込みを行わせるには,
継続して回り込ませる問題を \verb/mawarikomi*/ 環境で
くるむ必要があります。

\begin{verbatim}
    \begin{enumerate}
      \item
        \begin{mawarikomi}{図の横幅}{図}
          第1問
        \end{mawarikomi}
      \item 
        \begin{mawarikomi*}
          第2問
        \end{mawarikomi*}
      \item 第3問
      ...........
    \end{enumerate}
\end{verbatim}

下の例では,\verb/[l-2]/ として,回り込み行数を
計算値に比して2行少なく指定しています。

\begin{enumerate}
\item
\begin{mawarikomi}[l-2]{50pt}{%
    \begin{picture}(50,150)
      \put(0,0){\framebox(50,150){図}}
    \end{picture}}
ああああああああああああああああああああああああああああ
$\displaystyle\frac{\sqrt3}{2}$
\end{mawarikomi}
\item 
\begin{mawarikomi*}
いいいいいいいいいいいいいいいいいいいいいいいいい
いいいいいいいいいいいいいいいいいいいいいいいいいいいい
\end{mawarikomi*}
\item 
\begin{mawarikomi*}
ううううううううううううううううううううううううう
うううううううううううううううううううううううううううう
\end{mawarikomi*}
\item 
\begin{mawarikomi*}
えええええええええええええええええええええええええ
ええええええええええええええええええええええええええええ
ええええええええええええええええええええええええええええ
ええええええええええええええええええええええええええええ
ええええええええええええええええええええええええええええ
\end{mawarikomi*}
\item おおおおおおおおおおおおおおおおおおおおおおおおお
おおおおおおおおおおおおおおおおおおおおおおおおおおおお
おおおおおおおおおおおおおおおおおおおおおおおおおおおお
\end{enumerate}
\end{document}
