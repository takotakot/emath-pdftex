\documentclass{jarticle}
\usepackage{emathB}
\usepackage{showexample}

\begin{document}
分数 \texttt{67/555} を小数展開すると \texttt{0.1207207207...} と
循環小数になります.\textsf{emathB.sty} では,循環小数は
循環節を\texttt{(...)} でくくって表現します.この例では
\texttt{0.1(207)}と表すこととなります.
分数形式を循環小数形式に変換するマクロが
\verb/\FtoZ/ です.

\begin{showEx}{\cmd{FtoZ}}
\FtoZ{67/555}\tmp
\tmp
\end{showEx}

この結果を通常の表現形式にするのが \cmd{prZ} です.

\begin{showEx}{\cmd{prZ}}
\FtoZ{67/555}\tmp
$\prZ\tmp$
\end{showEx}

逆に循環小数形式を分数形式に変換するコマンドが \cmd{ZtoF} です.

\begin{showEx}{\cmd{ZtoF}}
\ZtoF{0.1(207)}\tmp
\tmp
\end{showEx}

この分数形式を通常の表現形式にするのが \cmd{prF} です.

\begin{showEx}{\cmd{prF}}
\ZtoF{0.1(207)}\tmp
$\prF\tmp$
\end{showEx}
\end{document}
