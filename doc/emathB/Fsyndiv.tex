\documentclass[a4j]{jarticle}
\usepackage{emathB}
\usepackage{showexample}

\begin{document}
\section{\cmd{Fsyndiv}の揃え}
\subsection{デフォルトの揃えを中央に変更}
\cmd{Fsyndiv}において,係数を右揃えにしていましたが,
分数と整数を縦に揃えるには右揃えよりも,
中央揃えが適切と思われますから,
デフォルトでは,中央揃えとするように仕様を変更しました。
\begin{showEx}(.54,.4){デフォルトは中央揃え}
$\Fsyndiv{3,1,0,2}{-2/3}$
\end{showEx}

\subsection{右揃えにするオプション}
右揃えにするには,\verb+<soroe=r>+を附加します。

\begin{showEx}(.54,.4){\texttt{<soroe=r>}}
$\Fsyndiv<soroe=r>{3,1,0,2}{-2/3}$
\end{showEx}

縦に揃ってないですか ?

これは,\TeX において,分数と整数を右揃えにする仕様です。
それを見ていただきましょう。

\begin{showEx}(.54,.4){\textsf{array}環境の右揃え}
$\begin{array}{rr}
  -2 & 2 \\
  \bunsuu25 & \bunsuu25
\end{array}$
\end{showEx}

\TeX は,これで「右揃え」をした,といっているのです。

数字を揃えたい,というご希望が出そうなので
\verb+<soroe=R>+オプションを用意しました。

\begin{showEx}(.54,.4){\texttt{<soroe=R>}}
$\Fsyndiv<soroe=R>{3,1,0,2}{-2/3}$
\end{showEx}

お気付きのことと思いますが,$\bunsuu{14}{9}$のように,
分子・分母の桁数が異なるときは中央揃えが標準ですが,
\verb+<soroe=r/R>+オプションをつけたときは,ここも
右揃えにこだわっています。
\newpage

\subsection{答行の縦位置}
答えの行に分数が含まれるとき,その分数と上の横罫線がついてしまいます。
これは\textsf{array}環境の仕様で,\verb+\arraystretch+を変更するなどの
対策もあります。

\begin{showEx}(.54,.4){\cmd{arraystretch}の変更}
\def\arraystretch{1.33}%
$\Fsyndiv{3,1,0,2}{-2/3}$
\end{showEx}

ただ,分数の上下の空き具合がいまひとつ,ということで\cmd{Fsyndiv}に
\verb+<kotaegyousityuu=..>+オプションをつける方法も用意しました。

\begin{showEx}(.54,.4){\texttt{<kotaegyousityuu=\bsityuu>}}
$\Fsyndiv<kotaegyousityuu=\bsityuu>%
  {3,1,0,2}{-2/3}$
\end{showEx}

ここで登場している\cmd{bsityuu}は,分数$\bunsuu12$の高さ,深さに
それぞれ\texttt{4pt, 3pt}を附加した支柱です。

上の例では,3行目の答え行のみに支柱を立てていますが,
3行全てに支柱を立てるオプションは\verb+<gyousityuu=..>+です。

\begin{showEx}(.54,.4){\texttt{<gyousityuu=\bsityuu>}}
$\Fsyndiv<gyousityuu=\bsityuu>%
  {3,1,0,2}{-2/3}$
\end{showEx}

前節の揃えオプションとの併用も可能です。

\begin{showEx}(.54,.4){縦横オプション}
$\Fsyndiv
  <kotaegyousityuu=\bsityuu,soroe=R>%
  {3,1,0,2}{-2/3}$
\end{showEx}
\end{document}
