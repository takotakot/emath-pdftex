\documentclass[a4j]{jarticle}
\usepackage{showexample}
\usepackage{emathB}
\usepackage[continue]{emathAe}

\def\KaitouTTL{\par\hfill (答)~}

\begin{document}
\title{分数計算マクロ\\
emathB.sty {\normalsize ver.0.14}\\使用例}
\author{tDB}
\date{2005/03/24}

\maketitle\thispagestyle{empty}
\begin{abstract}%
\parindent1\zw%
分数の四則計算結果を分数で表現するためのマクロ集です.
\LaTeX2e が前提です.

このマクロ集のマクロについてのご質問,バグ報告,修正・追加の提案等は
\begin{center}
http://emath.s40.xrea.com/
\end{center}
の掲示板へどうぞ。
\end{abstract}
\pagebreak
\pagenumbering{roman}%

\tableofcontents

\pagebreak

\pagenumbering{arabic}

\section{分数の表現}
このマクロでは,分数$\bunsuu{2}{3}$を文字列\verb+"2/3"+と表現します。
帯分数$2\bunsuu45$は文字列\verb+"2;4/5"+です。
この表現形式を通常の表現形式に変換するマクロが\cmd{prF}です

\begin{showEx}{\cmd{prF}}
\begin{gather*}
  \prF{2/3} \\
  \prF{2;4/5}
\end{gather*}
\end{showEx}

\cmd{prF}は与えた形式で表示しますが,
\cmd{prFK}は仮分数表示,\cmd{prFT}は帯分数表示をします。

\begin{showEx}{\cmd{prFK}, \cmd{prFT}}
\begin{gather*}
  \prFK{2;4/5} \\
  \prFT{14/5} \\
\end{gather*}
\end{showEx}

\section{分数の四則}
\subsection{加法}
分数の加法は\cmd{FAdd}で行います。
$\bunsuu14+\bunsuu16$を計算する例です。
\verb+\FAdd{1/4}{1/6}\kotae+により,制御綴\cmd{kotae}には,
文字列"5/12"が入ります。それを\cmd{prF}で$\bunsuu{5}{12}$と表示させます。

\begin{showEx}{\cmd{FAdd}}
  \FAdd{1/4}{1/6}\kotae
  $\prF\kotae$
\end{showEx}

\subsection{減法}
減法は\cmd{FSub}です。
$\bunsuu14-\bunsuu16$を計算する例です。

\begin{showEx}{\cmd{FSub}}
  \FSub{1/4}{1/6}\kotae
  $\prF\kotae$
\end{showEx}

\subsection{乗法}
乗法は\cmd{FMul}です。
$\bunsuu14\times\bunsuu25$を計算する例です。

\begin{showEx}{\cmd{FMul}}
  \FMul{1/4}{2/5}\kotae
  $\prF\kotae$
\end{showEx}

\subsection{除法}
除法は\cmd{FDiv}です。
$\bunsuu16\div\bunsuu12$を計算する例です。

\begin{showEx}{\cmd{FDiv}}
  \FDiv{1/6}{1/2}\kotae
  $\prF\kotae$
\end{showEx}

\subsection{累乗}
累乗は\cmd{FPower}です。
$\left(\bunsuu23\right)^{-2}$を計算する例です。

\begin{showEx}{\cmd{FPower}}
  \FPower{2/3}{-2}\kotae
  $\prF\kotae$
\end{showEx}

ただし,指数は整数でなければなりません。

\section{四則計算式}
\subsection{\cmd{Fcalc}}
複雑な式を\cmd{FAdd}, \cmd{FDiv}などを組み合わせて表現するのは面倒ですから,
\cmd{Fcalc}を用意しました。
$\bunsuu{1}{2}+\bunsuu{1}{3}\times2$を計算する例です。

\begin{showEx}(.8,.14){\cmd{Fcalc}}
  \Fcalc{(1/2)+(1/3)*(2)}=\kotae
  $\prF\kotae$
\end{showEx}

$\left(\bunsuu{1}{2}+\bunsuu{1}{3}\right)\times2$の場合は

\begin{showEx}(.8,.14){\cmd{Fcalc}}
  \Fcalc{((1/2)+(1/3))*(2)}=\kotae
  $\prF\kotae$
\end{showEx}

各項 \verb+1/2+, \verb+1/3+, \verb+2+ をすべて\verb+(  )+でくくって,
\verb+(1/2)+, \verb+(1/3)+, \verb+(2)+と表現しなければならない,
という仕様はちと厄介ではあります。

\cmd{Fcalc}の書式です。

\begin{boxnote}
\begin{verbatim}
\Fcalc#1=#2
  #1 : 計算式
  #2 : 結果を受け取る制御綴
\end{verbatim}
\end{boxnote}

\subsection{\cmd{Fkeisan}}
\cmd{Fcalc}の第1引数に与える計算式を普通の表現法で表すのが
\cmd{Fkeisan}です。

\begin{showEx}(.6,.34){\cmd{Fkeisan}}
  \Fkeisan{((1/2)+(1/3))*(2)}
\end{showEx}
\smallskip

\cmd{Fkeisan}に,\verb+<A>+オプションを付加すると
\begin{jquote}
\begin{verbatim}
\kaitou{\prF{計算結果}}
\end{verbatim}
\end{jquote}
が付加されますから,\textsf{emathAe.sty}
を用いてそれを処理することができます。
この文書では
\begin{jquote}
\begin{verbatim}
\usepackage[continue]{emathAe}

\def\KaitouTTL{\par\hfill (答)~}
\end{verbatim}
\end{jquote}
としてありますから,

\begin{showEx}(.6,.34){\cmd{Fkeisan}の\texttt{<A>}オプション}
  \Fkeisan<A>{((1/2)+(1/3))*(2)}
\end{showEx}
\smallskip

なお \cmd{Fkeisan}には,未知数を\texttt{X}とする一次方程式を解く機能
がありますから,

\begin{showEx}(.6,.34){\cmd{Fkeisan}の\texttt{<A>}オプション}
  \Fkeisan<A>{((0.5)+(1/3))*(2)=X}
\end{showEx}

あるいは

\begin{showEx}(.6,.34){\cmd{Fkeisan}の\texttt{<A>}オプション}
  \Fkeisan<A>{(X+(1/3))*(2)=(1)}
\end{showEx}

もっと複雑な例については,同梱の
\begin{jquote}
\begin{verbatim}
calcfrac.tex, newB14.tex, Fsyndiv.tex
\end{verbatim}
\end{jquote}
をご覧ください。

\section{循環小数}
分数 \texttt{67/555} を小数展開すると \texttt{0.1207207207...} と
循環小数になります.\textsf{emathB.sty} では,循環小数は
循環節を\texttt{(...)} でくくって表現します.この例では
\texttt{0.1(207)}と表すこととなります.
分数形式を循環小数形式に変換するマクロが
\verb/\FtoZ/ です.

\begin{showEx}{\cmd{FtoZ}}
\FtoZ{67/555}\tmp
\tmp
\end{showEx}

この結果を通常の表現形式にするのが \cmd{prZ} です.

\begin{showEx}{\cmd{prZ}}
\FtoZ{67/555}\tmp
$\prZ\tmp$
\end{showEx}

逆に循環小数形式を分数形式に変換するコマンドが \cmd{ZtoF} です.

\begin{showEx}{\cmd{ZtoF}}
\ZtoF{0.1(207)}\tmp
\tmp
\end{showEx}

この分数形式を通常の表現形式にするのが \cmd{prF} です.

\begin{showEx}{\cmd{prF}}
\ZtoF{0.1(207)}\tmp
$\prF\tmp$
\end{showEx}

\end{document}
