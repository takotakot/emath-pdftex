\documentclass[a4j,fleqn]{jarticle}
\usepackage[txC]{emathFx}
\usepackage{showexample}


\begin{document}
\section{黒いダイヤ}
\cmd{diamondsuit}は

\begin{showEx}{\cmd{diamondsuit}}
$\diamondsuit$
\end{showEx}
のように,中が白抜きとなりますが,これを黒く塗りつぶしたものは
標準の\LaTeX には用意されていないようです。
\textsf{txfonts}には,\cmd{vardiamondsuit}の名前で黒塗りのものがあります。

\begin{showEx}{\cmd{vardiamondsuit}}
$\vardiamondsuit$
\end{showEx}

\textsf{txfonts}を使用する場合は,\cmd{vardiamondsuit}を用いればよいのですが,
\textsf{cmfont}を主としてこの記号を用いたいとなると,フォントのつまみ食いを
することになります。\textsf{emathFx.sty v 0.17}に
その機能を持たせることにしました。
ただし,この記号のサイズは\LaTeX の\cmd{diamondsuit}のサイズとは一致しません。

\begin{showEx}{\cmd{diamondsuit}と\cmd{vardiamondsuit}}
\begin{gather*}
\diamondsuit\\
\vardiamondsuit
\end{gather*}
\end{showEx}

そこで,\cmd{diamondsuit}も\textsf{txfonts}のものを用いるため
\cmd{txdiamondsuit}を用意しました。

\begin{showEx}{\cmd{txdiamondsuit}}
\begin{gather*}
\txdiamondsuit\\
\vardiamondsuit
\end{gather*}
\end{showEx}

関連して

\begin{showEx}{\cmd{varheartsuit}など}
\begin{gather*}
\varheartsuit \\
\txheartsuit \\
\varclubsuit \\
\txclubsuit \\
\varspadesuit \\
\txspadesuit 
\end{gather*}
\end{showEx}
\end{document}
