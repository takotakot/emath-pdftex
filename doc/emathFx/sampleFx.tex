\documentclass[disablejfam,a4j]{jarticle}
\usepackage{epic,eepic}
\usepackage{emath}
\usepackage{showexample}
\usepackage{jquote}
\usepackage[eu,tx,xy,abx]{emathFx}

\begin{document}
\textsf{emathFx.sty}は,\textsf{txfonts, pxfonts}などのフォントの一部を
つまみ食いするためのファイルです。

まず,ロードするとき,つまみ食いをしたいフォントを表す記号を
オプションに指定します。例えば,\textsf{txfonts}をつまみ食いしたいときは
\begin{jquote}
\begin{verbatim}
  \usepackage[tx]{emathFx}
\end{verbatim}
\end{jquote}
とします。現時点で用意されているオプションは
\begin{jquote}
\begin{verbatim}
tx : txfonts
px : pxfonts
eu : euler
xy : Xy-pic パッケージ
abx: mathabx fonts
\end{verbatim}
\end{jquote}
の5種類です。
\clearpage

また,現時点でつまみ食いが定義されているのは,
\begin{enumerate}[(1)]
  \item 
\verb+\bekutoru+ の矢印の鏃の形

\begin{showpEx}(.7,.24){\cmd{bekutorukata}}
\begin{tabular}{ll}
cm & \bekutorukata{cm}\bekutoru{AB}\\
eu & \bekutorukata{eu}\bekutoru{AB}\\
tx & 
! \IfFileExists{txfonts.sty}{%
  \bekutorukata{tx}\bekutoru{AB}
!}{}
\\
abx & 
! \IfFileExists{mathabx.sty}{%
  \bekutorukata{abx}\bekutoru{AB}
!}{}
\\
xy & 
! \IfFileExists{xy.sty}{%
  \bekutorukata{xy}\bekutoru{AB}
!}{xy オプションは無効です}\\
\end{tabular}
\end{showpEx}

\item 
その他

\begin{jquote}
  \begin{tabular}{|l|l|l|c|c|}\hline
    フォント         & 文字・記号 & コマンド  & 結果 & cm フォント\\\hline
    \textsf{txfonts}&ギリシャ文字ω&\verb+\txomega+&$\txomega$&$\omega$\\\hline
    〃&パイ&\verb+\txpi+&$\txpi$&$\pi$\\\hline
    〃&ガンマ&\verb+\txgamma+&$\txgamma$&$\gamma$\\\hline
    〃& g & \verb+\txg+ & $\txg$ & $g$ \\\hline
    〃&積分記号\EMvphantom[5pt][3pt]{$\displaystyle\int$}
      &\verb+\txint+& $\displaystyle\txint_a^b$&$\displaystyle\int_a^b$
      \\\hline
    \textsf{euler} &積分記号\EMvphantom[6pt]{$\displaystyle\int$}&
      \verb+\euint+
      & $\displaystyle\euint_a^b$&$\displaystyle\int_a^b$
      \\\hline
    \textsf{mathabx} &積分記号\EMvphantom[6pt]{$\displaystyle\int$}&
      \verb+\abxint+
      & $\displaystyle\abxint_a^b$&$\displaystyle\int_a^b$
      \\\hline
    〃&等号付不等号\EMvphantom[6pt]{$\displaystyle\geqq$}&
      \verb+\abxgeqq+
      & $\displaystyle y\abxgeqq x$&$\displaystyle y \geqq x$
      \\\hline
    〃&和の記号\EMvphantom[6pt]{$\displaystyle\abxsum_k^n$}&
      \verb+\abxsum+
      & $\displaystyle \abxsum_{k=1}^n$&$\displaystyle \sum_{k=1}^n$
      \\\hline
  \end{tabular}
\end{jquote}
\end{enumerate}

この他につまみ食いをご希望になる記号などがあれば,
BBS に書き込んでください。
\end{document}
