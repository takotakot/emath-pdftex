\documentclass[a4j]{jarticle}
\usepackage{emathPh}
\usepackage{emathFx}
\usepackage{showexample}
\usepackage{jquote}

\begin{document}
\section{newFx21}
\textsf{emathFx.sty v 0.21}では,\verb+\bekutoru+コマンドにおける
鏃の形状を塗りつぶしたものとするために,\verb+\bekutorukata+コマンドの
引数に\verb+fill+を新設しました:
\begin{jquote}
\begin{verbatim}
\bekutorukata{fill}
\end{verbatim}
\end{jquote}
\bigskip

デフォルトの\verb+\bekutoru+です:
\begin{showEx}(.6,.34){デフォルト}
  \bekutoru{AB}
\end{showEx}

\verb+\bekutorukata{fill}+を宣言します:
\begin{showEx}(.6,.34){\cmd{bekutorukata}\texttt{\{fill\}}}
  \bekutorukata{fill}
  \bekutoru{AB}
\end{showEx}

鏃の形状は\verb+\changeArrowHeadSize+コマンドで変更可能です。\\
つぎは\verb+\changeArrowHeadSize[22.5]{1.2}+(長さを1.2倍に,
先端の開き角を40\Deg(default=36\Deg)に設定しています。):

\begin{showEx}(.6,.34){鏃の形状変更}
  \bekutorukata{fill}
  \changeArrowHeadSize[22.5]{1.2}
  \bekutoru{AB}
\end{showEx}

\begin{enumerate}[注 1.~]
  \item 矢印は\verb+\ArrowLine+を用いていますから,
\begin{jquote}
\begin{verbatim}
\usepackage{emathPh}
\end{verbatim}
\end{jquote}
    が必須です。
  \item 矢印の形状を変更するコマンド\verb+\bekutorukata+は
    \textsf{emathFx.sty}で定義されていますから
\begin{jquote}
\begin{verbatim}
\usepackage{emathFx}
\end{verbatim}
\end{jquote}
    も必須です。
  \item 文書全体を通して,ベクトル矢線の形状を変更するには,
\begin{jquote}
\begin{verbatim}
\begin{document}
\end{verbatim}
\end{jquote}
直後に,
\begin{jquote}
\begin{verbatim}
\bekutorukata{fill}
\changeArrowHeadSize[22.5]{1.2}
\end{verbatim}
\end{jquote}
    などを置きます。    
  \item デフォルトでは
\begin{jquote}
\begin{verbatim}
\bekutorukata{cm}
\end{verbatim}
\end{jquote}
    が実行されています。何らかの事情で初期状態に戻したければ
    上記コマンドを発行してください。
  \item 前版までは,\cmd{bekutorukata}の引数は
\begin{jquote}
\begin{verbatim}
cm (デフォルト)
abx
eu
px
tx
xy
\end{verbatim}
\end{jquote}
    のいずれかでしたが,これに\verb+fill+が加わったということです。
    上述の引数の効果については,丸ごとパック添付のドキュメント
    \texttt{sampleFx.tex}をご覧ください。
\end{enumerate}

\end{document}
