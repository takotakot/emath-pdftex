\documentclass[a4j,fleqn]{jarticle}
\usepackage[mbb]{emathFx}
\usepackage{showexample}
\begin{document}
\section{\texttt{[mbb]ロードオプション}}
\subsection{\cmd{mathmbb}}
\textsf{amsmath}パッケージに,
ブラックボードボールドを出力するコマンド
\cmd{mathbb}があります。

\begin{showEx}{\cmd{mathbb}}
$\mathbb{ABCXYZ}$
\end{showEx}

ただし,小文字・数字はサポートされていません。

\begin{showEx}{小文字はダメ}
$\mathbb{abc012}$
\end{showEx}

小文字・数字のブラックボードボールドが用意されているフォントも
いくつかありますが、そのひとつに
\textsf{mbboard}パッケージがあります。
ただし,\textsf{mbboard.sty}では,\textsf{amsmath}パッケージと
同一コマンド\cmd{mathbb}を用いていますから,
二者択一となってしまいます。

そこで \textsf{emathFx.sty} \texttt{v 0.33}において
ロードオプション\texttt{[mbb]}を付加することにより,\\
\cmd{mathmbb}コマンドで\textsf{mbboard}パッケージの
ブラックボードボールドを利用できるようにしました。

\begin{showEx}(.54,.4){\cmd{mathmbb}}
\begin{gather*}
\mathmbb{ABCXYZ}\\
\mathmbb{abc012}\\
\intertext{\cmd{mathbb}との併用確認}
\mathbb{ABCXYZ}
\end{gather*}
\end{showEx}

\subsection{問題番号に利用する例}
数字のブラックボードボールドを,問題番号に用いる例をご覧ください。

\begin{showEx}(.6,.34){使用例}
\begin{enumerate}[%
\Large$\expandafter\mathmbb 1$.~]
	\item あああ
	\item いいい
	\item ううう
	\item えええ
\end{enumerate}
\end{showEx}

\subsection{参考---ブラックボールドボールド出力一覧}
ブラックボードボールドを出力できるパッケージの出力一覧
\texttt{blackboard.pdf}を同梱しました。
興味のある方はご覧ください。
\end{document}
