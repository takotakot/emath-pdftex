\documentclass{jarticle}
\usepackage{amsmath}%   showProg に必須ではありません。
\usepackage{showProg}

\long\def\sample#1{%
\input{example/#1.tex}

これを実現しているソースリストは次のとおりです。

\begin{itembox}{\texttt{#1.tex}}
\verbatiminput{example/#1.tex}
\end{itembox}}


\begin{document}
\title{対訳 \TeX\\
showProg.sty {\normalsize ver.0.00α}}
\author{tDB}
\date{2002/11/09}

\maketitle\thispagestyle{empty}
\begin{abstract}%
\begin{center}
\LaTeX のソースリストとタイプセット結果との対比
\end{center}
\end{abstract}
\pagebreak
\pagenumbering{roman}%

\tableofcontents

\pagebreak
\pagenumbering{arabic}
\section{\textsf{showProg.sty}とは}
対訳形式で\TeX のソースとタイプセット結果を対比する
\textsf{showexample.sty}の左側(ソースリスト)に
乙部氏の作になる\textsf{program.sty}を使用する試みです。

\subsection{\textsf{showProg}環境}
まずは\textsf{program}環境を用いた\textsf{showProg}環境です。

\sample{ex01}

ここで,\verb+\begin{showProg}+の引数のうち,
\verb+(1,.9)+はソース部とタイプセット部の横幅
(単位は\verb+\linewidth+)で,
次の2つの引数は\textsf{program}環境に引き渡されます。
\clearpage

\subsection{行番号の変更}
ソース部の行番号を1からではなく,指定した番号からはじめるようにするには,
\textsf{program.sty}で\textsf{nprogram}環境が用意されていますが,
ここでは無精をして,\textsf{showProg}のオプション引数で調整することにしました。
次の例では,行番号を11から始めています。

\sample{ex02}
\clearpage

\subsection{行番号の継続}
\textsf{showProg}環境は,行番号をリセットして1から始めますが,
前の番号の続きにするために,
\textsf{program.sty}の\textsf{cprogram}環境に倣い,
\textsf{showcProg}環境を用意しました。
次の例では,行番号をList \ref{lineno}に続けています。

\sample{ex03}
\clearpage

\subsection{フッタ無しの様式}
長いリストを途中で切って解説していくようなときは,
まず,フッタ無しの様式で始めたいことがあります。

\textsf{program.sty}の\textsf{nfprogram}に倣い,
\textsf{shownfProg}環境を用意しました。
行番号の初期値を指定するオプションを付けることは可能です。

\sample{ex04}
\clearpage

\subsection{ヘッダ無しの様式}
フッタ無しの対がヘッダ無しで\textsf{shownhProg}, \textsf{shownhcProg}環境です。
下の例は,\textsf{shownhcProg}で行番号を続けています。

\sample{ex05}
\clearpage

\subsection{ボックスの種類変更}
タイプセットした結果を\textsf{ascmac.sty}で定義されている
\textsf{shadebox}環境を用いていますが,
これを変更することも可能です。次の例では,\textsf{screen}環境に変更しています。

\sample{ex06}

冒頭の1行がボックスの種類を指定しています。
\end{document}
