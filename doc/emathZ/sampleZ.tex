\documentclass[a4j]{jarticle}
  \usepackage{plext}
  \usepackage{showexample}
  \usepackage{emathZ}
  \usepackage{emathOld}
  \usepackage{emathR}

\unitlength1cm

\begin{document}
\title{樹形図描画マクロ\\
emathZ.sty {\normalsize ver.0.08}\\使用例}
\author{tDB}
\date{2002/11/20}

\maketitle\thispagestyle{empty}
\begin{abstract}%
\parindent1zw%
樹形図を描画するためのマクロです.
ただし,数学のプリント作成をするため,という目的です.
従って,emathP, emath などのパッケージの使用を前提とします.

このマクロ集のマクロについてのご質問,バグ報告,修正・追加の提案等は
\begin{center}
http://emath.s40.xrea.com/
\end{center}
の掲示板へどうぞ。
\end{abstract}
\pagebreak
\pagenumbering{roman}%

\tableofcontents

\pagebreak

\pagenumbering{arabic}

\section{基本的用法}
\subsection{\textsf{zyukeizu}環境と\cmd{eda}コマンド}
まずは,基本的な例からはじめます.

\showexample[基本例](0.4)(0.3){example/Zex00}

\noindent のように,\textsf{zyukeizu} 環境内で \cmd{eda} コマンドを
用います.上の例は2分岐ですが,\cmd{eda} を書き並べれば,多分岐も
可能です.

\showexample[多分岐](0.4)(0.3){example/Zex01}

逆に1本だけの枝も作れます.

\showexample[一分岐](0.4)(0.3){example/Zex02}
\clearpage

\subsection{多重分岐}
\textsf{zyukeizu}環境はネストさせることが出来ます.
\cmd{eda}コマンド内に,\textsf{zyukeizu}環境を記述することで
多重分岐の樹形図を描画することが出来ます.

\showexample[多重分岐](0.45)(0.45){example/Zex03}
\clearpage

枝の数は不均等でも構いません.

\showexample[不均等な分岐](0.45)(0.45){example/Zex04}
\clearpage

\section{枝の形状}
\subsection{枝の長さ}
枝の長さを調整するためのコマンドが \cmd{zyukeizuEdaNagasa} です.

{\showexample[枝の長さ](0.5)(0.4){example/Zex40}}

デフォルトは,\cmd{zyukeizuEdaNagasa}\texttt{\{30\}} です.
数値の単位は,ポイントで,枝の$x$軸方向(横組では水平方向)の
長さを整数で指定します.
\clearpage

\subsection{枝の開き}
枝の開き具合を調整するのが \cmd{zyukeizuEdaVpitch} コマンドです.

{\showexample[枝の開き具合](0.5)(0.4){example/Zex41}}

デフォルトは,\cmd{zyukeizuEdaVpitch}\texttt{\{0\}} です.

引数で与える数値は \cmd{zyukeizuEdaNagasa} と同じく,
pt を単位とする整数値です.

この数値が枝と枝の間の改行幅に追加されます.

\subsection{角型 --- トーナメント表形式}
枝は,デフォルトでは斜線になっていますが,これを縦横の線でトーナメント表
のようにするためのコマンドが \cmd{zyukeizuKata} です.

{\showexample[角型樹形図](0.5)(0.4){example/Zex00k}}

このコマンドの引数は
\begin{jquote}
s : 斜線 (デフォルト)\\
k : 角型
\end{jquote}
の2種類だけです.
\clearpage

\subsection{角型における枝の長さ}
\cmd{zyukeizuEdaNagasa} コマンドで指定される数値は,
縦線の左側の枝の長さと右側の枝の長さを加えたもので,
これを2等分して左と右の枝が作られます.
2等分ではない樹形図を作りたいときは \cmd{zyukeizuEdaNagasa} の
\texttt{[..]} オプションで左側の長さを指定します.

{\showexample[角型における長さ指定](0.5)(0.4){example/Zex43}}

\cmd{zyukeizuEdaNagasa}の書式です.
\begin{boxnote}
\begin{verbatim}
\zyukeizuEdaNagasa[#1]#2
    #1 : 角型樹形図の場合,縦線で区切られた左側の長さ
    #2 : 左・右を合わせた全体の長さ
\end{verbatim}
\end{boxnote}

\subsection{枝の線種}
枝の線種を変更するには,\cmd{sensyu}を再定義します。

\begin{showEx}{枝の線種変更}
\def\sensyu{\hasen}
色 %
\begin{zyukeizu}%
    \eda{ 赤 }%
    \eda{ 白 }%
\end{zyukeizu}%
\end{showEx}

\subsection{枝の太さ}
すべての枝の太さを変更するには\cmd{edahutosa}コマンドを使用します。

\begin{showEx}{枝の太さ変更}
\edahutosa{2pt}
色 %
\begin{zyukeizu}%
    \eda{ 赤 }%
    \eda{ 白 }%
\end{zyukeizu}%
\end{showEx}

一部の枝の太さを変更するには,当該の枝を描画する\cmd{eda}に\verb+<hutosa=..>+
オプションで指定します。

\begin{showEx}{一部の枝の太さ変更}
色 %
\begin{zyukeizu}%
    \eda<hutosa=2pt>{ 赤 }%
    \eda{ 白 }%
\end{zyukeizu}%
\end{showEx}


%\clearpage

\section{枝の傍への文字配置}
\subsection{\texttt{[...]} オプション}
確率の問題解説では,樹形図の枝に確率の値を付記したいことがあります.
そのために \cmd{eda} はオプション引数を取ることが出来ます.

\showexample[確率樹形図](0.6)(0.3){example/Zex30}

すなわち \cmd{eda} の \texttt{[..]} オプション内に,
枝に付記したい文字列を与えます.
しかし,この例では,上の枝と下の枝に付記した数値が重なっています.
配置オプションで文字の位置を調整します.

\subsection{配置オプション \texttt{[...]}}
\showexample[文字位置調整(1)](0.6)(0.3){example/Zex31}

\subsection{基準点調整オプション \texttt{(x,y)}}
下の枝は良いとして,上の枝の数値が枝にかかっています.
さらに調整します.

\showexample[文字位置調整(2)](0.6)(0.3){example/Zex32}

\subsection{基準点の変更}
文字を配置する基準点は,枝の中点です。
これを変更するのが\verb+<..>+オプションです。

\begin{showEx}(.6,.3){基準点の変更}
\begin{zyukeizu}%
  \eda[$\frac12$]<1>(0,6pt)[rb]{ 表 }%
  \eda[$\frac12$]<1>[rt]{ 裏 }%
\end{zyukeizu}%
\end{showEx}

オプションは0と1の間の数値で,デフォルトは0.5(枝の中点)となっています。

%\clearpage

\subsection{\cmd{eda} コマンドの書式}
\cmd{eda}の書式です.
\begin{boxnote}
\begin{verbatim}
\eda[#1]<#2>(#3,#4)[#5]#6
    #1      : 枝の傍に置く文字列(デフォルトは空文字列)
    #2      : #1 の文字列を置く位置 (0 〜 1) 
                (デフォルトは 0.5 すなわち枝の中点)
              または
                  hutosa=..
                  iti=..
    (#3,#4) : 文字列を置く位置の修正ベクトル 
                (デフォルトは (0,0))
    #5      : 文字列の配置 (l, c, r; t, b)
    #6      : 枝の末端に置く文字列
(注)#1〜#5は
    \sPut[#2]{枝の左端}{枝の右端}(#3,#4)[#5]{#1} として\sPutに渡されます。
\end{verbatim}
\end{boxnote}

\section{縦型}
\subsection{\textsf{plext} パッケージで縦組}
樹形図はデフォルトでは右の方向に枝が分岐していきます.
これを下方向に分岐させたい,という話しです.
一案として \cmd{rotatebox} で回転する,ということも考えられますが,
文字も回転してしまいます.文字部分を逆回転させるのもありますが...

ここでは,縦組を利用してみます.
さりとて,\textsf{tarticle} クラスを使うのは,数学では気乗りしません.
そこで,\textsf{plext} パッケージで \textsf{minipage}環境などに附加された
組方向オプションを利用してみました.

ただし,縦組の\verb+tpic specials+ はDVI-wareに依存します。

\begin{enumerate}[(1)]
  \item dviout.exe を使用する場合は,
\begin{jquote}
\begin{verbatim}
Option
    Setup Parameters
        Graphic
で tpic specials の
        tate
\end{verbatim}
\end{jquote}
にチェックを入れる必要があります。

  \item dvipdfm で PDF 化する際は
\begin{jquote}
\begin{verbatim}
\usepackage[EMdvipdfm]{emathZ}
\end{verbatim}
\end{jquote}
とロードオプションを附加する必要があります。

(この場合,dvi ファイルを dviout で見ると,線が乱れることでしょう。
上述の tate チェックをつけなければ dviout でも乱れません。)

  \item dvipsk+distiller で PDF化する際は,
\begin{jquote}
\begin{verbatim}
\usepackage{emathZ}
\end{verbatim}
\end{jquote}
としなければなりません。
\end{enumerate}

\showexample[基本例(縦)](0.6)(0.3){example/Zex00t}

\subsection{角型の縦組}

\showexample[角型(縦)](0.6)(0.3){example/Zex00kt}
\clearpage

\section{応用例}
最後に,数学の問題や解説で登場する樹形図をご覧いただきましょう.

\subsection{約数の個数}
下の例のソースリストは \texttt{example} フォルダ内の
\texttt{Zex20.tex} をご覧ください.

\newcommand{\fmtln}[3]{\makebox[1em][l]{$#2$} $\longrightarrow$
            $\makebox[2.6em][l]{$#1\cdot #2$}=\makebox[1.25em][r]{$#3$}$}

\begin{itemsquarebox}{$BLs?t$N8D?t(B}
72 $B$N@5$NLs?t$O2?8D$"$k$+!%(B
\end{itemsquarebox}

\begin{mawarikomi}{}{%
\begin{zyukeizu}%
\eda{\makebox[1em][l]{$1$}%
    \begin{zyukeizu}
        \eda{\fmtln{1}{1}{1}}%
        \eda{\fmtln{1}{3}{3}}%
        \eda{\fmtln{1}{3^2}{9}}%
    \end{zyukeizu}
}%
\eda{\makebox[1em][l]{$2$}%
    \begin{zyukeizu}
        \eda{\fmtln{2}{1}{2}}%
        \eda{\fmtln{2}{3}{6}}%
        \eda{\fmtln{2}{3^2}{18}}%
    \end{zyukeizu}
}%
\eda{\makebox[1em][l]{$2^2$}%
    \begin{zyukeizu}
        \eda{\fmtln{2^2}{1}{4}}%
        \eda{\fmtln{2^2}{3}{12}}%
        \eda{\fmtln{2^2}{3^2}{36}}%
    \end{zyukeizu}
}%
\eda{\makebox[1em][l]{$2^3$}%
    \begin{zyukeizu}
        \eda{\fmtln{2^3}{1}{8}}%
        \eda{\fmtln{2^3}{3}{24}}%
        \eda{\fmtln{2^3}{3^2}{72}}%
    \end{zyukeizu}
}%
\end{zyukeizu}}%
$72=2^3\times3^2$$B$G$"$k$+$i!$(B
\[2^3 \text{ $B$NLs?t(B } 1,~2,~2^2,~2^3\]
$B$N$*$N$*$N$K(B
\[3^2 \text{ $B$NLs?t(B } 1,~3,~3^2\]
$B$N$=$l$>$l$r3]$1$k$H!$(B72$B$N@5$NLs?t$,$9$Y$FF@$i$l$k!%(B
$B$7$?$,$C$F5a$a$k8D?t$O!$@Q$NK!B'$+$i(B
\[ 4\times3=\bm{12} \text{$B!J8D!K(B}\]
\end{mawarikomi}
\mawarikomiowari

\clearpage

\subsection{確率樹形図}
つぎは,確率樹形図の例です.ソースリストは \texttt{Zex21.tex} です.

\begin{itemsquarebox}{$B3NN(<y7A?^(B}
3$BK\$NEv$?$j$/$8$r4^$`(B10$BK\$N$/$8$,$"$k!%$3$N$/$8$r(BA, B$B$,$3$N=g$K(B1$BK\$:$D(B
$B0z$/$H$-!$(BB$B$,Ev$?$k3NN($r5a$a$h!%$?$@$7!$(BA$B$,0z$$$?$/$8$O$b$H$K(B
$BLa$5$J$$$H$9$k!%(B
\end{itemsquarebox}
{\zyukeizuEdaVpitch{10}%

\hspace{5zw}
\begin{zyukeizu}%
\eda[$\frac{3}{10}$](0,6pt)[b]{ $B!{(B %
	\begin{zyukeizu}%
	\eda[$\frac29$](0,4pt)[b]{ $B!{(B }%
	\eda[$\frac79$][t]{ $B!_(B }%
	\end{zyukeizu}%
}%
\eda[$\frac{7}{10}$](0,-2pt)[t]{ $B!_(B %
	\begin{zyukeizu}%
	\eda[$\frac39$](0,4pt)[b]{ $B!{(B }%
	\eda[$\frac69$][t]{ $B!_(B }%
	\end{zyukeizu}%
}%
\end{zyukeizu}%
}


\subsection{トーナメント表}
次いで,トーナメント表です.ソースリストは \texttt{Zex22.tex} です.

\begin{itemtbsquarebox}<c>{トーナメント表}{99 お茶の水女子大学}
\begin{mawarikomi}{}{\tate\zyukeizuKata{k}
\zyukeizuEdaVpitch{8}%
\zyukeizuEdaNagasa[10]{30}%
\begin{zyukeizu}
\eda{%
  \zyukeizuEdaNagasa[0]{20}%
  \begin{zyukeizu}%
  \eda{%
    \begin{zyukeizu}%
      \eda{}%
      \eda{}%
    \end{zyukeizu}%
  }%
  \eda{%
    \begin{zyukeizu}%
      \eda{}%
      \eda{}%
    \end{zyukeizu}%
  }%
  \end{zyukeizu}%
}%
\eda{%
  \zyukeizuEdaNagasa[0]{20}%
  \begin{zyukeizu}%
  \eda{%
    \begin{zyukeizu}%
      \eda{}%
      \eda{}%
    \end{zyukeizu}%
  }%
  \eda{%
    \begin{zyukeizu}%
      \eda{}%
      \eda{}%
    \end{zyukeizu}%
  }%
  \end{zyukeizu}%
}%
\end{zyukeizu}
}%
A, B, C, D の4つの県から2チームずつ,計8つの野球チームが
トーナメント形式で優勝を争うことになった.抽選で右図のように
対戦相手を決めるものとし,8チームの力は同等であるとする.
\end{mawarikomi}
\begin{enumerate}[(1)]
  \item 次の確率を求めよ.
  \begin{enumerate}[(i)]
    \item A県の2チームが1回戦で対戦する確率
    \item 1回戦の4試合がすべて同県勢の対戦になる確率
    \item 決勝戦以外では同県勢同士の対戦があり得ないような組合せになる
      確率
  \end{enumerate}
  \item 1回戦の4試合の中で同県勢同士の対戦になる試合数の期待値を求めよ.
\end{enumerate}
\end{itemtbsquarebox}\par


最後に少し大きい例です。
2002サッカーワールドカップの決勝トーナメント結果表です。

\ReadTeXFile{example/wc2T}
\end{document}
