\newcommand{\fmtln}[3]{\makebox[1em][l]{$#2$} $\longrightarrow$
            $\makebox[2.6em][l]{$#1\cdot #2$}=\makebox[1.25em][r]{$#3$}$}

\begin{itemsquarebox}{約数の個数}
72 の正の約数は何個あるか.
\end{itemsquarebox}

\begin{mawarikomi}{}{%
\begin{zyukeizu}%
\eda{\makebox[1em][l]{$1$}%
    \begin{zyukeizu}
        \eda{\fmtln{1}{1}{1}}%
        \eda{\fmtln{1}{3}{3}}%
        \eda{\fmtln{1}{3^2}{9}}%
    \end{zyukeizu}
}%
\eda{\makebox[1em][l]{$2$}%
    \begin{zyukeizu}
        \eda{\fmtln{2}{1}{2}}%
        \eda{\fmtln{2}{3}{6}}%
        \eda{\fmtln{2}{3^2}{18}}%
    \end{zyukeizu}
}%
\eda{\makebox[1em][l]{$2^2$}%
    \begin{zyukeizu}
        \eda{\fmtln{2^2}{1}{4}}%
        \eda{\fmtln{2^2}{3}{12}}%
        \eda{\fmtln{2^2}{3^2}{36}}%
    \end{zyukeizu}
}%
\eda{\makebox[1em][l]{$2^3$}%
    \begin{zyukeizu}
        \eda{\fmtln{2^3}{1}{8}}%
        \eda{\fmtln{2^3}{3}{24}}%
        \eda{\fmtln{2^3}{3^2}{72}}%
    \end{zyukeizu}
}%
\end{zyukeizu}}%
$72=2^3\times3^2$であるから,
\[2^3 \text{ の約数 } 1,~2,~2^2,~2^3\]
のおのおのに
\[3^2 \text{ の約数 } 1,~3,~3^2\]
のそれぞれを掛けると,72の正の約数がすべて得られる.
したがって求める個数は,積の法則から
\[ 4\times3=\bm{12} \text{(個)}\]
\end{mawarikomi}
\mawarikomiowari
