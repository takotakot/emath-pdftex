% 年度 1998
% 出題 0032 電気通信大学 後期
% 検索キーワード  コンピュータ
% 科目 数A

\documentclass[a4j,fleqn]{jarticle}
\usepackage{alltt}
\usepackage{emathN}
\usepackage{emathOld}
\usepackage{hako}
\pagestyle{empty}

\begin{document}
\hakosyokika
\begin{caprm}
\begin{nidan}{29em}{%
\small
\abovedisplayskip0pt
\belowdisplayskip0pt
\noindent
\begin{caprm}
\begin{nagarezu}[.45](32em,37\baselineskip)%
  \tansi{はじめ}%
  \sitahe%
  \nyuusyuturyoku{%
  \parbox{8\zw}{\ $S(1),\cdots,S(5)$の値を入力}}%
  \sitahe
  \syori{$J=1$}%
  \sitahe[label1]%
  \begin{handan}<.6><1.2>{ $\strut J>4$ ? }%
    \begin{migibunki}%
      \migihe<n>(50pt)"yes"%
      \sitahe%
      \tansi{おわり}%
    \end{migibunki}%
    \begin{sitabunki}%
      \sitahe"No"
    \syori{\parbox{5\zw}{\vspace{-.4\baselineskip}%
      \begin{align*}
        M&=J\\ K&=J+1
        \end{align*}}}%
    \sitahe[label2]%
    \begin{handan}<.6><1.2>{ $\strut K>5$ ? }%
    \begin{hidaribunki}%
      \hidarihe<n>"no"%
      \sitahe
        \begin{handan}<.6><1.2>{ $\strut S(M)<S(K)$ ? }%
          \begin{sitabunki}%
            \sitahe"no"%
            \syori{$M=K$}%
            \sitahe[matome1]%
            \pushcurP
          \end{sitabunki}%
          \begin{migibunki}%
            \migihe<n>"yes"%
            \sitahe<n>`matome1'
            \hidarihe`matome1'
          \end{migibunki}%
        \end{handan}%
        \popcurP
        \syori{K=K+1}%
        \sitahe<n>%
        \hidarihe<n>(60pt)%
        \uehe<n>`label2'%
        \migihe`label2'%
      \end{hidaribunki}%
      \begin{migibunki}%
        \migihe<n>(50pt)"yes"%
        \sitahe
      \syori{\parbox{5\zw}{\vspace{-.4\baselineskip}%
        \begin{align*}
            &B=S(J)\\
            &S(J)=S(M)\\
            &S(M)=B
          \end{align*}}}%
        \sitahe
        \nyuusyuturyoku{%
        \parbox{8\zw}{$S(1),\cdots,S(5)$の値を表示}}%
      \Put\migityuuten(0,0)[l]{ $\cdots\cdots(\ast)$}%
        \sitahe
        \syori{$J=J+1$}%
        \sitahe
        \hidarihe<n>(200pt)%
        \uehe<n>`label1'%
        \migihe`label1'%
      \end{migibunki}%
    \end{handan}%
    \end{sitabunki}%
  \end{handan}%
\end{nagarezu}%
\end{caprm}
}%
下のBASICによるプログラムは,
\begin{jquote}
$S(I)$が偶数のとき\\
 $S(I+1)=S(I)/2$ \\
$S(I)$が奇数のとき\\
 $S(I+1)=3\times S(I)-1$
\end{jquote}
という規則に従う数列を生成する.
\begin{alltt}
10 DIM S(10)
20 N=6:I=1
30 S(I)=N
40 PRINT S(I)
50 IF \Hako THEN N=N/2
  ELSE N=3*N-1
60 I=I+1
7O IF N<>1 GOTO 30
8O END
\end{alltt}
(注意:DIM S(10)という命令は,$S(1),S(2),\cdots\cdots,S(10)$という配列を
使えるようにするためのものである.\texttt{N<>1}は$N\neqq 1$を表す.)
\end{nidan}
\begin{Enumerate}[(1)]
  \item プログラム中の空欄に入るべき式を答えよ.ただし,
    正の数が与えられたとき,その小数部分を切り捨てた整数を求める関数INTを
    用いてよい.
  \item このプログラムにより表示される数を,表示される順にすべて答えよ.
\end{Enumerate}

次に,このプログラムで得られた数列のうち,S(1)からS(5)までの値を
入力として,上の流れ図に示される手順で計算を行う.
\begin{Enumerate*}
  \item 
    \begin{nidan}{21\zw}{%
      \begin{tabular}[t]{|l|*{5}{c|}} \hline
          & S(1) & S(2) & S(3) & S(4) & S(5) \\\hline
      2回目 &&&&& \\\hline
      4回目 &&&&& \\\hline
      \end{tabular}
    }\noindent
    図中の$(*)$を2回目と4回目に実行するときに表示される
    $S(1),S(2),\cdots\cdots,S(5)$の値をそれぞれ求め,
    右の表の形で解答欄(省略)に記入せよ.
    \end{nidan}
  \item 次の空欄に当てはまる言葉を答えよ.
    「この流れ図による処理は,一般に,入力として与えられた数列
    $S(1),~S(2),~\cdots\cdots,~S(5)$を\Hako に並べ替える働きをする.
\end{Enumerate*}
\end{caprm}
\end{document}
