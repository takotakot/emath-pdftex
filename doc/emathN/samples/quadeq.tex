\documentclass{jarticle}
\usepackage{emathN}

\begin{document}
2次方程式の解を求めるアルゴリズム

\noindent
\begin{caprm}
\fbox{%
\begin{nagarezu}(28em,16\baselineskip)%
  \tansi{始め}%
  \sitahe%
  \nyuusyuturyoku{$a$, $b$, $c$ を入力}%
  \sitahe
  \syori{$D~\longleftarrow~b^2-4ac$}%
  \sitahe
  \begin{handan}{ \strut$D:0$ }%
    \begin{sitabunki}%
      \sitahe"""$D=0$"
   	  \nyuusyuturyoku{\parbox{7zw}{ 重解を表示      }}%
	  \sitahe[matome]%
	  \tansi{終わり}%
    \end{sitabunki}%
    \begin{migibunki}%
    	\migihe<n>(50pt)"$D<0$"%
    	\sitahe(35pt)
    	\nyuusyuturyoku{\parbox{7zw}{異なる2つの虚数解を表示}}%
			\sitahe<n>`matome'%
			\hidarihe`matome'%
    \end{migibunki}%
    \begin{hidaribunki}%
    	\hidarihe<n>(50pt)"$D>0$"%
    	\sitahe(35pt)
    	\nyuusyuturyoku{\parbox{7zw}{異なる2つの実数解を表示}}%
    	\sitahe<n>`matome'%
    	\migihe`matome'%
    \end{hidaribunki}%
  \end{handan}%
\end{nagarezu}%
}%
\end{caprm}
\end{document}
