\documentclass[a4j]{jarticle}
\usepackage{emathN}
\usepackage{array}
\usepackage{emathOld}
\usepackage{showexample}
\usepackage{makeidx}

\makeindex%
\newcommand{\cindex}[1]{\index{#1@\texttt{\protect\cmd{#1}}}}

\begin{document}
\changeArrowHeadSize{1.5}
\title{流れ図描画マクロ\\
emathN.sty {\normalsize ver.0.15}\\使用例}
\author{tDB}
\date{2004/12/19}

\maketitle\thispagestyle{empty}
\begin{abstract}%
\parindent1\zw%
中学・高校で数学のプリントにつける流れ図の作成に必要な記号,コマンド,
環境を集めたマクロ集です.
\LaTeX2e が前提です.

このマクロ集のマクロについてのご質問,バグ報告,修正・追加の提案等は
\begin{center}
http://emath.s40.xrea.com/
\end{center}
の掲示板へどうぞ。
\end{abstract}
\pagebreak
\pagenumbering{roman}%

\tableofcontents

\pagebreak

\pagenumbering{arabic}

\section{このパッケージで取り扱う図形}
まずは右下の例をご覧ください.
このパッケージで取り扱う図形は左下の通りです.\bigskip

\begin{nidan}{12em}{\vspace{-.75\baselineskip}\par\begin{caprm}[l]
\fbox{%
\begin{nagarezu}(12em,17.5\baselineskip)(8em,0pt)%
  \tansi{$B;O$a(B}%
  \sitahe%
  \syori{$X~\longleftarrow~1$}%
  \sitahe[kurikaesi]%
  \syori{$Y~\longleftarrow~X^2$}%
  \sitahe
  \nyuusyuturyoku{X, Y $B$r=PNO(B}%
  \sitahe
  \syori{$X~\longleftarrow~X+1$}%
  \sitahe
  \begin{handan}{$\strut X<10$}%
    \begin{hidaribunki}%
    	\hidarihe<n>(40pt)"Yes""$X<10$"%
    	\uehe<n>`kurikaesi'%
    	\migihe`kurikaesi'%
    \end{hidaribunki}%
    \begin{sitabunki}%
      \sitahe"No""$X\geqq 10$"%
      \tansi{$B=*$o$j(B}%
    \end{sitabunki}%
  \end{handan}%
\end{nagarezu}%
}%
\end{caprm}
}%
\begin{tabular}[t]{|p{4\zw}|p{3\zw}|p{8\zw}|}\hline
\begin{nagarezu}(4\zw,2\baselineskip)\tansi{  }\end{nagarezu} 
& 端子 & 手順の開始・終了を表す.\\\hline
\begin{nagarezu}(4\zw,2\baselineskip)\nyuusyuturyoku{  }\end{nagarezu} 
& 入出力 & 数値などのデータの入力,計算結果の出力を表す.\\\hline
\begin{nagarezu}(4\zw,2\baselineskip)\syori{  }\end{nagarezu} 
& 処理 & 計算・代入などの処理を表す.\\\hline
\begin{nagarezu}(4\zw,2\baselineskip)\handan{  }\end{nagarezu} 
& 判断 & 条件の判断を表す.記号の中には比較条件を書き,
記号から出る流れ線に,その流れ線の順序に従う条件を付記する.\\\hline
\begin{nagarezu}(4\zw,4.5\baselineskip)\begin{kurikaesi}{  }{}\end{kurikaesi}
\end{nagarezu} 
& 繰返し & 繰り返し処理を表す。\\\hline
\begin{nagarezu}(4\zw,2\baselineskip)\sitahe\end{nagarezu} 
& 流れ線 & 上記の記号を結び,処理の流れを表す.\\\hline
\end{tabular}
\end{nidan}

これらの図形を組み合わせて流れ図を作りますが,
このパッケージでは,\textsf{nagarezu}環境の中に流れ図を記述します.

\section{端子記号}
端子記号を描画するコマンドは \cmd{tansi} です.(別名 \cmd{termbox})
\cindex{tansi}\cindex{termbox}

\showexample[\cmd{tansi}](0.55)(0.35){example/tansi01}

\textsf{nagarezu}環境の中に,§1で紹介した記号を配置していきます.
上の例では端子記号を1個だけ配置しました.
\textsf{nagarezu}環境は実質的には\textsf{picture}環境で,
座標原点は上辺中央にあります.(オプションで変更可能)
その書式は\index{nagarezu@nagarezu 環境}

\begin{boxnote}
\begin{verbatim}
\begin{nagarezu}[#1](#2,#3)(#4,#5)
  #1 : 原点の位置比率 (デフォルト値 0.5)
  #2 : 流れ図の横幅
  #3 :     縦幅
  #4 : 原点の位置を修正する x成分
  #5 : 原点の位置を修正する y成分
\end{verbatim}
\end{boxnote}

となっています.

\verb/(#2,#3)/は必須の引数で流れ図のサイズを指定します.
数値は単位を伴います.

デフォルトでは,原点は\verb/(#2,#3)/で指定された長方形の上辺中央にあります.
これを修正するオプション引数が \verb/(#4,#5)/ です.

長方形の左上を始点として \verb/(#4,#5)/ を成分とする
ベクトルの終点を原点とします.この成分も単位を伴う数値です.

例えば,\verb/\begin{nagarezu}(2cm,3cm)(0.5cm,-1cm)%/ としたときの
確保される領域と原点の位置は下のようになります.

\showexample[\cmd{tansi}](0.55)(0.35){example/tansi03}

\verb/[#1]/は旧版との互換性を保つために残してあります.

次に,端子の書式は

\begin{boxnote}
\begin{verbatim}
\tansi[#1]#2
  #1 : 塗りつぶしをするためのオプション引数で,
       塗りつぶしの濃さを 0.1〜1 の数値で指定します.
  #2 : 中に記述する文字列
\end{verbatim}
\end{boxnote}

となっています.

描画後のカレントポイントは,下辺の中点です.また,次の点が定義されています.
\begin{jquote}
\begin{verbatim}
\migityuuten   : 右側の半円の右端
\hidarityuuten : 左側の半円の左端
\uetyuuten     : 上辺の中点
\sitatyuuten   : 下辺の中点(カレントポイント)
\end{verbatim}
\end{jquote}
後で,必要となる点があれば,\cmd{nahuda}コマンドで保存しておきます.
その一例です.

\showexample[\cmd{nahuda}](0.55)(0.35){example/tansi02}

\cmd{nahuda} の書式です.\cindex{nahuda}

\begin{boxnote}
\begin{verbatim}
\nahuda#1#2
    #1 : 名札(ラベル)を付ける点
    #2 : ラベル名
\end{verbatim}
\end{boxnote}

ラベルを付けた点への戻り方は「流れ線」のところで解説します.

\section{入出力記号}
キーボードからの入力,画面への出力などを表す図形を描画するコマンドが
\cmd{nyuusyuturyoku}です.長たらしいコマンド名ですから,別名 \cmd{iobox}
も使用できます.\cindex{nyuusyuturyoku}\cindex{iobox}

\showexample[\cmd{nyuusyuturyoku}](0.55)(0.35){example/iobox01}

入出力記号中に複数行を記述する方法については,
次節「処理記号」で解説します.

描画後のカレントポイントは,下辺の中点です.また,次の点が定義されています.
\begin{jquote}
\begin{verbatim}
\migityuuten   : 右辺の中点
\hidarityuuten : 左辺の中点
\uetyuuten     : 上辺の中点
\sitatyuuten   : 下辺の中点(カレントポイント)
\end{verbatim}
\end{jquote}

\section{処理記号}
処理記号を描画するコマンドは \cmd{syori} です.(別名 \cmd{opbox})
\cindex{syori}\cindex{opbox}

\showexample[\cmd{syori}](0.6)(0.3){example/syori01}

処理記号の中に複数行を入れたいときは,\textsf{minipage}環境
あるいは \verb/\parbox/コマンドなどを用います.

\showexample[複数行](0.6)(0.3){example/syori02}

複数行が数式の場合は,\textsf{align*}環境などを用いるのもあります.

\showexample[\textsf{align}環境](0.6)(0.3){example/syori03}

ただし,\textsf{align*}環境を用いると,
上に空白がつきますので \verb/\vspace/ で詰める方がよさそうです.

\showexample[\textsf{align}環境の頭部空白](0.6)(0.3){example/syori04}

描画後のカレントポイントは,下辺の中点です.また,次の点が定義されています.
\begin{jquote}
\begin{verbatim}
\migityuuten   : 右辺の中点
\hidarityuuten : 左辺の中点
\uetyuuten     : 上辺の中点
\sitatyuuten   : 下辺の中点(カレントポイント)
\end{verbatim}
\end{jquote}
\cindex{migityuuten}\cindex{hidarityuuten}
\cindex{uetyuuten}\cindex{sitatyuuten}

\section{流れ線}
処理の流れを示すのに「流れ線」を用います.流れ線の方向は4種類あります.
オプションによって,矢印をつけないもの,二重線を引くことも可能です.
\begin{center}
\begin{tabular}[t]{|p{5\zw}|p{8\zw}|}\hline
コマンド & 流れ線 \\\hline
\verb/\sitahe/\par\verb/\downto/
&\vspace{1\zh}\par\begin{nagarezu}(8\zw,4\zh)\sitahe\end{nagarezu}\\\hline
\verb/\migihe/\par\verb/\rightto/
&\vspace{1\zh}\par\begin{nagarezu}(8\zw,4\zh)\migihe\end{nagarezu}\\\hline
\verb/\hidarihe/\par\verb/\leftto/
&\vspace{1\zh}\par\begin{nagarezu}(8\zw,4\zh)\hidarihe\end{nagarezu}\\\hline
\verb/\uehe/\par\verb/\upto/
&\vspace{1\zh}\par\begin{nagarezu}(8\zw,4\zh)\uehe\end{nagarezu}\\\hline
\end{tabular}
\end{center}
書式は4つのコマンドすべて共通ですから,\cmd{sitahe} について解説します.
\cindex{sitahe}\cindex{uehe}\cindex{migihe}\cindex{hidarihe}
\cindex{downto}\cindex{upto}\cindex{rightto}\cindex{leftto}

\begin{boxnote}
\begin{verbatim}
\sitahe<#1>(#2)[#3]"#4""#5"`#6'
  #1:線の形状指定オプション
      a = 矢印をつける(デフォルト)
      n = 矢印なし
      r = 逆方向に矢印
      w = 二重線
      (重複指定可能) 
  #2:線の長さ(デフォルト値 20pt)
  #3:ラベル
  #4:矢線の上に置く文字列
  #5:矢線の下に置く文字列
  #6:流れ線の行き先ラベル
\end{verbatim}
\end{boxnote}

\subsection{流れ線の形状}
単に \cmd{sitahe} とすると,長さ20pt の下向き矢印を描画します.
\texttt{<n>} オプションをつけると矢印なしの線分となります.

\showexample[\cmd{sitahe<n>}](0.55)(0.4){example/sen01}

\texttt{<w>} オプションで二重線となります.\texttt{<n>} と併用する例です.

\showexample[\cmd{migihe<wn>}](0.5)(0.44){example/sen02}

\subsection{流れ線の長さ}
流れ線の長さはデフォルトでは 20pt となっています.
これを変更するには \texttt{(...)} オプションを用います.
長さは単位を伴った数値で指定します.

\showexample[\cmd{sitahe(4\zh)}](0.55)(0.4){example/sen03}

\subsection{流れ線に文字列の添付}
流れ線の傍に文字列を添付するオプションが \texttt{"..."} です.
\begin{jquotation}
\cmd{sitahe}, \cmd{uehe} では,第1の\texttt{"..."} で流れ線の左中央部に,
第2の\texttt{"..."}で流れ線の右中央部に文字列を出力します.

\cmd{hidarihe}, \cmd{migihe} では,第1の\texttt{"..."} で流れ線の上中央部に,
第2の\texttt{"..."}で流れ線の下中央部に文字列を出力します.
\end{jquotation}

\showexample[流れ線に文字列を添付](0.55)(0.39){example/sen05}

\subsection{流れ線のラベル}
\texttt{[...]} オプションで指定したラベル名に,
流れ線の中点の座標が保存されます.

\subsection{流れ線の行き先指定}
流れ線の行き先をラベルによって指定することができます.
\texttt{`...'} オプションです.
ただし,このスタイルファイルでは斜線は引きませんから,
\begin{jquotation}
\cmd{sitahe}, \cmd{uehe} では,流れ線の終点の$y$座標がラベルの点の
$y$座標と等しくなります.

\cmd{migihe}, \cmd{hidarihe} では,流れ線の終点の$x$座標がラベルの点の
$x$座標と等しくなります.

\showexample[流れ線の行き先指定](0.55)(0.39){example/sen04}

\end{jquotation}

\section{判断記号}
条件を与えて分岐させる環境が \textsf{handan} です(別名 \textsf{ifbox}).
コマンドではなく,環境仕立てになっています.
\index{handan@handan 環境}\index{ifbox@ifbox 環境}

また,右・左・下方向の分岐もそれぞれ
\begin{jquote}
\textsf{migibunki}, \textsf{hidaribunki}, \textsf{sitabunki}
\end{jquote}
環境内に記述します.
\index{migibunki@migibunki 環境}
\index{hidaribunki@hidaribunki 環境}
\index{sitabunki@sitabunki 環境}

\showexample[\textsf{handan}環境](0.53)(0.41){example/handan01}

書式は
\begin{boxnote}
\begin{verbatim}
\begin{handan}[#1]<#2><#3>#4{%
  #1 : 塗りつぶしの濃さ
  #2 : 横方向伸縮率(デフォルト値 1)
  #3 : 縦方向伸縮率(デフォルト値 1)
  #4 : 判断記号内の文字列
\end{verbatim}
\end{boxnote}

描画後のカレントポイントは,下辺の中点です.また,次の点が定義されています.
\begin{jquote}
\begin{verbatim}
\migityuuten   : 右端
\hidarityuuten : 左端
\uetyuuten     : 上端
\sitatyuuten   : 下端
\end{verbatim}
\end{jquote}

\section{繰り返し記号}
\verb+FOR .... NEXT+ などの繰り返しを表現するには,
\textsf{kurikaesi}環境を用います。

使用例をご覧ください。1から10までの和を求める流れ図です。

\begin{showEx}(.6,.34){\textsf{kurikaesi}環境}
\begin{nagarezu}%
  (7em,18\baselineskip)
\tansi{始め}
\sitahe
\syori{S=0}
\sitahe
\begin{kurikaesi}{N=1 to 10}{N}
\sitahe
\syori{S=S+K}
\sitahe
\end{kurikaesi}
\sitahe
\iobox{Sを出力}
\sitahe
\tansi{終り}
\end{nagarezu}%
\end{showEx}

すなわち繰り返し部分を \textsf{kurikaesi}環境内に記述します。
\textsf{kurikaesi}環境の書式です。

\begin{boxnote}
\begin{verbatim}
\begin{kurikaesi}[#1](#2)#3#4{%
  #1 : 塗りつぶしの濃さ
  #2 : 横幅
  #3 : 始めのボックス内に記述する内容
  #4 : 終りのボックス内に記述する内容
\end{verbatim}
\end{boxnote}
\index{kurikaesi@kurikaesi 環境}

繰り返しボックスを塗りつぶすオプションを使用してみましょう。

\begin{showEx}(.6,.34){塗りつぶしオプション}
\begin{caprm}
\begin{nagarezu}(7em,18\baselineskip)
\tansi{始め}
\sitahe
\syori{$S=0$}
\sitahe
\begin{kurikaesi}[.5]{$N=1$ to 10}{N}
\sitahe
\syori{$S=S+N$}
\sitahe
\end{kurikaesi}
\sitahe
\iobox{Sを出力}
\sitahe
\tansi{終り}
\end{nagarezu}%
\end{caprm}
\end{showEx}

\bigskip

繰り返し部分全部を塗りつぶしたいときは \textsf{kurikaesi*}環境を用います。
\index{kurikaesi*@kurikaesi* 環境}

\begin{showEx}(.6,.34){繰り返し部分全部を塗りつぶす}
\begin{caprm}
\begin{nagarezu}(7em,18\baselineskip)
\tansi{始め}
\sitahe
\syori{$S=0$}
\sitahe
\begin{kurikaesi*}[.5]{$N=1$ to 10}{N}
\sitahe
\syori{$S=S+N$}
\sitahe
\end{kurikaesi*}
\sitahe
\iobox{Sを出力}
\sitahe
\tansi{終り}
\end{nagarezu}%
\end{caprm}
\end{showEx}

もう少し複雑な例です。

A(1), A(2), ..... , A(6)のうち,最大値 MAX を出力する例です。

\begin{showEx}(.54,.4){\cmd{pushcurrentP}}
\begin{caprm}
\begin{nagarezu}(15\zw,22.5\baselineskip)
  \termbox{始め}
  \downto
  \opbox{$MAX=A(1)$}
  \downto
  \begin{repbox*}%
      [.5](12\zw){I=2,3,4,5,6}{I}%
    \downto
    \begin{ifbox}<.7><1.4>%
          {\strut$A(I)>MAX$}%
      \begin{branch}{d}%
        \downto"Yes"%
        \opbox{$MAX=A(I)$}%
        \downto[skip]%
        \pushcurrentP
      \end{branch}%
      \begin{branch}{r}%
        \rightto<n>"No"%
        \downto<n>`skip'%
        \leftto`skip'%
      \end{branch}%
    \end{ifbox}%
  \end{repbox*}%
  \downto
  \iobox{MAX}
  \downto
  \termbox{終り}
\end{nagarezu}%
\end{caprm}
\end{showEx}

このリストには \cmd{pushcurrentP} という新しいコマンドが登場しています。
これは \texttt{handan}環境(別名 \texttt{ifbox}環境)が終了した後,
流れ図をどこに続けるかを指定するコマンドで,\cmd{pushcurrentP}直前の
流れ線に続けるように指定したことになります。

なお,\textsf{repbox}は\textsf{kurikaesi}の別名です。
\index{repbox@repbox 環境}

ついでですから,最小値も欲張ってみましょうか。
ソースリストは \texttt{example}ディレクトリの\texttt{rep1.tex}を見てください。

\begin{center}
\begin{caprm}
\begin{nagarezu}(22.5\zw,27\baselineskip)(7.5\zw,0pt)
  \termbox{始め}
  \downto
  \opbox(10\zw){$MAX=A(1)$}
  \opbox(10\zw){$MIN=A(1)$}
  \downto
  \begin{kurikaesi}(11\zw){I=2,3,4,5,6}{I}
    \downto
    \begin{ifbox}<.7><1.4>{\strut$A(I)>MAX$}
      \begin{sitabunki}
        \downto(61pt)"Yes"
        \opbox{$MAX=A(I)$}
        \downto[skipminmax]
        \pushcurrentP
      \end{sitabunki}
      \begin{migibunki}
        \rightto<n>(40pt)"No"
        \downto
        \begin{ifbox}<.7><1.4>{\strut$A(I)<MIN$}
          \begin{branch}{d}
            \downto"Yes"
            \opbox{$MIN=A(I)$}%
            \downto<n>`skipminmax'
            \leftto`skipminmax'
            \pushcurrentP
          \end{branch}%
          \begin{branch}{r}
            \rightto<n>"No"
            \downto<n>`skipminmax'
            \leftto`skipminmax'
          \end{branch}
        \end{ifbox}
      \end{migibunki}
    \end{ifbox}
  \end{kurikaesi}
  \downto
  \iobox{MAX, MIN}
  \downto
  \termbox{終り}
\end{nagarezu}%
\end{caprm}

\end{center}

\clearpage


\printindex
\end{document}
