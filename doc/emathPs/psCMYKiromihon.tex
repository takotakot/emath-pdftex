\documentclass[a4j]{jarticle}
\usepackage[cmyk]{emathPs}
\def\iromihon#1{%
\begin{pszahyou*}[borderwidth=2pt,ul=8mm,xscale=2.5,Sitayohaku=1.6\zh,
    Migiyohaku=1\zw,Hidariyohaku=1\zw](0,1)(0,1)\relax
\Nuritubusi[nuriiro=#1]{\trueLT\trueLB\trueRB\trueRT}%
\Takakkei<iro=Black>{\trueLT\trueLB\trueRB\trueRT}%
\Put{(.5,0)}(0,-3pt)[t]{$\mathstrut$#1}%
\end{pszahyou*}}
\begin{document}
%\color[named]{Black}
\parindent=0pt
\texttt{dvipsname.def}で定義されている色

\iromihon{GreenYellow}
\iromihon{Yellow}
\iromihon{Goldenrod}
\iromihon{Dandelion}
\iromihon{Apricot}\\
\iromihon{Peach}
\iromihon{Melon}
\iromihon{YellowOrange}
\iromihon{Orange}
\iromihon{BurntOrange}\\
\iromihon{Bittersweet}
\iromihon{RedOrange}
\iromihon{Mahogany}
\iromihon{Maroon}
\iromihon{BrickRed}\\
\iromihon{Red}
\iromihon{OrangeRed}
\iromihon{RubineRed}
\iromihon{WildStrawberry}
\iromihon{Salmon}\\
\iromihon{CarnationPink}
\iromihon{Magenta}
\iromihon{VioletRed}
\iromihon{Rhodamine}
\iromihon{Mulberry}\\
\iromihon{RedViolet}
\iromihon{Fuchsia}
\iromihon{Lavender}
\iromihon{Thistle}
\iromihon{Orchid}\\
\iromihon{DarkOrchid}
\iromihon{Purple}
\iromihon{Plum}
\iromihon{Violet}
\iromihon{RoyalPurple}\\
\iromihon{BlueViolet}
\iromihon{Periwinkle}
\iromihon{CadetBlue}
\iromihon{CornflowerBlue}
\iromihon{MidnightBlue}\\
\iromihon{NavyBlue}
\iromihon{RoyalBlue}
\iromihon{Blue}
\iromihon{Cerulean}
\iromihon{Cyan}\\
\iromihon{ProcessBlue}
\iromihon{SkyBlue}
\iromihon{Turquoise}
\iromihon{TealBlue}
\iromihon{Aquamarine}\\
\iromihon{BlueGreen}
\iromihon{Emerald}
\iromihon{JungleGreen}
\iromihon{SeaGreen}
\iromihon{Green}\\
\iromihon{ForestGreen}
\iromihon{PineGreen}
\iromihon{LimeGreen}
\iromihon{YellowGreen}
\iromihon{SpringGreen}\\
\iromihon{OliveGreen}
\iromihon{RawSienna}
\iromihon{Sepia}
\iromihon{Brown}
\iromihon{Tan}\\
\iromihon{Gray}
\iromihon{Black}
\iromihon{White}
\newpage

日本の伝統色(色の小辞典)\\
(財団法人日本色彩研究所編/福田邦夫著 読売新聞社発行)
より(一部)

\iromihon{桜色}%
\iromihon{一斤染}%
\iromihon{肌色}%
\iromihon{枯色}%
\iromihon{香色}\\
\iromihon{白茶}%
\iromihon{黄蘗色}%
\iromihon{鳥の子色}
\iromihon{練色}%
\iromihon{若葉色}\\
\iromihon{柳色}%
\iromihon{抹茶色}%
\iromihon{浅緑}%
\iromihon{白緑}%
\iromihon{青磁色}\\
\iromihon{水浅葱}%
\iromihon{水色}%
\iromihon{瓶覗}%
\iromihon{空色}%
\iromihon{藤色}\\
\iromihon{薄色}%
\iromihon{乳白色}%
\iromihon{灰白色}%

その他

\iromihon{PastelBlue}%
\end{document}
