\documentclass[a4j]{jarticle}
\usepackage{enumerate}%       必須ではありません。enumerate.sty を読み込むときは yenumerateの前に
\usepackage[dvips]{graphicx}% 必須ではありません。当文書で図を取り込むため
\usepackage{emathEy}
\usepackage{jquote}%          これも,当文書のためにのみ必要です。
\usepackage{showexample}%     これも,当文書のためにのみ必要です。

\title{\textsf{emathYe.sty} \\{\normalsize v 0.00}\\
使用例}
\author{tDB}
\date{2005/09/03}

\begin{document}
\maketitle\thispagestyle{empty}
\begin{abstract}%
\parindent1\zw%
\textsf{enumerate}環境で、項目を横に並べるための環境を
いくつか用意したものです。

このマクロ集のマクロについてのご質問,バグ報告,修正・追加の提案等は
\begin{center}
http://emath.s40.xrea.com/
\end{center}
の掲示板へどうぞ。
\end{abstract}

\tableofcontents

\pagebreak
\pagenumbering{arabic}
\pagebreak

\def\bunsuu{\displaystyle\frac}%

\def\labelenumii{(\arabic{enumii})}
\section{概略}
\begin{enumerate}
\item 横に2個並べる\textsf{edaenumerate}環境
\begin{edaenumerate}
  \item ああ
  \item いい
  \item ううう
  \item えええ
  \item えええええ
  \item お
  \item お
  \item ええええ
  \item えええ
\end{edaenumerate}

\item 横に3個並べる\textsf{edaenumerate}環境

  \textsf{edaenumerate}環境は,デフォルトでは横に2問ずつ並べます。
  この個数を変更するのが\verb+<..>+オプションで,この問題では,
  \begin{quote}
  \verb+\begin{enumerate}<3>+
  \end{quote}
  としています。
\begin{edaenumerate}<3>
  \item ああ
  \item いい
  \item ううう
  \item えええ
  \item えええええ
  \item お
  \item お
  \item ええええ
  \item えええ
\end{edaenumerate}

\item 横に並べる個数を自動的に決める\textsf{yokoenumerate}環境

  \textsf{edaenumerate}が横に並べる小問数を指定する方式なのに対して,
  \textsf{yokoenumerate}は,小問の横幅を計量し最長のもので版面を均等割して
  横に並べる個数を求めます。
\begin{yokoenumerate}
  \item ああ
  \item いい
  \item ううう
  \item えええ
  \item えええええ
  \item お
  \item お
  \item ええええ
  \item えええ
\end{yokoenumerate}

\item 横にベタ詰で並べる\textsf{betaenumerate}環境

  \textsf{edaenumerate}, \textsf{yokoenumerate}はいずれも,
  小問の横幅は均一です。それに対し,\textsf{betaenumerate}は,
  小問をベタ詰していきます。したがって複数行になるとき,
  小問番号が縦に揃いません。
\begin{betaenumerate}
  \item ああ
  \item いい
  \item ううう
  \item えええ
  \item えええええ
  \item お
  \item お
  \item ええええ
  \item えええ
\end{betaenumerate}

\clearpage

% 以下で使用している図
%           golfer.ps, tiger.ps
% は Ghostscript に含まれるものです。

\item \textsf{edaenumerate}環境で図版を並べてみます。

以下で使用している図
\begin{verbatim}
  golfer.eps, tiger.eps
\end{verbatim}
は Ghostscript に含まれるものです。

  \begin{edaenumerate}
    \item \mbox{}\par\includegraphics[width=30mm,clip]{golfer.eps}
    \item \mbox{}\par\includegraphics[width=30mm,clip]{tiger.eps}
    \item \mbox{}\par\includegraphics[width=30mm,clip]{tiger.eps}
    \item \mbox{}\par\includegraphics[width=30mm,clip]{golfer.eps}
  \end{edaenumerate}

\makeatletter
\def\fgcaption{\def\@captype{figure}\caption}
\makeatother

\item 項目番号を取り去り,図番号を付けて見ます。\bigskip
  \begin{edaenumerate}
    \item[] 
    	\hfill\includegraphics[width=30mm,clip]{golfer.eps}\hfill\null
            \fgcaption{golfer}
    \item[] 
    	\hfill\includegraphics[width=30mm,clip]{tiger.eps}\hfill\null
            \fgcaption{tiger}
\edakaigyou\bigskip

    \item[] 
    	\hfill\reflectbox{\includegraphics[width=30mm,clip]{tiger.eps}}\hfill\null
            \fgcaption{tiger}
    \item[] 
    	\hfill\reflectbox{\includegraphics[width=30mm,clip]{golfer.eps}}\hfill\null
            \fgcaption{golfer}
  \end{edaenumerate}
\end{enumerate}
\clearpage

\section{小問を横に並べる。}
\subsection{横に並べる個数指定}
小問を横に並べる環境 \textsf{edaenumerate} です。

{
\renewcommand{\labelenumi}{\Alph{enumi}.}%
\renewcommand{\labelenumii}{【\arabic{enumii}】}%
\renewcommand{\labelenumiii}{(\arabic{enumiii})}%

\begin{enumerate}
\item 一番多いのは,横に2問ずつ並べるタイプでしょうか。

\showexample[\textsf{edaenumerate}環境](.95)(0.9){example/eda01}
\clearpage

\item 横に3問ずつ並べたいときは \texttt{<..>} オプションを使います。

\showexample[\texttt{edaenumerate<3>}](.95)(0.9){example/eda02}

なお,\LaTeXe で \textsf{enumerate} に [...] オプションを併用する場合は
\begin{quote}\begin{verbatim}
      \begin{edaenumerate}<3>[(1)]
\end{verbatim}
\end{quote}
のように,\verb/<..>/ オプションを先に付けてください。

\pagebreak

\item 小問の一部は2問ずつ,一部は3問ずつ並べたい,という例です。

\showexample[\texttt{変則}](.95)(0.9){example/eda05}
\pagebreak

\item 小問の位置(左インデント)を変更するには (...) オプションを用います。

\showexample[edaenumerate(...)](.95)(0.9){example/eda06}

%\pagebreak

\pagebreak

\item テスト問題などでは,縦方向に空白を作りたいことがあります。
\verb+<...>+ オプションに
\begin{jquote}
\texttt{gyoukan=2cm}
\end{jquote}
などと行間の長さを指定します。

\begin{showEx}{\texttt{<gyoukan=..>}}
\begin{enumerate}[1.~]
  \item 次の問いに答えよ。
    \begin{edaenumerate}%
    <gyoukan=2cm>[(1)]
      \item aaa
      \item bbb
      \item ccc
      \item ddd
    \end{edaenumerate}
\end{enumerate}
\end{showEx}

横に並べる小問数も指定したければ,オプションを
\begin{jquote}
\begin{verbatim}
<retusuu=3,gyoukan=2cm>
\end{verbatim}
\end{jquote}
などと記述します。
\index{retusuu=}\index{gyoukan=}
\pagebreak

均等割りなら,
\verb+\vfill+ も使えますし,
\begin{jquote}
\begin{verbatim}
<gyoukan=\vfill>
\end{verbatim}
\end{jquote}
と指定するのもありです。

%\showexample[\cmd{vfill} で縦方向の空白](.95)(.9){example/eda07}
\begin{itembox}{\cmd{vfill} で縦方向の空白}
\verbatiminput{example/eda07}
\end{itembox}

をタイプセットすると,下のようになります。

\begin{enumerate}
\item $B<!$NLd$KEz$($h!%(B
    \begin{edaenumerate}<gyoukan=\vfill>
    \item aaa
    \item bbb
    \item ccc
    \item ddd
    \end{edaenumerate}
\end{enumerate}

\clearpage

また,横に並べた小問の間に縦罫線を入れたいことがあります。
解答の余白を縦 \texttt{2cm} として,縦罫線を入れる例です。

\showexample[小問の間に縦罫線](.95)(.9){example/eda04}

\item eda... の書式です。
    \begin{enumerate}
    \item 環境
        \begin{verbatim}
edaenumerate<横に並べる小問の個数(デフォルトは2)>
            (小問の左インデント)
            [enumerate のオプションと同じ(LaTeX2eのみ)]
        \end{verbatim}
    \item コマンド
        \begin{verbatim}
\item<横に並べる小問の個数
    (デフォルトは \begin{edaenumerate}の指定値)>
     (小問の横幅)
     [\item のオプションと同じ]
        \end{verbatim}
    \end{enumerate}
\end{enumerate}
}

\subsection{ベタ並べ}
\textsf{edaenumerate}環境は,横に並べる個数を指定しますが,
個数は指定せず,ベタに並べるのが\textsf{betaenumerate}環境です。

\showexample[betaenumerate](.6)(.99){example/enumb01}

余白量はデフォルトで
\begin{jquote}
\begin{verbatim}
\def\betalabelwidth{3em}
\end{verbatim}
\end{jquote}
としてあります。
これを再定義して小さくすれば更に詰め込むことができます。
\clearpage

\subsection{\textsf{yokoenumerate}}
\textsf{betaenumerate}環境は追い込み式の横並べですが,
項目幅は項目ごとに異なります。
長さが最大の項目幅で横並べするのが \textsf{yokoenumerate} 環境です。

\begin{showEx}(1,1){\textsf{yokoenumerate}環境}
\begin{enumerate}[1.~]
\item \textsf{betaenumerate}環境
\begin{betaenumerate}
  \item aaa
  \item b
  \item ccccc
  \item ddddddddddd
  \item ee
  \item ffff
  \item ggggg
  \item hhhhhh
\end{betaenumerate}

\item \textsf{yokoenumerate}環境
\begin{yokoenumerate}
  \item aaa
  \item b
  \item ccccc
  \item ddddddddddd
  \item ee
  \item ffff
  \item ggggg
  \item hhhhhh
\end{yokoenumerate}
\end{enumerate}
\end{showEx}

この環境と\textsf{edaenumerate}環境との違いは,
\textsf{edaenumerate}環境は横に並べる項目数をこちらが指定するのに対して,
\textsf{yokoenumerate}環境は項目幅と版面との兼ね合いで\TeX が決めてくれる
という違いです。

\section{指定した項目番号に\texttt{*}などを附加}
問題集などで,一部の問題に\texttt{*}印などをつけようという話です。

\begin{showEx}{\cmd{item*}}
\begin{enumerate}[1.~]
  \item 次の問いに答えよ。
  \begin{edaenumerate}[%
    \protect\preitem(1)]
    \item* aaa
    \item bbb
    \item ccc
    \item* ddd
  \end{edaenumerate}
\end{enumerate}
\end{showEx}

すなわち,書式指定に\cmd{preitem}を置き,
\texttt{*}を付けたい項目をアスタリスク付の \cmd{item*}
で記述します。

\texttt{*}を番号の後ろにおきたければ

\begin{showpEx}{\cmd{item**}}
!\begin{enumerate}[1.~]
! \item 次の問いに答えよ。
  \begin{edaenumerate}[(1){%
  \makebox[0pt][l]\postitem}~]
    \item** aaa
    \item bbb
    \item ccc
    \item** ddd
  \end{edaenumerate}
!\end{enumerate}
!\end{showpEx}
書式指定に\cmd{postitem}を加え,アスタリスクを2個つけた\cmd{item**}を
用います。

また,後置記号のありなしで番号部分の横幅が揃うように
\begin{jquote}
\begin{verbatim}
  \begin{edaenumerate}[(1){\makebox[0pt][l]\postitem}~]
\end{verbatim}
\end{jquote}
と,細工をしています。

\texttt{*}以外の記号を使いたいときは,\cmd{item*}, \cmd{item**} に
\verb+[..]+オプションをつけます。

\begin{showpEx}{\cmd{item*[..]}}
!\begin{enumerate}[1.~]
! \item 次の問いに答えよ。
  \begin{edaenumerate}[%
    \protect\preitem(1)]
    \item*[\dag] aaa
    \item bbb
    \item ccc
    \item* ddd
  \end{edaenumerate}
!\end{enumerate}
!\end{showpEx}

後置する場合です。

\begin{showpEx}{\cmd{item**[..]}}
!\begin{enumerate}[1.~]
! \item 次の問いに答えよ。
  \begin{edaenumerate}[(1){%
  \makebox[0pt][l]\postitem}~]
    \item**[\dag] aaa
    \item bbb
    \item ccc
    \item** ddd
  \end{edaenumerate}
!\end{enumerate}
!\end{showpEx}

前と後ろの両方に記号をつけたい,ということは起こりそうもありませんが,
可能ではあります。

\begin{showpEx}{前後置}
!\begin{enumerate}[1.~]
! \item 次の問いに答えよ。
  \begin{edaenumerate}[%
  \protect\preitem(1){%
  \makebox[0pt][l]\postitem}~]
    \item*[*][\dag] aaa
    \item* bbb
    \item**[\dag] ccc
    \item*[][\ddag] ddd
  \end{edaenumerate}
!\end{enumerate}
!\end{showpEx}
と,\cmd{item*}に\verb+[前置記号][後置記号]+と2つのオプションをつけます。
この場合,前置記号が`\texttt{*}'であっても,明示する必要があります。

以上の例はすべて\textsf{edaenumerate}環境で示しましたが,
一般の\textsf{enumerate}環境,各種の横並び\textsf{enumerate}環境でも
有効です。

\end{document}
