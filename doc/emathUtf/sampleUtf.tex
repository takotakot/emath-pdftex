\documentclass[a4j]{jarticle}
\usepackage{emath}
\usepackage{emathUtf}
\usepackage{showexample}
\usepackage{type1cm}% PS, PDF 作成には必要
%
\begin{document}
\title{\cmd{ajMaru}などで相互参照を行うマクロ\\
emathUtf.sty {\normalsize ver.0.00}\\使用例}
\author{tDB}
\date{2005/09/10}

\maketitle\thispagestyle{empty}
\begin{abstract}%
\parindent1\zw%
utf パッケージで定義されている丸付き数字を呼び出すマクロ
\cmd{ajMaru}などでは,相互参照がうまくいきません。
\verb+\label{foo}+で定義したラベルを\verb+\ajMaru{\ref{foo}}+
で参照するとエラーとなります。

そこで,\cmd{ajMaru}を再定義して相互参照を行うことを可能としました。

ただし,このマクロは\textsf{utf}を前提としますから
\begin{jquote}
\begin{verbatim}
OS 依存
Dvi-ware 依存
環境依存
\end{verbatim}
\end{jquote}
です。

なお,\textsf{utf.sty}は\textsf{emathUtf.sty}の中で強制的に読み込まれます。
また,\textsf{otf.sty}とは併用できません。
\bigskip

このマクロ集のマクロについてのご質問,バグ報告,修正・追加の提案等は
\begin{center}
http://emath.s40.xrea.com/
\end{center}
の掲示板へどうぞ。
\end{abstract}
%\pagebreak
%\pagenumbering{roman}%

\tableofcontents

\pagebreak

\pagenumbering{arabic}

\section{\cmd{ajMaru}}
\subsection{enumerate環境の番号に使用した場合}
次の例では,
\begin{jquote}
\begin{verbatim}
\begin{enumerate}[m]
\end{verbatim}
\end{jquote}
により,番号付けには\cmd{maru}が用いられますが,
\begin{jquote}
\begin{verbatim}
\let\maru\ajMaru
\end{verbatim}
\end{jquote}
としてありますから,\cmd{ajMaru}が使用されます。

\begin{showEx}{\cmd{ajMaru}による相互参照}
\let\maru\ajMaru
次の選択肢から選べ。
ただし,\ref{Eii}において...
\begin{enumerate}[m]
  \item aaa
  \item \label{Eii}bbb
  \item ccc
\end{enumerate}
\end{showEx}

ただし,\verb+\ref{ラベル名}+で,丸付き数字が返されましたが,
\textsf{enumerate}環境の設定によっては,丸の中の数値だけが返されます。

\begin{showEx}{\cmd{labelenumi}による場合}
\let\maru\ajMaru
次の選択肢から選べ。
ただし,\ref{E2}において...
\def\labelenumi{\maru{\theenumi}}
\begin{enumerate}
  \item aaa
  \item \label{E2}bbb
  \item ccc
\end{enumerate}
\end{showEx}

丸付き数字で参照するには,
\begin{jquote}
\begin{verbatim}
  \maru{\ref{E2}}
\end{verbatim}
\end{jquote}
としなければなりません。
\newpage

\subsection{数式番号に使用した場合}
数式番号に対しても
\begin{jquote}
\begin{verbatim}
\let\maru\ajMaru
\end{verbatim}
\end{jquote}
とすることで,\cmd{ajMaru}が用いられます。

\begin{showEx}{数式番号に使用}
\let\maru\ajMaru
\begin{gather}
  2x+3y=5 \label{eq1} \\
  3x-2y=1 \label{eq2}
\end{gather}
\eqref{eq1}, \eqref{eq2}より
\end{showEx}

参照は,\cmd{eqref}により丸付き番号が,
\cmd{ref}により,丸の中の数値が参照されます。

\section{\cmd{ajMaruKaku}}
\cmd{ajMaru}のみならず,\cmd{ajMaruKaku}などに対しても
相互参照を行うことができます。

\begin{showEx}{\cmd{ajMaruKaku}による相互参照}
\let\maru\ajMaruKaku
次の選択肢から選べ。
ただし,\ref{MKii}において...
\begin{enumerate}[m]
  \item aaa
  \item \label{MKii}bbb
  \item ccc
\end{enumerate}
\end{showEx}
\newpage

\section{ローマ数字}
ローマ数字は,英文字(I, V, X, ...)などを組み合わせてあらわすのが正書法ですが,
全角一文字幅に収めたい,というご意見もあるようです。

\begin{showEx}{\cmd{ajRoman}による相互参照}
\let\maru\ajRoman
\begin{enumerate}[m]
  \item aaa
  \item \label{Rmii}bbb
  \item \ref{Rmii}において,ccc
\end{enumerate}
\end{showEx}

次は,小文字で(..)でくくった形式です。

\begin{showEx}{\cmd{ajKakkoroman}による相互参照}
\let\maru\ajKakkoroman
\begin{enumerate}[m]
  \item aaa
  \item \label{rmii}bbb
  \item \ref{rmii}において,ccc
\end{enumerate}
\end{showEx}
\newpage

関連して,算用数字を括弧でくくり等幅にする例です。

\begin{showEx}{\cmd{ajKakko}による相互参照}
\let\maru\ajKakko
\begin{enumerate}<syokiti=7>[m]
  \item aaa
  \item \label{nii}bbb
  \item \ref{nii}において,ccc
  \item ddd
\end{enumerate}
\end{showEx}

さらに脱線して,算用数字の桁数にかかわらず,全角幅で

\begin{showEx}{\cmd{ajarabic}による相互参照}
\let\maru\ajarabic
\begin{enumerate}<syokiti=7>[m]
  \item aaa
  \item \label{nii}bbb
  \item \ref{nii}において,ccc
  \item ddd
\setcounter{enumi}{122}
  \item zzz
\end{enumerate}
\end{showEx}

ここまでくると悪趣味かな。
\end{document}
