\section{空間図形}
\subsection{円柱}

\showexample[円柱](0.9)(0.4){example/entyu01}

\subsection{円錐}

\showexample[円錐](0.9)(0.45){example/ensui01}

\subsection{円錐台}

\showexample[円錐台](0.9)(0.4){example/ensui02}

\subsection{四面体}

\showexample[四面体](0.9)(0.6){example/simentai}

\subsection{四角錐}

\showexample[四角錐](0.9)(0.6){example/kakusui1}

\subsection{平行六面体}

\showexample[平行六面体](0.9)(0.6){example/tamenta1}

基本ベクトルの数値を変更すると,直方体・立方体も作れます.
頂点の記号はエディタの一括置換機能を用いることを想定しています.

\showexample[平行六面体](0.9)(0.6){example/tamenta2}
