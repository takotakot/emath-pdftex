\documentclass[a4j]{jarticle}
\usepackage{longtable}
\usepackage{multido}
\usepackage{emathMw}
\usepackage{emathP}
\usepackage{emathBk}
\usepackage{emathR}
\usepackage{showexample}
\usepackage{makeidx}
\usepackage{type1cm}% PS, PDF 作成には必要
\usepackage{emathPh}
\usepackage{bekutorukata}
%
% しおり作成
%\usepackage[dvips]{hyperref}
%
\makeindex

\newcommand{\cindex}[1]{\index{#1@\texttt{\protect\cmd{#1}}}}
\newdimen\mydimena
\newdimen\mydimenb

\makeatletter
\@ifundefined{texorpdfstring}{\def\texorpdfstring#1#2{#1}}{}%
\makeatother


\unitlength1cm

%\includeonly{SCandL}

\begin{document}
\title{picture 環境支援マクロ集\\
emathP.sty {\normalsize ver.0.66}\\
emathPb.sty {\normalsize ver.0.01}\\
emathPg.sty {\normalsize ver.0.00}\\
emathPh.sty {\normalsize ver.2.45}\\
emathPk.sty {\normalsize ver.0.94}\\
emathPxy.sty {\normalsize ver.0.31}\\
emathT.sty {\normalsize ver.0.42}\\
使用例}
\author{tDB}
\date{2005/11/02}

\maketitle\thispagestyle{empty}
\begin{abstract}%
\parindent1zw%
中学・高校で数学のプリントにつける図の作成に必要な記号,コマンド,
環境を集めたマクロ集です.
%\LaTeXe が前提です.

このマクロ集のマクロについてのご質問,バグ報告,修正・追加の提案等は
\begin{center}
http://emath.s40.xrea.com/
\end{center}
の掲示板へどうぞ。
\end{abstract}
\pagebreak
\pagenumbering{roman}%

\tableofcontents

\pagebreak

\pagenumbering{arabic}

%\section{直線図形}
\section{\textsf{zahyou}環境}
\subsection{なぜ\textsf{zahyou}環境か}
\LaTeX で座標平面を描画するには,\textsf{picture}環境があります。
ただし,負の数を扱うには,\textsf{picture}環境の引数を
計算しなければなりません。
例えば,$-3<x<3$, $-2<y<2$の範囲を描画するには

\begin{showEx}(.5,.44){\textsf{picture}環境}
\unitlength=8mm
\begin{picture}(6,4)(-3,-2)%
\put(-3,0){\vector(1,0){6}}%
\put(0,-2){\vector(0,1){4}}%
\end{picture}%
\end{showEx}

引数の与え方が面倒であることと,
座標軸の描画もまとめて面倒を見てしまおう,
ということで\textsf{zahyou}環境です。

\subsection{\textsf{zahyou}環境}

\begin{showEx}(.5,.44){\textsf{zahyou}環境}
\unitlength=8mm
\begin{zahyou}(-3,3)(-2,2)%
\end{zahyou}%
\end{showEx}

\textsf{picture}環境とは,引数の与え方が異なり,
\begin{jquote}
\begin{verbatim}
(xmin, xmax)(ymin, ymax)
\end{verbatim}
\end{jquote}
と,$x$, $y$の区間を与えます。

なお,この環境内では,次の変数が定義されています。
\begin{jquote}
\begin{verbatim}
\xmin xの区間の最小値
\xmax xの区間の最大値
\ymin yの区間の最小値
\ymax yの区間の最大値

\O    原点
\XMAX x軸の右端の点
\XMIN x軸の左端の点
\YMAX y軸の上端の点
\YMIN y軸の下端の点
\RT   描画領域右上のコーナー
\RB   描画領域右下のコーナー
\LT   描画領域左上のコーナー
\LB   描画領域左下のコーナー
\end{verbatim}
\end{jquote}
\bigskip

座標軸を描画する必要がない場合に対しては
\textsf{zahyou*}環境を用意してあります。

\begin{showEx}(.5,.44){\textsf{zahyou*}環境}
\unitlength=8mm
\begin{zahyou*}(-3,3)(-2,2)%
  \Drawline{\LB\RT\RB\LT\LB}
  \En\O{2}%
\end{zahyou*}%
\end{showEx}

上記リスト中,\cmd{Drawline}は,指定点を折れ線で結びます。
\verb+\En\O{2}+は点\cmd{O}(原点)を中心とする半径2の円を描画しています。
詳細は後述します。

\textsf{zahyou}環境には,細かい微調整をするためのオプションがありますが,
これについては後述します。

\section{直線図形}
\subsection{折れ線}
折れ線を描画するのに,\textsf{epic.sty} で
定義されている \cmd{drawline} を用いることができます.

\showexample[\cmd{drawline}](0.55)(0.35){example/oresen01}

このスタイルファイルでは,点を \cmd{def}\cmd{A\{(3,0)\}}などと
変数で表すことにしています.その場合は,\cmd{drawline} は使えません.
代わりに \cmd{Drawline} を用意しました.このコマンドは内部で
\cmd{drawline} を呼び出していますから,\textsf{epic.sty} を必要とします.

\showexample[\cmd{Drawline}](0.55)(0.35){example/oresen02}
\cindex{Drawline}\bigskip

\cmd{Drawline} の書式は

\begin{boxnote}
\begin{verbatim}
\Drawline#1
    #1 : 点列
\end{verbatim}
\end{boxnote}

\subsection{点線}
点線を描画するには\textsf{eepic.sty}で定義されている\cmd{dottedline}を用います。

\begin{showEx}(.64,.3){\cmd{dottedline}}
\begin{zahyou}[ul=5mm](-3,3)(-3,3)
  \dottedline{.2}(-2,2)(2,-1)
\end{zahyou}
\end{showEx}

\subsubsection{\cmd{emDottedline}}
\textsf{emath}では,点列を一つにまとめた扱いますので,
\cmd{emDottedline}を定義しました。

\begin{showEx}(.64,.3){\cmd{emDottedline}}
\begin{zahyou}[ul=5mm](-3,3)(-3,3)
  \tenretu{A(-2,2)w;B(2,-1)e}
  \emDottedline{\A\B}
\end{zahyou}
\end{showEx}

\cmd{dottedline}では,第1引数に点を打つ間隔を指定しますが,\cmd{unitlength}を単位とする
無名数であることが使い難いので,\cmd{emDottedline}では省略することとしました。
デフォルトは\verb+3pt+としてありますが,これを変更するには\verb+<G=..>+オプションを用います。

\subsubsection{\texttt{<G=..>}オプション}
さて,その\verb+<G=..>+オプションを用いて,もう少し点を稠密にして見ましょう。

\begin{showEx}(.64,.3){\texttt{<G=..>}オプション}
\begin{zahyou}[ul=5mm](-3,3)(-3,3)
  \tenretu{A(-2,2)w;B(2,-1)e}
  \emDottedline<G=2pt>{\A\B}
\end{zahyou}
\end{showEx}
\newpage

\subsubsection{\texttt{<C=..>}オプション}
\cmd{dottedline}には,配置する点の形状を変更するオプション\verb+[...]+が用意されています。

\cmd{emDottedline}では,\verb+<C=..>+オプションを用意しました。

\begin{showEx}(.64,.3){\texttt{<C=..>}オプション}
\begin{zahyou}[ul=5mm](-3,3)(-3,3)
  \tenretu{A(-2,2)w;B(2,-1)e}
  \emDottedline<C=$*$,G=5pt>{\A\B}
\end{zahyou}
\end{showEx}

小さい円(黒塗りの)にしますか。

\begin{showEx}(.64,.3){小さい円を配置}
\begin{zahyou}[ul=5mm](-3,3)(-3,3)
  \ukansan{1pt}\tyokkei
  \tenretu{A(-2,2)w;B(2,-1)e}
  \emDottedline<C=\circle*\tyokkei>{\A\B}
\end{zahyou}
\end{showEx}

ここで登場しているコマンド\verb+\ukansan#1#2+は,
\verb+#1+に与えられた,単位付の長さを,\cmd{unitlength}を単位とする無名数に換算した結果を
\verb+#2+の制御綴に与えるものです。

\subsubsection{応用例}
奥村先生の掲示板 \verb+Q&A+ に「行列の点々」という投稿がありました。
\textsf{emath}を用いる一案です。

\setbox0=\hbox{$b_{n-1}+d_{n-1}$}\setlength{\templa}{\wd0}\edef\wdtbl{\the\templa}%
\setlength{\templa}{2\wd0}\edef\xlen{\mumeisuu\templa}
\setlength{\templa}{\ht\strutbox+\dp\strutbox+20pt}
\edef\ylen{\mumeisuu\templa}
\def\PQ{(0,-3)(\xlen,-\ylen)}%
\[
\setlength{\arraycolsep}{0pt}%
  \left[
    \begin{array}{*8{C{\wdtbl}}}
    a_1 &b_1+d_1  &c_1    &   &   &   &   \\
      &a_2    &b_2+d_2  &c_2    &   &   &   \\
      &   &\EMcell<\emDottedline\PQ>{c}{a_3}
          &\EMcell<\emDottedline\PQ>{c}{b_3+d_3}
          &\EMcell<\emDottedline\PQ>{c}{c_3}    &   &   \\[20pt]
      &   &   &   &   &   & \\
      &   &   &   &a_{n-1}  &b_{n-1}+d_{n-1}&c_{n-1}  \\
      &   &   &   &   &a_n    &b_n+d_n  & c_n
    \end{array}
  \right]
\]

ただし,このコマンドは\textsf{multido.sty}で定義されている\cmd{multido}を用います。
つぎの\cmd{Dottedline}は,\cmd{multido}は使いません。

\subsubsection{旧定義 \cmd{Dottedline}}
旧コマンドは \cmd{Dottedline} も有効です.

\showexample[\cmd{Dottedline}](0.55)(0.35){example/tensen01}

このコマンドは内部で \textsf{eepic.sty} の \cmd{dottedline} を
呼び出しています.書式もほぼ同様で

\begin{boxnote}
\begin{verbatim}
\Dottedline#1#2
    #1 : dotgap
    #2 : 点列
\end{verbatim}
\end{boxnote}

\subsection{破線}
\subsubsection{\texorpdfstring{\cmd{Dashline}}{Dashline}}
破線を引くコマンドは \cmd{Dashline} です.\cindex{Dashline}

\showexample[\cmd{Dashline}](0.55)(0.35){example/hasen01}

このコマンドは内部で \textsf{eepic.sty} の \cmd{dashline} を
呼び出しています.書式もほぼ同様で

\begin{boxnote}
\begin{verbatim}
\Dashline[#1]#2#3
    #1 : stretch
    #2 : dashlength
    #3 : 点列
\end{verbatim}
\end{boxnote}
です.\texttt{[dashdotgap]} はサポートしていません.
\texttt{[stretch]} で代用してください.

\showexample[\cmd{Dashline}\texttt{[stretch]}](0.55)(0.35){example/hasen02}

\subsubsection{\texorpdfstring{\cmd{hasen}}{hasen}}
\cmd{dashline}の第1引数は \cmd{unitlength}の値によって変えなければなりません。

\begin{showEx}{\cmd{Dashline}と\cmd{unitlength}}
\unitlength=10mm
\begin{picture}(3,3)
\dashline{.2}(0,0)(3,3)
\end{picture}\\
\unitlength=1mm
\def\B{(30,30)}
\begin{picture}(30,30)
\dashline{2}(0,0)(30,30)
\end{picture}
\end{showEx}

これは面倒ですから,\cmd{hasen}を作りました。

\begin{showEx}{\cmd{hasen}}
\unitlength=10mm
\begin{picture}(3,3)
\hasen(0,0)(3,3)
\end{picture}\\
\unitlength=1mm
\begin{picture}(30,30)
\hasen(0,0)(30,30)
\end{picture}
\end{showEx}

また,\cmd{dashline}における第1引数は不要とし,
\cmd{hasen}の形状はオプション引数で指定する方式を取りました。
\cmd{hasen}の書式です。

\begin{boxnote}
\begin{verbatim}
\hasen[#1](x1,y1)(x2,y2).....(xN,yN)
    #1 : L=(破線の実線部分の長さ) デフォルト値=1mm
       : G=(破線のギャップの長さ) デフォルト値=0.9mm
    (x1,y1)...(xN,yN) : 折れ線の頂点列
\end{verbatim}
\end{boxnote}\cindex{hasen}

オプション引数を与えて破線の形状を変更する例です。

\begin{showEx}{\cmd{hasen}の形状変更}
\unitlength=10mm
\begin{picture}(3,3)
\hasen[L=2mm,G=2mm](0,0)(3,3)
\end{picture}
\end{showEx}

ただし,ギャップの長さは必ずしも指定した長さにはなりません。
というのは,破線の両端における実線部分が指定した長さとなるように
ギャップを調整しているからです。

細かいことですが,\cmd{drawline}で描画した直線と
\cmd{dashline}で描画した破線の位置がずれることがあります。
これを修正するのも\cmd{hasen}を開発した目的の一つです。

\subsubsection{\texorpdfstring{\cmd{Hasen}}{Hasen}}
\cmd{drawline}, \cmd{dashline} に対して,\cmd{Drawline}, \cmd{Dashline}
があるように,\cmd{hasen} に対しても \cmd{Hasen} を定義しています。
その書式は

\begin{boxnote}
\begin{verbatim}
\Hasen[#1]#2
    #1 : L=(破線の実線部分の長さ) デフォルト値=1mm
       : G=(破線のギャップの長さ) デフォルト値=0.9mm
    #2 : 破線で結ぶ点列
\end{verbatim}
\end{boxnote}\cindex{Hasen}

\begin{showEx}(.58,.36){\cmd{Hasen}}
\begin{zahyou}[ul=10mm](-.5,2.5)(-.5,1.5)
\tenretu{A(0,1)w;B(2,1)e;C(2,0)s}
\Hasen{\A\B\C}
\end{zahyou}
\end{showEx}
複数の点列を定義する\cmd{tenretu}など,
まだ説明していないコマンドが登場してしまいましたが,
ここでは折れ線を破線で引く例としてご覧ください。

\subsubsection{\texorpdfstring{\cmd{Hasen*}}{Hasen*}}
2点を破線で結ぶコマンド\cmd{hasen}, \cmd{Hasen}は,
両端から実線部分が始まりますが,これを両端からは空白部分が
始まるようにしたものが\verb+*+つきコマンドです。

下の例では,実線ABと破線BCの境界がBではなくなってしまいます。

\begin{showEx}(.6,.345){\cmd{Hasen}}
\begin{zahyou}[ul=10mm](-.5,1.5)(-2.5,2.5)
\tenretu{A(1,2)e;B(1,1)e;C(1,-1)e;D(1,-2)e}
\kuromaru{\A;\B;\C;\D}
\Drawlines{\A\B;\C\D}
\Hasen{\B\C}
\end{zahyou}
\end{showEx}

\cmd{Hasen*}と比較してみてください。
\begin{showEx}(.6,.345){\cmd{Hasen*}}
\begin{zahyou}[ul=10mm](-.5,1.5)(-2.5,2.5)
\tenretu{A(1,2)e;B(1,1)e;C(1,-1)e;D(1,-2)e}
\kuromaru{\A;\B;\C;\D}
\Drawlines{\A\B;\C\D}
\Hasen*{\B\C}
\end{zahyou}
\end{showEx}
\cindex{Hasen*}


\subsection{鎖線}
鎖線をひくコマンドは \cmd{Chainline}です。\cindex{Chainline}

{\unitlength8mm
\showexample[\cmd{Chainline}](0.64)(0.3){example/sasen01}}

その書式は
\begin{boxnote}
\begin{verbatim}
\Chainline[#1][#2]#3#4
    #1 : 一つの線分の長さ
    #2 : 線分と線分の間の長さ
    #3 : 始点
    #4 : 終点
\end{verbatim}
\end{boxnote}

\subsection{複数の折れ線}
複数の折れ線を描画するコマンド \cmd{Drawlines} もあります。
\cindex{Drawlines}
複数の折れ線を`;'で区切ります。

\showexample[\cmd{Drawlines}](0.55)(0.35){example/oresen03}

折れ線の線種を変更するには,\verb/<...>/ オプションを用います。

\showexample[\cmd{線種の変更}](0.55)(0.35){example/oresen04}

\cmd{Drawlines}の書式です。

\begin{boxnote}
\begin{verbatim}
\Drawlines<#1>#2
  <#1>  : <sensyu=\dashlines[40]{.1}>などによる
          線種のローカルな変更
   #2   : 複数の点列を`;'で区切る
\end{verbatim}
\end{boxnote}

\subsection{矢線}
矢線を引くコマンドは2種類用意してあります。

\cmd{yasen}は,成分を指定します。

もうひとつの\cmd{ArrowLine}は,始点と終点を指定します。

\subsubsection{\texorpdfstring{\cmd{yasen}}{yasen}}
まずは成分を与えて矢線を描画するコマンド \cmd{yasen} を使用する一例です:

\begin{showEx}(.54,.4){\cmd{yasen}}
  \begin{zahyou*}[ul=10mm](0,4)(0,3)
    \put(0,0){\kousi43}%
    \put(1,0){\yasen(-1,3)}%
    \put(3,0){\yasen(1,3)}%
  \end{zahyou*}
\end{showEx}

このコマンドは,矢線の傍に文字列を配置するオプションをもっています。

\begin{showEx}(.54,.4){\texttt{<..>}オプション}
  \begin{zahyou*}[ul=10mm](0,4)(0,3)
    \put(0,0){\kousi43}%
    \put(1,0){%
      \yasen<[ne]{\beku a}>(-1,3)}%
    \put(3,0){%
      \yasen<[nw]{\beku b}>(1,3)}%
  \end{zahyou*}
\end{showEx}

すなわち\verb+<...>+オプションに,文字列を与えます。その際,\cmd{Put}の
配置オプションを附加することができます。
なお,文字列を配置する基準点は,矢線の中点です。
これを動かすのが\verb+[..]+オプションです。

\begin{showEx}(.54,.4){\texttt{[..]}オプション}
  \begin{zahyou*}[ul=10mm](0,4)(0,4)
    \put(0,0){\kousi43}%
    \put(1,0){%
      \yasen[1]<[n]{\beku a}>(-1,3)}%
    \put(3,0){%
      \yasen[.8]<[nw]{\beku b}>(1,3)}%
  \end{zahyou*}
\end{showEx}

\verb+[..]+オプションのデフォルト値は\verb+0.5+,すなわち中点です。
これを\verb+[1]+とすれば,終点,\verb+[0]+とすれば,始点がそれぞれ
文字列配置の基準点となります。

\cmd{yasen}の書式です。

\begin{boxnote}
\begin{verbatim}
\yasen[#1]<#2>(#3,#4)
\Yasen[#1]<#2>#3
  #1 : ラベルを置く位置(デフォルト=0.5, 中点)
  #2 : ラベル(位置指定を含めて)
  (#3,#4) : 成分(始点は \put (\Put) で指定)
\end{verbatim}
\end{boxnote}
\cindex{yasen}

成分を,まとめてひとつにして扱うものが \cmd{Yasen} です。
\cindex{Yasen}

\begin{showEx}(.54,.4){\cmd{Yasen}}
  \begin{zahyou*}[ul=10mm](0,4)(0,3)
    \def\avec{(1,3)}
    \put(0,0){\kousi43}%
    \put(0,0){\Yasen\avec}%
    \put(3,0){\Yasen\avec}%
  \end{zahyou*}
\end{showEx}

\cmd{Yasen}に対しても,\verb+[...]+, \verb+<...>+ オプションが
\cmd{yasen}と同様に使用できます。


\subsubsection{\texorpdfstring{\cmd{ArrowLine}}{ArrowLine}}
矢線を引くコマンドが \cmd{ArrowLine} です.

\showexample[\cmd{ArrowLine}](0.55)(0.35){example/yasen01}

矢線の傍に文字列を配置することも可能です。

\begin{showEx}(.54,.4){\cmd{ArrowLine}の文字列配置オプション}
  \begin{zahyou*}[ul=10mm](0,4)(-1,4)
    \tenretu{A(0,0)s;B(3,0)s;
      C(4,3)n;D(1,3)n}
    \Drawline{\B\C\D}
    {\thicklines
    \ArrowLine<putstr=[s]{\beku b}>
      \A\B
    \ArrowLine<putstr=[nw]{\beku d}>
      \A\D}%
  \end{zahyou*}
\end{showEx}

すなわち,\verb+<...>+オプションとして,\verb+putstr+の右辺値に,
文字列を配置オプションとともに記述します。配置基準点(デフォルトは中点)
を変更するには,キーワード\verb+putpos+を用います。

\begin{showEx}(.54,.4){\cmd{ArrowLine}の文字列配置基準点の変更}
  \begin{zahyou*}[ul=10mm](0,4)(-1,4)
    \tenretu{A(0,0)s;B(3,0)s;
      C(4,3)n;D(1,3)n}
    \Drawline{\B\C\D\A}
    {\thicklines
    \ArrowLine<putpos=0.9,
      putstr=[n]{\beku b}>
        \A\B}
  \end{zahyou*}
\end{showEx}

\cmd{ArrowLine}の書式です。

\begin{boxnote}
\begin{verbatim}
\ArrowLine<#1>[#2]#3#4
  #1 : key=val
         sensyu=
         putstr=      (矢線の傍に置く配置オプション+文字列)
         putpos=      (文字列の配置基準位置:デフォルトは0.5, 中点)
  #2 : 矢印を置く位置(デフォルト = 1 すなわち終点)
       ただし #2=b のときは,両端に矢印
  #3 : 始点
  #4 : 終点
\end{verbatim}
\end{boxnote}
\cindex{ArrowLine}

\subsubsection{始点・終点位置の微調整}
\cmd{ArrowLine}で引かれる矢線の始点・終点を少しずらしたいときがあります。
例えば

\begin{showEx}(.64,.3){\cmd{ArrowLine}}
\begin{zahyou*}[ul=10mm](-.5,2.5)(-.5,1.5)
  \tenretu{A(0,0)w;B(2,1)e}
  \Kuromaru\B
  \ArrowLine\A\B
\end{zahyou*}
\end{showEx}

\noindent
のような場合,矢印が終点の黒丸にめり込んでいます。
矢印の終点を左下に少しずらしたいときなどのために
\begin{jquote}
\begin{verbatim}
<Henvi=..>    始点に対する修正ベクトル
<Henvii=...>  終点に対する修正ベクトル
\end{verbatim}
\end{jquote}
オプションを新設しました。
右辺値はベクトルで,成分は単位を伴った長さです。
右辺値には`,'が入りますから,右辺値全体を\verb+{...}+でくくる必要があります。

\begin{showEx}(.64,.3){\texttt{<Henvii=..>}オプション}
\begin{zahyou*}[ul=10mm](-.5,2.5)(-.5,1.5)
  \tenretu{A(0,0)w;B(2,1)e}
  \Kuromaru\B
  \ArrowLine<Henvii={(-.8pt,-.4pt)}>\A\B
\end{zahyou*}
\end{showEx}

なお
\begin{jquote}
\begin{verbatim}
<henvi=..>    始点に対する修正ベクトル
<henvii=...>  終点に対する修正ベクトル
\end{verbatim}
\end{jquote}
も同機能ですが,右辺値の成分は\cmd{unitlength}を単位とする無名数で与えます。

\subsubsection{鏃の形状}
鏃のサイズを変更することもできます.例えば
\begin{jquote}
\begin{verbatim}
\changeArrowHeadSize{1.5}
\end{verbatim}
\end{jquote}
とすれば,鏃の大きさは1.5倍になります。\cindex{changeArrowHeadSize}

{\showexample[\cmd{changeArrowHeadSize}](0.65)(0.3){example/yasen02}}

鏃の開き具合も調整できます.デフォルトは\cindex{ArrowHeadAngle}
\begin{jquote}
\begin{verbatim}
\def\ArrowHeadAngle{18}%
\end{verbatim}
\end{jquote}
となっています.これを 30 とすると,鏃は正三角形になります.

{\showexample[\cmd{ArrowHeadAngle}](0.65)(0.3){example/yasen03}}

サイズ・開き具合の両方を同時に変更するときは
\begin{jquote}
\begin{verbatim}
\changeArrowHeadSize[30]{3}
\end{verbatim}
\end{jquote}

と\cmd{changeArrowHeadSize}のオプションで開き角を指定することもできます。

\showexample[\cmd{ArrowHeadAngle}](0.65)(0.3){example/yasen07}

\cmd{ArrowLine}で引かれる矢線において,鏃は二等辺三角形を塗りつぶしていますが,
その底辺に窪みをつけることも可能です。
窪みの深さの「二等辺三角形の高さを1としての比率」
を\cmd{ArrowHeadPit}で指定します。

次の図では,座標軸が2個描画されますが,上が従来の形状,下が窪みをつけた形状です。

\begin{showEx}(.5,.44){\cmd{ArrowHeadPit}}
\begin{zahyou}[ul=8mm](-3,3)(-3,3)
  \ArrowLine\LB\RT
\end{zahyou}

\def\ArrowHeadPit{0.25}
\begin{zahyou}[ul=8mm](-3,3)(-3,3)
  \ArrowLine\LB\RT
\end{zahyou}
\end{showEx}

\cmd{changeArrowHeadSize}に\verb+<...>+オプションを用いて,
窪みを指定することもできます。
\begin{jquote}
\begin{verbatim}
  \def\ArrowHeadPit{.25}
と
  \changeArrowHeadSize<.25>{1}
\end{verbatim}
は同値です。前者の方が簡潔ですが,矢印の長さ・開き角を変更する際には後者の方が
便利でしょうか。

そんな馬鹿な,という極端な例です:

\begin{showEx}(.5,.44){\cmd{changeArrowSize}による指定}
\changeArrowHeadSize[30]<0.3333>{4}
\bekutorukata{fill}

\bekutoru{AB}

\begin{zahyou}[ul=8mm](-3,3)(-3,3)
  \ArrowLine\LB\RT
\end{zahyou}
\end{showEx}

\end{jquote}

なお,これら矢線では鏃が塗りつぶされますが,これを枠線のみ描画させるには
\begin{jquote}
\begin{verbatim}
\renewcommand\ArrowHeadType{l}
\end{verbatim}
\end{jquote}
とします。
デフォルトは \verb/\newcommand\ArrowHeadType{f}/ としてあります.
\cindex{ArrowHeadType}

\showexample[塗りつぶさない鏃](0.55)(0.35){example/yasen06}

\subsubsection{矢印の位置}
矢印を途中に付けるためには,\texttt{[...]} オプションで,
矢印を置く位置を指定します.

{\showexample[\cmd{ArrowLine[0.5]}](0.65)(0.4){example/yasen05}}

また,矢印を両向きにつけたいときは\verb+[b]+オプションをつけます。

\begin{showEx}{両向き矢印}
\unitlength10mm\relax
\begin{picture}(2,1)
\def\A{(0,.5)}
\def\B{(2,.5)}
\ArrowLine[b]\A\B
\end{picture}
\end{showEx}

\subsubsection{矢線を点線・破線}

矢線の線種を変更するには \cmd{ArrowLine} に\verb+<sensyu=...>+ オプションを
付与します。

\begin{showEx}(.54,.4){矢線を点線で}
\unitlength10mm\relax
\begin{picture}(2,1)
\def\A{(0,.5)}
\def\B{(2,.5)}
\ArrowLine%
  <sensyu=\protect\dottedline{.2}>%
  \A\B
\end{picture}
\end{showEx}

\begin{showEx}(.54,.4){矢線を破線で}
\unitlength10mm\relax
\begin{picture}(2,1)
\def\A{(0,.5)}
\def\B{(2,.5)}
\ArrowLine%
  <sensyu=\protect\hasen>%
  \A\B
\end{picture}
\end{showEx}

\subsubsection{複数の矢線}
複数の矢線を引くコマンドが \cmd{ArrowLines} です。
始点・終点を`;'で区切って並べます。

\begin{showEx}{\cmd{ArrowLines}}
\unitlength=10mm
\begin{picture}(4,2)
\def\A{(0,0)}
\def\B{(3,0)}
\def\C{(4,2)}
\def\D{(1,2)}
\ArrowLines[.5]{%
  \A\B;\B\C;\C\D;\D\A}
\end{picture}
\end{showEx}

\subsection{多角形}
多角形を描画するコマンドが\cmd{Takakkei}です。\cindex{Takakkei}
\bigskip

\showexample[\cmd{多角形}](0.55)(0.35){example/polygon2}

\cmd{Takakkei}の始点Oと終点Bを結ぶ線分が強制的に付加されます。
\bigskip

\subsection{正多角形,極座標→直交座標}
円周上に点をとるときは,極座標形式の方が便利です.
これを直交座標に変換するコマンド \cmd{kyokuTyoku} です.
使用例として,正六角形を描画してみました.\cindex{kyokuTyoku}

\showexample[\cmd{kyokuTyoku}](0.6)(0.35){example/polygon0}

\begin{jquote}
\cmd{kyokuTyoku(1,90)}\cmd{A}
\end{jquote}
で,極座標 (1,90) を直交座標に変換した (0,1) が \cmd{A} に
セットされます.\cmd{kyokuTyoku} の書式です.

\begin{boxnote}
\begin{verbatim}
\kyokuTyoku(#1,#2)#3
    (#1,#2) : 極座標
    #3      : 変換した直交座標を代入するコントロールシーケンス
\end{verbatim}
\end{boxnote}

\begin{enumerate}[(1)]
  \item このスタイルファイルでは,原則として,角の単位は六十分法で表します.
  \item \cmd{kyokuTyoku}は点を一つずつ定義しますが,
    複数の点列を定義するコマンド
\begin{jquote}
\begin{verbatim}
\rtenretu, \rtenretu*
\end{verbatim}
\end{jquote}
    もあります。
  \item 正多角形を描画するための\textsf{emPoly.sty}もあります。
\end{enumerate}

\subsection{角の丸い多角形}
多角形を描画するコマンド\cmd{Takakkei}の
バリエーション\cmd{ovalTakakkei}コマンドは
多角形のコーナーを丸くするコマンドです。
まずは,\cmd{Takakkei}の使用例からご覧ください。

\begin{showEx}(.54,.4){\cmd{Takakkei}}
  \begin{zahyou*}%
      [ul=10mm,Ueyohaku=1em,
        Hidariyohaku=1em,%
        Sitayohaku=1em]%
      (0,4)(0,4)
    \tenretu{A(1,0)s;B(4,1)e;%
      C(2,4)n;D(0,3)w}
    \Takakkei{\A\B\C\D}
  \end{zahyou*}
\end{showEx}

コーナーを切り取って,円弧で結びます。
切り取る線分の長さを\cmd{ovalTakakkei}の第1引数に与えます。
単位を伴った数値を指定します。

\begin{showEx}(.54,.4){\cmd{ovalTakakkei}}
  \begin{zahyou*}%
      [ul=10mm,Ueyohaku=1em,
        Hidariyohaku=1em,%
        Sitayohaku=1em]%
      (0,4)(0,4)
    \tenretu*{A(1,0)s;B(4,1)e;%
      C(2,4)n;D(0,3)w}
    \ovalTakakkei{5mm}{\A\B\C\D}
  \end{zahyou*}
\end{showEx}


\subsection{分点}
線分を内分(外分)する点を求めるコマンド \cmd{Bunten} です.
\sankaku{ABC} の辺BCの中点MとAを結ぶ線分を描画する例です.

\showexample[\cmd{Bunten}](0.55)(0.35){example/bunten01}

\cmd{Bunten} の書式は
\begin{boxnote}
\begin{verbatim}
\Bunten#1#2#3#4#5
    #1 : 端点1
    #2 : 端点2
    #3#4 : 内(外)分比
    #5 : 分点
\end{verbatim}
\end{boxnote}
すなわち \verb+#1+ と \verb+#2+ を結ぶ線分を \verb+#3:#4+ に
分ける点の座標を \verb+#5+ にセットします.

特に$1:1$の内分点,すなわち中点を求めるコマンド\cmd{Tyuuten}も便利です。

\showexample[\cmd{Bunten}](0.55)(0.35){example/bunten02}

\subsection{格子}
碁盤の目状の街路図などは良く登場します.
格子を描画するコマンド \cmd{kousi} です.\cindex{kousi}

\showexample[\cmd{kousi}](0.4)(0.4){example/kousi01}

縦横のサイズを変更したいときは,
\texttt{(横サイズ,縦サイズ)} オプションを指定します.
サイズの単位は \cmd{unitlength} です.

\showexample[\cmd{kousi(..,..)オプション}](0.45)(0.5){example/kousi02}

\cmd{kousi} の書式です.

\begin{boxnote}
\begin{verbatim}
\kousi(#1,#2)#3#4
    #1 : 横方向1区画の長さ(デフォルト=1)
    #2 : 縦方向1区画の長さ(デフォルト=1)
    #3 : 横方向のブロック数
    #4 : 縦方向のブロック数
置く位置は \put で指定する.
\end{verbatim}
\end{boxnote}

格子を点線・破線で引いてみます。

\showexample[格子線を破線で引く](0.6)(0.34){example/kousi03}

%外枠だけは実線でとなると,

%\showexample[外枠は実線で](0.6)(0.34){example/kousi04}

\subsection{さいころ}
さいころの目を表すコマンド \cmd{saikoro} を用意しました.
\cindex{saikoro}\index{さいころ}

\showexample[さいころ](0.45)(0.45){example/saikoro}

ただし,このコマンドは\textsf{ascmac.sty}で定義されている
\cmd{keytop}コマンドを使用していますから,このスタイルファイルを
使用することが前提です.
さらに,このスタイルファイルは \textsf{emathP} から読み込まれる
\textsf{eepic.sty}と相性が悪いことがあります.その対策を
\textsf{itembbox.sty}で施していますので,\textsf{ascmac.sty}に
引き続き \textsf{itembbox.sty}も読み込んでおく必要があります.


%\section{文字列}
\section{$BJ8;zNs(B}
\subsection{$BJ8;zNs(B}
\subsubsection{\texorpdfstring{\cmd{Put}}{Put}}
\LaTeX $B$N(B \textsf{picture} $B4D6-$K$*$$$F!$J8;zNs$rG[CV$9$k%3%^%s%I$O(B
\cmd{put}$B$G$9!#(B

\textsf{emath} $B$G$O!$E@$N0LCV$r(B \cmd{A} $B$J$I$NJQ?tL>$G;XDj$7$^$9!#(B
$B$=$3$G(B \cmd{Put} $B$NEP>l$H$J$j$^$9!#(B

\begin{showEx}{\cmd{Put}}
\begin{picture}(3,2)
\def\A{(0,0)}
\def\B{(3,2)}
\Drawline{\A\B}
\Put\A{A}
\Put\B{B}
\end{picture}
\end{showEx}

$B$&$^$/$J$$$G$9$M!#(B
$BC1$K(B \verb+\Put\A{A}+ $B$G$O!$J8;z(B`A'$B$N:82<$,;XDj$7$?E@(B\verb+A(0,0)+$B$K(B
$B$/$k$h$&$KG[CV$5$l$^$9!#$3$l$r=$@5$9$k$N$K(B
\begin{jquote}
\begin{verbatim}
(dx,dy)[pos]
\end{verbatim}
\end{jquote}
$B%*%W%7%g%s$rMQ0U$7$^$7$?!#<!@a0J9_$3$N2r@b$r$7$^$9$,!$$H$j$"$($:(B
\cmd{Put}$B$N=q<0$r$4Mw$/$@$5$$!#(B

\begin{boxnote}
\begin{verbatim}
\Put#1(#2,#3)[#4]#5
    #1 : $BJ8;zNs$rCV$/0LCV!J:BI8!K(B
    (#2,#3) : $B0LCV$N=$@5%Y%/%H%k!JD9$5$NC10L$,I,MW$G$9!#!K(B
    #4 : $BG[CV(B (l,r,t,b)
    #5 : $BJ8;zNs(B
\end{verbatim}
\end{boxnote}\cindex{Put}

\subsubsection{$BJ8;z0LCV$ND4@0(B}
$BJ8;zNs0LCV$NHyD4@0$r9T$&Nc$G$9!#(B

\showexample[$B0LCV$ND4@0A0(B](0.55)(0.4){example/string02}

\showexample[$B0LCV$ND4@08e(B](0.55)(0.4){example/string03}

A$B$ND4@0$O(B \texttt{(0,2pt)} $B$N;XDj$K$h$j!$D:E@(BA$B$N>e(B \texttt{2pt} $B$N(B
$B$H$3$m$,4p=`$K$J$j$^$9!#(B\texttt{[b]}$B;XDj$K$h$j!$4p=`E@$,J8;z$N(B
\texttt{bottom}$B$H$J$k$h$&$KG[CV$5$l$^$9!#(B
\texttt{(x,y)} $B$N(B \texttt{x, y} $B$OC10L$r$D$1$??tCM$G$9!#(B
0$B$@$1$OC10L$r$D$1$J$/$F$b$h$$!$$H$7$F$"$j$^$9!#(B

$BJ8;z0LCV$ND4@0NL(B\verb/(#2,#3)/$B$OD>8r:BI8@.J,$G$9$,!$(B
\verb/[r](#2,#3)/$B$H$9$l$P!$6K:BI8;XDj$H$_$J$5$l$^$9!#(B
A$B$N0LCV$rJQ$($F$_$^$9!#(B

\showexample[$BD4@0NL$r6K:BI8;XDj(B](0.55)(0.4){example/string07}

\subsubsection{$BJ8;zNs0LCV$N4J0W;XDj(B}
$B0LCV$ND4@0$r5-=R$9$k$N$OLLE]$J$N$G!$4J0W;XDj%*%W%7%g%s$rMQ0U$7$^$7$?!#(B
$B$=$N=q<0$G$9!#(B

\begin{boxnote}
\begin{verbatim}
\Put#1[#2]#3
    #1 : $BJ8;zNs$rCV$/0LCV!J:BI8!K(B
    #2 : $B0LCV;XDj%*%W%7%g%s(B
      n = $BKL(B  $B!J>e!K(B
      nw= $BKL@>!J:8>e!K(B
      w = $B@>(B  $B!J:8!K(B
      sw= $BFn@>!J:82<!K(B
      s = $BFn(B  $B!J2<!K(B
      se= $BFnEl!J1&2<!K(B
      e = $BEl(B  $B!J1&!K(B
      ne= $BKLEl!J1&>e!K(B
    #3 : $BJ8;zNs(B
\end{verbatim}
\end{boxnote}

$B$3$l$rMQ$$$k$H!$@h$NNc$O<!$N$h$&$K5-=R$G$-$^$9!#(B

\showexample[$B0LCV$N4J0W;XDj%*%W%7%g%s(B](0.55)(0.4){example/string04}

$B$9$J$o$A!$E@$+$i$_$FJ8;zNs$r$*$/$Y$-J}0L$r;XDj$7$^$9!#(B

$B!JCm!K(B\cmd{Put} $B$N@5$7$$%3%^%s%IL>$O(B \cmd{emathPut} $B$G$9!#(B
\cindex{emathPut}$B%3%^%s%IL>(B \cmd{Put}$B$,B>$N%9%?%$%k%U%!%$%k$H(B
$B6%9g$7$?>l9g$O!$(B\cmd{emathPut} $B$r$*;H$$$/$@$5$$!#(B

\subsubsection{$BJ8;zNs$N2sE>(B}
\verb/\Put[#1]/ $B$K$OA0@a$N$[$+<!$N%*%W%7%g%s$,$"$j$^$9!#(B
\begin{boxnote}
\begin{verbatim}
background=white $BGX7J$rGr$/EI$k(B
kaiten    =ddd   $BJ8;z$r2sE>$9$k(B
houkou    =vvv   $B2sE>3Q$r4V@\E*$K;XDj(B
from      =PPP   $B<!$N(B to=QQQ $B$HJ;$;$F2sE>3Q$r4V@\E*$K;XDj(B
to        =QQQ
\end{verbatim}
\end{boxnote}

$B<!$O(B \texttt{kaiten=ddd} $B$GJ8;z$r2sE>$5$;$^$9!#(B
$B2sE>3Q(B \texttt{ddd} $B$OId9fIU$-$NO;==J,K!$G$G$9!#(B

\begin{showEx}(.62,.32){\texttt{kaiten=}}
\unitlength6mm\footnotesize
\begin{zahyou}(-2.5,2.5)(-2.5,2.5)
\tenretu*{A(0,-1);B(1,0)}
\Put\A[w]{$-1$}
\Put\B[s]{1}
\Tyokusen\A\B{}{}
\Put[kaiten=45]\B(0,1mm)[lb]{%
$y=x-1$}
\end{zahyou}
\end{showEx}

$B2sE>3Q$rJ}8~%Y%/%H%k$K$h$C$FM?$($k%*%W%7%g%s(B
\texttt{houkou=vvv} $B$G$9!#(B

\begin{showEx}(.64,.3){\texttt{houkou=}}
\unitlength6mm\footnotesize
\begin{zahyou}(-2.5,2.5)(-2.5,2.5)
\tenretu*{A(0,2);B(1,0)}
\Put\A[w]{2}
\Put\B[s]{1}
\Tyokusen\A\B{}{}
\Subvec\B\A\AB
\Put[houkou=\AB]\A(1mm,0)[lb]{%
$y=2-2x$}
\end{zahyou}
\end{showEx}

2$BE@$rM?$($F2sE>3Q$r;X<($9$kNc$G$9!#(B
\begin{showEx}(.64,.3){\texttt{from= , to= }}
\unitlength6mm\footnotesize
\begin{zahyou}(-2.5,2.5)(-2.5,2.5)
\tenretu*{A(0,2);B(1,0)}
\Put\A[w]{2}
\Put\B[s]{1}
\Tyokusen\A\B{}{}
\Put[from=\A,to=\B]\A(1mm,0)[lb]{%
$y=2-2x$}
\end{zahyou}
\end{showEx}

\texttt{background=white}$B$HJ;MQ$7$F$_$^$9!#(B

\begin{showEx}(.64,.3){\texttt{background=white}$B$HJ;MQ(B}
\unitlength6mm\footnotesize
\begin{zahyou}(-2.5,2.5)(-2.5,2.5)
\tenretu*{O(0,0);A(0,2);B(1,0)}
\Put\A[w]{2}
\Put\B[s]{1}
\En*\O{2}
\Tyokusen\A\B{}{}
\Put[background=white,from=\A,to=\B]\A(%
1mm,0)[lb]{$y=2-2x$}
\end{zahyou}
\end{showEx}

$BJ8;z$N30B&$NM>Gr$O(B \verb/\fboxsep/ $B$K$h$jD4@0$G$-$^$9!#(B

\begin{showEx}(.64,.3){\cmd{fboxsep}$B$K$h$kM>GrD4@0(B}
\unitlength6mm\footnotesize
\begin{zahyou}(-2.5,2.5)(-2.5,2.5)
\tenretu*{O(0,0);A(0,2);B(1,0)}
\Put\A[w]{2}
\Put\B[s]{1}
\En*\O{2}
\Tyokusen\A\B{}{}
{\fboxsep=0pt
\Put[background=white,from=\A,to=\B]\A(%
1mm,0)[lb]{$y=2-2x$}}
\end{zahyou}
\end{showEx}

\subsubsection{$BJ8;zNs$+$i4p=`E@$X$NLp@~(B}
$BJ8;zNs$rCV$-$?$$IU6a$G?^$,9~$_F~$C$F$$$k>l9g!$(B
$BJ8;zNs$r>/$7N%$l$?0LCV$K$*$$$F!$$=$3$+$i3:Ev8D=j$K(B
$BLp@~$r0z$/$?$a$N%3%^%s%I$,(B \cmd{PutStr} $B$G$9!#(B

\showexample[\cmd{PutStr}](0.58)(0.36){example/string11}

$BLp@~$r1_8L$K$7$?$$$H$-$O(B \verb+to+ $B$N8e$m$K(B \verb+[$BH>7B(B]+ $B%*%W%7%g%s$r(B
$B$D$1$^$9!#H>7B$OL5L>?t$GC10L$O(B \verb+\unitlength+ $B$G$9!#(B

\showexample[\cmd{PutStr}](0.58)(0.36){example/string06}

$BH>7B$KIi$NCM$r;XDj$9$k$H!$Lp0u$,Ii$N2sE>$rI=$9J}8~$K$D$-$^$9!#(B

\showexample[\cmd{PutStr}](0.58)(0.36){example/string12}


\paragraph{\texttt{addvec=}$B%*%W%7%g%s(B}
$BLp0u$N@hC<$,6J@~$KKdKW$7$F$$$k$N$,5$$K$J$k>l9g$NBP:v$G$9!#(B
$BLp0u$N@hC<$N0LCV$rHyD4@0$9$k$?$a$K$N%*%W%7%g%s$,(B\texttt{[addvec=..]}$B$G$9!#(B
$B>e$NNc$G!$Lp0u$N@hC<$r>eJ}$K(B\texttt{1.1pt}$B;}$A>e$2$F$_$^$9!#(B

\begin{showEx}(.6,.34){\cmd{PutStr}}
  \begin{zahyou}[ul=7.5mm](-2,3)(-2,3)
    \PutStr{(1,2)}(0,0)[b]{A(1, 1)}%
      to[addvec={(0,1.1pt)}]{(1,1)}
    {\Thicklines
      \En\O{1.414}
      \kTyokusen\O{45}{}{}
    }%
  \end{zahyou}
\end{showEx}

$B$9$J$o$A!$(B\texttt{addvec}$B$N1&JUCM$KLp0u$N0LCV$r=$@5$9$k%Y%/%H%k$rM?$($^$9$,!$(B
$B$=$N@.J,$O!$C10L$rH<$C$??tCM$G$9!#(Bx$B@.J,!$(By$B@.J,$N4V$K(B`,'$B$,F~$j$^$9$+$i!$(B
\verb+addvec={(0,1.1pt)}+$B$H!$1&JUA4BN$r(B\verb+{ }+$B$G$/$/$C$F$*$/I,MW$,$"$j$^$9!#(B
%\clearpage

\paragraph{$B6K:BI87A<0$N;XDjK!(B}
$B$^$?!$@~I}$,:Y$/$F$bBP>]E@$K9u4]$r$D$1$?$j$7$?>l9g$bLp0u$,KdKW$7$^$9!#(B

\begin{showEx}(.6,.34){\cmd{PutStr}}
  \begin{zahyou}[ul=7.5mm](-2.5,3)(-2.5,3)
    \def\A{(1,1.732)}
    \PutStr{(-1,2)}(0,0)[r]%
      {A(1, $\sqrt3$)}%
      to\A
    \En\O{2}
    \kTyokusen\O{60}{}{}
    \Kuromaru\A
  \end{zahyou}
\end{showEx}

$B$3$N>l9g$O!$HyD4@0%Y%/%H%k$r6K:BI87A<0$GM?$($kJ}K!$bMQ0U$7$^$7$?!#(B

\begin{showEx}(.6,.34){\cmd{PutStr}}
  \begin{zahyou}[ul=7.5mm](-2.5,3)(-2.5,3)
    \def\A{(1,1.732)}
    \PutStr{(-1,2)}(0,0)[r]%
      {A(1, $\sqrt3$)}%
      to[addvec={r(1.2pt,180)}]\A
    \En\O{2}
    \kTyokusen\O{60}{}{}
    \Kuromaru\A
  \end{zahyou}
\end{showEx}

$B$9$J$o$A!$(B\verb+r(.,.)+$B$H!$@hF,$K(B`r'$B$rIm2C$7$^$9!#(B

\paragraph{$B1_8L$K$9$k>l9g$NH>7B;XDj(B}
$BLp@~$r1_8L$K$9$k$K$O!$(B\verb+[...]+$B%*%W%7%g%s$K!$(B
\verb+[hankei=..]+$B$rIm2C$7$^$9!#(B

\begin{showEx}(.6,.34){\cmd{PutStr}}
  \begin{zahyou}[ul=7.5mm](-2.5,3)(-2.5,3)
    \def\A{(1,1.732)}
    \PutStr{(-1,2.5)}(0,0)[r]%
      {A(1, $\sqrt3$)}%
      to[hankei=-2,addvec={r(1.2pt,120)}]\A
    \En\O{2}
    \kTyokusen\O{60}{}{}
    \Kuromaru\A
  \end{zahyou}
\end{showEx}

$B1&JUCM$O!$(B\cmd{unitlength}$B$rC10L$H$9$kL5L>?t$G$"$k$N$O!$(B
$B:#$^$G$N;EMM$r0z$-$:$C$F$$$^$9!#(B

\paragraph{\cmd{PutStr} $B$N(B \texttt{arrowheadsize=}$B%*%W%7%g%s(B}
\cmd{PutStr}$B$O!$%G%U%)%k%H$G$OLp0u$,$D$-$^$9!#(B

$BLp0u$N%5%$%:$rJQ99$9$k$?$a$K$N%*%W%7%g%s$,(B\verb+arrowheadsize=..+$B$G$9!#(B
$BFC$K1&JUCM$r(B0$B$H$9$l$P!$Lp0u$,$D$+$J$$$3$H$H$J$j$^$9!#(B

\begin{showEx}(.64,.3){\cmd{PutStr}}
\begin{zahyou}[ul=10mm](-1.2,2.5)(-1.2,1.5)
  \En\O{1}
  \PutStr{(1.2,.5)}[e]{(1, 0)}to
    [hankei=-1,arrowheadsize=0]%
    {(1,0)}
\end{zahyou}
\end{showEx}

\paragraph{\cmd{PutStr}$B$N=q<0(B}
\cmd{PutStr}$B$N=q<0$G$9!#(B\cindex{PutStr}

\begin{boxnote}
\begin{verbatim}
\PutStr#1[#2]#3to[#4]#5
   #1 : $BJ8;zNs$rCV$/0LCV(B
   #2 : #1 $B$+$i8+$F$NJ}0L!J%G%U%)%k%H(B = e $B!K(B
   #3 : $BJ8;zNs(B
   #4 : $BJ8;zNs$+$i=P$kLp0u$r1_8L$K$7$?$$$H$-$O(B
        $B$=$NH>7B$r;XDj$9$k!#(B
        key=val $B$N7A<0$b2DG=(B
          hankei=..
          addvec=..
          arrowheadsize=..
   #5 : $BJ8;zNs$+$i=P$kLp0u$N=*E@(B
\PutStr#1(#2,#3)[#4]#5to[#6]#7
   #1$B!A(B#5 : \Put $BJ8$HF1$8(B
   #6 : $BJ8;zNs$+$i=P$kLp0u$r1_8L$K$7$?$$$H$-$O(B
        $B$=$NH>7B$r;XDj$9$k!#(B
   #7 : $BJ8;zNs$+$i=P$kLp0u$N=*E@(B
\end{verbatim}
\end{boxnote}

\if0
\subsubsection{$BJ8;zNs$NGrH4$-(B}
\cmd{Put}$B$N0z?t$NKAF,$K(B\verb/[background=white]/$B%*%W%7%g%s$rIU2C$9$k$3$H$K$h$j(B
$BJ8;zNs$rGrH4$-$GG[CV$9$k$3$H$,$G$-$^$9!#(B

$BGX7J$rGr$/EI$k$K$O!$(B
\begin{verbatim}
   \colorbox{white}{....}
\end{verbatim}
$B$,$"$j$^$9$,!$J8;z$r2sE>$5$;$k$N$K(B
\begin{verbatim}
   \rotate{..}{\colorbox{white}{...}}
\end{verbatim}
$B$H$7$?$H$-!$3Q$NBg$-$5$K$h$C$F$OGX7J$K:/@W$,;D$C$F$7$^$&$3$H$,$"$j$^$9!#(B
$B$=$3$G!$(B\texttt{Tpic specials} $B$rMQ$$$FGX7J$rGrEI$j$9$k$3$H$K$7$^$7$?!#(B

\begin{showEx}(.64,.3){\texttt{background=white}}
\unitlength6mm\footnotesize
\drawaxisfalse
\begin{zahyou}(-2.5,2.5)(-2.5,2.5)
\def\O{(0,0)}
\En*\O{2}
\Put[background=white]\O(0,0){$S$}
\end{zahyou}
\end{showEx}
\fi



\subsection{$BJ#?t$NE@$NDj5A$H%i%Y%kIU$1(B}
\subsubsection{\texorpdfstring{\cmd{tenretu}}{tenretu}}
$BJ#?t$NE@$NDj5A$H%i%Y%kIU$1$rF1;~$KJRIU$1$h$&$H$$$&$N$,(B
\cmd{tenretu}$B%3%^%s%I$G$9!#(B\cindex{tenretu}

\showexample[\cmd{tenretu}](0.55)(0.4){example/string05}

\begin{boxnote}
\begin{verbatim}
 $BJ#?t$NE@$NDj5A$H%i%Y%k$D$1(B
 \tenretu#1
 \tenretu*#1$B!JDj5A$N$_(B  $B%i%Y%kIU$1$O%*%W%7%g%s!K(B
   #1 $B$OE@Ns$r(B `;' $B$G6h@Z$C$?Ns(B
     #1$B$O(B
       [##1]##2(##3,##4)##5
     $B$N7A<0$GE@Ns$r(B `;' $B$G6h@Z$k!#(B
       ##1 : $B%*%W%7%g%s$G!$E@$N0LCV$KCV$/J8;zNs(B
             $B!J>JN,;~$O!$(B##2 $B$HF1$8!K(B
       ##2 : \##2 $B$H$$$&JQ?t$ND:E@L>(B
       [r] : $B$r$D$1$k$H6K:BI87A<0$H$_$J$9(B
       [s] : $B$r$D$1$k$HAjBP0\F0$H$_$J$9(B
       (##3,##4) : $BD:E@$N:BI8(B
       ##5 : $BD:E@$N6aK5$KCV$/J8;zNs$NG[CV%*%W%7%g%s(B
             [$BJ}3Q(B] $B%*%W%7%g%s$K8B$j(B
             $B6h@Z;R(B `[', `]' $B$r>JN,2DG=(B
  \edef\##2{(##3,##4)}$B$H$7$F!$E@(B\##2 $B$,Dj5A$5$l!$(B
  \Put $B$K(B
       \Put\##2##5{##1}
  $B$H$7$F0z$-EO$5$l$k!#!K(B
\end{verbatim}
\end{boxnote}
\cindex{tenretu}

\subsubsection{\texorpdfstring{\cmd{tenretu*}}{tenretu*}}
\texttt{*}$B$rIm2C$7$?(B \cmd{tenretu*} $B$O!$E@$rDj5A$9$k$@$1$G!$(B
$B%i%Y%kIU$1$O$7$^$;$s!#(B

\showexample[\cmd{tenretu}](0.55)(0.4){example/string13}
\cindex{tenretu*}

\subsubsection{\texorpdfstring{\cmd{oresen}}{oresen}}
$B99$K$=$l$i$r@^$l@~$G7k$s$G$7$^$*$&!$(B
$B$H$$$&$N$,(B\cmd{oresen}$B$G$9!#(B\cindex{oresen}

\showexample[\cmd{oresen}](0.55)(0.4){example/string08}

\verb/<sensyu=...>/$B%*%W%7%g%s$rMxMQ$7$F!$@^$l@~$rE@@~!&GK@~!&:?@~$K(B
$B$9$k$3$H$b2DG=$G$9!#(B

\showexample[$B@~<o$NJQ99!J:?@~!K(B](0.55)(0.4){example/string09}

\showexample[$B@~<o$NJQ99!JGK@~!K(B](0.55)(0.4){example/string10}

\cmd{oresen}$B$N=q<0$G$9!#(B

\begin{boxnote}
\begin{verbatim}
\oresen<#1>#2
  <#1> : $B@~<o$rJQ99$9$k$H$-$N%*%W%7%g%s(B
     sensyu=\dashline[40]{.1}
     sensyu=\chainline[.4][.2]
     $B$J$I(B
  #2 $B$O@^$l@~$ND:E@Ns$r(B `;' $B$G6h@Z$C$?Ns(B
    #2$B$O(B
      [##1]##2(##3,##4)##5
    $B$N7A<0$GE@Ns$r(B `;' $B$G6h@Z$k!#(B
      ##1 : $B%*%W%7%g%s$G!$E@$N0LCV$KCV$/J8;zNs(B
            $B!J>JN,;~$O!$(B##2 $B$HF1$8!K(B
      ##2 : \##2 $B$H$$$&JQ?t$ND:E@L>(B
      [r] : $B$r$D$1$k$H6K:BI87A<0$H$_$J$9(B
      [s] : $B$r$D$1$k$HAjBP0\F0$H$_$J$9(B
      (##3,##4) : $BD:E@$N:BI8(B
      ##5 : $BD:E@$N6aK5$KCV$/J8;zNs$NG[CV%*%W%7%g%s(B
            [$BJ}3Q(B] $B%*%W%7%g%s$K8B$j(B
            $B6h@Z;R(B `[', `]' $B$r>JN,2DG=(B
 \edef\##2{(##3,##4)}$B$H$7$F!$E@(B\##2 $B$,Dj5A$5$l!$(B
 \Put $B$K(B
      \Put\##2##5{##1}
 $B$H$7$F0z$-EO$5$l$k!#!K(B
\end{verbatim}
\end{boxnote}

\subsubsection{\texorpdfstring{\cmd{rtenretu}}{rtenretu}}
\cmd{tenretu(*)}$B$K$h$kE@$NDj5A$OD>8r:BI8$rA0Ds$H$7$F$$$^$9!#(B
\verb+[r]+$B%*%W%7%g%s$G6K:BI87A<0$K$9$k$3$H$,$G$-$^$9$,!$(B
$BB?$/$NE@$r6K:BI87A<0$GDj5A$9$k$K$O!$HQ;($G$9!#(B

$B$=$3$G!$$9$Y$F$NE@$,6K:BI87A<0$G$"$k$H$-$K;HMQ$9$k$?$a$K(B
\cmd{rtenretu(*)}$B$r:n$j$^$7$?!#(B
\begin{showEx}(.6,.345){\cmd{rtenretu}}
\begin{zahyou}[ul=10mm](-1.2,1.5)(-1.2,1.5)
\small
\rtenretu{A(1,0)ne;B(1,60)ne;C(1,120)nw;
D(1,180)nw;E(1,240)sw;F(1,300)se}
\Drawline{\A\B\C\D\E\F\A}
\En\O{1}
\end{zahyou}
\end{showEx}
\cindex{rtenretu}

$BCf?4$,86E@0J30$N$H$-$O(B\verb+[...]+$B%*%W%7%g%s$G;XDj$7$^$9!#(B
\begin{showEx}(.6,.345){\cmd{rtenretu[$B6K(B]}}
\begin{zahyou}[ul=10mm](-0.2,2.5)(-1.2,1.5)
\small
\def\Tyuusin{(1,0)}
\rtenretu[\Tyuusin]{A(1,0)ne;B(1,60)ne;
C(1,120)nw;D(1,180)nw;E(1,240)sw;F(1,300)se}
\Drawline{\A\B\C\D\E\F\A}
\En\Tyuusin{1}
\end{zahyou}
\end{showEx}

\verb+*+$BIU$N(B\cmd{rtenretu*}$B$H(B\cmd{rtenretu}$B$N4X78$O!$(B
\cmd{tenretu*}$B$H(B\cmd{tenretu}$B$N4X78$HF1$8$G$9!#(B
\cindex{rtenretu*}

\subsubsection{$B:BI8$K7W;;<0$r5-=R(B}
\cmd{tenretu}$B$K(B\verb+<perl>+$B%*%W%7%g%s$rIU2C$9$k$H!$(B
$B:BI8$K7W;;<0$r5-=R$9$k$3$H$,$G$-$^$9!#(B
$B$?$@$7!$(Bperl$B$H$NO"7H5!G=$rA0Ds$H$7$^$9$+$i!$(B
\textsf{samplePp.tex}$B$r$4Mw$/$@$5$$!#(B

\subsection{$B@~J,$KJ8;zNs(B}
\subsubsection{$B@~J,$ND9$5I=5-(B}
$B@~J,$NN>C<$r1_8L$G7k$S!$$=$NCf1{It$KD9$5$J$I$r5-F~$9$k$?$a$N(B
$B%3%^%s%I(B \cmd{HenKo} $B$G$9!#(B
\cindex{HenKo}

\showexample[\cmd{HenKo}](0.55)(0.35){example/HenKo01}

\cmd{HenKo}$B$O4pK\E*$K$O!$(B
\begin{jquote}
    $B@~J,$NN>C<$HCf1{It$KCV$/J8;zNs(B
\end{jquote}
$B$N(B3$B8D$N0z?t$rM?$($^$9!#$J$*!$M?$($kN>C<$NE@$N=g=x$r5UE>$5$;$k$H!$(B
$B1_8L$HJ8;zNs$,@~J,$NH?BPB&$KI=<($5$l$^$9!#(B

\showexample[$BC<E@$N=g=x(B](0.55)(0.35){example/HenKo04}

$B=q<0$O(B

\begin{boxnote}
\begin{verbatim}
\HenKo[#1]<#2>#3#4#5

   #1: $B8L$rE@@~$K$9$k>l9g!$E@$N8D?t(B(*$B$r;XDj$7$?>l9g$O!$0lG$(B)
   #2: key=val
         henkoH=.. $BJU$H8L$N5wN%(B($BC10LIU?tCM(B) $B%G%U%)%k%HCM(B=1.6ex
         henkohi=..$B1&JUCM$OL5L>?t$G!$(B
                   $BJU$H8L$N5wN%$r%G%U%)%k%H$N2?G\$K$9$k$+$r;XDj(B
         putoption=.. 
         henkosep=.. ($BGrH4$-%\%C%/%9$N(B \fboxsep)
         yazirusi=a/r/b
         henkomozikaiten=1/-1
         linewidth=..
         dash=..
         henkotype=0/1 (0:$B1_8L!J%G%U%)%k%H!K(B, 1:$BBJ1_(B, 2:$B@^$l@~(B)
   #3,#4 : $BN>C<$NE@(B
   #5 : $BG[CV$9$kJ8;zNs(B
\end{verbatim}
\end{boxnote}
$B0z?t$,B?$$$G$9$,!$I,?\$N$b$N$O!$(B
$B>e$NNc$N$h$&$K(B \verb+#3, #4, #5+ $B$@$1$G$9!#(B

$B@~J,$H8L$N4V3V$rD4@0$9$k$K$O!$(B\verb+henkoH=..+$B%*%W%7%g%s$rMQ$$$^$9!#(B
$B%G%U%)%k%HCM$N(B\verb+1.6ex+$B$+$iBg$-$/$7$F$_$^$7$g$&!#(B

\showexample[$BJ8;z$H@~J,$H$N4V3V(B](0.55)(0.35){example/HenKo02}

$B8L$NItJ,$rE@@~$K$9$k$K$O(B \verb+[#1]+ $B$G!$8L>e$KCV$/E@$N8D?t$r;XDj$7$^$9!#(B

\showexample[$B8L$rE@@~$G(B](0.55)(0.35){example/HenKo03}

$B6KC<$JOC$7!$$3$l$r(B \verb+[0]+ $B$H;XDj$9$l$P!$1_8L$OIA$+$l$^$;$s!#(B

\showexample[$BJ8;z$N$_(B](0.55)(0.35){example/HenKo05}

$B$J$*!$J8;zNs$O@~J,$NCf1{$KCV$+$l$^$9$,!$$3$l$r$I$A$i$+$K4s$;$k%3%^%s%I$,(B
\cmd{sPut} $B$G$9!#(B

\showexample[\cmd{sPut}](0.55)(0.35){example/sPut01}

$B=q<0$O(B\cindex{sPut}
\begin{boxnote}
\begin{verbatim}
\sPut[#1]#2#3(#4,#5)[#6]#7
    #1 : $BHfN((B (0~1)
    #2 : $B;OE@(B
    #3 : $B=*E@(B
   (#4,#5) : $B0LCV$N=$@5%Y%/%H%k(B $BC10LI,?\(B
    #6 : \makebox $B$N(B [..] $B%*%W%7%g%s(B
    #7 : $BJ8;zNs(B
#2 $B$+$i(B #3 $B$X8~$+$&@~J,$N(B #1 $BG\$N0LCV(B(\X)$B$K!$(B
    \Put\X(#4,#5)[#6]#7 $B$H$7$FJ8;zNs(B(#7)$B$rCV$/!#(B
\end{verbatim}
\end{boxnote}

$B!JCm!K(Bdviout.exe $B$K$h$k0u:~$G!$JU$NK5$K5-$7$?D9$5$NJ8;z$J$I$,(B
$B9u$$%\%C%/%9$K$J$C$F$7$^$&>l9g$O!$(Bdviout $B$N(B 
        graphic --- color specials
$B@_Dj$r(B
        replace(def) $B$^$?$O(B replace(bak)
$B$K$7$F$_$F$/$@$5$$!#(B

\subsubsection{\texorpdfstring{\cmd{HenKo}}{HenKo} : $BJ8;zNs$NG[CVD4@0(B}
\cmd{HenKo}$B$G@~J,$NK5$KCV$/J8;zNs$,D9$/$J$C$?$H$-$J$I(B
$BJ8;z0LCV$rD4@0$7$?$$$H$-$,$"$j$^$9!#(B
$B$=$N0lNc$G$9!#(B

\begin{showEx}(.65,.29){$B=$@5A0(B}
\unitlength10mm\footnotesize
\drawaxisfalse
\begin{zahyou}(-.5,2.5)(-.5,2.5)
\Thicklines
\oresen{A(0,2)n;B(2,0)s}
\thinlines
\HenKo\A\B{$2+\sqrt3$}
\end{zahyou}
\end{showEx}

$BJ8;zNs(B$2+\sqrt{3}$$B$,@~J,$K$+$+$C$F$$$^$9$N$G!$:8$KF0$+$7$^$9!#(B
$BD4@0$KF~$kA0$K!$%G%U%)%k%H$NJ8;zNsG[CV$K$D$$$F=R$Y$^$9!#(B

$B4p=`E@$O1_8L$NCfE@$G!$J8;zNs%\%C%/%9$NCf?4$,$3$N4p=`E@$K$/$k$h$&$K(B
$BG[CV$5$l$^$9!#(B
$B$G$O!$4p=`E@$r:8$K(B4mm$BF0$+$7$F$_$^$7$g$&!#(B
\begin{showEx}(.65,.29){$BJ?9T0\F0(B}
\unitlength10mm\footnotesize
\drawaxisfalse
\begin{zahyou}(-.5,2.5)(-.5,2.5)
\Thicklines
\oresen{A(0,2)n;B(2,0)s}
\thinlines
\HenKo<putoption={(-4mm,0)}>\A\B%}>%
{$2+\sqrt3$}
\end{zahyou}
\end{showEx}
\verb/<putoption={(-4mm,0)}>/$B$NItJ,$,J?9T0\F0$r$9$k$?$a$N%*%W%7%g%s$G!$(B
\cmd{Put}$B$K%*%W%7%g%s$H$7$F0z$-EO$5$l$^$9!#(B
$B$9$J$o$A!$1_8L$NCfE@$r(B\verb/\Q/$B$H$7$F(B
\begin{verbatim}
    \Put\Q(-4mm,0){$2+\sqrt3$}
\end{verbatim}
$B$,<B9T$5$l$^$9!#(B

$BJ8;zNs$,D9$$$H$-$O!$J8;zNs$r2sE>$7$F@~J,$HJ?9T$K$9$kJ}$,$h$$$G$7$g$&$+!#(B
\begin{verbatim}
  <henkomozikaiten=1> $B%*%W%7%g%s(B
\end{verbatim}
$B$G$9!#(B

\begin{showEx}(.65,.29){$B2sE>(B}
\unitlength10mm\footnotesize
\drawaxisfalse
\begin{zahyou}(-.5,2.5)(-.5,2.5)
\tenretu{A(0,2)n;B(2,0)s}
\HenKo<henkomozikaiten=1>\A\B%
{$2+\sqrt3$}
{\Thicklines\Drawline{\A\B}}
\end{zahyou}
\end{showEx}

$B%*%W%7%g%s$N1&JUCM$r(B$-1$$B$H$9$l$P!$J8;z$O(B180$BEY2sE>$7$^$9!#(B

\begin{showEx}(.65,.29){$B5U2sE>(B}
\unitlength10mm\footnotesize
\drawaxisfalse
\begin{zahyou}(-.5,2.5)(-.5,2.5)
\tenretu{A(0,2)n;B(2,0)s}
\HenKo<henkomozikaiten=-1>\A\B%
{$2+\sqrt3$}
{\Thicklines\Drawline{\A\B}}
\end{zahyou}
\end{showEx}

\subsubsection{$B8L$KLp0u(B}
$B8L$NC<Kv$KLp0u$r$D$1$k%*%W%7%g%s$,(B\verb+<yazirusi=a>+$B%*%W%7%g%s$G$9!#(B

\begin{showEx}{\texttt{<yazirusi=a>}$B%*%W%7%g%s(B}
\begin{zahyou*}(0,3)(0,3)
\def\A{(1,1)}%
\def\B{(2,2)}%
\Put\A{\makebox(0,0)[r]{A }}%
\Put\B{\makebox(0,0)[l]{ B}}%
\HenKo<yazirusi=a>\A\B{$r$}%
\Drawline{\A\B}
\end{zahyou*}
\end{showEx}

$B5U8~$-$K$D$1$k$K$O(B

\begin{showEx}{\texttt{<yazirusi=r>}$B%*%W%7%g%s(B}
\begin{zahyou*}(0,3)(0,3)
\def\A{(1,1)}%
\def\B{(2,2)}%
\Put\A{\makebox(0,0)[r]{A }}%
\Put\B{\makebox(0,0)[l]{ B}}%
\HenKo<yazirusi=r>\A\B{$r$}%
\Drawline{\A\B}
\end{zahyou*}
\end{showEx}

$BN>8~$-$K$D$1$k$K$O(B

\begin{showEx}{\texttt{<yazirusi=b>}$B%*%W%7%g%s(B}
\begin{zahyou*}(0,3)(0,3)
\def\A{(1,1)}%
\def\B{(2,2)}%
\Put\A{\makebox(0,0)[r]{A }}%
\Put\B{\makebox(0,0)[l]{ B}}%
\HenKo<yazirusi=b>\A\B{$r$}%
\Drawline{\A\B}
\end{zahyou*}
\end{showEx}

\subsubsection{\texorpdfstring{\cmd{HenKo}}{HenKo}$B$NCfE@!$Cf?4(B}
\cmd{HenKo}$B%3%^%s%IH/9T8e!$1_8L$NCfE@!$1_8L$NCf?40LCV$rCN$j$?$$$3$H$,$"$j$^$9!#(B

\paragraph{\cmd{HenKoTyuuten}}
\cmd{HenKo}$B$GIA2h$5$l$k1_8L$NCfE@$O(B\cmd{HenKoTyuuten}$B$KJ]B8$5$l$F$$$^$9!#(B

\begin{showEx}(.6,.34){\cmd{HenKoTyuuten}}
  \begin{zahyou*}[ul=8mm](-.5,3)(-.5,2.5)
    \tenretu{A(0,0)nw;B(3,2)ne}
    \Drawline{\A\B}
    \HenKo\A\B{}
    \Kuromaru\HenKoTyuuten
  \end{zahyou*}
\end{showEx}

\paragraph{\cmd{HenKoTyuusin}}
\cmd{HenKo}$B$GIA2h$5$l$k1_8L$NCf?4$r(B\cmd{HenKoTyuusin}$B$KJ]B8$7$^$9!#(B

\begin{showEx}(.6,.34){\cmd{HenKoTyuuten}}
  \begin{zahyou*}[ul=8mm](-1,4)(-1,4)
    \tenretu{A(0,0)sw;B(2,0)se}
    \Drawline{\A\B}
    \HenKo\A\B{}
    \Kuromaru\HenKoTyuusin
    \Hasen{\A\HenKoTyuusin\B}
    \Enko<hasen={[.5][.8]}>\HenKoTyuusin%
      {tuukaten=\A}%
      {hazimeten=\B}{owariten=\A}
  \end{zahyou*}
\end{showEx}

%\clearpage

\paragraph{$B1~MQNc(B}
\cmd{HenKoTyuuten}, \cmd{HenKoTyuusin}$B$rMxMQ$7$?Nc$G$9!#(B

\begin{showEx}(.7,.24){$B1~MQNc(B}
  \begin{zahyou*}[ul=8mm](-.5,2.5)(-.5,4.5)
    \tenretu{A(2,4)n;B(1,2)se;C(0,0)s}
    \Drawline{\A\C}
    \HenKo\A\B{}
    \Bunten\HenKoTyuusin\HenKoTyuuten{.95}{.05}\P
    \Bunten\HenKoTyuusin\HenKoTyuuten{1.05}{-.05}\Q
      \Drawline{\P\Q}
    \HenKo\B\C{}
    \Bunten\HenKoTyuusin\HenKoTyuuten{.95}{.05}\P
    \Bunten\HenKoTyuusin\HenKoTyuuten{1.05}{-.05}\Q
      \Drawline{\P\Q}
  \end{zahyou*}
\end{showEx}

\subsubsection{\cmd{HenKo}$B$N7A>u$$$m$$$m(B}
\cmd{HenKo}$B$N(B\verb+<henkotype=..>+$B%*%W%7%g%s$K$h$j!$(B
$B1_8LItJ,$N%P%j%(!<%7%g%s$K$D$$$F@bL@$7$^$9(B.

$B$3$N@a$G$O!$(B\verb+\HenKo+$B$NJ}$,>GE@$G$9$+$i!$(B
$BJU$N$[$&$OGK@~IA2h$H$7$F$*$-$^$9!#(B

\paragraph{$B?'$D$1(B}
$B1_8L$NItJ,$K?'$r$D$1$k$K$O(B \verb+<henkocolor=...>+ $B%*%W%7%g%s$rMQ$$$^$9!#(B

\begin{showEx}(.7,.24){\texttt{<henkocolor=..>}$B%*%W%7%g%s(B}
\begin{zahyou*}[ul=6mm](0,5)(0,4)
  \tenretu{A(0,1)w;B(4,2)e}
  \Hasen{\A\B}
  \HenKo<henkocolor=red>\A\B{$a$}
\end{zahyou*}
\end{showEx}

\paragraph{\texttt{<henkotype=..>}$B%*%W%7%g%s(B}
$B%G%U%)%k%H$G$O!$JU$NN>C<$r1_8L$G7k$S$^$9$,!$(B
$B$3$N7A>u$rJQ$($k%*%W%7%g%s$,(B \verb+<henkotype=..>+$B%*%W%7%g%s$G$9!#(B

\subparagraph{\texttt{<henkotype=ellipse>}}
$B1_8L$G$O$J$/!$BJ1_8L$K$9$k%*%W%7%g%s$,(B
\begin{jquote}
\begin{verbatim}
<henkotype=1> $B$^$?$O(B <henkotype=ellipse>
\end{verbatim}
\end{jquote}
$B$G$9!#(B

\begin{showEx}(.7,.24){\texttt{<henkotype=ellipse>}$B%*%W%7%g%s(B}
\begin{zahyou*}[ul=6mm](0,5)(0,4)
  \tenretu{A(0,1)w;B(4,2)e}
  \Hasen{\A\B}
  \HenKo<henkotype=ellipse>\A\B{$a$}
\end{zahyou*}
\end{showEx}

\subparagraph{\texttt{<henkotype=triangle>}}
$B1_8L$G$O$J$/!$JU$NN>C<E@!&J8;zNs$rG[CV$9$kE@$r7k$V;03Q7A$rIA2h$9$k(B
$B%*%W%7%g%s$,(B
\begin{jquote}
\begin{verbatim}
<henkotype=2> $B$^$?$O(B <henkotype=triangle>
\end{verbatim}
\end{jquote}
$B$G$9!#(B

\begin{showEx}(.7,.24){\texttt{<henkotype=triangle>}$B%*%W%7%g%s(B}
\begin{zahyou*}[ul=6mm](0,5)(0,4)
  \tenretu{A(0,1)w;B(4,2)e}
  \Hasen{\A\B}
  \HenKo<henkotype=triangle>\A\B{$a$}
\end{zahyou*}
\end{showEx}

\subparagraph{\texttt{<henkotype=parallel>}}
$B1_8L$G$O$J$/!$JU$HJ?9T$J@~J,$rIA2h$9$k%*%W%7%g%s$,(B
\begin{jquote}
\begin{verbatim}
<henkotype=3> $B$^$?$O(B <henkotype=parallel>
\end{verbatim}
\end{jquote}
$B$G$9!#(B

\begin{showEx}(.7,.24){\texttt{<henkotype=parallel>}$B%*%W%7%g%s(B}
\begin{zahyou*}[ul=6mm](0,5)(0,4)
  \tenretu{A(0,1)w;B(4,2)e}
  \Hasen{\A\B}
  \HenKo<henkotype=parallel>\A\B{$a$}
\end{zahyou*}
\end{showEx}

\paragraph{\texttt{yazirusi=a/r/b}$B%*%W%7%g%s(B}

$B$3$N7A>u$O!$@~J,(BAB$B$ND9$5$rI=$9$H$-$J$I$KMxMQ$G$-$=$&$G$9!#(B
$B$=$7$F!$(B\cmd{HenKo}$B$GIA2h$5$l$kJ}$N@~J,$K$OLp0u$r$D$1$k$3$H$,B?$$$h$&$G$9!#(B

\begin{showEx}(.7,.24){\texttt{<yazirusi=b>}$B%*%W%7%g%s(B}
\begin{zahyou*}[ul=6mm](0,5)(0,4)
  \tenretu{A(0,1)w;B(4,2)e}
  \Hasen{\A\B}
  \HenKo<henkotype=parallel,yazirusi=b>\A\B{$a$}
\end{zahyou*}
\end{showEx}

\verb+\HenKo<yazirusi=..>\A\B{...}+$B$K$*$1$k(B\verb+yazirusi=..+$B$N1&JUCM$O(B
\begin{jquote}
\begin{verbatim}
a: $BE@(B\A$B$+$iE@(B\B$B$K8~$+$&8~$-(B
r: $BE@(B\B$B$+$iE@(B\A$B$K8~$+$&8~$-(B
b: $BN>8~$-(B
\end{verbatim}
\end{jquote}
$B$N$$$:$l$+$G$9!#(B

\paragraph{\texttt{<henkosideb=..,henkosidet=..>} $B%*%W%7%g%s(B}
$B$3$N>l9g!$$5$i$KJd=u@~(B --- $B@~J,$NN>C<$H(B\cmd{HenKo}$B$K$h$C$F(B
$B0z$+$l$kJB9T@~J,$NN>C<$r7k$V@~J,!J<c431dD9$7$F(B)$B$,M_$7$/$J$k$+$b$7$l$^$;$s!#(B

\begin{showEx}(.7,.24){\texttt{<henkosideb=..,henkosidet=..>}$B%*%W%7%g%s(B}
\begin{zahyou*}[ul=6mm](0,5)(0,4)
  \tenretu{A(0,1)w;B(4,2)e}
  \Hasen{\A\B}
  \HenKo<henkotype=parallel,yazirusi=b,%
    henkosideb=0,henkosidet=1.5>\A\B{$a$}
\end{zahyou*}
\end{showEx}

$B$3$N>l9g!$Jd=u@~$O!$JU$NC<E@$H(B\cmd{HenKo}$B$K$h$C$F0z$+$l$k@~J,!JLp@~!K$N(B
$BC<E@$r7k$V@~J,$r(B
\begin{jquote}
\begin{verbatim}
  \henkosideb : 1-\henkosideb
$B$H(B
  \henkosidet : 1-\henkosidet
$B$KJ,$1$k(B2$BE@$r7k$V@~J,(B
\end{verbatim}
\end{jquote}
$B$H$J$j$^$9!#(B

$B$J$*!$(B\verb+<henkocolor=..>+$B%*%W%7%g%s$K$h$k?'$E$1$O(B\verb+\HenKo+$B$K$h$k(B
$B@~J,!JN>8~$-Lp@~!K$KBP$7$F$N$_F/$-$^$9!#(B

\begin{showEx}(.7,.24){\texttt{<henkocolor=..>}$B%*%W%7%g%s(B}
\begin{zahyou*}[ul=6mm](0,5)(0,4)
  \tenretu{A(0,1)w;B(4,2)e}
  \Hasen{\A\B}
  \HenKo<henkotype=parallel,yazirusi=b,%
    henkocolor=red,%
    henkosideb=0,henkosidet=1.5>\A\B{$a$}
\end{zahyou*}
\end{showEx}

$BJd=u@~$KBP$9$k?'$E$1$O(B\verb+<henkosidecolor=..>+$B%*%W%7%g%s$G9T$$$^$9!#(B

\begin{showEx}(.7,.24){\texttt{<henkosidecolor=..>}$B%*%W%7%g%s(B}
\begin{zahyou*}[ul=6mm](0,5)(0,4)
  \tenretu{A(0,1)w;B(4,2)e}
  \Hasen{\A\B}
  \HenKo<henkotype=parallel,yazirusi=b,%
    henkocolor=red,henkosidecolor=green,%
    henkosideb=0,henkosidet=1.5>\A\B{$a$}
\end{zahyou*}
\end{showEx}

$B$^$5$+$M$'!A(B\verb+(^^$B!6(B+

\subparagraph{\texttt{<henkotype=bracket>}}
$B1_8L$G$O$J$/!$Bg3g8L$K$9$k%*%W%7%g%s$,(B
\begin{jquote}
\begin{verbatim}
<henkotype=1> $B$^$?$O(B <henkotype=bracket>
\end{verbatim}
\end{jquote}
$B$G$9!#(B

\begin{showEx}(.7,.24){\texttt{<henkotype=bracket>}$B%*%W%7%g%s(B}
\begin{zahyou*}[ul=6mm](0,5)(0,4)
  \tenretu{A(0,1)w;B(4,2)e}
  \Hasen{\A\B}
  \HenKo<henkotype=bracket>\A\B{$a$}
\end{zahyou*}
\end{showEx}

\paragraph{$B3Q$r4]$/(B}
$B%3!<%J!<$r4]$/$9$k%*%W%7%g%s$,(B\verb+<Oval=..>+$B%*%W%7%g%s$G$9!#(B
$B1&JUCM$O!$C10L$rH<$C$?D9$5$G%3!<%J!<$N;MJ,1_$NH>7B$r;XDj$7$^$9!#(B

\begin{showEx}(.7,.24){\texttt{<Oval=...>}$B%*%W%7%g%s(B}
\begin{zahyou*}[ul=6mm](0,5)(0,4)
  \tenretu{A(0,1)w;B(4,2)e}
  \Hasen{\A\B}
  \HenKo<henkotype=bracket,Oval=4pt,henkocolor=cyan>
    \A\B{$a$}
\end{zahyou*}
\end{showEx}

$BCm!'(B\verb+<henkotype=brace>+$B$O$"$j$^$;$s!#(B
$B8e=R$N(B\cmd{rotUbrace}$B%3%^%s%I$r$4Mw$/$@$5$$(B.

\paragraph{\texttt{<agezoko=..>}$B%*%W%7%g%s(B}
\cmd{HenKo}$B$NC<E@$rJU$+$iIb$+$;$?$$$H$-$,$"$j$^$9!#(B
$B$=$N$?$a$N%*%W%7%g%s$G$9!#(B
$B1&JUCM$OL5L>?t$GC10L$O(B\cmd{unitlength}$B$G$9!#(B
$BC10L$r$D$1$??tCM$G;XDj$7$?$$$H$-$O(B\verb+<Agezoko=..>+$B%*%W%7%g%s$rMQ$$$^$9!#(B

\begin{showEx}(.7,.24){\texttt{<agezoko=>}$B%*%W%7%g%s(B}
\begin{zahyou*}[ul=6mm](0,5)(0,4)
  \tenretu{A(0,1)w;B(4,2)e}
  \Hasen{\A\B}
  \HenKo<henkotype=bracket,Agezoko=5pt>\A\B{$a$}
\end{zahyou*}
\end{showEx}

\paragraph{\texttt{<agezokovi=..,agezokovii=..>}$B%*%W%7%g%s(B}
\cmd{HenKo}$B$NC<E@$r%*%W%7%g%s$N1&JUCM$@$1$:$i$9%*%W%7%g%s$G$9!#(B
$B1&JUCM$O%Y%/%H%k$G!$@.J,$OL5L>?t(B---$BC10L$O(B\cmd{unitlength}$B$G$9!#(B
$BC10L$r$D$1$??tCM$G;XDj$7$?$$$H$-$O(B\verb+<Agezokovi=..,\Agezokovii=..>+$B%*%W%7%g%s$rMQ$$$^$9!#(B
$BKvHx(B\verb+`i'+$B$,;OE@$K!$(B\verb+`ii'+$B$,=*E@$KBP$9$kJd@5%Y%/%H%k$G$9!#(B

\begin{showEx}(.7,.24){\texttt{<agezokovi=>}$B%*%W%7%g%s(B}
\begin{zahyou*}[ul=6mm](0,5)(0,2)
  \tenretu{A(0,1)nw;B(4,1)ne}
  \Hasen{\A\B}
  \HenKo<henkotype=bracket>\A\B{}
  \HenKo<henkotype=bracket,henkocolor=red,%
    Agezokovi={(-2pt,0)},Agezokovii={(2pt,0)},%
    henkoH=3pt>\A\B{}
\end{zahyou*}
\end{showEx}


\subsubsection{$BJU$K(B\texorpdfstring{\cmd{brace}}{brace}}
\cmd{underbrace}, \cmd{overbrace}$B$r(B
$B79$$$?@~J,$KBP$7$F;HMQ$9$k%3%^%s%I$,(B
\begin{jquote}
  \cmd{rotUbrace}, \cmd{rotObrace}
\end{jquote}
$B$G$9!#(B

$B$^$:$O!$(B\cmd{rotUbrace}$B$N=q<0$+$i(B
\begin{boxnote}
\begin{verbatim}
\rotUbrace[#1]#2#3#4
     $BE@(B#2$B$+$iE@(B#3$B$X8~$+$&M-8~@~J,$N!$(B
       $B?J9TJ}8~1&B&$K(B \underbrace $B$r$D$1!$(B
       $BCf1{2<It$KJ8;zNs(B#4$B$rG[CV$9$k!#(B
       $B!J(B#4 $B$O?t<0%b!<%IFb$H2r<a$5$l$k!#(B--- \scriptstyle $B!K(B
     \underbrace $B5-9f$HM-8~@~J,$N4V3V$r6u$1$?$$$H$-$O(B#1$B$K(B
        depth=3pt
     $B$J$I$H!$4V3V$rC10LIU$-?tCM$G;XDj$9$k!#(B
\end{verbatim}
\end{boxnote}
\cindex{rotUbrace}

$B$G$O!$4JC1$J;HMQNc$G$9!#(B
\begin{showEx}(.55,.39){\cmd{rotUbrace}}
\footnotesize
\begin{zahyou}[ul=4mm](-1,11)(-1,6)%
\tenretu*{A(0,5);B(10,0)}%
\Put\A[w]{5}\Put\B[s]{10}%
\kuromaru[2pt]{\A;(2,4);(4,3);
  (6,2);(8,1);\B}%
\Tyokusen\A\B{}{}%
\rotUbrace\A\B{\textstyle 6$B8D(B}%
\end{zahyou}
\end{showEx}
$B$9$J$o$AE@(BA(0, 5)$B$+$iE@(BB(10, 0)$B$X8~$+$&@~J,$N2<$K(B
\cmd{underbrace}$B$r$D$1!$$=$NCf1{It$K3J;RE@$N8D?t$r<($7$F$$$^$9!#(B
$B$3$NJ8;zNs$O%G%U%)%k%H$G$O(B\cmd{scriptstyle}$B$G>.$5$9$.$^$9$N$G!$(B
\cmd{textstyle}$B$rIU2C$7$F$$$^$9!#(B

\cmd{underbrace}$B$r@~J,$+$i>/$7N%$7$?$$!$$H$$$&>lLL$b$"$k$G$7$g$&!#(B
$B$=$N$?$a$N%*%W%7%g%s$,(B
\begin{jquote}
  \verb+[depth=...]+
\end{jquote}
$B$G$9!#1&JUCM$OC10L$rH<$C$??tCM$G$9!#(B

\begin{showEx}(.55,.39){\texttt{[depth=...}$B%*%W%7%g%s(B}
\footnotesize
\begin{zahyou}[ul=4mm](-1,11)(-1,6)%
\tenretu*{A(0,5);B(10,0)}%
\Put\A[w]{5}\Put\B[s]{10}%
\kuromaru[2pt]{\A;(2,4);(4,3);
(6,2);(8,1);\B}%
\Tyokusen\A\B{}{}%
\rotUbrace[depth=3pt]\A\B%
  {\textstyle 6$B8D(B}%
\end{zahyou}
\end{showEx}

$B4p=`$H$J$k@~J,$OM-8~@~J,$H$7$F07$o$l$^$9!#(B
$B>e$NNc$G!$8~$-$rF~$lJQ$($k$H(B

\begin{showEx}(.55,.39){$BM-8~@~J,$N8~$-$r5U(B}
\footnotesize
\begin{zahyou}[ul=4mm](-1,11)(-1,6)%
\tenretu*{A(0,5);B(10,0)}%
\Put\A[w]{5}\Put\B[s]{10}%
\kuromaru[2pt]{\A;(2,4);(4,3);
(6,2);(8,1);\B}%
\Tyokusen\A\B{}{}%
\rotUbrace\B\A{\textstyle 6$B8D(B}%
\end{zahyou}
\end{showEx}

\cmd{brace}$B5-9f$,@~J,(BAB$B$K4X$7$FBP>N$J0LCV$KIU$-$^$9!#(B
$BJ8;z$N8~$-$,$*$+$7$$$G$9$+!#(B
$B$3$l$O(B\cmd{underbrace}$B$r2sE>$5$;$F$$$k$+$i$G!$(B
$BJ8;z$N8~$-$r5U$K$9$k$K$O!$(B\cmd{overbrace}$B$rMQ$$$l$P$h$$$G$7$g$&!#(B
$B$H$$$&$3$H$G!$$D$.$O(B\cmd{rotObrace}$B$NOC$7$K0\$j$^$9!#(B

\begin{showEx}(.55,.39){\cmd{rotObrace}}
\footnotesize
\begin{zahyou}[ul=4mm](-1,11)(-1,6)%
\tenretu*{A(0,5);B(10,0)}%
\Put\A[w]{5}\Put\B[s]{10}%
\kuromaru[2pt]{\A;(2,4);(4,3);
(6,2);(8,1);\B}%
\Tyokusen\A\B{}{}%
\rotObrace\A\B{\textstyle 6$B8D(B}%
\end{zahyou}
\end{showEx}

$B@~J,$H(B\cmd{overbrace}$B5-9f$rN%$9$?$a$N%*%W%7%g%s$O(B
\verb![height=...]!$B$G$9!#(B

\begin{showEx}(.55,.39){\texttt{[height=...]$B%*%W%7%g%s(B}}
\footnotesize
\begin{zahyou}[ul=4mm](-1,11)(-1,6)%
\tenretu*{A(0,5);B(10,0)}%
\Put\A[w]{5}\Put\B[s]{10}%
\kuromaru[2pt]{\A;(2,4);(4,3);
(6,2);(8,1);\B}%
\Tyokusen\A\B{}{}%
\rotObrace[height=3pt]\A\B{\textstyle 6$B8D(B}%
\end{zahyou}
\end{showEx}

\cmd{rotObrace}$B$N=q<0$G$9!#(B
\begin{boxnote}
\begin{verbatim}
\rotObrace[#1]#2#3#4
     $BE@(B#2$B$+$iE@(B#3$B$X8~$+$&M-8~@~J,$N!$(B
       $B?J9TJ}8~:8B&$K(B \overbrace $B$r$D$1!$(B
       $BCf1{>eIt$KJ8;zNs(B#4$B$rG[CV$9$k!#(B
       $B!J(B#4 $B$O?t<0%b!<%IFb$H2r<a$5$l$k!#(B--- \scriptstyle $B!K(B
     \overbrace $B5-9f$HM-8~@~J,$N4V3V$r6u$1$?$$$H$-$O(B#1$B$K(B
        height=3pt
     $B$J$I$H!$4V3V$rC10LIU$-?tCM$G;XDj$9$k!#(B
\end{verbatim}
\end{boxnote}
\cindex{rotObrace}

\subsubsection{$BEyJU5-9f(B}
2$B$D$N@~J,$ND9$5$,Ey$7$$$H$-$K!$=DK@$r0z$$$?$j$7$FI=8=$9$k$?$a$N(B
$B%3%^%s%I(B \cmd{Touhenkigou} $B$G$9!#(B\cindex{Touhenkigou}

\showexample[\cmd{Touhenkigou}](0.55)(0.4){example/Touhen01}

$B=q<0$G$9!#(B
\begin{boxnote}
\begin{verbatim}
\Touhenkigou[#1]<#2><#3>(#4)#5#6
     #1 : $B5-9f!J%G%U%)%k%H$O(B | $B!K(B
     #2 : $B8D?t(B
     #3 : #2$B$GJ#?t$r;XDj$7$?>l9g$N5-9f4V4V3V!J%G%U%)%k%H(B0.5pt$B!K(B
     #4 : $B0LCV!J%G%U%)%k%H$O(B0.5$B!$$9$J$o$ACfE@!K(B
     #5, #6 : $B@~J,$NN>C<(B
\end{verbatim}
\end{boxnote}

$BJU>e$KCV$/=DK@$r(B2$BK\!J(B3$BK\!K$K$7$?$$$H$-$O(B \verb+<#2>+ $B%*%W%7%g%s$rMQ$$$^$9!#(B

\showexample[\cmd{Touhenkigou}](0.55)(0.4){example/Touhen02}

$BJ#?t$N@~J,$KEyJU5-9f$r$D$1$k%3%^%s%I$,(B \cmd{touhenkigou} $B$G$9!#(B
$B@~J,$r(B`;'$B$G6h@Z$C$FJB$Y$^$9!#(B

\showexample[\cmd{touhenkigou}](0.55)(0.4){example/Touhen03}

$B=q<0$G$9!#(B\cindex{touhenkigou}
\begin{boxnote}
\begin{verbatim}
\touhenkigou[#1]<#2>#3
        #1 : $BJU>e$KCV$/5-9f!J%G%U%)%k%H$O(B | $B!K(B
        #2 : $B8D?t(B
        #3 : $B@~J,Ns!J6h@Z;R$O(B`;'$B!K(B
\end{verbatim}
\end{boxnote}

\subsubsection{$BJ?9T5-9f(B}
\begin{showEx}(.59,.35){\cmd{heikoukigou}}
\unitlength5mm\small
\begin{picture}(6,6)
\tenretu{A(1,1)s;B(4,5)n;
  C(3,1)s;D(6,5)n}
\heikoukigou{\A\C;\B\D}
\heikoukigou[2]{\A\B;\C\D}
\end{picture}
\end{showEx}

\begin{enumerate}[(1)]
  \item $B$9$J$o$A0lHV4JC1$J;HMQK!$O(B
\begin{jquote}
\begin{verbatim}
\heikoukigou{\P\Q;\R\S}
\end{verbatim}
\end{jquote}
$B$J$I$H!$?7%3%^%s%I(B\cmd{heikoukigou}$B$N0z?t$K(B
$BJ?9T5-9f$r$D$1$?$$@~J,$r(B`;'$B$G6h@Z$C$FJB$Y$^$9!%(B
  \item (1)$B$N>l9g!$5-9f(B`$>$'$B$O(B1$B8D$@$1$D$-$^$9$,!$$3$l$r(B2$B8D$K$7$?$1$l$P(B
\begin{jquote}
\begin{verbatim}
\heikoukigou[2]{\P\Q;\R\S}
\end{verbatim}
\end{jquote}
$B$H!$(B\cmd{heikoukigou}$B$K(B\verb+[2]+$B%*%W%7%g%s$r$D$1$^$9!%(B
  \item $B$5$i$K5-9f$N%5%$%:!$4V3V!$0LCV$J$I$rD4@0$7$?$$$H$-$O(B
    \verb+[...]+ $B%*%W%7%g%s$K(B
\begin{jquote}
\begin{verbatim}
key=val,key=val,....
\end{verbatim}
\end{jquote}
$B$N7A$N%*%W%7%g%s$rNs5-$7$^$9!%$I$NMM$J%*%W%7%g%s$,$"$k$+$O!$(B
$B<!$N=q<0$r$4Mw$/$@$5$$!%(B
\end{enumerate}

\begin{boxnote}
\begin{verbatim}
$BJ?9T5-9f(B
$B@~J,$NCf1{$K!$5-9f(B `>' $B$rG[CV$9$k!#(B
 \heikoukigou[#1]#2
   #1 : $B5-9f$N8D?t(B
     $B$^$?$O(B key=val
        heikoukigoukosuu=$B5-9f$N8D?t!J%G%U%)%k%HCM$O(B1$B!K(B
        heikoukigouiti  =$B5-9f$N0LCV!J%G%U%)%k%HCM$O(B0.5, 
                                    $B$9$J$o$A@~J,$NCfE@!K(B
        heikoukigoukankaku=$B5-9f$rJ#?tG[CV$9$k$H$-$N4V3V(B
                          $B!J%G%U%)%k%HCM$O(B1mm$B!K(B
        heikoukigousize =$B5-9f$NBg$-$5!J%G%U%)%k%HCM$O(B2$B!K(B
   #2 : $B@~J,$NNs$r(B`;'$B6h@Z$j$GNs5-$9$k!%(B
\end{verbatim}
\end{boxnote}

\subsection{$B3Q$NFbIt$K5-9f(B}
\subsubsection{\texorpdfstring{\cmd{Kakukigou}}{Kakukigou}}
$B3Q$NFbIt$K1_8L$J$I$rIA$-!$3Q$NBg$-$5$J$I$rI=<($5$;$k%3%^%s%I(B\\
\cmd{Kakukigou} $B$G$9!#(B\cindex{Kakukigou}

\showexample[\cmd{Kakukigou}](0.6)(0.35){example/KAKUki01}

\cmd{Kakukigou} $B$N4pK\E*$J;HMQK!$O(B
\begin{jquote}
$B3Q$r9=@.$9$k(B3$BE@!J??$sCf$,3Q$ND:E@!K$H3QFb$KCV$/5-9f!JJ8;zNs!K(B
\end{jquote}
$B$N(B4$B$D$N0z?t$rM?$($^$9!#=q<0$O(B

\begin{boxnote}
\begin{verbatim}
$B3Q$NFbIt$K1_8L$rIA$-!$$=$NK5$K5-9f$J$I$rCV$/!#(B
  \Kakukigou[#1]<#2>#3#4#5<#6><#7>
    #1 : $B1_8L$N>e$KCV$/5-9f(B
     #1=a $B$N$H$-$O3Q5-9f$KLp0u$r$D$1$k!#(B
     #1=r $B$N$H$-$O3Q5-9f$K5U8~$-$NLp0u$r$D$1$k!#(B
     #1=b $B$N$H$-$O3Q5-9f$KN>8~$-$NLp0u$r$D$1$k!#(B
     #1=| $B$N$H$-$O!$1_8LCf1{$K1_8L$H?bD>$JC;$$@~J,(B
    #2 : $B1_8L$N8D?t(B
    #3#4#5 : $B3Q(BBAC 
       #4 $B$,3Q$ND:E@(B
       $BH>D>@~(B #4#3 $B$r(B $B2sE>$7$F(B #4#5 $B$K=E$M$k2sE>$rI=<($9$k!#(B
    #6 : $B1_8L$HD:E@$N5wN%78?t!J%G%U%)%k%HCM(B 1 $B!K(B
       hankei=$B!JL5L>?t!K(B $BH>7B$N;XDj(B
       Hankie=$B!JC10LIU!K(B     $B!7(B
       siteiten=         $B1_8L$,;XDjE@$rDL2a(B
       hasen=[$BGK@~$ND9$5(B][$BGK@~$N4V3V(B] $B1_8L$rGK@~$G(B
    #7 : $B1_8L$NCf?4$H3Q$ND:E@0LCV!J%G%U%)%k%HCM(B 0 $B!K(B
    $B0J2<!$(B\emathPut $B$KB3$/!#(B
  \Kakukigou* : $B1_8L$NFbIt$rEI$j$D$V$7$^$9!#(B
\end{verbatim}
\end{boxnote}

\verb+#3+ $B$H(B \verb+#5+ $B$rF~$lBX$($k$H!$(B

\showexample[$B3Q$N8~$-(B](0.6)(0.35){example/KAKUki09}

$B$9$J$o$A!$H>D>@~(B \verb+#4#3+ $B$r@5$N8~$-$K2sE>$7$F(B \verb+#4#5+ $B$K(B
$B=E$M$k2sE>$rI=<($7$^$9!#(B

$B1_8L$N0LCV$rD4@0$9$k$K$O(B \verb+<#6>+ $B%*%W%7%g%s$rMQ$$$^$9!#(B
$B%G%U%)%k%H$r(B1$B$H$7$F!$(B1$B$h$jBg$-$/$9$l$PD:E@$+$iN%$l!$(B
1$B$h$j>.$5$/$9$k$HD:E@$K6a$E$-$^$9!#(B

\showexample[$B1_8L$N0LCV(B](0.64)(0.3){example/KAKUki06}

$B1_8L$NH>7B$rD>@\;XDj$9$k$3$H$b2DG=$G$9!#(B

\showexample[$B1_8L$NH>7B;XDj(B](0.64)(0.3){example/KAKUki13}

\verb/hankei=/$B$N1&JUCM$NC10L$O(B\verb/\unitlength/$B$G$9!#(B
$B$3$l$r(B\verb/\unitlength/$B$K0MB8$7$J$$CM$G;XDj$9$k$K$O!$(B
\verb/Hankei=/$B$H$7$^$9!#(B

\showexample[$B1_8L$NH>7B;XDj!JC10LIU!K(B](0.64)(0.3){example/KAKUki14}

$B1_8L$,DL2a$9$Y$-0lE@$r;XDj$9$kJ}K!$b$"$j$^$9!#(B

\showexample[$B1_8L$NDL2aE@;XDj(B](0.64)(0.3){example/KAKUki15}

$B1_8L$N6J$,$j6q9g$O!$(B\verb+<#7>+ $B%*%W%7%g%s$GD4@0$7$^$9!#(B
$B%G%U%)%k%H$O(B 0 $B$G$9$,!$(B1$B$K6a$E$1$k$HH>7B$,>.$5$/$J$C$F$$$-$^$9!#(B

\showexample[\cmd{Kakukigou}](0.64)(0.3){example/KAKUki07}

$B1_8L$NCf1{It$K1_8L$K?bD>$J=D@~$r(B1$BK\F~$l$kI=<(K!$b$"$j$^$9!#(B
\texttt{[|]} $B%*%W%7%g%s$G$9!#(B

\showexample[$B1_8L$NCf?4$K=DK@(B](0.64)(0.3){example/KAKUki11}

$B1_8L$KLp0u$r$D$1$k$K$O(B \cmd{Kakukigou} $B$K(B \verb+[a]+ $B%*%W%7%g%s$r$D$1$^$9!#(B

\showexample[$B1_8L$KLp0u(B](0.64)(0.3){example/KAKUki08}

\verb+[a]+ $B%*%W%7%g%s$G$O!$Lp0u$O@5$N2sE>J}8~$K$D$-$^$9!#(B
$B$3$l$r5U8~$-$K$D$1$?$$$H$-$O!$(B\verb+[a]+ $B%*%W%7%g%s$KBe$($F(B
\verb+[r]+ $B%*%W%7%g%s$r$D$1$^$9!#(B

\showexample[$B5U8~$-$KLp0u(B](0.64)(0.3){example/KAKUki10}

\verb+[b]+ $B%*%W%7%g%s$r$D$1$k$H!$N>8~$-$KLp0u$,$D$-$^$9!#(B

\showexample[$BN>8~$-$KLp0u(B](0.64)(0.3){example/KAKUki21}

$B1_8L$NFbIt$rEI$j$D$V$9$K$O!$(B\verb+\Kakukigou*+$B%3%^%s%I$rMQ$$$^$9!#(B

\showexample[$B1_8L$NFbIt$rEI$j$D$V$9(B](0.64)(0.3){example/KAKUki23}

$BEI$j$D$V$7$NG;EY$r;XDj$9$k$K$O(B\verb+[...]+$B%*%W%7%g%s$rMQ$$$^$9!#(B
0$B!JGr!K$H(B1$B!J??9u!K$N4V$N?t$r;XDj$7$^$9!#(B

\showexample[$B1_8L$NFbIt$r9u$GEI$j$D$V$9(B](0.64)(0.3){example/KAKUki24}

$BJ#?t$N3Q$r6hJL$9$k$?$a!$(B
$B3QFb$N1_8L$r(B2$B=E!$(B3$B=E$K$9$k$3$H$,$G$-$^$9!#(B\texttt{<..>}$B%*%W%7%g%s$G$9!#(B

\showexample[$B1_8L$rFs=E$K(B](0.9)(0.5){example/KAKUki02}

$B5U$K(B \texttt{<..>} $B%*%W%7%g%s$K(B \texttt{<0>} $B$H;XDj$9$l$P!$(B
$B3QFb$K1_8L$OIA$+$:!$5-9fN`$@$1$rCV$/$3$H$b$G$-$^$9!#(B

\showexample[$B1_8L$J$7(B](0.9)(0.5){example/KAKUki03}

$B1_8L$rGK@~$GIA2h$9$k$3$H$b2DG=$G$9!#(B

\showexample[$BGK@~$N1_8L(B](0.9)(0.5){example/KAKUki18}

$B$5$i$KLp0u$rIU$1$k$3$H$b2DG=$G$9!#(B

\showexample[$BGK@~$N1_8L$KLp0u(B](0.9)(0.5){example/KAKUki22}

$B:G8e$K!$3QFb$KCV$/J8;zNs$N0LCVD4@0$G$9!#(B
$B$3$l$O!$(B\cmd{Put}$B$NJ8;zNs0LCVD4@0%*%W%7%g%s$HF1$8$G$9!#(B

\showexample[\cmd{Kakukigou}](0.64)(0.3){example/KAKUki16}

\noindent
$B$H4J0W;XDj%*%W%7%g%s(B\verb/[e]/$B$b;H$($^$9$,!$(B
$B3QFb$O69$$$G$9$+$i(B

\showexample[\cmd{Kakukigou}](0.64)(0.3){example/KAKUki17}

\noindent
$B$J$I$H:Y$+$/;XDj$7$?J}$,NI$$$G$7$g$&!#(B
$B<XB-$J$,$iJ8;zNs$rG[CV$9$k4p=`E@$O!$1_8L$NCfE@$G$9!#(B

$BEy$7$$3Q$KF1$85-9f$r$D$1$k$?$a$N%3%^%s%I$,(B \cmd{toukakukigou} $B$G$9!#(B
$B3Q$r(B `;' $B$G6h@Z$C$FJB$Y$k$H$$$&$N$O!$(B\cmd{touhenkigou} $B$HF1MM$G$9!#(B

\showexample[\cmd{toukakukigou}](0.55)(0.39){example/KAKUki19}

\cmd{toukakukigou*}$B$H(B`$*$'$B$r$D$1$k$H3QFb$rEI$j$D$V$7$^$9!#(B

\showexample[\cmd{toukakukigou*}](0.55)(0.39){example/KAKUki20}

$B=q<0$G$9!#(B\cindex{toukakukigou}\cindex{toukakukigou*}

\begin{boxnote}
\begin{verbatim}
$BJ#?t$N3Q$K3Q5-9f$r$D$1$k!#(B
  \toukakukigou[#1]<#2>#3
  \toukakukigou*[#1]<#2>#3
    #1 : $B1_8L$N>e$KCV$/5-9f(B
     #1=| $B$N$H$-$O!$1_8LCf1{$K1_8L$H?bD>$JC;$$@~J,(B
     *$BIU$N$H$-$O!$EI$j$D$V$7$NG;EY(B(0$B!A(B1)
    #2 : $B1_8L$N8D?t(B
    #3 : $B3Q$NNs(B $B6h@Z$j;R$O(B`;'
\end{verbatim}
\end{boxnote}

\subsubsection{$BD>3Q5-9f(B}
$BFC$KD>3Q$rI=$95-9f$G$9!#(B

\showexample[\texttt{<..>}$B%*%W%7%g%s(B](0.55)(0.4){example/KAKUki04}

$B=q<0$O(B\cindex{Tyokkakukigou}

\begin{boxnote}
\begin{verbatim}
\Tyokkakukigou[#1](#2)#3#4#5
        #1      : $BD>3Q5-9fFb$rEI$j$D$V$9$H$-$NG;$5(B
        #2      : $BD>3Q5-9f$N%5%$%:(B
        #3#4#5  : $BD>3Q(B(#4$B$,D:E@(B)
\end{verbatim}
\end{boxnote}

$BD>3Q5-9f$N%5%$%:$O(B \verb+(#2)+ $B$G;XDj$G$-$^$9!#(B
$B%G%U%)%k%H$O(B \texttt{5(pt)} $B$G$9!#(B\texttt{10pt} $B$K$7$?$$$H$-$O!$(B
\texttt{(10)} $B$H;XDj$7$^$9!#(B

\showexample[\texttt{<..>}$B%*%W%7%g%s(B](0.55)(0.4){example/KAKUki05}

$B6u4V?^7A$J$IJ#;($J?^7A$G!$D>3Q5-9f$rEI$j$D$V$7$?$$$H$-$O!$(B
\verb/[#1]/ $B%*%W%7%g%s$KEI$j$D$V$7$NG;EY(B(0$B!A(B1)$B$r;XDj$7$^$9!#(B

\showexample[\texttt{<..>}$B%*%W%7%g%s(B](0.95)(0.6){example/KAKUki12}

$BJ#?t$N3Q$KD>3Q5-9f$r$D$1$k%3%^%s%I(B\cmd{tyokkakukigou}$B$b$"$j$^$9!#(B
\cindex{tyokkakukigou}

\subsubsection{$B0lHL3Q(B}
360\Deg $B$r1[$($k3Q$K3Q5-9f$r$D$1$k%3%^%s%I(B \cmd{ippankaku} $B$N(B
$B4pK\E*$J;HMQK!$O!$0z?t$K3QEY$rM?$($k$@$1$G$9!#(B

\begin{showEx}{\cmd{ippankaku}}
\unitlength8mm\small
\begin{zahyou}(-3,3)(-3,3)%
\ippankaku{1230}%
\kHantyokusen{(0,0)}{1230}%
\end{zahyou}
\end{showEx}
\bigskip

\paragraph{\cmd{rasenP}}
$B$3$N5-9f$NE,Ev$J$H$3$m$KJ8;zNs$rG[CV$9$k$?$a$K!$(B
$B3QEY$r;XDj$7$FMf@{>e$NE@$r5a$a$k%3%^%s%I(B \cmd{rasenP} $B$r(B
$BMQ0U$7$^$7$?!#(B\cmd{emathPut}$B$HJ;MQ$7$FJ8;zNs$rG[CV$7$^$9!#(B

\begin{showEx}{\cmd{rasenP}}
\unitlength8mm\small
\begin{zahyou}(-3,3)(-3,3)%
\ippankaku{1230}%
\rasenP{1140}\P
\kHantyokusen{(0,0)}{1230}%
\Put\P[ne]{1230\Deg}%
\end{zahyou}
\end{showEx}
%\clearpage

\paragraph{$BMf@{$N7A>uJQ99(B}
$BMf@{(B $r=a\theta+b$ $B$N78?t(B $a$, $b$ $B$N%G%U%)%k%HCM$O(B
$a=0.02$, $b=0.2$ $B$H$7$F$"$j$^$9!#$3$l$rJQ99$9$k$?$a$N%*%W%7%g%s$,(B
\begin{jquote}
\begin{verbatim}
  \ippankaku<#1><#2>
  #1 : a $B$r(B #1 $BG\$9$k!#(B
  #2 : b $B$r(B #2 $BG\$9$k!#(B
\end{verbatim}
\end{jquote}

$B$=$N8z2L$O(B

\begin{showEx}{\texttt{<...>}$B%*%W%7%g%s(B}
\unitlength8mm\small
\begin{zahyou}(-3,3)(-3,3)%
\ippankaku<1.5><2.5>{1230}%
\kHantyokusen{(0,0)}{1230}%
\end{zahyou}
\end{showEx}

\paragraph{$B;O@~JQ99%*%W%7%g%s(B}
$B;O@~$N0LCV$rJQ99$9$k$K$O(B \texttt{[...]} $B%*%W%7%g%s$rMQ$$$^$9!#(B

\begin{showEx}{\texttt{[...]}$B%*%W%7%g%s(B}
\unitlength8mm\small
\begin{zahyou}(-3,3)(-3,3)%
\ippankaku[60]{1230}%
\def\O{(0,0)}%
\kHantyokusen{(0,0)}{60}%
\kHantyokusen{(0,0)}{1230}%
\end{zahyou}
\end{showEx}
\bigskip

\paragraph{$B3Q$,Ii$N>l9g$N=hM}(B}
$B3Q$,Ii$N>l9g$O!$FbIt=hM}$G(B $r=a\theta+b$ $B$N(B $a$ $B$NId9f$r5UE>$5$;$F$$$^$9!#(B

\begin{showEx}{$BIi$N3Q(B}
\unitlength8mm\small
\begin{zahyou}(-3,3)(-3,3)%
\ippankaku{-1230}%
\kHantyokusen{(0,0)}{-1230}%
\end{zahyou}
\end{showEx}
\bigskip

\paragraph{$BLp0u$N@)8f(B}
$BLp0u$N@)8f$O(B \verb/(#4)/ $B%*%W%7%g%s$G9T$$$^$9!#(B
$B%G%U%)%k%H$O=*C<$KLp0u$r$D$1$^$9!#(B
$B$3$l$r5U$K;OC<$K$D$1$?$$$H$-$O(B \verb/(s)/ $B%*%W%7%g%s$r$D$1$^$9!#(B
\begin{showEx}{\texttt{(s)}$B%*%W%7%g%s(B}
\unitlength8mm\small
\begin{zahyou}(-3,3)(-3,3)%
\ippankaku(s){1230}%
\kHantyokusen{(0,0)}{1230}%
\end{zahyou}
\end{showEx}
\bigskip

$BN>C<$K$D$1$k$K$O(B \verb/(b)/ $B%*%W%7%g%s$rMQ$$$^$9!#(B
\begin{showEx}{\texttt{(b)}$B%*%W%7%g%s(B}
\unitlength8mm\small
\begin{zahyou}(-3,3)(-3,3)%
\ippankaku(b){1230}%
\kHantyokusen{(0,0)}{1230}%
\end{zahyou}
\end{showEx}
\bigskip

$BLp0u$r$D$1$?$/$J$$$H$-$O(B \verb/(n)/ $B%*%W%7%g%s$G$9!#(B
\begin{showEx}{\texttt{(n)}$B%*%W%7%g%s(B}
\unitlength8mm\small
\begin{zahyou}(-3,3)(-3,3)%
\ippankaku(n){1230}%
\kHantyokusen{(0,0)}{1230}%
\end{zahyou}
\end{showEx}
\bigskip

\paragraph{$BN>C<$K(B\cmd{Kuromaru}, \cmd{Siromaru}}
$BN>C<$K9u4](B,$BGr4]$rG[CV$7$?$$$H$-$O!$(B
$B$i$;$s>e$N(B1$BE@$r5a$a$k(B \cmd{rasenP} $B%3%^%s%I$rMQ$$$^$9!#(B
\begin{showEx}(.54,.4){$-690\Deg\leqq \theta<0\Deg$}
\small
\begin{zahyou*}[ul=8mm](-3,3)(-3,3)%
\drawXYaxis
\ippankaku<1.5><2.5>(n){-690}%
\kHantyokusen{(0,0)}{-690}%
\rasenP{0}\Start\Siromaru\Start
\rasenP{-690}\End\Kuromaru\End
\rasenP{-640}\P
\Put\P[ne]{$-690\Deg$}%
\end{zahyou*}
\end{showEx}
%\clearpage

\paragraph{\cmd{ippankaku} $B$N=q<0(B}
\cmd{ippankaku} $B$N=q<0$G$9!#(B

\begin{boxnote}
\begin{verbatim}
\ippankaku<#1><#2>[#3](#4)#5
        #1 : a $B$NG\N((B
        #2 : b $B$NG\N((B
        #3 : $B;O$a3Q!JO;==J,K!!K(B
        #4 : $BLp0u$NG[CV(B
          e = $B=*C<!J%G%U%)%k%H!K(B
          s = $B;OC<(B
          b = $BN>C<(B
          n = $B$J$7(B
        #5 : $B=*$j3Q!JO;==J,K!!K(B
\end{verbatim}
\end{boxnote}

\subsection{$B?t<0$K(B\textsf{picture}$B4D6-$rJ;CV(B}
\subsubsection{\textsf{sikipicture}$B4D6-(B}
$B?t<0$K(B\textsf{picture}$B4D6-$rJ;@_$7(B,$B?t<0$r>~$jN)$F$h$&$H$$$&$N$,(B
\textsf{sikipicture}$B4D6-$G$9!#$3$l$O!$<B<A(B\textsf{zahyou*}$B4D6-$G$9!#(B
$B$7$?$,$C$F!$$=$N4D6-Fb$K$O!$(B\textsf{zahyou*}$B4D6-Fb$K5-=R$G$-$k(B
$B$b$N$O$9$Y$F5-=R2DG=$G$9!#(B

\begin{showEx}{\textsf{sikipicture}$B4D6-(B}
a\begin{sikipicture}{%
  x^2,
  +,
  2,
  x,
  +,
  \protect\bunsuu12}
\Takakkei{\LT\LB\RB\RT}
\put(0,0){\color{red}\Kuromaru\O}%
\end{sikipicture}z
\end{showEx}

$B:BI8NN0h$r3NG'$9$k$?$a!$;M6y$r7k$V;M3Q7A$H(B
$B:BI886E@$K@V4]0u$rIA2h$7$F$*$-$^$7$?!#(B

\begin{enumerate}[$BCm(B1.~]
  \item \verb+\begin{sikipicture}{...}+$B$N0z?tFb$K(B\cmd{bunsuu}$B$J$I$N%3%^%s%I$r(B
    $BF~$l$k$H$-$O(B\verb+\protect+$B$r$+$V$;$F$*$/I,MW$,$"$j$^$9!#(B
  \item \TeX $B$,G'CN$9$k%5%$%:$O>e?^$N;M3Q7A$G$9!#(B
    \cmd{put}$B$rMQ$$$F$3$N;M3Q7A$N30It$KJ8;zNs$rG[CV$9$k$3$H$O2DG=$G$9$,!$(B
    \TeX $B$O$=$NB8:_$rG'CN$7$^$;$s$+$i!$A08e$NJ8>O$H$+$V$j$^$9!#(B
    \verb+\vspace{...}+$B$J$I$GD4@0$9$kI,MW$,$"$j$^$9!#(B
  \item \textsf{sikipicture}$B4D6-$N(B\cmd{unitlength}$B$O!$(B1pt$B!J8GDj!K$G$9!#(B
  \item \textsf{sikixypos}$B$bF15A8l$H$7$F;H$($^$9!#(B
\end{enumerate}

\subsubsection{\cmd{sikiBi}, \cmd{sikiTi}}
\verb+\begin{sikipicture}{...}+$B$N0z?tFb$G!$J,3d5-=R$5$l$?3F9`$NCf1{2<It(B
$B$NE@$N:BI8$,(B
\begin{jquote}
\begin{verbatim}
\sikiBi, \sikiBii, \sikiBiii, .....
\end{verbatim}
\end{jquote}
$B$KDj5A$5$l$F$$$^$9!#$=$3$K@V4]$rBG$C$F$_$^$7$g$&!#(B

\begin{showEx}{\cmd{sikiBi},...}
a\begin{sikipicture}{%
  x^2,
  +,
  2,
  x,
  +,
  \protect\bunsuu12}
\put(0,0){\color{red}\kuromaru{%
  \sikiBi;\sikiBii;\sikiBiii;%
  \sikiBiv;\sikiBv;\sikiBvi}}%
\end{sikipicture}z
\end{showEx}

$+$$B$N2<$N@V4]$,$:$l$F$$$^$9$M!#(B
$B$3$l$O!$Fs9`1i;;;R!$4X781i;;;R$K$ON>B&$K%"%-$,F~$k$3$H$K$h$k$:$l$G$9!#(B
$B@5$7$$0LCV$,M_$7$1$l$P!$(B\verb+\begin{sikipicture}{...}+$B$N0z?tFb$N1i;;;RN>B&$K(B
\verb+{}+$B$rJd$C$F$d$j$^$9!#(B

\begin{showEx}{$BFs9`1i;;;R$KBP$9$kJd@5(B}
a\begin{sikipicture}{%
  x^2,
  {}+{},
  2,
  x,
  {}+{},
  \protect\bunsuu12}
\put(0,0){\color{red}\kuromaru{%
  \sikiBi;\sikiBii;\sikiBiii;%
  \sikiBiv;\sikiBv;\sikiBvi}}%
\end{sikipicture}z
\end{showEx}

$x^2$$B$N2<$b$:$l$F$k$C$F(B $B!)(B
$B$3$l$O$:$l$F$$$^$;$s!#(B$x^2$$B$NCf1{2<$G$9!#Dl$N(B$x$$B$NCf1{2<$,M_$7$$$J$i!$(B
$x$$B$H$=$N$Y$->h$rJ,N%$7$FM?$($^$9!#(B

$B$D$$$G$K!$3F9`$NCf1{>e$K$bNP4]$r$D$1$F$_$^$9!#(B
$B!J(B\verb+\sikiTi, \sikiTii, .....+$B!K(B

\begin{showEx}{$BFs9`1i;;;R$KBP$9$kJd@5(B}
a\begin{sikipicture}{%
  x,
  {}^2,
  {}+{},
  2,
  x,
  {}+{},
  \protect\bunsuu12}
\put(0,0){\color{red}\kuromaru{%
  \sikiBi;\sikiBii;\sikiBiii;%
  \sikiBiv;\sikiBv;\sikiBvi;%
  \sikiBvii}}%
\put(0,0){\color{green}\kuromaru{%
  \sikiTi;\sikiTii;\sikiTiii;%
  \sikiTiv;\sikiTv;\sikiTvi;%
  \sikiTvii}}%
\end{sikipicture}z
\end{showEx}

\subsubsection{\cmd{sikixposi}, \cmd{sikiyhposi}, \cmd{sikiydposi}}
$B$J$*!$4]0u$r$D$1$?E@$N(B$x$$B:BI8$O(B
\begin{jquote}
\begin{verbatim}
\sikixposi, \sikixposii, .....
\end{verbatim}
\end{jquote}
$B$G$9!#$^$?(B$y$$B:BI8$O!$@V4]$N$[$&$,(B
\begin{jquote}
\begin{verbatim}
\sikiydposi, \sikiydposii, .....
\end{verbatim}
\end{jquote}
$BNP4]$N$[$&$,(B
\begin{jquote}
\begin{verbatim}
\sikiyhposi, \sikiyhposii, .....
\end{verbatim}
\end{jquote}
$B$H$J$C$F$$$^$9!#(B

\begin{showEx}{\cmd{sikixpos..}}
a\begin{sikipicture}{%
  x,
  {}^2,
  {}+{},
  2,
  x,
  {}+{},
  \protect\bunsuu12}
\Takakkei{\LT\LB\RB\RT}
\put(0,0){\color{red}\kuromaru{%
  (\sikixposi,\ymin);%
  (\sikixposii,\ymin);%
  (\sikixposiii,\ymin)}}%
\put(0,0){\color{green}\kuromaru{%
  (\sikixposi,\ymax);%
  (\sikixposii,\ymax);%
  (\sikixposiii,\ymax)}}%
\end{sikipicture}z
\end{showEx}

\subsubsection{\cmd{sikixlposi}, \cmd{sikixrposi}}
\cmd{sikixposi}$B$J$I$O3F9`$NCf1{(B$x$$B:BI8$rI=$7$^$9$,!$(B
$B3F9`!&:81&C<$N(B$x$$B:BI8$O$=$l$>$l(B

\begin{jquote}
\begin{verbatim}
\sikixlposi, \sikilposii, .....
\sikixrposi, \sikirposii, .....
\end{verbatim}
\end{jquote}
$B$J$I$G$9!#(B

\begin{showEx}{\cmd{sikixlpos..}, \cmd{sikixrpos..}}
a\begin{sikipicture}{%
  x,
  {}^2,
  {}+{},
  2,
  x,
  {}+{},
  \protect\bunsuu12}
\Takakkei{\LT\LB\RB\RT}
\put(0,0){\color{red}\kuromaru{%
  (\sikixlposiv,\sikiyhposiv);%
  (\sikixrposiv,\sikiyhposiv)}}%
\put(0,0){\color{green}\kuromaru{%
  (\sikixposiv,\sikiydposiv)}}%
\end{sikipicture}z
\end{showEx}

\subsubsection{$B;HMQNc(B}
\paragraph{$B<0E83+$N@bL@?^(B}
$B<0$rE83+$9$k$H$-$NN.$l$r@bL@$9$k?^$G$9(B.

\begin{showEx}(1,.9){$B<0E83+$N@bL@?^(B}
\vskip 24pt

\begin{sikipicture}{(,a,+,b,),\kern1em(,x,+,y,)=,ax,+,ay,+,bx,+,by}
  \HenKo{(\sikixposvii,\ymax)}{(\sikixposii,\ymax)}{\maru1}
  \HenKo<henkoH=15pt>{(\sikixposix,\ymax)}{(\sikixposii,\ymax)}{\maru2}
  \HenKo{(\sikixposiv,\ymin)}{(\sikixposvii,\ymin)}{\maru3}
  \HenKo<henkoH=15pt>{(\sikixposiv,\ymin)}{(\sikixposix,\ymin)}{\maru4}
  \Put{(\sikixposxi,\ymax)}(0,0)[b]{\maru1}
  \Put{(\sikixposxiii,\ymax)}(0,0)[b]{\maru2}
  \Put{(\sikixposxv,\ymin)}(0,0)[t]{\maru3}
  \Put{(\sikixposxvii,\ymin)}(0,0)[t]{\maru4}
\end{sikipicture}
\vskip 10pt

\mbox{}
\end{showEx}

\paragraph{$BItJ,@QJ,$NN.$l(B}
$BItJ,@QJ,$NN.$l$r<($9$N$K$b;H$($^$9(B.

\begin{showEx}(1,.9){$BITDj@QJ,$NN.$l(B}
\vskip15pt

\begin{sikipicture}{%
  \dint{}{},f'(x),g(x),dx=,f(x),g(x),-,\dint{}{},f(x),g'(x),dx
}
  \HenKo<henkoH=15pt,yazirusi=r>\sikiTv\sikiTii{\footnotesize $B@QJ,$9$k(B}
  \HenKo<henkoH=15pt,yazirusi=r>\sikiTix\sikiTv{\footnotesize $B$=$N$^$^(B}
  \HenKo<henkoH=15pt,yazirusi=a>\sikiBiii\sikiBvi{\footnotesize $B$=$N$^$^(B}
  \HenKo<henkoH=15pt,yazirusi=a>\sikiBvi\sikiBx{\footnotesize $BHyJ,$9$k(B}
\end{sikipicture}
\vskip10pt
\mbox{}
\end{showEx}

\cmd{HenKo}$B$K$h$k1_8L$r<0$+$i>/$7N%$7$?$$$H$-$O!$(B\cmd{HenKo}$B$KBP$7(B
\verb+<agezoko=..>+$B%*%W%7%g%s$r$D$1$^$9!#(B

\begin{showEx}(1,.9){$BITDj@QJ,$NN.$l(B}
\vskip18pt

\begin{sikipicture}{%
  \dint{}{},f'(x),g(x),dx=,f(x),g(x),-,\dint{}{},f(x),g'(x),dx
}
  \HenKo<henkoH=15pt,yazirusi=r,agezoko=3>\sikiTv\sikiTii{\footnotesize $B@QJ,$9$k(B}
  \HenKo<henkoH=15pt,yazirusi=r,agezoko=3>\sikiTix\sikiTv{\footnotesize $B$=$N$^$^(B}
  \HenKo<henkoH=15pt,yazirusi=a,agezoko=3>\sikiBiii\sikiBvi{\footnotesize $B$=$N$^$^(B}
  \HenKo<henkoH=15pt,yazirusi=a,agezoko=3>\sikiBvi\sikiBx{\footnotesize $BHyJ,$9$k(B}
\end{sikipicture}
\vskip10pt
\mbox{}
\end{showEx}

\paragraph{$BCLOC<<(B No.107 $B$U$?$?$S(B}
\texttt{newPh232.tex}$B$G$b<h$j>e$2$?;HMQNc$r:FEY<h$j>e$2$F$_$^$9!#(B\\
$B%=!<%9%j%9%H$O(B\texttt{re107a.tex}$B$G$9!#(B

\begin{shadebox}
\ReadTeXFile{re107a.tex}
\vskip12pt
\mbox{}
\end{shadebox}
\bigskip

$B1&B&$NItJ,!$A02s$O(B\cmd{phkasen}$B$G%"%s%@!<%i%$%s$r0z$-$^$7$?$,!$(B
$B:#2s$O(B\cmd{HenKo}$B$G(B
\begin{jquote}
\begin{verbatim}
<henkotype=bracket>
\end{verbatim}
\end{jquote}
$B%*%W%7%g%s$rMQ$$$F$_$^$7$?!#(B
\pagebreak

\subsubsection{\textsf{bunpicture}$B4D6-(B}
\textsf{sikipicture}$B4D6-$O!$?t<0$KBP$9$k$b$N$G(B
\begin{jquote}
\begin{verbatim}
\begin{sikipicture}{.....}
\end{verbatim}
\end{jquote}
$B$N0z?t(B\verb+{.....}+$B$O?t<0%b!<%I$KF~$k$H2>Dj$5$l$F$$$^$9!#(B

$B$3$l$KBP$7$F!$%F%-%9%H%b!<%I$KF~$k$b$N$r(B\textsf{bunpicture}$B4D6-$H>N$9$k$3$H$H$7$^$9!#(B

\begin{showEx}(1,.9){\textsf{bunpicture}$B4D6-(B}
\begin{bunpicture}{%
    \underline{It}, is natural  ,\underline{that she should think}, so.}
  \HenKo<yazirusi=r,henkotype=bracket,Oval=4pt,Agezoko=2pt>\bunBi\bunBiii{}
\end{bunpicture}
\vskip5pt

\mbox{}
\end{showEx}

$B$5$i$K<j$,9~$s$G(B

\begin{showEx}(1,.9){$B1~MQNc(B}
\mbox{}\vskip1\baselineskip
\begin{bunpicture}{It ,is natural, that ,she should think so,.}
  \HenKo<henkotype=bracket,yazirusi=r,Agezoko=2pt,putoption={(0,1pt)[b]}>%
    \bunTiii\bunTi{\footnotesize It$B$O(Bthat$B0J2<$rI=$7$^$9(B}
  \HenKo<henkotype=bracket,yazirusi=a,Agezoko=2pt>%
    {(\bunxposiii,\ymin)}{(\bunxlposiv,\ymin)}{}
  \Add\ymin{-2}\yy
\put(0,0){\color{red}%
  \ArrowLine{(\bunxlposiv,\yy)}{(\bunxrposiv,\yy)}%
  \HenKo<henkotype=bracket,yazirusi=r,henkoH=10pt,Oval=5pt,%
        putoption={(0,-2pt)[t]},Agezoko=2pt>%
    {(\bunxlposii,\ymin)}{(\bunxrposiv,\ymin)}%
    {\color{black}\footnotesize that$B0J2<$,(BIt$B$N8e$KB3$-$^$9(B}
  \ArrowLine{(\bunxlposii,\yy)}{(\bunxrposii,\yy)}%
}%
\end{bunpicture}
\vskip1\baselineskip
\mbox{}
\end{showEx}



%\section{円・楕円}
\section{円・楕円}
\subsection{円}
\subsubsection{円}
\paragraph{中心と半径を指定}
円を描画するには,\LaTeX で \cmd{circle} がありますが,
直径を指定することになっています.半径の方が使いやすいので,
\cmd{En} を用意しました.次のように使います.\cindex{En}

\showexample[\cmd{En}](0.55)(0.35){example/en01}
\bigskip

すなわち,\cmd{En} は,中心と半径の2つの引数をとります.

\paragraph{半径の間接指定}
半径を指定する代わりに,円周上の1点を指定して描画する方法もあります。

\showexample[円周上の1点を指定](0.55)(0.35){example/en22}
\bigskip

\paragraph{直径を指定}
2点を指定して,それを結ぶ線分を直径とする円を描画するコマンド\cmd{EnT}
もあります。

\showexample[直径を指定](0.55)(0.35){example/en23}
\bigskip

\subsubsection{円の破線描画}
円を破線で描画するには,

\showexample[円周を破線で](0.55)(0.35){example/en15}

と\verb/<hasen=[破線の長さ][破線の間隔]>/オプションを与えます。

\begin{boxnote}
\begin{verbatim}
\En<#1>#2#3
  <#1>: hasen=xx で円弧を破線で描画するときの
                  [破線の長さ][破線の間隔]
   #2 : 中心の座標
   #3 : 半径
\end{verbatim}
\end{boxnote}


\subsubsection{円弧}
円弧を描画するために,\cmd{Enko} を用意しました.\cindex{Enko}

\showexample[\cmd{Enko}](0.55)(0.4){example/en04}

すなわち,\cmd{Enko} は,
\begin{jquote}
\begin{verbatim}
\Enko{中心}{半径}{開始角}{終了角}
\end{verbatim}
\end{jquote}
中心,半径,開始角,終了角と4つの引数を取ります.
\bigskip

円弧を破線で描画するには,\verb/<#1>/ に
\begin{verbatim}
   <hasen=[破線の長さ][破線の間隔]>
\end{verbatim}
オプションを与えます。

\showexample[円弧を破線で](0.55)(0.4){example/en16}

点線で描画するには,\verb/<#1>/に
\begin{verbatim}
   <ten=周上に置く点の個数>
\end{verbatim}
オプションを与えます。

\showexample[円弧を点線で](0.55)(0.4){example/en21}

\cmd{Enko}の書式です。

\begin{boxnote}
\begin{verbatim}
% 円弧
\Enko<#1>#2#3#4#5
  #1 : オプション
         hasen=xx で円弧を破線で描画するときの
                  [破線の長さ][破線の間隔]
                オプションを与える。
         ten=xx 円周を点線で描画するときの周上の点の個数
         yazirusi=a : 正方向に矢印をつける
                 =r : 負方向に矢印をつける
                 =b : 両方向に矢印をつける
                 =n : 矢印をつけない
  #2 : 中心
  #3 : 半径を直接与えるか
         tuukaten=xx として,円弧の周上の一点を与える
  #4 : 始め角を直接与えるか
        hazimeten=xx として,中心を始点,xx を終点とするベクトルの
         方向角を 始め角とするように指定する。
  #5 : 終り角を直接与えるか
         owariten=xx として,中心を始点,xx を終点とするベクトルの
         方向角を 終り角とするように指定する。
\end{verbatim}
\end{boxnote}

始め角,終り角を六十分法の角度で与える代わりに,
点を指定して,中心から指定した点に向かうベクトルの
方向角で与えるオプションの利用例です。

\showexample[両端の間接的指定](0.55)(0.4){example/en17}

半径も円周上の一点を指定することにより,間接的に与えることもできます。

\showexample[半径の間接的指定](0.55)(0.4){example/en18}

\subsubsection{矢印付きの円弧(1) 偏角指定}
円弧に矢印をつけるには,\cmd{Enko}に\verb+<yazirusi=a>+オプションを
つけます。

\showexample[\texttt{<yazirusi=a>}](0.55)(0.4){example/en10}

矢印を逆向きにしたいときは,オプションの右辺値を`r'とします。

\showexample[\texttt{<yazirusi=r>}](0.55)(0.4){example/en11}

また,オプションの右辺値を`b'とすれば,両向きの矢印がつきます。

\showexample[\texttt{<yazirusi=b>}](0.55)(0.4){example/en19}

さらには,\verb+<hasen=..>+オプションと併用も可能です。

\showexample[\texttt{<yazirusi=b>}](0.55)(0.4){example/en20}

\subsubsection{矢印付きの円弧(2) 端点指定}
矢印付きの円弧を,半径と端点を指定して描画するのが
\cmd{ArrowArc} コマンドです.\cindex{ArrowArc}

\showexample[\cmd{ArrowArc}](0.55)(0.4){example/en12}

%\verb+[r]+ オプションをつけると,
%回転が負の方向となるように描画されます.

半径に負の値を与えると,
回転が負の方向となるように描画されます.

\showexample[回転の向き](0.55)(0.4){example/en13}


\begin{boxnote}
\begin{verbatim}
\ArrowArc[#1]#2#3#4
      #1 : 
      #2 : 半径(負のときは回転が負となる)
      #3 : 始点
      #4 : 終点
\end{verbatim}
\end{boxnote}

このコマンドを造った理由の一つは,
図が込み入って来たりして,文字列を該当個所に置けないとき,
ゆったりしたところに文字列を置き,そこから矢線を引こう,
というねらいがあります.少し曲げた方が見やすいのではないか,
と思います.

\showexample[\cmd{ArrowArc}の一応用](0.55)(0.4){example/en14}

\subsubsection{等弧記号}
二つの円弧の長さが等しいことを表すのに,
円弧の中央部分に短い縦線を入れるコマンドが
\cmd{Toukokigou}です。

\begin{showEx}(.6,.345){\cmd{Toukokigou}}
\begin{zahyou}[ul=15mm](-0.2,2.5)(-1.2,1.5)
\small
\def\O{(1,0)}
\rtenretu[\O]{A(1,0)ne;B(1,60)ne;
C(1,90)nw;D(1,150)nw}
\Enko\O{1}{0}{60}
\Enko\O{1}{90}{150}
\Toukokigou<2>\O\A\B
\Toukokigou<2>\O\C\D
\Drawlines{\A\O\B;\C\O\D}
\end{zahyou}
\end{showEx}

書式は
\begin{boxnote}
\begin{verbatim}
\Toukokigou<#1>#2#3#4
  #1 : 中央に配置する短い縦棒の個数(デフォルト値=1)
  #2 : 円弧の中心
  #3 : 弧の端点1
  #4 : 弧の端点2
\end{verbatim}
\end{boxnote}
\cindex{Toukokigou}

\subsubsection{扇形}
扇形を描画するために,\cmd{ougigata} を用意しました.\cindex{ougigata}

\showexample[\cmd{ougigata}](0.55)(0.4){example/en02}

すなわち,\cmd{ougigata} は,半径,開始角,終了角と3つの引数を取ります.

\begin{boxnote}
\begin{verbatim}
\ougigata#1#2#3
  #1 : 半径を直接与えるか
       tuukaten=xx として,円弧の周上の一点を与える
  #2 : 始め角を直接与えるか
       hazimeten=xx として,中心を始点,xx を終点とするベクトルの
       方向角を 始め角とするように指定する。
  #3 : 終り角を直接与えるか
       owariten=xx として,中心を始点,xx を終点とするベクトルの
       方向角を 終り角とするように指定する。
 (中心は \put (\emathPut) で指定する.)
\end{verbatim}
\end{boxnote}

\subsubsection{弓形}
弓形を描画するために,\cmd{yumigata} を用意しました.\cindex{yumigata}

\showexample[\cmd{yumigata}](0.55)(0.4){example/en03}

すなわち,\cmd{yumigata} は,半径,開始角,終了角と3つの引数を取ります.

\begin{boxnote}
\begin{verbatim}
\yumigata#1#2#3
  #1 : 半径を直接与えるか
       tuukaten=xx として,円弧の周上の一点を与える
  #2 : 始め角を直接与えるか
       hazimeten=xx として,中心を始点,xx を終点とするベクトルの
       方向角を 始め角とするように指定する。
  #3 : 終り角を直接与えるか
       owariten=xx として,中心を始点,xx を終点とするベクトルの
       方向角を 終り角とするように指定する。
 (中心は \put (\emathPut) で指定する.)
\end{verbatim}
\end{boxnote}

\subsection{楕円}
\subsubsection{楕円}
軸が水平,垂直な楕円を描画するコマンドは \cmd{Daen} です.\cindex{Daen}

\showexample[\cmd{Daen}](0.45)(0.45){example/en05}

中心,横方向半径,縦方向半径 と三つの引数をとります.

\begin{boxnote}
\begin{verbatim}
\Daen#2#3#4
        #1 : 中心の座標
        #2 : 横軸方向の半径
        #3 : 縦軸方向の半径
\end{verbatim}
\end{boxnote}

\subsubsection{楕円弧}
楕円の一部を描画する \cmd{Daenko} です.\cindex{Daenko}

\showexample[\cmd{Daenko}](0.45)(0.45){example/en06}

\begin{boxnote}
\begin{verbatim}
\Daenko<#1>#2#3#4#5
        #1 : key=val
             hasen=[破線の長さ][破線の間隔]
             yazirusi=a : 正方向に矢印をつける
                     =r : 負方向に矢印をつける
                     =b : 両方向に矢印をつける
                     =n : 矢印をつけない
        #2 : 横軸方向の半径
        #3 : 縦軸方向の半径
        #4 : 始め角
        #5 : 終り角
             (中心は \put (\emathPut) で指定する.)
\end{verbatim}
\end{boxnote}

\subsubsection{破線}
(楕)円(弧)を破線で描画するには \cmd{Daenko} に
\verb+<hasen=..>+オプションを与えます.

\showexample[楕円弧の破線](0.5)(0.45){example/en07}

破線の長さを変更するには,オプションの値を変更します.
上の図を標準として比率を指定します.

\showexample[破線の長さ](0.5)(0.45){example/en08}

破線の間隔を調整するには,第2の \texttt{[..]}オプションです.
やはり,標準に対する比率です.

\showexample[破線の間隔](0.5)(0.45){example/en09}

なお,横軸方向の半径と縦軸方向の半径を同じ値にすれば,円(弧)を
破線で描画することもできます.

\subsubsection{矢印}
(楕)円(弧)に矢印をつけるには \cmd{Daenko} に
\verb+<yazirusi=..>+オプションを与えます.

\begin{showEx}{楕円弧に矢印}
\begin{picture}(4,2)%
\Put{(2,1)}{%
\Daenko<yazirusi=b>%
  {2}{1}{0}{90}}%
\end{picture}
\end{showEx}

\subsubsection{回転記号}
楕円弧に矢印を付ける機能を利用して,
回転軸の周りに回転を表す記号を付けることができます。

\begin{showEx}(.5,.44){\cmd{kaitenkigou}}
\begin{zahyou}[ul=10mm](-2,2)(-1,1)
\Put{(1.5,0)}{\kaitenkigou}
\end{zahyou}
\end{showEx}

$y$軸の周りの回転を表すには,\verb+[90]+オプションを付けます。

\begin{showEx}(.5,.44){$y$軸の周りの回転}
\begin{zahyou}[ul=10mm](-2,2)(-2,2)
\Put{(0,1.5)}{\kaitenkigou[90]}
\end{zahyou}
\end{showEx}

ここで,矢印は正の回転を表すようにつきますが,これを負の回転を表すように
つけるためには
\begin{jquote}
\begin{verbatim}
<muki=r>
\end{verbatim}
\end{jquote}
オプションを用います。

\begin{showEx}(.54,.4){矢印を逆向き}
\begin{zahyou}[ul=10mm](-1,3.5)(-.5,.5)
  \Put{(3,0)}{\kaitenkigou<muki=r>}
\end{zahyou}
\end{showEx}

\verb+<..>+オプションは,本来は倍率指定オプションです。

\begin{showEx}(.54,.4){倍率指定}
\begin{zahyou}[ul=10mm](-1,3.5)(-.5,.5)
  \Put{(3,0)}{\kaitenkigou<1.5>}
\end{zahyou}
\end{showEx}

倍率指定と矢印向きなど他のオプションと併用したいときは
\verb+<bairitu=..>+オプションを用います。

\begin{showEx}(.54,.4){倍率指定}
\begin{zahyou}[ul=10mm](-1,3.5)(-.5,.5)
  \Put{(3,0)}{%
    \kaitenkigou<bairitu=1.5,muki=r>}
\end{zahyou}
\end{showEx}

楕円のサイズは,デフォルトでは
\begin{jquote}
\begin{verbatim}
tyouhankei=3mm
tanhankei=1.5mm
\end{verbatim}
\end{jquote}
としてあります。これを変更するオプションです:

\begin{showEx}(.54,.4){サイズ変更}
\begin{zahyou}[ul=10mm](-1,3.5)(-.5,.5)
  \Put{(3,0)}{%
    \kaitenkigou<tanhankei=3mm>}
\end{zahyou}
\end{showEx}

楕円の右端を一部切って矢印をつけていますが,
切る場所を変えるオプションを紹介します。

\begin{showEx}(.54,.4){矢印の位置変更}
\begin{zahyou}[ul=10mm](-1,3.5)(-.5,.5)
  \Put{(3,0)}{\kaitenkigou%
    <hazimekaku=-165,owarikaku=165>}
\end{zahyou}
\end{showEx}

デフォルト値は
\begin{jquote}
\begin{verbatim}
hazimekaku=15, owarikaku=345
\end{verbatim}
\end{jquote}
となっています。

回転オプションと併用するときは,上記の指定角は回転する前の状況での値です。
上の図を$90\Deg$回転してみましょう。

\begin{showEx}(.54,.4){回転との併用}
\begin{zahyou}[ul=10mm]%
    (-1,3.5)(-.5,1.5)
  \Put{(0,1)}{\kaitenkigou%
    <hazimekaku=-165,%
    owarikaku=165>[90]}
\end{zahyou}
\end{showEx}

最後に,\cmd{kaitenkigou}の書式です。

\begin{boxnote}
\begin{verbatim}
直線のまわりに回転させることを表す記号
\kaitenkigou<#1>[#2]
   #1 : 倍率
     または key=val
            bairitu= (倍率)デフォルト値 : 1
            muki = r/n                      n
                r で,負の向き
                n で,正の向き
            hazimekaku=                     15
            owarikaku=                     345
            tyouhankei=                      3mm
            tanhankei=                     1.5mm
   #2 : 回転角
   位置は \emathPut で指定
\end{verbatim}
\end{boxnote}

直線$y=x$の周りに回転させる記号を,2倍のサイズで描いてみます。

\begin{showEx}(.5,.44){$y$軸の周りの回転}
\begin{zahyou}[ul=10mm](-2,2)(-2,2)
\Put{(1.8,1.8)}{\kaitenkigou<2>[45]}
\kTyokusen\O{45}{}{}
\end{zahyou}
\end{showEx}



%\section{円・直線の交点}
\section{円・直線の交点}
直線と直線,円と直線,円と円の交点を求めるコマンド類です.
コマンド名は \verb+\*and*+ の形で \texttt{*} のところは

\begin{jquote}
\begin{verbatim}
    C : 円
    L : 2点を与えた直線
    l : 1点と方向ベクトルを与えた直線
    k : 1点と方向角を与えた直線
\end{verbatim}
\end{jquote}
のいずれかで,
\begin{jquote}
\begin{verbatim}
\CandC
\CandL
\Candl
\Candk
\LandL
\Landl
\landl
\Landk
\kandk
\end{verbatim}
\end{jquote}
の9種類があります.

\subsection{2直線の交点}
\subsubsection{2直線の交点(1) \texorpdfstring{\cmd{LandL}}{LandL}}

\cmd{LandL} の使用例です.\cindex{LandL}

\showexample[\cmd{LandL}](0.55)(0.4){example/LandL01}

\begin{boxnote}
\begin{verbatim}
\LandL#1#2#3#4#5
    2点 #1, #2 を通る直線と
    2点 #3, #4 を通る直線の交点を #5 に与える.
\end{verbatim}
\end{boxnote}

\subsubsection{2直線の交点(2) \texorpdfstring{\cmd{Landl}}{Landl}}
\sankaku{ABC} の辺AB上に $\kaku{BCD}=30\Deg$ となる点Dを求める例です.
\cindex{Landl}

\showexample[\cmd{Landl}](0.55)(0.4){example/LandL02}

\begin{boxnote}
\begin{verbatim}
\Landl#1#2#3#4#5
    2点 #1, #2 を通る直線と
    点 #3 を通り,方向ベクトルが #4 の直線の交点を #5 に与える.
\end{verbatim}
\end{boxnote}

\subsubsection{2直線の交点(3) \texorpdfstring{\cmd{landl}}{landl}}
直線を1点と方向ベクトルで与える場合の,交点を求めるコマンドが
\cmd{landl} です.\cindex{landl}

\showexample[\cmd{landl}](0.6)(0.34){example/LandL03}

\begin{boxnote}
\begin{verbatim}
\landl#1#2#3#4#5
    点 #1 を通り, 方向ベクトルが #2 の直線と
    点 #3 を通り,方向ベクトルが #4 の直線の交点を #5 に与える.
\end{verbatim}
\end{boxnote}

\subsubsection{2直線の交点(4) \texorpdfstring{\cmd{Landk}}{Landk}}
直線を1点と方向角($x$軸の正の向きとなす角---六十分法)で与えた場合です。
まずは \cmd{kandk} の使用例です。\cindex{Landk}

\showexample[\cmd{kandk}](0.54)(0.4){example/LandL04}

\begin{boxnote}
\begin{verbatim}
\kandk#1#2#3#4#5
    点 #1 を通り, 方向角が #2 の直線と
    点 #3 を通り,方向角が #4 の直線の交点を #5 に与える.
\end{verbatim}
\end{boxnote}

ついで,\cmd{Landk} の使用例です。\cindex{Landk}

\showexample[\cmd{Landk}](0.54)(0.4){example/LandL05}

\begin{boxnote}
\begin{verbatim}
\Landk#1#2#3#4#5
    2点 #1, #2 を通る直線と
    点 #3 を通り,方向角が #4 の直線の交点を #5 に与える.
\end{verbatim}
\end{boxnote}

\subsubsection{垂線の足}
三角形の頂点から対辺(またはその延長)に下した垂線の足を求める
コマンド \cmd{Suisen} です.\cindex{Suisen}

\showexample[\cmd{Suisen}](0.55)(0.35){example/suisen01}

\begin{boxnote}
\begin{verbatim}
\Suisen#1#2#3#4
    点 #1 から直線 #2#3 へ下ろした垂線の足を #4 にセット
\end{verbatim}
\end{boxnote}

関連して,直線が1点と方向ベクトルまたは方向角で与えられている場合に
用いるコマンドがそれぞれ
\cmd{mSuisen}, \cmd{kSuisen} です。使用例を一つずつあげます。
\cindex{mSuisen}
\cindex{kSuisen}

\begin{showEx}(.5,.44){\cmd{mSuisen}}
\unitlength9mm\footnotesize
\begin{zahyou}(-3,3)(-3,3)
\tenretu{A(-2,-1)s;P(-1,2)n}
\def\houkouV{(1,1)}
\mSuisen\P\A\houkouV\Q
\kuromaru{\A;\P;\Q}
\Put\Q[se]{Q}
\Drawline{\P\Q}
\Tyokkakukigou\P\Q\A
\mTyokusen\A\houkouV{}{}
\end{zahyou}
\end{showEx}

\begin{showEx}(.5,.44){\cmd{kSuisen}}
\unitlength9mm\footnotesize
\begin{zahyou}(-3,3)(-3,3)
\tenretu{A(-2,-1)s;P(-1,2)n}
\def\kaku{60}
\kSuisen\P\A\kaku\Q
\kuromaru{\A;\P;\Q}
\Put\Q[s]{Q}
\Drawline{\P\Q}\Tyokkakukigou\P\Q\A
\kTyokusen\A\kaku{}{}
\end{zahyou}
\end{showEx}

\subsubsection{直線に関する対称点}
前節の垂線を発展させて,点の直線に関する対称点を求めるコマンド
\begin{jquote}
\begin{verbatim}
\Taisyouten 2点を通る直線
\mTaisyouten 1点と方向ベクトルを指定した直線
\kTaisyouten 1点と方向角を指定した直線
\end{verbatim}
\end{jquote}
を新設しました。以下,その使用例を並べます。
\cindex{Taisyouten}
\cindex{mTaisyouten}
\cindex{kTaisyouten}

\begin{showEx}(.5,.44){\cmd{Taisyouten}}
\unitlength9mm\footnotesize
\begin{zahyou}(-3,3)(-3,3)
\tenretu{A(-2,-1)s;B(1,1)s;P(-1,2)n}
\Taisyouten\P\A\B\Q
\kuromaru{\A;\B;\P;\Q}
\Put\Q[s]{Q}
\Drawline{\P\Q}
\Tyokusen\A\B{}{}
\end{zahyou}
\end{showEx}

\begin{showEx}(.5,.44){\cmd{mTaisyouten}}
\unitlength9mm\footnotesize
\begin{zahyou}(-3,3)(-3,3)
\tenretu{A(-2,-1)s;P(-1,2)n}
\def\houkouV{(1,1)}
\mTaisyouten\P\A\houkouV\Q
\kuromaru{\A;\P;\Q}
\Put\Q[s]{Q}
\Drawline{\P\Q}
\mTyokusen\A\houkouV{}{}
\end{zahyou}
\end{showEx}

\begin{showEx}(.5,.44){\cmd{kTaisyouten}}
\unitlength9mm\footnotesize
\begin{zahyou}(-3,3)(-3,3)
\tenretu{A(-2,-1)s;P(-1,2)n}
\def\kaku{60}
\kTaisyouten\P\A\kaku\Q
\kuromaru{\A;\P;\Q}
\Put\Q[s]{Q}
\Drawline{\P\Q}
\kTyokusen\A\kaku{}{}
\end{zahyou}
\end{showEx}
\bigskip

対称の中心(垂線の足)が必要なときは,オプション引数 \verb/[#4]/ に
その点を受取る制御綴を与えておきます。

\begin{showEx}(.5,.44){対称の中心}
\unitlength9mm\footnotesize
\begin{zahyou}(-3,3)(-3,3)
\tenretu{A(-2,-1)s;B(1,1)s;P(-1,2)n}
\Taisyouten\P\A\B[\H]\Q
\kuromaru{\A;\B;\P;\Q;\H}
\Put\Q[s]{Q}
\Tyokkakukigou\P\H\A
\touhenkigou<2>{\P\H;\H\Q}
\Drawline{\P\Q}
\Tyokusen\A\B{}{}
\end{zahyou}
\end{showEx}

この機能は,\cmd{mTaisyouten}, \cmd{kTaisyouten} にも使用できます。

\begin{showEx}(.5,.44){\cmd{mTaisyouten}}
\unitlength9mm\footnotesize
\begin{zahyou}(-3,3)(-3,3)
\tenretu{A(-2,-1)s;P(-1,2)n}
\def\houkouV{(1,1)}
\mTaisyouten\P\A\houkouV[\H]\Q
\kuromaru{\A;\P;\Q;\H}
\Put\Q[s]{Q}
\Drawline{\P\Q}
\Tyokkakukigou\P\H\A
\touhenkigou<2>{\P\H;\H\Q}
\mTyokusen\A\houkouV{}{}
\end{zahyou}
\end{showEx}

\begin{showEx}(.5,.44){\cmd{kTaisyouten}}
\unitlength9mm\footnotesize
\begin{zahyou}(-3,3)(-3,3)
\tenretu{A(-2,-1)s;P(-1,2)n}
\def\kaku{60}
\kTaisyouten\P\A\kaku[\H]\Q
\kuromaru{\A;\P;\Q;\H}
\Put\Q[s]{Q}
\Drawline{\P\Q}
\Tyokkakukigou\P\H\A
\touhenkigou<2>{\P\H;\H\Q}
\kTyokusen\A\kaku{}{}
\end{zahyou}
\end{showEx}

\subsection{円と直線の交点}
\subsubsection{円と直線の交点(1) \texorpdfstring{\cmd{CandL}}{CandL}}
\sankaku{ABC}の中線AMの延長が\sankaku{ABC}の外接円と交わる点Dを求めます.
\cindex{CandL}

\showexample[\cmd{CandL}](0.5)(0.45){example/CandL01}

\begin{boxnote}
\begin{verbatim}
\CandL#1#2#3#4#5#6
\CandL*#1#2#3#4#5#6
        点 #1 を中心とし,半径 #2 の円と
        2点 #3, #4 を通る直線との交点を #5 と #6 にセットする.
円と直線の2つの交点のどちらを #5 とするかについては
  \CandL の場合
    2つの交点のうち,x座標の小さい方が #5
    2つの交点のx座標が一致するときは,y座標の小さい方が #5
  \CandL* の場合
    円の中心(#1)と2つの交点(#5, #6)で作られる三角形の周を
        #1 → #5 → #6 → #1
    とたどる回り方が正の回転となるように定める。
    (三角形がつぶれる場合は,\CandL の定め方に従う)
\end{verbatim}
\end{boxnote}

\subsubsection{円と直線の交点(2) \texorpdfstring{\cmd{Candl}}{Candl}}
原点中心,半径1の円と,点(2,0)を通り傾き$-\bunsuu13$の直線との交点を求めます.\cindex{Candl}

\showexample[\cmd{Candl}](0.5)(0.45){example/CandL02}

\begin{boxnote}
\begin{verbatim}
\Candl#1#2#3#4#5#6
        点 #1 を中心とし,半径 #2 の円と
        点 #3 を通り, 方向ベクトルが #4 の直線
        との交点を #5 と #6 にセットする.
    2つの交点のうち,どちらを #5 とするかは \CandL と同じ。
\end{verbatim}
\end{boxnote}

\subsubsection{円と直線の交点(3) \texorpdfstring{\cmd{Candk}}{Candk}}
原点中心,半径1の円と,点(2,0)を通り傾き方向角15\Deg の直線
との交点を求めます.\cindex{Candk}

\showexample[\cmd{Candl}](0.5)(0.45){example/CandL03}

\begin{boxnote}
\begin{verbatim}
\Candk#1#2#3#4#5#6
        点 #1 を中心とし,半径 #2 の円と
        点 #3 を通り, 方向角が #4 の直線
        との交点を #5 と #6 にセットする.
    2つの交点のうち,どちらを #5 とするかは \CandL と同じ。
\end{verbatim}
\end{boxnote}

\subsection{円と円の交点 \texorpdfstring{\cmd{CandC}}{CandC}}
線分BCの長さが7のとき,Bを中心とする半径5の円と,Cを中心とする半径3の
円との交点をAとすれば,3辺の長さが7, 5, 3の三角形がえられます.
\cindex{CandC}

\showexample[\cmd{CandC}](0.5)(0.42){example/CandC01}

\begin{boxnote}
\begin{verbatim}
\CandC#1#2#3#4#5#6
        点 #1 を中心,半径 #2 の円と
        点 #3 を中心,半径 #4 の円との交点を#5と#6にセット
円と円の2つの交点のどちらを #5 とするかについては
  \CandC の場合
    2つの交点のうち,x座標の小さい方が #5
    2つの交点のx座標が一致するときは,y座標の小さい方が #5
  \CandC* の場合
    円の中心(#1)と2つの交点(#5, #6)で作られる三角形の周を
        #1 → #5 → #6 → #1
    とたどる回り方が正の回転となるように定める。
    (三角形がつぶれる場合は,\CandC の定め方に従う)
\end{verbatim}
\end{boxnote}


%\section{だ円・直線の交点}
\section{楕円と直線の交点}
\subsection{\texorpdfstring{\cmd{EandL}}{EandL}}
円と直線の交点を求めるコマンドを発展させ,
楕円と直線の交点を求めるコマンドが\cmd{EandL}です。
その書式は

\begin{boxnote}
\begin{verbatim}
\def\EandL#1#2#3#4#5#6#7{%
  #1 : 楕円の中心
  #2 : x軸方向の半径
  #3 : y軸方向の半径
  #4 : 直線上の点1
  #5 : 直線上の点2
  #6 : 交点1を受け取る制御綴
  #7 : 交点2を受け取る制御綴
\end{verbatim}
\end{boxnote}
\bigskip

使用例です。2点A($-1$, 0), B(2, 1)を通る直線と
楕円$\bunsuu{x^2}{4}+y^2=1$との交点P, Qを求めます。

\begin{showEx}{\cmd{EandL}}
\begin{zahyou}[ul=10mm]%
    (-2.5,2.5)(-1.5,1.5)
  \tenretu{A(-1,0)nw;B(2,1)nw}
  \kuromaru{\A;\B}
  \Put\B[syaei=xy,xpos={[ne]},
    ypos={[ne]}]{}
  \Daen\O{2}{1}\Tyokusen\A\B{}{}
  \EandL\O{2}{1}\A\B\P\Q
  \Put\P[s]{P}\Put\Q[s]{Q}
  \kuromaru{\P;\Q}
\end{zahyou}
\end{showEx}
%\clearpage

\subsection{\texorpdfstring{\cmd{Eandl}}{Eandl}}
直線を1点と方向ベクトルで与える場合です。

楕円$\bunsuu{(x-2)^2}{4}+(y-1)^2=1$と,
点A(0, 1)を通り方向ベクトルが(2, 1)の直線の交点を求めます。
この場合,点Aは楕円上にありますから,交点の一方PとAは一致します。

\begin{showEx}{\cmd{Eandl}}
\begin{zahyou}[ul=10mm]%
    (-.5,4.5)(-.5,2.5)
  \tenretu{A(0,1)nw;C(2,1)se}
  \Put\C[syaei=xy,ypos={[ne]}]{}
  \def\m{(2,1)}%
  \Kuromaru{\A}
  \Daen\C{2}{1}
  \mTyokusen\A\m{}{}
  \Eandl\C{2}{1}\A\m\P\Q
  \Put\P[se]{P}\Put\Q[s]{Q}
  \kuromaru{\P;\Q}
\end{zahyou}
\end{showEx}

\subsection{\texorpdfstring{\cmd{Eandk}}{Eandk}}
直線を1点と方向角(六十分法)で与える場合です。

楕円$\bunsuu{(x-2)^2}{4}+(y-1)^2=1$と,
点A(1, 2)を通り方向角が$-45\Deg$である直線との交点を求めます。


\begin{showEx}{\cmd{Eandk}}
\begin{zahyou}[ul=10mm]%
    (-.5,4.5)(-.5,2.5)
  \tenretu{A(1,2)ne;C(2,1)ne}
  \Put\A[syaei=xy]{}
  \Put\C[syaei=xy]{}
  \def\m{(2,1)}%
  \Kuromaru{\A}
  \Daen\C{2}{1}
  \kTyokusen\A{-45}{}{}
  \Eandk\C{2}{1}\A{-45}\P\Q
  \Put\P[s]{P}\Put\Q[n]{Q}
  \kuromaru{\P;\Q}
\end{zahyou}
\end{showEx}

\subsection{楕円の接線}
\subsubsection{\texorpdfstring{\cmd{DaennoSessen}}{DaennnoSessen}}
楕円の周上の点における接線の方向ベクトルを求めるコマンドです。

\begin{boxnote}
\begin{verbatim}
\DaennoSessen#1#2#3#4#5{%
 #1 : 楕円の中心
 #2 : x軸方向の半径
 #3 : y軸方向の半径
 #4 : 接点
 #5 : 接線の方向ベクトルを受け取る制御綴
\end{verbatim}
\end{boxnote}
\bigskip

楕円$\bunsuu{(x-1)^2}{9}+\bunsuu{y^2}{4}=1$上の点A
$\left(\bunsuu{5}{2},~\sqrt{3} \right)$
における接線を引きます。

\begin{showEx}(.54,.4){\cmd{EandL}}
\begin{zahyou}[ul=5mm]%
      (-2.5,5)(-2.5,4)
  \tenretu{A(2.5,1.732)ne;[]C(1,0)}
  \Kuromaru\A
  \Daen\C{3}{2}
  \DaennoSessen\C{3}{2}\A\uvec
  \mTyokusen\A\uvec{}{}
\end{zahyou}
\end{showEx}
%\clearpage

\subsubsection{\texorpdfstring{\cmd{DaenniSessen}}{DaenniSessen}}
つぎは,楕円の外部の点から楕円に引いた接線の接点を求めます。

\begin{boxnote}
\begin{verbatim}
\DaenniSessen#1#2#3#4#5#6{%
 #1 : 楕円の中心
 #2 : x軸方向の半径
 #3 : y軸方向の半径
 #4 : 楕円の外部の点
 #5 : 接点1を受け取る制御綴
 #6 : 接点2を受け取る制御綴
\end{verbatim}
\end{boxnote}
\bigskip

楕円$\bunsuu{(x-3)^2}{9}+\bunsuu{(y+2)^2}{4}=1$に
点A(6, 1)から接線を引きます。

\begin{showEx}{\cmd{DaennnoSessen}}
\begin{zahyou}[ul=5mm]%
    (-1,7)(-5,2)
  \tenretu{A(6,1)nw;C(3,-2)se}
  \Put\C[syaei=xy,xpos={[se]}]{}
  \Put\A[syaei=xy,xpos={[ne]}]{}
  \Kuromaru{\A}
  \Daen\C{3}{2}
  \DaenniSessen\C{3}{2}\A\P\Q
  \kuromaru{\P;\Q;\C}
  \Tyokusen\A\P{}{}
  \Tyokusen\A\Q{}{}
\end{zahyou}
\end{showEx}
%\clearpage

\subsubsection{\texorpdfstring{\cmd{Earg}}{Earg}}
楕円の媒介変数表示
\begin{align*}
  x&=x_0+a\cos\theta\\
  y&=y_0+b\sin\theta
\end{align*}
において,周上の点($x$, $y$)を指定して$\theta$を求めるマクロです。

\begin{boxnote}
\begin{verbatim}
\Earg#1#2#3#4#5{%
 #1 : 楕円の中心
 #2 : x軸方向の半径
 #3 : y軸方向の半径
 #4 : 周上の点
 #5 : 媒介変数の値(六十分法)
\end{verbatim}
\end{boxnote}
\bigskip

\begin{showEx}(.54,.4){\cmd{Earg}}
\begin{zahyou*}[ul=5mm](-3,5)(-4,6)
  \tenretu{[]C(1,-1);A(2,5)n}
  \DaenniSessen\C{3}{2}\A\P\Q
  \Put\P[nw]{P}\Put\Q[ne]{Q}
  \kuromaru{\P;\Q}
  \Earg\C{3}{2}\P\argP
  \Earg\C{3}{2}\Q\argQ
  \Put\C{\Daenko<hasen=[.5][.5]>%
    {3}{2}{\argQ}{\argP}}
  \Add\argQ{360}\argQQ
  \Put\C{\Daenko{3}{2}{\argP}{\argQQ}}
  \Drawline{\P\A\Q}
\end{zahyou*}
\end{showEx}


%\section{塗りつぶし(1)}
\section{$BEI$j$D$V$7(B(1)}
\subsection{$BB?3Q7AFbIt$NEI$j$D$V$7(B}
$BB?3Q7A$NFbIt$rEI$j$D$V$9%3%^%s%I(B \cmd{Nuritubusi} $B$G$9!%(B
\cindex{Nuritubusi}

\showexample[$BB?3Q7A$NEI$j$D$V$7(B](0.51)(0.44){example/shade01}

\cmd{Nuritubusi} $B$N=q<0$O(B

\begin{boxnote}
\begin{verbatim}
\Nuritubusi[#1]#2
  #1 : $BEI$kG;$5(B 
      ( 0 $B$H(B 1 $B$N4V$N?t!%(B 0 $B$O??$CGr!$(B1$B$O??$C9u!%(B
        $B%G%U%)%k%H$O(B 0.5)
  #2 : $BB?3Q7A$NE@Ns!JJD$8$F$$$J$1$l$P$J$j$^$;$s!%(B)
\end{verbatim}
\end{boxnote}

\subsection{$B1_FbIt$NEI$j$D$V$7(B}
$B1_$NFbIt$rEI$j$D$V$9$N$,(B \cmd{En*} $B%3%^%s%I$G$9!%(B\cindex{En*}

\showexample[$B1_$NEI$j$D$V$7(B](0.5)(0.45){example/shade03}

\begin{boxnote}
\begin{verbatim}
\En*[#1]#2#3
    #1 : $BEI$kG;$5(B 
      ( 0 $B$H(B 1 $B$N4V$N?t!%(B 0 $B$O??$CGr!$(B1$B$O??$C9u!%(B
        $B%G%U%)%k%H$O(B 0.5)
        #2 : $BCf?4$N:BI8(B
        #3 : $BH>7B(B
\end{verbatim}
\end{boxnote}

$B%*%W%7%g%s0z?t(B\texttt{[..]}$B$GG;$5$r;XDj$9$k$3$H$,$G$-$^$9!%(B
$B%G%U%)%k%H$O(B0.5$B$G$9!%FC$K(B0$B$r;XDj$9$k$HGrH4$-$H$J$j$^$9!%(B

\showexample[$BGrH4$-(B](0.5)(0.45){example/shade04}

\subsection{$B@p7A$NEI$j$D$V$7(B}
\showexample[$B@p7A$NEI$j$D$V$7(B](0.5)(0.45){example/shade05}
\cindex{ougigata*}

\begin{boxnote}
\begin{verbatim}
\ougigata*[#1]<#2>#3#4#5
  #1 : $BEI$j$D$V$7$NG;$5(B
  #2 : $B6-3&@~IA2h%*%W%7%g%s(B
        <border=1> $B$G6-3&@~$rIA2h$9$k!#(B
  #3 $B!A(B #5 : \ougigata $B$N(B #1$B!A(B#3 $B$HF1$8(B
  $B!JCf?4$O(B \put (\Put) $B$G;XDj$9$k!%!K(B
\end{verbatim}
\end{boxnote}

\subsection{$B5]7A$NEI$j$D$V$7(B}
\showexample[$B5]7A$NEI$j$D$V$7(B](0.5)(0.45){example/shade06}
\cindex{yumigata*}

\begin{boxnote}
\begin{verbatim}
\yumigata*[#1]<#2>#3#4#5
  #1 : $BEI$j$D$V$7$NG;$5(B
  #2 : $B6-3&@~IA2h%*%W%7%g%s(B
        <border=1> $B$G6-3&@~$rIA2h$9$k!#(B
  #3 $B!A(B #5 : \yumigata $B$N(B #1$B!A(B#3 $B$HF1$8(B
  $B!JCf?4$O(B \put (\Put) $B$G;XDj$9$k!%!K(B
\end{verbatim}
\end{boxnote}


\subsection{$BBJ1_$NEI$j$D$V$7(B}
\showexample[$BBJ1_$NEI$j$D$V$7(B](0.5)(0.45){example/shade07}
\cindex{Daen*}

\begin{boxnote}
\begin{verbatim}
\Daen*[#1]#2#3#4
    #1 : $BEI$kG;$5(B 
      ( 0 $B$H(B 1 $B$N4V$N?t!%(B 0 $B$O??$CGr!$(B1$B$O??$C9u!%(B
        $B%G%U%)%k%H$O(B 0.5)
        #2 : $BCf?4$N:BI8(B
        #3 : $B2#<4J}8~$NH>7B(B
        #4 : $B=D<4J}8~$NH>7B(B
\end{verbatim}
\end{boxnote}


\subsection{$B5]7A!JBJ1_8L!K$NEI$j$D$V$7(B}
\showexample[$B5]7A!JBJ1_8L!K$NEI$j$D$V$7(B](0.5)(0.45){example/shade08}
\cindex{Daenko*}

\begin{boxnote}
\begin{verbatim}
\Daenko*[#1]#2#3#4#5
    #1 : $BEI$kG;$5(B 
         ( 0 $B$H(B 1 $B$N4V$N?t!%(B 0 $B$O??$CGr!$(B1$B$O??$C9u!%(B
         $B%G%U%)%k%H$O(B 0.5)
    #2 : $B2#<4J}8~$NH>7B(B
    #3 : $B=D<4J}8~$NH>7B(B
    #4 : $B;O$a3Q(B
    #5 : $B=*$j3Q(B
        $B!JCf?4$O(B \put (\Put) $B$G;XDj$9$k!%!K(B
\end{verbatim}
\end{boxnote}

\subsection{$B%+%i!<;XDj(B}
\edef\saveUL{\the\unitlength}%
\unitlength=1pt\relax
\verb+\En*+, \verb+\Nuritubusi+$B$K$*$$$F!$?'$r$D$1$k$H$-$O>/!9Lq2p$J$3$H$,$"$j$^$9!#(B
\subsubsection{\cmd{color}$B$K$h$kEI$j?';XDj(B}
$B<!$NNc$G$O!$(B
\begin{showEx}(.64,.3){\cmd{color}$B$K$h$kEI$j?';XDj(B}
\begin{picture}(50,50)
  \color{red}\put(25,25){\circle*{40}}
\end{picture}
\end{showEx}

$B$5$F!$$"$J$?$N4D6-$G$O1_$O2??'$GEI$j$D$V$5$l$F$$$^$9$+!#(B
$B<B$O!$$=$l$O(B \verb+dvi-ware+$B$K0MB8$9$k$N$G$9!#(B\textsf{emath}$B$G$O(B
\begin{jquote}
\begin{verbatim}
 .tex ----> .dvi by platex
 .dvi ----> .ps  by dvipsk
(.ps  ----> .pdf by Distiller)
\end{verbatim}
\end{jquote}
$B$rI8=`$H$7$F$$$^$9$,!$$3$NJ}<0$G:n@.$5$l$k(B .ps(, .pdf) $B$O??$C9u$KEI$j$D$V$5$l$F$$$^$9!#(B
$B@V$O$+$1$i$b8+$($^$;$s!#(B

\begin{description}
  \item[$BCm(B] $B$3$N$"$?$j!$$5$i$KJ,@O$r$7$F$*$-$^$9!#(B
  
    \verb+\color{red}\circle*{..}+$B$OJ,@O$9$k$H(B
    \begin{jquote}
\begin{verbatim}
\color{red}\special{bk}\circle{..}
\end{verbatim}
    \end{jquote}
    $B$H$J$j$^$9!#$3$N=g=x$r0lItF~$lBX$($F(B
    \begin{jquote}
\begin{verbatim}
\special{bk}\color{red}\circle{..}
\end{verbatim}
    \end{jquote}
    $B$H$9$k$H!$(B\texttt{dvipsk}$B$G:n$i$l$k(B .ps $B%U%!%$%k$b1_$NFbIt$,@V$/$J$j$^$9!#(B

\begin{showEx}(.7,.24){\cmd{color}$B%3%^%s%IH/9T$N%?%$%_%s%0(B}
\begin{picture}(50,50)
  \put(25,25){\color{red}\special{bk}\circle{40}}
\end{picture}
\begin{picture}(50,50)
  \put(25,25){\special{bk}\color{red}\circle{40}}
\end{picture}
\end{showEx}

2$B$D$N1_$,IA2h$5$l$^$9$,!$(B\texttt{dvipsk}$B$GJQ49$7$?(B .ps $B%U%!%$%k$G$O!$1_$NFbIt$O(B
\begin{jquote}
$B>e$O9u(B\\
$B2<$O@V(B
\end{jquote}
$B$GEI$j$D$V$5$l$^$9!#(B

\end{description}
\subsubsection{\cmd{En*}$B$N(B\texttt{[nuriiro=..]}$B%*%W%7%g%s(B}
$B$=$3$G!$(B\textsf{emath}$B$G$O!$(B\verb+En*+$B$J$I$K(B\verb+[nuriiro=...]+$B%*%W%7%g%s$r$D$1$k$3$H$G(B
$BFbIt$K?'$r$D$1$k$3$H$K$7$F$$$^$9!#(B


\begin{showEx}(.64,.3){\cmd{En*}$B$N(B\texttt{[nuriiro=..]}$B%*%W%7%g%s(B}
\begin{picture}(50,50)
  \En*[nuriiro=red]{(25,25)}{20}
\end{picture}
\end{showEx}

\begin{description}
  \item[$BCm(B] $B8=;~E@$G$O!$(B\verb+[nuriiro=..]+$B%*%W%7%g%s$G$O$J$/!$(B\verb+\color+$B%3%^%s%I$rMQ$$$F$b!$(B
    $BF1MM$N7k2L$rF@$k$3$H$,$G$-$^$9!#(B

\begin{showEx}(.64,.3){$B;CDj;EMM(B}
\begin{picture}(50,50)
  \color{red}\En*{(25,25)}{20}
\end{picture}
\end{showEx}

$B$7$+$7!$>-MhE*$K$O$3$NJ}<0$OGK;:$9$kM=46$,$"$j$^$9$N$G!$(B\verb+[nuriiro=..]+$B%*%W%7%g%s$N;HMQ$r(B
$B?d>)$7$^$9!#(B
\end{description}

\subsubsection{$B1_<~$N?'$O!)(B}
$B1_$NFbIt$O?'IU$1$,$G$-$^$7$?$,!$1_<~$O9u$N$^$^$G$9$M$'!A!#(B
$B$3$l$O(B\verb+tpic specials+$B$N;EMM$+$bCN$l$^$;$s!#$$$d!$$=$b$=$b(B\verb+tpic specials+$B$O(B
$B%+%i!<BP1~$7$F$$$k$s$G$7$g$&$+!#@u3X$K$7$F$o$+$j$^$;$s!#(B\verb+(^^$B!6(B+

$B$=$3$G2?$H$+$N>eEI$j$r$9$k$3$H$H$7$^$9!#(B

\begin{showEx}(.64,.3){\cmd{En*}$B$N(B\texttt{[border=..]}$B%*%W%7%g%s(B}
\begin{picture}(50,50)
  \En*[nuriiro=red,border=red]{(25,25)}{20}
\end{picture}
\end{showEx}

\verb+[border=red]+$B%*%W%7%g%s$G1_<~$r@V$G>eEI$j$7$^$7$?!#(B
$B1_<~$b@V$K$J$C$?$+$K8+$($^$9$,!$$h$/8+$k$H85$N9u$,0lIt$O$_=P$7$F$$$^$9$M!#(B
$B@V$N>eEI$j$r$b$&>/$7@~$rB@$/$9$l$PNI$$$+$b$7$l$^$;$s$,!%!%!%!%!%!%!%(B

$B$7$+$7!$O+B?$/$7$F2?$H$d$i$H$$$&5$$b$7$^$9!#(B
$B%+%i!<$O$d$O$j(B\verb+PostScript+$B$G07$&$Y$-$b$N$G$7$g$&$+!#(B

\begin{description}
  \item[$BCm(B] \verb+tpic specials+$B$G$b(B
    \begin{jquote}
\begin{verbatim}
$B1_!&BJ1_(B
$B!!$H(B
$BB?3Q7A(B
\end{verbatim}
    \end{jquote}
    $B$G$O!$7k2L$,0[$J$j!$A0<T$O6-3&$,9u$GIA2h$5$l$^$9$,!$(B
    $B8e<T$O6-3&$OIA2h$5$l$^$;$s!#(B
    $B$I$&$b!$(B\verb+tpic specials+$B$G%+%i!<$r07$&$N$O5$$,?J$_$^$;$s$M!#(B

\begin{showEx}(.64,.3){\cmd{Nuritubusi}$B$N>l9g(B}
\begin{picture}(50,50)
  \Nuritubusi[nuriiro=red]{%
    (0,0)(50,0)(50,50)(10,40)}
\end{picture}
\end{showEx}
    
\end{description}

\subsubsection{\texttt{PostScript}$B$G$O(B}
$B0J>e$N9M;!$+$i!$%+%i!<EI$j$D$V$7$O(B\texttt{PostScript --- pszahyou$B4D6-(B}$B$r(B
$B;HMQ$9$k$N$,$h$5$=$&$G$9!#(B
\unitlength=\saveUL\relax

%\begin{showEx}(.64,.3){\textsf{pszahyou}$B4D6-(B}
%\begin{pszahyou*}(0,50)(0,50)
%  \En*[nuriiro=red]{(25,25)}{20}
%\end{pszahyou*}
%\end{showEx}


%\section{斜線塗り(1)}
\section{$B<P@~EI$j(B(1)}
$BA0@a$N%3%^%s%I$NKvHx$K99$K(B \texttt{*} $B$rIU2C$9$k$H!$<P@~EI$j$H$J$j$^$9!%(B

\subsection{$BB?3Q7A(B}
$B$^$:$O!$B?3Q7A$NFbIt$K<P@~$r0z$$$F$_$^$9!%(B

\showexample[$BB?3Q7A$NFbIt$K<P@~(B](0.5)(0.45){example/hatch01}

\cmd{Nuritubusi*} $B$N=q<0$O(B\cindex{Nuritubusi*}

\begin{boxnote}
\begin{verbatim}
\Nuritubusi*[#1]<#2>#3
  #1 : $B<P@~$NJ}8~3Q(B
      ( -90 $B$H(B 90 $B$N4V$N?t!JC10L$OEY!K!%%G%U%)%k%H$O(B 45)
  #2 : $B<P@~$N4V3V(B $B!J%G%U%)%k%H$O(B 0.125 $B!K(B
  #3 : $BB?3Q7A$NE@Ns!JJD$8$F$$$J$1$l$P$J$j$^$;$s!%(B)
\end{verbatim}
\end{boxnote}\medskip

\subsection{$B1_(B}
$B1_$NFbIt$r<P@~EI$j$9$kNc$G$9!%(B
\verb+[-45]+ $B%*%W%7%g%s$r$D$1$F!$<P@~$NJ}8~3Q$rJQ99$7$F$_$^$7$?!%(B

\showexample[$B1_$NFbIt$K<P@~(B](0.45)(0.45){example/hatch02}

\cmd{En**} $B$N=q<0$G$9!%(B\cindex{En**}

\begin{boxnote}
\begin{verbatim}
\En**[#1]<#2>#3#4
    #1 : $B<P@~$NJ}8~3Q(B
      ( -90 $B$H(B 90 $B$N4V$N?t!JC10L$OEY!K!%%G%U%)%k%H$O(B 45)
    #2 : $B<P@~$N4V3V(B $B!J%G%U%)%k%H$O(B 0.125 $B!K(B
    #3 : $BCf?4$N:BI8(B
    #4 : $BH>7B(B
\end{verbatim}
\end{boxnote}

\subsection{$B@p7A$J$I(B}
\cmd{Daen**}, \cmd{yumigata**}, \cmd{ougigata**}, \cmd{Daenko**},
\cmd{Nuritubusi*} $B$bF1MM$G$9!%(B
\cindex{Daen**}\cindex{yumigata**}\cindex{ougigata**}
\cindex{Daenko**}\cindex{Nuritubusi*}

$BBeI=E*$K(B\verb/\yumigata**/ $B$N=q<0$r5-$7$^$9!#(B

\begin{boxnote}
\begin{verbatim}
\yumigata**[#1]<#2>#3#4#5
  #1 : $B<P@~EI$j$N79<P3Q!J%G%U%)%k%H$O(B45$B!K(B
  #2 : $B%*%W%7%g%s(B
        <border=1> $B$G6-3&@~$rIA2h$9$k!#(B
        <syanurisiteiten=xx> $B$G<P@~72$,;XDjE@(Bxx$B$rDL$k$h$&$K$9$k!#(B
        <syanurikankaku=xx> $B<P@~$N4V3V$r;XDj$9$kL5L>?t!J%G%U%)%k%H$O(B 0.125$B!K(B
          #2 $B$KC1$KL5L>?t$r5-=R$9$l$P!$<P@~$N4V3V$rJQ99$9$k;X<($H2r<a$5$l$k!#(B
  #3 $B!A(B #5 : \yumigata $B$N(B #1$B!A(B#3 $B$HF1$8(B

$B!JCm!K;O$a3Q$H=*$j3Q$NM?$(J}$N=g=x$K$D$$$F(B
    2$B$D$N3Q$r;XDj$9$k$H!$1_$O(B2$B$D$N5]7A$KJ,3d$5$l$^$9!#(B
    $B$=$N$I$A$i$,IA2h$5$l$k$N$+$H$$$&OC$G$9!#(B
    $B;O$a3Q$G;XDj$5$l$?F07B$r@5$N8~$-$K!J;~7W$N?K$HH?BP8~$-!K(B
    $B2sE>$7$F=*$j3Q$G;XDj$7$?F07B$H=E$J$k$^$G$rIA2h$7$^$9!#(B
\end{verbatim}
\end{boxnote}

\showexample[$B5]7A$N<P@~EI$j(B](0.5)(0.45){example/hatch04}

\showexample[$B@p7A$N<P@~EI$j(B](0.5)(0.45){example/hatch03}

$B$?$@$7!$@p7A$NCf?43Q$O(B180$BEY0J2<$G$J$1$l$P$J$j$^$;$s!%(B
$BM%3Q$N@p7A$N<P@~EI$j$O!$1_$r<P@~EI$j$7$?8e!$M>J,$N@p7A$rGrEI$j$7$^$9!%(B

\showexample[$B@p7A$N<P@~EI$j!JM%3Q$N>l9g!K(B](0.5)(0.45){example/hatch05}

\showexample[$B5]7A!JBJ1_8L!K$N<P@~EI$j(B](0.5)(0.45){example/hatch06}

\subsection{$B3J;R%;%k$N$L$j$D$V$7(B}
$B3J;R$rIA2h$9$k(B \cmd{kousi} $B$K(B
\begin{jquote}
\begin{verbatim}
$B%;%k$rEI$k(B
$B%;%k$N:BI8$r;XDj$7$F$N(B \put, 
$B9u4]G[CV(B
\end{verbatim}
\end{jquote}
$B$J$I$N$r<B8=$9$k(B\texttt{<nuri=..>}$B%*%W%7%g%s$N@bL@$G$9!#(B
\verb+<nuri=..>+$B$N1&JU$K!$EI$j$D$V$7$r9T$&%;%k$N:82<%3!<%J!<$N(B
$B3J;R:BI8$rM?$($^$9!#(B

\begin{showEx}(.64,.3){$BEI$j$D$V$7(B}
\begin{zahyou*}[ul=4mm](0,8)(0,8)
\kousi<nuri={(1,2)}>(2,1.5){4}{5}
\end{zahyou*}
\end{showEx}

\verb+<nuri=..>+$B$N1&JU$O!$%3%s%^$r4^$_$^$9$+$i!$(B\verb+{...}+$B$G(B
$B$/$/$C$F$*$/I,MW$,$"$j$^$9!#G;EY;XDj$O(B\verb+[..]+$B%*%W%7%g%s$H$$$&$N$O(B
\cmd{Nuritubusi}$B$HF1MM$G$9!#(B

\begin{showEx}(.64,.3){$BG;EY;XDj(B}
\begin{zahyou*}[ul=4mm](0,8)(0,8)
\kousi<nuri={[1](3,1)}>(2,1.5){4}{5}
\end{zahyou*}
\end{showEx}

\verb+<nuri=..(x,y)>+$B$K$*$1$k(B\verb+..+$B$NItJ,$O(B\cmd{Nuritubusi}$B$N(B
$B%*%W%7%g%s$H$7$F0z$-EO$5$l$^$9!#@V$GEI$j$D$V$7$F$_$^$7$g$&!#(B

\begin{showEx}(.64,.3){$B%+%i!<EI$j(B}
\begin{zahyou*}[ul=4mm](0,8)(0,8)
\kousi<nuri={[nuriiro=red](3,0)}>(2,1.5){4}{5}
\end{zahyou*}
\end{showEx}

\noindent
$B$H$$$&$3$H$J$l$P!$<P@~EI$j$O(B

\begin{showEx}(.64,.3){$B<P@~EI$j(B}
\begin{zahyou*}[ul=4mm](0,8)(0,8)
\kousi<nuri={*(1,4)}>(2,1.5){4}{5}
\end{zahyou*}
\end{showEx}

$B<P@~$N3QEY!$4V3V;XDj$b(B

\begin{showEx}(.64,.3){$B<P@~3QEY!$4V3V(B}
\begin{zahyou*}[ul=4mm](0,8)(0,8)
\kousi<nuri={*[-45]<.5>(0,2)}>(2,1.5){4}{5}
\end{zahyou*}
\end{showEx}

\cmd{Nuritubusi}$B$N%*%W%7%g%s$HF1MM$G$"$k$3$H$,G<F@$$$?$@$1$?$G$7$g$&$+!#(B

$BJ#?t$N%;%k$KBP$7$FEI$j$r;XDj$9$k$K$O!$(B\verb+<nuri=..>+$B$N1&JU$K!$(B
$B>e5-$N;XDj$r(B`;'$B$G6h@Z$C$FJB$Y$l$P<B8=$G$-$^$9!#(B

\begin{showEx}(.64,.3){$BJ#?t;XDj(B}
\begin{zahyou*}[ul=4mm](0,8)(0,8)
\kousi<nuri={%
  (1,2);%
  [1](3,1);%
  [nuriiro=red](3,0);%
  *(1,4);%
  *[-45]<.5>(0,2)%
}>(2,1.5){4}{5}
\end{zahyou*}
\end{showEx}


\subsection{$B6-3&@~$H<P@~$N4V$r6u$1$k(B}
$BNN0h$N6-3&$r4^$^$J$$$3$H$r<($9$N$K!$(B
$B<P@~$H6-3&$r$/$C$D$1$:!$>/$76u$1$kN.57$,$"$j$^$9!#(B

$B$3$l$r<B8=$9$k$K$O$$$/$D$+$NJ}K!$,$"$j$^$9$,!$(B
$BEDCf(B $BE0(B $B$5$s$,(B\verb+BBS #354+$B$KEj9F$5$l$?J}K!(B
\begin{jquote}\relax
(1) $B$R$H$^$:EI$j$D$V$7$r9T$&(B\\
(2) $B6-3&@~$rB@$a$NGr$$@~$G>e=q$-(B\\
(3) $B2~$a$F6-3&@~$r0z$/(B
\end{jquote}
$B$OM-NO$J<jK!$G$9!#(B

$B$=$N%"%$%G%"$rD:$$$F%^%/%m2=$7$F$_$^$7$?!#(B
\textsf{emathPh.sty v 1.07}$B$G(B
\begin{jquote}
\begin{verbatim}
syasentanmatu=...$B!JC10LIU$ND9$5!K(B
\end{verbatim}
\end{jquote}
$B%*%W%7%g%s$K$h$jGrEI$j$NI}$r;XDj$9$k$3$H$K$7$F$_$^$7$?!#(B
$B6-3&@~$rCf?4$H$7$F!$$=$N:81&$K;XDj$7$?I}$rM-$9$kGr$$6J@~$rIA2h$7$^$9!#(B

$B!J<P@~$NC<E@$H6-3&@~$N5wN%$,;XDj$7$?D9$5$H$J$j!$(B
$BGrEI$j$N6J@~$NB@$5$O;XDj$7$?I}$N(B2$BG\$H$J$j$^$9!#!K(B

\begin{showEx}(.6,.34){\texttt{syasentanmatu=..}$B%*%W%7%g%s(B}
\begin{zahyou*}[ul=6mm](0,5)(0,4)
  \tenretu*{A(0,0);B(5,0);C(4,4);D(1,3)}
  \Nuritubusi*<syasentanmatu=1mm>%
    {\A\B\C\D\A}
  \Drawline{\A\B\C\D\A}
\end{zahyou*}
\end{showEx}

\subsection{2$B1_$N6&DLItJ,$K<P@~(B}
2$B1_$N6&DLItJ,$r<P@~EI$j$9$k$N$O!$7k9=LLE]$G$9!#(B

$B$^$:$O<P@~EI$j$G$O$J$/!$%Y%?EI$j$N>l9g$r8!F$$7$^$9!#(B
$B$3$l$O!$6&DL89$G(B2$BJ,3d$7$?$=$l$>$l$N5]7A$rEI$j$D$V$7$^$9!#(B

\begin{showEx}(.6,.34){2$B1_$N6&DLItJ,$N%Y%?EI$j(B}
\unitlength12mm\drawaxisfalse
\begin{zahyou}(-1.5,1.5)(-1.5,2.5)
  \tenretu*{A(0,0);B(0,1)}
  \edef\hankeiA{1}
  \edef\hankeiB{1}
  \En\A\hankeiA
  \En\B\hankeiB
  \CandC\A\hankeiA\B\hankeiB\P\Q
  \Put\A{\yumigata*\hankeiA{%
    hazimeten=\Q}{owariten=\P}}
  \Put\B{\yumigata*\hankeiB{%
    hazimeten=\P}{owariten=\Q}}
\end{zahyou}
\end{showEx}

$B$3$l$,<P@~EI$j$H$J$k$H(B

\begin{showEx}(.6,.34){2$B1_$N6&DLItJ,$N<P@~EI$j(B(1)}
\unitlength12mm\drawaxisfalse
\begin{zahyou}(-1.5,1.5)(-1.5,2.5)
  \tenretu*{A(0,0);B(0,1)}
  \edef\hankeiA{1}
  \edef\hankeiB{1}
  \En\A\hankeiA
  \En\B\hankeiB
  \CandC\A\hankeiA\B\hankeiB\P\Q
  \Put\A{\yumigata**\hankeiA{%
    hazimeten=\Q}{owariten=\P}}
  \Put\B{\yumigata**\hankeiB{%
    hazimeten=\P}{owariten=\Q}}
\end{zahyou}
\end{showEx}
2$B$D$N5]7A$N<P@~$,$D$J$,$j$^$;$s!#(B

$B$=$3$G!$<P@~EI$j$K(B
\begin{verbatim}
  syanurisiteiten=
\end{verbatim}
$B$H$$$&=q<0$G!$(B
$B<P@~72$,DL2a$9$Y$-(B1$BE@$r;XDj$9$k$?$a$N%*%W%7%g%s$rIU2C$7$^$7$?!#(B
$B$=$N8z2L$O(B

\begin{showEx}(.6,.34){2$B1_$N6&DLItJ,$N<P@~EI$j(B(1)}
\unitlength12mm\drawaxisfalse
\begin{zahyou}(-1.5,1.5)(-1.5,2.5)
  \tenretu*{A(0,0);B(0,1)}
  \edef\hankeiA{1}
  \edef\hankeiB{1}
  \En\A\hankeiA
  \En\B\hankeiB
  \CandC\A\hankeiA\B\hankeiB\P\Q
  \Put\A{\yumigata**<%
    syanurisiteiten=\Q>\hankeiA{%
    hazimeten=\Q}{owariten=\P}}
  \Put\B{\yumigata**<%
    syanurisiteiten=\Q>\hankeiB{%
    hazimeten=\P}{owariten=\Q}}
\end{zahyou}
\end{showEx}

$B0l8+$D$J$,$C$?$h$&$K8+$($^$9$,!$$h$/L\$r6E$i$9$H$A$g$C$H$:$l$F$$$^$9!#(B
$B$3$l$,5$$K$J$l$P!$<P@~EI$j$9$kNN0h$N6-3&@~$rB?3Q7A6a;w$7$F!$(B
$BB?3Q7A$r<P@~EI$j$9$k%3%^%s%I(B \cmd{Nuritubusi} $B$r;H$&$3$H$K$J$j$^$7$g$&!#(B
$B1_8L$r@^$l@~6a;w$9$k$K$O!$(B
\begin{verbatim}
  $n$$B<!4X?t$N%0%i%U$r@^$l@~6a;w$9$k(B \KinziOresen
  $B0lHL$N4X?t$N%0%i%U$r@^$l@~6a;w$9$k(B \yKinziOresen
  $BG^2pJQ?tI=<($N6J@~$r@^$l@~6a;w$9$k(B \bKinziOresen
\end{verbatim}
$B$J$I$rMQ$$$l$P2DG=$G$9$,!$$3$N:]!$1_8L$r@^$l@~6a;w$9$k%3%^%s%I(B
$B$r:n$C$F$_$^$7$?!#L>IU$1$F(B \cmd{KinziEnko}
$B=q<0$O$"$H$G5-=R$9$k$3$H$K$7$F!$$H$j$"$($:;HMQ8z2L$r8+$F$_$^$7$g$&!#(B

\begin{showEx}(.65,.29){2$B1_$N6&DLItJ,$N<P@~EI$j(B(1)}
\unitlength12mm\drawaxisfalse
\begin{zahyou}(-1.1,1.1)(-1.5,2.5)
  \tenretu*{A(0,0);B(0,1)}
  \edef\hankeiA{1}\edef\hankeiB{1}
  \En\A\hankeiA\En\B\hankeiB
  \CandC\A\hankeiA\B\hankeiB\P\Q
  \Put\P[w]{P}\Put\Q[e]{Q}
  \KinziEnko\A\hankeiA{hazimeten=\Q}{%
    owariten=\P}\oresenA
  \KinziEnko\B\hankeiB{hazimeten=\P}{%
    owariten=\Q}\oresenB
  \edef\oresen{\Q\oresenA\P\oresenB\Q}
  \Nuritubusi*\oresen
\end{zahyou}
\end{showEx}
$B$-$l$$$K<P@~EI$j$,$G$-$?$h$&$G$9!#(B

\cmd{KinziEnko} $B$N=q<0$G$9!#(B

\begin{boxnote}
\begin{verbatim}
$B1_8L$r6a;w$9$k@^$l@~(B
\KinziEnko<#1>#2#3#4#5#6
  #1 : $B9o$_CM(B
  #2 : $BCf?4(B
  #3 : $BH>7B$rD>@\M?$($k$+(B
       tuukaten=xx $B$H$7$F!$1_8L$N<~>e$N0lE@$rM?$($k(B
  #4 : $B;O$a3Q$rD>@\M?$($k$+(B
       hazimeten=xx $B$H$7$F!$Cf?4$r;OE@!$(Bxx $B$r=*E@$H$9$k%Y%/%H%k$N(B
       $BJ}8~3Q$r(B $B;O$a3Q$H$9$k$h$&$K;XDj$9$k!#(B
  #5 : $B=*$j3Q$rD>@\M?$($k$+(B
       owariten=xx $B$H$7$F!$Cf?4$r;OE@!$(Bxx $B$r=*E@$H$9$k%Y%/%H%k$N(B
       $BJ}8~3Q$r(B $B=*$j3Q$H$9$k$h$&$K;XDj$9$k!#(B
  #6 : $B6a;w@^$l@~$r<u$1<h$k@)8fDV(B
\end{verbatim}
\end{boxnote}

\subsection{$BGK@~$K$h$k<P@~EI$j(B}\label{S-hasen}
$B<P@~$rGK@~$G0z$/$K$O(B
\begin{jquote}
\verb/\def\sensyu{\hasen}%/
\end{jquote}
$B$J$I$H$7$^$9!%(B

\showexample[$BGK@~$K$h$k<P@~EI$j(B](0.5)(0.45){example/hatch08}

$B$J$*!$<P@~$r(B \verb/\color{lightgray}/ $B$J$I$H$9$k$N$bM-NO$J<jK!$G$9!%(B
$B$?$@$7!$(Bgray $B$NG;EY$O%W%j%s%?4D6-$K$h$C$FD4@0$9$kI,MW$,$"$j$^$9!%(B

\showexample[$B%0%l!<%9%1!<%k$K$h$k<P@~EI$j(B](0.5)(0.45){example/hatch09}

\subsection{$B<P@~EI$j$N@)Ls>r9`(B}
%\begin{jquote}
\begin{description}
\item[\textgt{$B<P@~EI$j$N@)Ls>r9`(B}]
        $B<P@~$NJ}8~3Q$r(B$\theta$$B$H$7$F!$D>@~72(B $y\cos\theta-x\sin\theta=k$ 
        $B$,NN0h$K$h$C$F$-$j<h$i$l$k@~J,$OC10l$NO"7k$J@~J,$G$"$k$3$H!%(B

    $B!!!!BLL\$J>l9g$NE57?$O!$%I!<%J%C%D$G$9!%$3$l$O$I$N$h$&$JJ}8~$G@Z$C$F$b(B
        $BCf?4$rDL$kD>@~$,%I!<%J%C%D$K$h$C$F@Z$j<h$i$l$kItJ,$O#28D$N@~J,$K(B
        $B$J$C$F$7$^$7$^$9!%$3$N$h$&$JNN0h$N<P@~EI$j$O$$$C$Z$s$K$O$G$-$^$;$s!%(B
        $B%I!<%J%C%D$N>l9g$O!$Bg$-$$1_$NFbItA4It$r0lC6<P@~EI$j$7$?$&$($G!$(B
        $B>.$5$$1_$NFbIt$rGrEI$j$9$k$3$H$G<B8=$G$-$^$9!%(B
\end{description}
%\end{jquote}


%\section{三角形の五心}
\section{$B;03Q7A$N8^?4(B}
\subsection{$B=E?4(B}

\showexample[$B=E?4(B](0.55)(0.4){example/zyusin01}
\cindex{Zyuusin}

\begin{boxnote}
\begin{verbatim}
\Zyuusin#1#2#3#4
    #1,#2,#3 $B$rD:E@$H$9$k;03Q7A$N=E?4$r(B #4 $B$K%;%C%H$7$^$9!%(B
\end{verbatim}
\end{boxnote}

\subsection{$B30?4(B}
$B;03Q7A$N30?4$r5a$a$k%3%^%s%I(B \cmd{Gaisin} $B$G$9!%(B

\showexample[$B30?4(B](0.55)(0.4){example/gaisin01}
\cindex{Gaisin}

\begin{boxnote}
\begin{verbatim}
\Gaisin#1#2#3#4
    #1,#2,#3 $B$rD:E@$H$9$k;03Q7A$N30?4$r(B #4 $B$K%;%C%H$7$^$9!%(B
\end{verbatim}
\end{boxnote}

$B;03Q7A$N30@\1_$rIA$/%3%^%s%I(B \cmd{Gaisetuen} $B$b$"$j$^$9!%(B
\cindex{Gaisetuen}

\showexample[$B30@\1_(B](0.55)(0.4){example/gaisin02}

\begin{boxnote}
\begin{verbatim}
\Gaisetuen#1#2#3
    #1,#2,#3 $B$rD:E@$H$9$k;03Q7A$N30@\1_$rIA2h$7$^$9!%(B
        $B30?4$O(B \vGaisin $B$K%;%C%H!$H>7B$O(B \lR
\end{verbatim}
\end{boxnote}
\cindex{vGaisin}\cindex{lR}

\subsection{$BFb?4(B}
$B;03Q7A$NFb?4$r5a$a$k%3%^%s%I(B \cmd{Naisin} $B$G$9!%(B
\cindex{Naisin}

\showexample[$BFb?4(B](0.55)(0.4){example/naisin01}

\begin{boxnote}
\begin{verbatim}
\Naisin#1#2#3#4
    #1,#2,#3 $B$rD:E@$H$9$k;03Q7A$NFb?4$r(B #4 $B$K%;%C%H$7$^$9!%(B
\end{verbatim}
\end{boxnote}


$B;03Q7A$NFb@\1_$rIA$/%3%^%s%I(B \cmd{Naisetuen} $B$b$"$j$^$9!%(B
\cindex{Naisetuen}

\showexample[$BFb@\1_(B](0.55)(0.4){example/naisin02}

\begin{boxnote}
\begin{verbatim}
\Naisetuen#1#2#3
    #1,#2,#3 $B$rD:E@$H$9$k;03Q7A$NFb@\1_$rIA2h$7$^$9!%(B
        $BFb?4$O(B \vNaisin $B$K%;%C%H!$H>7B$O(B \lr
\end{verbatim}
\end{boxnote}
\cindex{vNaisin}\cindex{lr}

\subsection{$BK5?4(B}
$B;03Q7A$NK5?4$r5a$a$k%3%^%s%I(B \cmd{Bousin} $B$G$9!%(B
\cindex{Bousin}

\showexample[$BK5?4(B](0.55)(0.4){example/bousin01}

\begin{boxnote}
\begin{verbatim}
\Bousin#1#2#3#4
    #1,#2,#3 $B$rD:E@$H$9$k;03Q7A$N!$(B
      $B"\(BA$BFb$K$"$kK5?4$r(B #4 $B$K%;%C%H$7$^$9!%(B
\end{verbatim}
\end{boxnote}


$B;03Q7A$NK5@\1_$rIA$/%3%^%s%I(B \cmd{Bousetuen} $B$b$"$j$^$9!%(B
\cindex{Bousetuen}

\showexample[$BK5@\1_(B](0.55)(0.4){example/bousin02}

\begin{boxnote}
\begin{verbatim}
\Bousetuen#1#2#3
    #1,#2,#3 $B$rD:E@$H$9$k;03Q7A$NK5@\1_$rIA2h$7$^$9!%(B
        $BK5?4$O(B \vBousin $B$K%;%C%H!$H>7B$O(B \BousetuenHankei
\end{verbatim}
\end{boxnote}
\cindex{vBousin}\cindex{BousetuenHankei}

$B;03Q7A$NFb@\1_!$(B3$B$D$NK5@\1_$rIA2h$9$kNc$G$9!#(B

\showexample[$BFb@\1_$HK5@\1_(B](0.9)(0.9){example/bousin03}

\subsection{$B?b?4(B}
$B;03Q7A$N?b?4$r5a$a$k%3%^%s%I$OMQ0U$7$F$"$j$^$;$s!%(B
\cmd{Suisen} $B$G?b@~$r0z$-!$(B\cmd{LandL} $B$G8rE@$r5a$a$k$3$H$G(B
$B?b?4$,F@$i$l$^$9!%(B

\showexample[$B?b?4(B](0.55)(0.4){example/suisin01}

\subsection{$B3Q$NFsEyJ,@~(B}
$B;03Q7A$NFsEyJ,@~$HBPJU$N8rE@$r5a$a$k$K$O!$(B
$B%3%^%s%I(B\cmd{Nitoubunsen}$B$rMQ$$$^$9!#(B
$B$=$N=q<0$G$9!#(B

\begin{boxnote}
\begin{verbatim}
$B;03Q7A$NFsEyJ,@~$HBPJU$N8rE@$r5a$a$k!#(B
\Nitoubunsen[#1]#2#3#4#5
  $B3Q(B#2#3#4$B!J(B#3$B$,3Q$ND:E@!K$r(B2$BEyJ,$9$kD>@~$,JU(B#2#4$B$H8r$o$kE@$r(B#5$B$KM?$($k!#(B
  #1$B$rM?$($?$H$-$O!$303Q$N(B2$BEyJ,@~$r(B#1$B$KM?$($k!#(B
\end{verbatim}
\end{boxnote}
%\clearpage

$B$^$:$O!$Fb3Q$NFsEyJ,@~$G$9!#(B

\begin{showEx}(1,1){\cmd{Nitoubunsen}}
\begin{zahyou*}[ul=10mm](-1,5)(-1,3)\footnotesize
  \tenretu{A(3,2)n;B(0,0)s;C(4,0)s}
  \Nitoubunsen\B\A\C\D
  \Put\D[s]{D}
  \Drawline{\A\B\C\A\D}
  \Kakukigou<0>\B\A\D(0,0)[c]{$\bullet$}
  \Kakukigou<0>\D\A\C(0,0)[c]{$\bullet$}
\end{zahyou*}
\end{showEx}
%\clearpage

$B303Q$NFsEyJ,@~$HBPJU!J$N1dD9!K$H$N8rE@$rF@$k$K$O!$(B
$B%*%W%7%g%s0z?t(B\verb+[#1]+$B$K8rE@$r<u$1<h$k@)8fDV$rM?$($^$9!#(B

\begin{showEx}(1,1){$B303Q$NFsEyJ,@~(B}
\begin{zahyou*}[ul=10mm](-1,12)(-1,4)\footnotesize
  \tenretu{A(3,2)n;B(0,0)s;C(4,0)s}
  \Nitoubunsen[\E]\B\A\C\D
  \Put\D[s]{D}
  \Put\E[s]{E}
  \Hantyokusen\B\C
  \Hantyokusen\B\A
  \Drawlines{\A\B\C\A\D;\A\E}
  \Kakukigou<0>\B\A\D(0,0)[c]{$\bullet$}
  \Kakukigou<0>\D\A\C(0,0)[c]{$\bullet$}
  \Bunten\A\B{-1}{2}\T
  \Kakukigou<0>\C\A\E(0,0)[c]{$\times$}
  \Kakukigou<0>\E\A\T(0,0)[c]{$\times$}
\end{zahyou*}
\end{showEx}


%\section{三角形の決定}
\section{$B;03Q7A$N7hDj(B}
\subsection{$B;0JU(B}
$B;0JU$ND9$5$,$o$+$C$F$$$k$H$-$K!$$=$N;03Q7A$r$I$N$h$&$KIA2h$9$k$+!$(B
$B$H$$$&OC$G$9!%(B

$B6qBNNc$H$7$F(B $\mathrm{AB=7,~BC=5,~CA=3}$ $B$H$7$^$7$g$&!%(B
$B$^$:(B A(0,0), B(7,0) $B$H$7$^$9!%(B
$BE@(BC$B$O!$(BA$B$rCf?4$H$9$kH>7B(B3$B$N1_$H!$(BB$B$rCf?4$H$9$kH>7B(B5$B$N1_$H$N8rE@(B
$B$H$7$F5a$a$^$9!%(B

\showexample[$B;03Q7A$N7hDj(B(1)$B;0JU(B](0.55)(0.8){example/sankaku1}

\subsection{$BFs3QTsJU(B}
$\mathrm{BC=5,~\kaku{B}=60\Deg,~\kaku{C}=45\Deg}$
$B$H$7$^$7$g$&!%(B
$B$3$l$O!$(BB(0,0), C(5,0) $B$H$7$?>e$G(B
\begin{jquote}
    $BE@(BB$B$rDL$j!$J}8~3Q(B $60\Deg$ $B$ND>@~$H(B\\
    $BE@(BC$B$rDL$j!$J}8~3Q(B $135\Deg$ $B$ND>@~(B
\end{jquote}
$B$N8rE@$H$7$F(BA$B$r5a$a$^$9!%(B

\showexample[$B;03Q7A$N7hDj(B(2) $BFs3QTsJU(B](0.8)(0.8){example/sankaku2}

\subsection{$BFsJUTs3Q(B}
$\mathrm{AB=3,~BC=5,~\kaku{B}=60\Deg}$
$B$H$7$^$7$g$&!%(B
$B$3$l$O(B B(0,0), C(5,0) $B$H$7$?>e$G(B A $B$N:BI8$r6K:BI8"*D>8r:BI8JQ49$G(B
$B5a$a$^$9!%$9$J$o$A(B
\begin{jquote}
\begin{verbatim}
    \kyokuTyoku(3,60)\A
\end{verbatim}
\end{jquote}

\showexample[$B;03Q7A$N7hDj(B(3) $BFsJUTs3Q(B](0.8)(0.8){example/sankaku3}


%\section{正弦定理・余弦定理}
\section{正弦定理・余弦定理}
\subsection{正弦定理}
正弦定理を用いて三角形の辺・角,外接円の半径を求めるコマンドを
解説します.
一例として,
\begin{jquote}
$A=120\Deg$,~B=15\Deg,~c=10 
\end{jquote}
である三角形を BC を底辺として描画することを考えてみます.
そのために $\mathrm{BC}=a$ を正弦定理で求めます.

\resetcounter{equation}
まずは $C=45\Deg$ と $c=10$ から
\[ \bunsuu{c}{\sin C}=2R \]
を用いて外接円の半径を求めます.そのためのコマンドが
\cmd{seigenR}です.書式は\cindex{seigenR}\cindex{lR}\cindex{lRR}
\begin{boxnote}
\begin{verbatim}
\seigenR#1#2
    #1 : 向かい合った辺・角のうち辺
    #2 :             角
  外接円の直径が \lRR, 半径が \lR にセットされる.
\end{verbatim}
\end{boxnote}
ここでは,\verb/\seigenR{10}{45}/ とします.

次いで,
\[ \bunsuu{a}{\sin A}=2R \]
を用いて $a$ を求めます.そのためのコマンドが \cmd{seigen} です.
\cindex{seigen}
\begin{boxnote}
\begin{verbatim}
\seigen#1#2
   #1 : 角
   #2 : 角と向かい合った辺の長さを受取るコントロールシーケンス
\end{verbatim}
\end{boxnote}
ここでは \verb/\seigen{120}\la/ として \verb/\la/ に $a$ が
セットされます.

関連して,逆に辺を与えて角を求めるコマンドが \cmd{Seigen} です.
\cindex{Seigen}
\begin{boxnote}
\begin{verbatim}
\Seigen[#1]#2#3
  #2 : a
  #3 : A を受取るコントロールシーケンス
  #1 = e のときは,角A(鋭角)を求める.(デフォルト)
  #1 = d のときは,角A(鈍角)を求める.
\end{verbatim}
\end{boxnote}

では,この節のはじめに取り上げた三角形を実際に描画してみましょう.

\showexample[正弦定理](0.9)(0.6){example/seigen01}

\subsection{余弦定理}
2辺夾角がわかっているときに,第3辺を求めるコマンドが
\cmd{yogen} です.\cindex{yogen}
\begin{boxnote}
\begin{verbatim}
2辺夾角から第3辺の長さを求める.
\yogen#1#2#3#4
  #1=b, #2=c, #3=A → a を#4にセットする.
\end{verbatim}
\end{boxnote}
では,このコマンドを用いて
\[ b=2,~c=3,~A=60\Deg \]
である三角形を BC を底辺として描画してみます.

\showexample[余弦定理](0.9)(0.4){example/yogen01}

関連して,3辺の長さがわかっているとき,
角を求めるコマンドが \cmd{Yogen} です.\cindex{Yogen}
\begin{boxnote}
\begin{verbatim}
3辺の長さから角の余弦を求める.
\Yogen[#1]#2#3#4#5
   cos(A)=(b^2+c^2-a^2)/(2bc)
   #2=a, #3=b, #4=c 結果は #5 にセット
   #1=a のときは角を求める.(単位は度)
\end{verbatim}
\end{boxnote}


%\section{ベクトル}
\section{$B%Y%/%H%k(B}
\subsection{$B%Y%/%H%k1i;;(B}
$BE@$N:BI8$r0LCV%Y%/%H%k$N@.J,$H8+$F!$%Y%/%H%k1i;;$r$9$k%3%^%s%I$r(B
$BMQ0U$7$^$7$?!%(B

\begin{boxnote}
\begin{list}{}{\leftmargin7zw\labelwidth7zw}
\item[$BOB!'(B]\cmd{Addvec}\verb+#1#2#3+$B!'(B
    2$B$D$N%Y%/%H%k(B\verb+#1, #2+$B$NOB%Y%/%H%k$r(B\verb+#3+$B$KM?$($^$9!%(B

\item[$B:9!'(B]\cmd{Subvec}\verb+#1#2#3+$B!'(B
    2$B$D$N%Y%/%H%k(B\verb+#1, #2+$B$N:9%Y%/%H%k$r(B\verb+#3+$B$KM?$($^$9!%(B

\item[$B%9%+%i!<G\!'(B]\cmd{Mulvec}\verb+#1#2#3+$B!'(B
    $B%Y%/%H%k(B\verb+#2+ $B$N%9%+%i!<(B\verb+#1+$BG\$r(B\verb+#3+$B$KM?$($^$9!%(B

\item[$BBg$-$5!'(B] \cmd{Absvec}\verb+#1#2+:
    $B%Y%/%H%k(B\verb+#1+ $B$NBg$-$5$r(B \verb+#2+ $B$KM?$($^$9!%(B

\item[$BJ}8~3Q!'(B] \cmd{Argvec}\verb+#1#2+:
    $B%Y%/%H%k$,(B x$B<4$N@5$N8~$-$H$J$93Q$r(B \verb+#2+ $B$KM?$($^$9!%(B

\item[$BK!@~%Y%/%H%k!'(B]\cmd{Nvec}\verb+#1#2+:
    $B%Y%/%H%k(B \verb+#1+ $B$K?bD>$JC10L%Y%/%H%k$r(B \verb+#2+ $B$KM?$($^$9!%(B

\item[$B2sE>!'(B] \cmd{Rotvec}\verb+[#1]<#2>#3#4#5+:
    $B%Y%/%H%k(B \verb+#3+ $B$r(B $B3Q(B \verb+#4+ $B$@$12sE>$7$?%Y%/%H%k$r(B \verb+#5+$B$K(B
    $BM?$($^$9!%(B\verb+[#1]+ $B$r;XDj$7$?>l9g$O!$D9$5$r;XDj$7$?CM$K$7$^$9!%(B
    $B$^$?!$(B\verb+<#2>+ $B$r;XDj$7$?>l9g$O!$D9$5$r85$N%Y%/%H%k$N(B \verb+#2+ $BG\(B
    $B$K$7$^$9!%(B

\item[$B@.J,!'(B] \cmd{vecXY}\verb+#1#2#3+:
    $B%Y%/%H%k(B \verb+#1+ $B$N(Bx$B@.J,$r(B \verb+#2+ $B$X!$(By$B@.J,$r(B \verb+#3+ $B$XCj=P$7$^$9!%(B
\end{list}
\end{boxnote}
\cindex{Addvec}\cindex{Subvec}\cindex{Mulvec}\cindex{Absvec}
\cindex{Argvec}\cindex{Nvec}\cindex{Rotvec}\cindex{vecXY}

\subsection{$BJ?9T;MJU7A(B}
A(2,3), B(1,1), C(4,1) $B$rD:E@$H$9$kJ?9T;MJU7A(B ABCD $B$r:n?^$9$k$K$O(B
3$BE@$N:BI8(B \cmd{A}, \cmd{B}, \cmd{C} $B$r0LCV%Y%/%H%k$H8+$F(B
\begin{quote}
    $\mbox{\cmd{A}}+\mbox{\cmd{C}}-\mbox{\cmd{B}}$
\end{quote}
$B$H$7$F(B \cmd{D} $B$r5a$a$^$9!%(B

\showexample[$BJ?9T;MJU7A(B](0.6)(0.55){example/sikakkei}

\subsection{$B2sE>(B}
$B%Y%/%H%k$r2sE>$5$;$k%3%^%s%I(B \cmd{Rotvec} $B$rMQ$$$F(B
$B2sE>$r9T$&$3$H$,$G$-$^$9!%$3$3$G$O$5$i$K0lHL2=$7$?(B
\cmd{Kaiten} $B$rMQ$$$F!$(B
$B;XDj$7$?(B2$BE@(B A, B $B$r7k$V@~J,$r0lJU$H$9$k@5;03Q7A$r:n?^$7$^$9!%(B
\cindex{Kaiten}

\showexample[$B2sE>(B](0.55)(0.35){example/kaiten01}

\cmd{Kaiten} $B%3%^%s%I$N=q<0$G$9!%(B
\begin{boxnote}
\begin{verbatim}
   \Kaiten[#1]<#2>#3#4#5#6
      #1 : $BD9$5;XDj(B
      #2 : $BD9$5$NG\N(;XDj(B
      #3 : $B2sE>$NCf?4(B
      #4 : $B2sE>$5$;$kE@(B
      #5 : $B2sE>3Q(B
      #6 : $B7k2L$N:BI8$r<u$1<h$k%3%s%H%m!<%k%7!<%1%s%9(B
\end{verbatim}
\end{boxnote}


%\section{座標平面}
\section{座標平面}
\subsection{数直線}
目盛付きの数直線を描画するコマンドが \cmd{suutyokusen} です.
\cindex{suutyokusen}

\showexample[数直線](0.5)(0.45){example/sutyoku1}

\begin{boxnote}
\begin{verbatim}
   \suutyokusen[#1](#2)#3#4
       #1 : [+] としたときは正の目盛には + をつける.
       #2 : 目盛の刻み値(デフォルトは 1)
       #3 : x の下限
       #4 : x の上限
   描画位置は \put 文で数直線の原点位置を指定する.
\end{verbatim}
\end{boxnote}

\subsection{連立不等式の解を図表示}
連立方程式の解を数直線上に図表示する話です。

{\unitlength10mm\drawaxisfalse
\begin{showEx}(.8,.8){例1}
  連立不等式
    \renritu{&x+1<0 \\ &x^2<4}
  の解は

\begin{zahyou}(-5,5)(-.5,1.5)%
\ArrowLine{(\xmin,0)}{(\xmax,0)}%
\Put{(\xmax,0)}[s]{$x$}%
\Put{(0,0)}[s]{$0$}%
\Put{(-1,0)}[s]{$-1$}%
\Put{(-2,0)}[s]{$-2$}%
\Put{(2,0)}[s]{$2$}%
\KTkukan{(,-1);(-2,2)}{(-2,-1)}
\end{zahyou}
\end{showEx}

\begin{showEx}(.8,.8){例2}
  連立不等式
    \renritu{&x^2>x \label{E2-1}\\ &x^2\leqq 4\label{E2-2}}
の解は

\begin{zahyou}(-5,5)(-.5,1.5)%
\ArrowLine{(\xmin,0)}{(\xmax,0)}%
\Put{(\xmax,0)}[s]{$x$}%
\Put{(0,0)}[s]{$0$}%
\Put{(1,0)}[s]{$1$}%
\Put{(-2,0)}[s]{$-2$}%
\Put{(2,0)}[s]{$2$}%
\KTkukan[E2-1;E2-2]{{(,0)|(1,)};[-2,2]}{[-2,0)|(1,2]}
\end{zahyou}
\end{showEx}
}

\cmd{KTkukan}の書式です。

\begin{boxnote}
\begin{verbatim}
\KTkukan[#1]#2#3
    #1 : 各区間のラベル指定オプション
    #2 : 各区間を`;'区切りで並べる
    #3 : 結果の区間

    区間は
        開区間   : (-3,5)
        閉区間   : [-3,5]
        半開区間 : (-3,5], [-3,5)
        無限区間 : (,-3), (5,)
      などと表す。
      2つの区間の和集合は (,-3)|(5,) などのように,`|'を用いる

    ラベル指定オプションは,
        デフォルトは [auto] で,\maru1, \maru2, .....が付与される。
        各不等式に \label がついているときは,ラベル名を`;'区切りで並べる。
        [] と指定すれば,区間にラベルはつかない。
        下の \kukantakasa を用いて,自由に配置してもよい。

    区間を表す横罫線の y座標は \kukantakasa で,そのデフォルト値は 0.5
        層を重ねるときは,その 2, 3, ... 倍となる。
\end{verbatim}
\end{boxnote}

\subsection{\textsf{zahyou} 環境}
座標軸を描きます.

基本的な使用法は

\begin{jquote}
\begin{verbatim}
\begin{zahyou}(xの下限, xの上限)(yの下限, yの上限)
\end{verbatim}
\end{jquote}

例えば,

\showexample[座標平面](0.5)(0.45){example/zahyo01}

\textsf{zahyou}環境は,実質 \textsf{picture} 環境です.
ここへ,1次関数,2次関数のグラフを描こうという魂胆です.

なお,座標軸を描画したくないときは,\verb/*/付の
\textsf{zahyou*}環境を使用します。
下の例は,\cmd{drawXaxis}で$x$軸だけを描画しています。
\cindex{drawaxisfalse}\cindex{drawXaxis}

\showexample[\cmd{drawaxisfalse}](0.5)(0.45){example/zahyo08}

$y$軸のみを描画する\cmd{drawYaxis}も定義されています。
\cindex{drawYaxis}

\subsection{座標軸描画のタイミング}
座標軸を描画するタイミングは,zahyou環境の最後ですが,
座標軸を描画した後に,座標軸上の点に\cmd{Siromaru}で白丸をつけたいときなどは
うまくありません。下の図のように,白丸の中を座標軸が突き抜けてしまいます。

\showexample[座標軸描画のタイミング](.6)(.34){example/zahyo15}

そのようなときは\verb+zahyou*+環境を用いて,
適切なタイミングで\cmd{drawXYaxis}を発行して,座標軸を別途描画します。
\cindex{drawXYaxis}

\showexample[\cmd{drawXYaxis}](.6)(.34){example/zahyo16}

\subsection{\textsf{zahyou}環境のオプション}
\textsf{zahyou}環境については,一番初めに紹介しましたが,
細かい点を修正するオプションについて説明します。

\textsf{zahyou}環境の細かい指定をするのに,
オプション引数を用意しています。これらは
\begin{jquote}
\begin{verbatim}
key = val
\end{verbatim}
\end{jquote}
の形式で,コンマで区切ることで複数のオプションを指定できます。
\subsubsection{\texorpdfstring{\cmd{unitlength}}{unitlength}の指定}
\textsf{picture}環境の単位長(\cmd{unitlength})は,デフォルトでは
\verb/1pt/となっています。これを変更するオプションが
\begin{jquote}
\begin{verbatim}
ul=...
\end{verbatim}
\end{jquote}
で,右辺値は単位を伴った長さです。

\begin{showEx}(.5,.44){\texttt{ul=...}オプション}
\begin{zahyou}[ul=8mm]%
  (-3,3)(-2,2)%
  \xmemori{1}%
\end{zahyou}%
\end{showEx}

\begin{enumerate}[注 1.~]
  \item \TeX が扱える実数値は
\begin{jquote}
\begin{verbatim}
\maxdimen=16383.99999pt % the largest legal <dimen>
\end{verbatim}
\end{jquote}
が上限とされています。\textsf{emathPh}における計算もこの制限を受けます。
距離計算では平方計算が必要ですが,$128^2$でオーバーフローしてしまいます。
従って座標の値を大きくしないように,\cmd{unitlength}を\verb/1cm/前後に
しておく方が良いでしょう。
  \item
\begin{jquote}
\begin{verbatim}
\unitlength8mm
\begin{zahyou}....

\end{zahyou}
\end{verbatim}
\end{jquote}
と
\begin{jquote}
\begin{verbatim}
\begin{zahyou}[ul=8mm]....

\end{zahyou}
\end{verbatim}
\end{jquote}
との大きな違いは,\cmd{unitlength}変更の有効範囲です。

前者は\textsf{zahyou}環境が終わった後でもこの変更が有効(残ってしまう)
のに対して,後者はこの変更が当該\textsf{zahyou}環境内のみに有効である,
すなわちこの\textsf{zahyou}環境が終われば,\cmd{unitlength}は以前の値に
戻ることです。

\textsf{zahyou}の\verb+[...]+オプションによる変更はすべて当該
\textsf{zahyou}環境内に限定されます。
\end{enumerate}

\subsubsection{座標軸の名称変更}
デフォルトでは,
\begin{jquote}
原点には O\\
横軸には $x$\\
縦軸には $y$
\end{jquote}
が表示されますが,これを変更するオプションが
\begin{jquote}
\begin{verbatim}
\gentenkigou=
\yokozikukigou=
\tatezikukigou=
\end{verbatim}
\end{jquote}
で,右辺値は文字列です。

原点記号を変更します。
\begin{showEx}(.5,.44){原点記号の変更}
\begin{zahyou}%
  [%
    ul=8mm,
    gentenkigou=O$'$
  ]%
  (-3,3)(-2,2)%
  \xmemori{1}%
\end{zahyou}%
\end{showEx}

座標軸名称の変更例です。
\begin{showEx}(.5,.44){座標軸名の変更}
\begin{zahyou}%
  [%
    ul=8mm,
    yokozikukigou=$a$,
    tatezikukigou=$b$
  ]%
  (-3,3)(-2,2)%
  \xmemori{1}%
\end{zahyou}%
\end{showEx}

\subsubsection{軸記号の配置オプション}
デフォルトでは,
\begin{jquote}
原点記号は 左下\\
横軸名は   下\\
縦軸名は   左
\end{jquote}
に表示されますが,これを変更するオプションが
\begin{jquote}
\begin{verbatim}
\gentenhaiti=
\yokozikuhaiti=
\tatezikuhaiti=
\end{verbatim}
\end{jquote}
で,右辺値は\cmd{Put}のオプションと同形式です。
原点記号はデフォルトでは左下に配置されますが,
これを右下に変更してみましょう。

\begin{showEx}(.5,.44){原点記号の配置変更}
\begin{zahyou}%
  [%
    ul=8mm,
    gentenhaiti={[se]}
  ]%
  (-3,3)(-2,2)%
  \xmemori{1}%
\end{zahyou}%
\end{showEx}

\cmd{Put}の配置オプション \verb+[se]+を\texttt{gentenhaiti=}の
右辺値に指定しますが,\verb+]+が\textsf{zahyou}環境に対する\verb+[...]+
オプションの終了と解釈されるのを防ぐため\verb+{[se]}+と,\verb+{ }+ で
括っておく必要があります。

次に座標軸名の配置を変更する例です。

\begin{showEx}(.5,.44){座標軸名の配置変更}
\begin{zahyou}%
  [%
    ul=8mm,
    yokozikuhaiti={[e]},
    tatezikuhaiti={(-1pt,0)[r]}
  ]%
  (-3,3)(-2,2)%
  \xmemori{1}%
\end{zahyou}%
\end{showEx}

この場合,座標軸名$x$, $y$は\textsf{zahyou}環境の外に飛び出しています。
これを防ぐために,周囲に少し余白を付けておきたい,
というのが後述の
\begin{jquote}
\begin{verbatim}
migiyohaku=
hidariyohaku=
ueyohaku=
sitayohaku=
\end{verbatim}
\end{jquote}
オプションです。

\subsubsection{矢印のサイズ変更}
座標軸の矢印を含め,\textsf{zahyou}環境内の矢印のサイズを変更する
オプションが
\begin{jquote}
\begin{verbatim}
arrowheadsize=
\end{verbatim}
\end{jquote}
です。

\begin{showEx}(.5,.44){矢印のサイズ変更}
\begin{zahyou}%
  [%
    ul=8mm,
    arrowheadsize=2.5
  ]%
  (-3,3)(-2,2)%
  \xmemori{1}%
\end{zahyou}%
\end{showEx}

矢印が大きくなりすぎましたか。
\verb+arrowheadsize+の右辺値を適当に変更してください。
この数値は,現在のサイズの何倍にするかを指定するものです。

逆に矢印がいらない場合は,つぎの\verb/zikusensyu/オプションを用います。

\subsubsection{軸の線種変更}
座標軸には,デフォルトで矢印が付加されていますが,
軸の線種を変更することで矢印を付けないようにすることができます。

\begin{showEx}(.5,.44){軸の矢印なし}
\begin{zahyou}%
  [%
    ul=8mm,
    zikusensyu=\drawline
  ]%
  (-3,3)(-2,2)%
  \xmemori{1}%
\end{zahyou}%
\end{showEx}

軸を太くしてみましょうか。

\begin{showEx}(.5,.44){軸を太目に}
\begin{zahyou}%
  [%
    ul=8mm,
    zikusensyu=\thicklines\drawline
  ]%
  (-3,3)(-2,2)%
\end{zahyou}%
\end{showEx}

軸を太くし,矢印も付けるには,線種をデフォルトの
\cmd{arrowdrawline}にして\cmd{thicklines}などをかぶせます。

\begin{showEx}(.54,.4){軸を太目に,矢印も}
\begin{zahyou}[%
  ul=8mm,
  zikusensyu=\thicklines\arrowdrawline,
  arrowheadsize=2
  ]%
  (-3,3)(-2,2)%
\end{zahyou}%
\end{showEx}

\subsubsection{描画領域の周辺に余白}
描画領域の周辺に文字を配置するためなどに余白をとっておきたいことがあります。
そのためのオプションが
\begin{jquote}
\begin{verbatim}
migiyohaku
hidariyohaku
ueyohaku
sitayohaku
\end{verbatim}
\end{jquote}
です。

まず基準の確認です。\TeX が認識している\textsf{zahyou}環境を\cmd{fbox}で
枠をつけてみます。

\begin{showEx}(.5,.44){基準サイズ}
\fboxsep=0pt\fbox{%
\begin{zahyou}%
  [%
    ul=8mm,
    yokozikuhaiti={[e]},
    tatezikuhaiti={[n]}
  ]%
  (-3,3)(-1,3)%
  \xmemori{1}%
  \def\Fx#1#2{\Mul{#1}{#1}#2}
  \yGurafu\Fx{}{}
\end{zahyou}}%
\end{showEx}
座標軸のラベルが枠外にはみだしています。

上と右に余白を付けて見ます。

\begin{showEx}(.5,.44){上と右に余白}
\fboxsep=0pt\fbox{%
\begin{zahyou}%
  [%
    ul=8mm,
    yokozikuhaiti={[e]},
    tatezikuhaiti={[n]},
    ueyohaku=.75,
    migiyohaku=.5
  ]%
  (-3,3)(-1,3)%
  \def\Fx#1#2{\Mul{#1}{#1}#2}%
  \Put{(1,1)}[syaei=xy]{}%
  \yGurafu\Fx\xmin\xmax%
  \Put{(1.732,3)}[n]{$y=x^2$}%
\end{zahyou}}%
\end{showEx}

枠内に納まりましたね。

\verb+ueyohaku+などの右辺値は,無名数で,単位は \cmd{unitlength} です。

これに対して
\begin{jquote}
\begin{verbatim}
Migiyohaku=
Hidariyohaku=
Ueyohaku=
Sitayohaku=
Yohaku=
\end{verbatim}
\end{jquote}
は,右辺値に単位を伴う長さを与えます。

\begin{showEx}(.5,.44){\cmd{Ueyohaku}}
\fboxsep=0pt\fbox{%
\begin{zahyou}%
  [%
    ul=8mm,
    yokozikuhaiti={[e]},
    tatezikuhaiti={[n]},
    Ueyohaku=15pt,
    Migiyohaku=10pt
  ]%
  (-3,3)(-1,3)%
  \def\Fx#1#2{\Mul{#1}{#1}#2}%
  \Put{(1,1)}[syaei=xy]{}%
  \yGurafu\Fx\xmin\xmax%
  \Put{(1.732,3)}[n]{$y=x^2$}%
\end{zahyou}}%
\end{showEx}

\subsubsection{縦横比の変更}
\textsf{zahyou}環境では,縦と横の単位長は同一になっています。
これを別々に設定するためのオプション
\begin{jquote}
\begin{verbatim}
yscale, xscale
\end{verbatim}
\end{jquote}
があります。\cindex{xscale}\cindex{yscale}

ただし,この機能は\textsf{emathPxy.sty}で定義されていますから,
\begin{jquote}
\begin{verbatim}
\usepackage{emathPxy}
\end{verbatim}
\end{jquote}
としておく必要があります。




次のリストは\verb/yscale=.25/として,$y$軸方向を1/4に縮め,
$y=x^2$ ($-4<x<4$)のグラフを描画しています。

\begin{showEx}{\texttt{<yscale=...>}オプション}
\begin{zahyou}%
  [ul=6mm,yscale=.25]%
  (-4,4)(-4,16)
\def\Fx#1#2{\Mul{#1}{#1}#2}
\yGurafu\Fx\xmin\xmax
\zahyouMemori[g]<dx=1,dy=4>
\end{zahyou}
\end{showEx}

グラフ描画コマンド \cmd{yGurafu} については後述します。
\bigskip

次は横長の格子を描く例です。
\begin{jquote}
\begin{verbatim}
ul=3mm,xscale=3,yscale=2
\end{verbatim}
\end{jquote}
というオプションで
\begin{jquote}
\begin{verbatim}
ul=3mm で \unitlength は 3mm
x軸方向の単位長は ul×xscale 9mm で \xunitlength が 9mm
y軸方向の単位長は ul×yscale 6mm で \yunitlength が 6mm
\end{verbatim}
\end{jquote}
に設定されます。

\begin{showEx}{横長の格子}
\begin{zahyou*}[%
    ul=3mm,xscale=3,%
    yscale=2](0,4)(0,5)%
  \Put\O{\kousi{4}{5}}%
  \Kuromaru{(2,3)}%
\end{zahyou*}%
\end{showEx}

\subsection{\textsf{zahyou}環境の縦方向配置}
\bgroup
\def\kizyunsen{{\color{red}\relax
        \unitlength=1pt\relax
        \begin{picture}(0,0)\relax
          \Drawline{(0,0)(260,0)}%
        \end{picture}}}%
\fboxsep=0pt\relax
まずは,
\begin{jquote}
\textsf{picture(zahyou)}環境で作成した図,\\
\textsf{tabular(array)}環境で作成した表
\end{jquote}
を横に並べたときの縦位置に関する話から始めます。
%\clearpage

\subsubsection{デフォルトの確認}
まずはデフォルトの状態の確認です。

\begin{showEx}(1,.8){デフォルト}
\kizyunsen
あいう%
\fbox{\begin{zahyou}[ul=10mm](-2,2)(-1,5)
  \En\O{1}
\end{zahyou}}%
xyz%
\begin{tabular}{|c|c|}\hline
  1 & 1 \\\hline
  1 & 1 \\\hline
\end{tabular}%
か漢字%
\end{showEx}

赤の横線が基準線(ベースライン)です。\TeX は,この線を文字通り基準として
文字,表,図などボックスを配置して行きます。まずは,文字にご注目ください。
`y'は基準線から下にはみ出す形で配置されています。
すなわち「深さ」をもっています。全角文字も良く見ると一部基準線から下に
はみ出しています。

さて,図ですが,\textsf{picture}環境の下辺を基準線に重ねる,
というのが\LaTeX の仕様です。

表のほうは,デフォルトでは,縦方向中央揃えとなります。
デフォルトでは,といったのは,オプションで変更可能ということで,
次節でそれを見ていきます。
%\clearpage

\subsubsection{下揃え}
表の下辺を基準線に重ねるのが,\textsf{tabular, array}の
\verb+[b]+オプションです。

\begin{showpEx}(1,.8){下揃え}
!\kizyunsen
!あいう%
!\fbox{\begin{zahyou}[ul=10mm](-2,2)(-1,5)
!  \En\O{1}
!\end{zahyou}}%
!xyz%
\begin{tabular}[b]{|c|c|}\hline
!  1 & 1 \\\hline
!  1 & 1 \\\hline
!\end{tabular}%
!か漢字%
\end{showpEx}

ソースリストは,デフォルトのものから変更した行だけを記します。

\subsubsection{上揃え}
逆に,表の上辺を基準線に重ねるのが,\textsf{tabular, array}の
\verb+[t]+オプションです。
\textsf{zahyou}環境にも\verb+[haiti=t]+オプションがあります。

\begin{showpEx}(1,.8){上揃え}
!\kizyunsen
!あいう%
\fbox{\begin{zahyou}[ul=10mm,haiti=t](-2,2)(-1,5)
!  \En\O{1}
!\end{zahyou}}%
!xyz%
\begin{tabular}[t]{|c|c|}\hline
!  1 & 1 \\\hline
!  1 & 1 \\\hline
!\end{tabular}%
!か漢字%
\end{showpEx}

\textsf{zahyou}環境の上辺をさらに上に引っ張って文字の高さと揃えたい,
というご要望もありそうです。
そのために\verb+haiti=t/c/b+オプションには,そのあとに補正量を与えることが
できるようにしてあります。

\begin{showpEx}(1,.8){上揃え--さらに調整}
!\kizyunsen
!あいう%
\fbox{\begin{zahyou}[ul=10mm,haiti=t+1\zh](-2,2)(-1,5)
!  \En\O{1}
!\end{zahyou}}%
!xyz%
!\begin{tabular}[t]{|c|c|}\hline
!  1 & 1 \\\hline
!  1 & 1 \\\hline
!\end{tabular}%
!か漢字%
\end{showpEx}

正の補正量で上方に,負の補正量で下方に移動します。

\subsubsection{中央揃え}
表を縦方向センタリングするのが,\textsf{tabular, array}の
\verb+[c]+オプションです。\\
\textsf{zahyou}環境にも\verb+[haiti=c]+オプションがあります。

\begin{showpEx}(1,.8){中央揃え}
!\kizyunsen
!あいう%
\fbox{\begin{zahyou}[ul=10mm,haiti=c](-2,2)(-1,5)
!  \En\O{1}
!\end{zahyou}}%
!xyz%
\begin{tabular}[c]{|c|c|}\hline
!  1 & 1 \\\hline
!  1 & 1 \\\hline
!\end{tabular}%
!か漢字%
\end{showpEx}

\textsf{zahyou}環境には,\verb+[haiti=x]+オプションもあります。
これは$x$軸を基準線に重ねます。


\begin{showpEx}(1,.8){横軸揃え}
!\kizyunsen
!あいう%
\fbox{\begin{zahyou}[ul=10mm,haiti=x](-2,2)(-1,5)
!  \En\O{1}
!\end{zahyou}}%
!xyz%
\begin{tabular}[c]{|c|c|}\hline
!  1 & 1 \\\hline
!  1 & 1 \\\hline
!\end{tabular}%
!か漢字%
\end{showpEx}

\textsf{emathPh.sty v 1.70}までは,\verb+[haiti=c]+オプションで,
$x$軸を基準線に重ねていましたが,\textsf{v 1.71}で上記のように
修正いたしました。
\egroup

\subsection{\textsf{zahyou}環境の書式}
\begin{boxnote}
\begin{verbatim}
    \begin{zahyou}[#1](#2,#3)(#4,#5)
      #1 : key=val の形式で,key には次のものが使えます。
          yokozikukigou デフォルトは $x$
          tatezikukigou デフォルトは $y$
          gentenkigou   デフォルトは O
          yokozikuhaiti デフォルトは (0,-3pt)[rt]
          tatezikuhaiti デフォルトは (-3pt,0)[rt]
          gentenhaiti   デフォルトは [sw]
          ul            \unitlength を変更します。デフォルトは 1pt
          yscale        デフォルトは1 --> emathPxy.sty
          xscale        デフォルトは1 --> emathPxy.sty
          arrowheadsize 矢印のサイズ(その時点のものに対する比率)
          zikusensyu    座標軸の線種
                        デフォルトは \arrowdrawline
          ueyohaku      右辺値は無名数(単位は\unitlength)
          sitayohaku
          migiyohaku
          hidariyohaku
          Ueyohaku      右辺値は単位を伴う長さ
          Sitayohaku
          Migiyohaku
          Hidariyohaku
          haiti         t/c/b/x
      (#2,#3) : xの範囲
      (#4,#5) : yの範囲
\end{verbatim}
\end{boxnote}
\index{zahyou@zahyou 環境}

\subsection{目盛り}
\subsubsection{座標軸上に等間隔の目盛り}
座標軸に等間隔に目盛りを打つコマンドが \cmd{zahyouMemori} です.
\cindex{zahyouMemori}

\showexample[座標目盛り](0.95)(0.65){example/zahyo04}

\begin{boxnote}
\begin{verbatim}
\zahyouMemori[#1][#2]<#3>
   #1 : g : グリッド
        + : 格子点に +マーク
        o : 格子点に黒丸
        z : 座標軸上の格子点に +マーク
   #2 : n : グリッドのみで,目盛り数値を打たない.
   #3 : 刻み
        key=val の形式 ---> emathPxy.sty
           key は
            dx= xの刻み値
            dy= yの刻み値
            xo= xの基準値
            yo= yの基準値
\end{verbatim}
\end{boxnote}

目盛りの間隔はデフォルトでは1ですが,これを2に変更してみましょう。

\begin{showEx}(.5,.44){間隔2の目盛線}
\small
\begin{zahyou}[ul=4mm](-5,5)(-5,5)
\zahyouMemori<2>
\end{zahyou}
\end{showEx}

$x$軸と$y$軸とで間隔を変えることもできます。

\begin{showEx}{\texttt{<dx=..,dy=..>}オプション}
\small
\begin{zahyou}%
  [ul=3mm](-7,7)(-5,5)
\zahyouMemori<dx=3,dy=2>
\end{zahyou}
\end{showEx}

\cmd{xscale}, \cmd{yscale} と併用して

\begin{showEx}{\texttt{\cmd{xscale}と併用}オプション}
\small
\begin{zahyou}%
  [ul=8mm,yokozikuhaiti={[n]},%
    xscale=0.017453]%
  (-5,400)(-1.5,1.5)
\zahyouMemori<dx=90>
\end{zahyou}
\end{showEx}

基準点はデフォルトでは原点ですが,
これを変更するには,\verb+<xo=...,yo=...>+オプションを用います。

\begin{showEx}{基準点の変更}
\small
\begin{zahyou}%
  [ul=8mm,%
    yokozikuhaiti={[n]},%
    gentenhaiti={[se]},%
    xscale=0.017453]%
  (-60,360)(-1.5,1.5)
\zahyouMemori<dx=90,xo=-30>
\end{zahyou}
\end{showEx}

\subsubsection{グリッド線}
\texttt{[g]}オプションをつけると方眼を描きます.

\showexample[方眼 \& 目盛り](0.95)(0.65){example/zahyo05}

グリッド線の線種を変更するには,
\verb+<sensyu=...>+ オプションを用います。

\showexample[線種の変更](0.95)(0.65){example/zahyo14}

\texttt{[+]}オプションでは,格子点に + マークをつけます。

\showexample[格子点に + ](0.95)(0.65){example/zahyo11}

\texttt{[o]}オプションでは,格子点に黒丸をつけます。

\showexample[格子点に $\bullet$ ](0.95)(0.65){example/zahyo12}

なお,目盛りの数値は要らないというときは第2の\texttt{[n]}オプションを
つけます.


\showexample[方眼](0.45)(0.5){example/zahyo07}

\subsubsection{軸上に目盛り}
また,座標軸上の指定した位置に目盛りを打つコマンドが
\begin{jquote}
\cmd{xmemori}, \cmd{ymemori}
\end{jquote}
です.

\showexample[目盛り](0.45)(0.4){example/zahyo06}

目盛りの位置をずらしたいときは,
\verb+[..]+オプションに\cmd{emathPut}の配置オプションを記述します。

\begin{showEx}(1,1){\cmd{xmemori}の\texttt{[...]}オプション}
\begin{zahyou}[ul=10mm](-5,5)(-1,1)
  \xmemori[b]{1}
  \xmemori[t]{2}
  \xmemori[{[ne]}]{3}
  \xmemori[{(-2pt,-3pt)[rt]}]{-1}
\end{zahyou}
\end{showEx}

\begin{showEx}(.64,.3){\cmd{ymemori}の\texttt{[...]}オプション}
\begin{zahyou}[ul=10mm](-1,1)(-2,4)
  \ymemori[l]{1}
  \ymemori[r]{2}
  \ymemori[{[sw]}]{3}
  \ymemori[{(-2pt,-3pt)[rt]}]{-1}
\end{zahyou}
\end{showEx}

\cmd{xmemori}, \cmd{ymemori}の書式は

\begin{boxnote}
\begin{verbatim}
座標軸上任意位置に目盛
\xmemori[#1]<#2>#3
\ymemori[#1]<#2>#3
   #1 : t = 座標軸の下に目盛
        b =         上
        l =         右
        r =         左
        \emathPut の配置指定オプション
   #2 : 目盛の文字
       (省略すれば,#3 に指定したものが代用されます.)
   #3 : 目盛の位置
\end{verbatim}
\end{boxnote}

\cmd{xscale}と併用した例です。

\begin{showEx}(.5,.44){\cmd{xscale}と併用}
\footnotesize
\begin{zahyou}[ul=5mm](-5,5)(-1.5,2)
\zahyouMemori[g][n]<dx=1.57,dy=.5>
\ymemori{1}
\ymemori{-1}
\xmemori<\pi>{3.14}
\xmemori<-\pi>{-3.14}
\end{zahyou}
\end{showEx}

\texttt{dx}で指定した間隔で目盛を打つために \cmd{xmemori} の
機能を強化した \cmd{xMemori} が \textsf{emathPxy.sty} で定義されています。

下のリストは,$x$軸方向$\mbox{\cmd{Pis}}=\pi/6$間隔で目盛を入れています。

\begin{showEx}(.5,.44){\cmd{xmemori}}
\footnotesize
\begin{zahyou}[ul=6mm](-1,7)(-1.5,2)
\zahyouMemori[g][n]<%
dx=\Pis,dy=.5,sensyu=\drawline>
\ymemori{1}
\ymemori{-1}
\xMemori<\frac{\pi}{6}>{1}
\xMemori<\frac{\pi}{3}>{2}
\xMemori<\frac{\pi}{2}>{3}
\xMemori<\pi>{6}
\xMemori<\frac{3\pi}{2}>{9}
\xMemori<2\pi>(\Pis){12}
\end{zahyou}
\end{showEx}

新設した\cmd{xMemori}, \cmd{yMemori}の書式です。

\begin{boxnote}
\begin{verbatim}
座標軸上任意位置に目盛
\xMemori[#1]<#2>(#3)#4
\yMemori[#1]<#2>(#3)#4
  #1 : t = 座標軸の下に目盛(デフォルト)
       b =         上
  #2 : 目盛の文字
       (省略すれば,#4 に指定したものが代用されます.)
  #3 : 単位の長さ(デフォルト値=\zahyouMemoriで指定したdx)
  #4 : 単位の長さの#4倍の位置に目盛を打つ
\end{verbatim}
\end{boxnote}

さらに,\verb+<xo=...,yo=....>+ オプションと併用した例です。

\begin{showEx}(.5,.44){\texttt{xo,yo}オプションと併用}
\footnotesize
\begin{zahyou}[%
ul=6mm,gentenhaiti={[se]}](%
-1,7)(-1.5,2)
\zahyouMemori[+][n]%
  <xo=-\Pis,dx=\Pih>
\ymemori{1}
\ymemori{-1}
\xMemori<-\frac{\pi}{6}>{0}
\xMemori<\frac{\pi}{3}>{1}
\xMemori<\frac{5\pi}{6}>{2}
\xMemori<\frac{4\pi}{3}>{3}
\xMemori<\frac{11\pi}{6}>{4}
\end{zahyou}
\end{showEx}

\subsection{座標軸への垂線}
平面上に点をプロットしたとき,その点から座標軸に下ろした垂線を
破線で描画したいときがあります。
この目的のために \cmd{Put} にオプションを追加しました。

\begin{boxnote}
\begin{verbatim}
\emathPut#1[#2].....
  #1 : 文字列を置く位置(座標)
  #2 : 方位
          syaei=x|y|xy
          houi=n|nw|w|sw|s|se|e|ne
          xlabel=..
          ylabel=..
          xpos=..
          ypos=..
          syaeisensyu=..
\end{verbatim}
\end{boxnote}

従来,\verb/[#2]/ オプションは,文字列の点に対する方位を指定するものでした。
これはそのまま有効ですが,ここに
\begin{verbatim}
    syaei=xy
\end{verbatim}
とすれば,点から両座標軸への垂線を描画することができます。

\begin{showEx}(.6,.34){\texttt{[syaei=xy]}}
\unitlength10mm\footnotesize
\begin{zahyou}(-1,3)(-1,3)
  \tenretu{A(1,2)ne}
  \Put\A[syaei=xy]{}
\end{zahyou}
\end{showEx}

\begin{verbatim}
    syaei=x
\end{verbatim}
とすれば,$x$軸への垂線が描画されます。

\begin{showEx}(.6,.34){\texttt{[syaei=x]}}
\unitlength10mm\footnotesize
\begin{zahyou}(-1,3)(-1,3)
  \tenretu{A(1,2)ne}
  \Put\A[syaei=x]{}
\end{zahyou}
\end{showEx}

従来の方位オプションもここに記述したいときは
\begin{verbatim}
  houi=..
\end{verbatim}
と記述します。

\begin{showEx}(.6,.34){\texttt{[houi=..]}}
\unitlength10mm\footnotesize
\begin{zahyou}(-1,3)(-1,3)
  \def\A{(1,2)}
  \Put\A[syaei=xy,houi=sw]{A}
\end{zahyou}
\end{showEx}


座標軸上のラベルですが,デフォルトでは指定点の $x$, $y$座標が表示されます。
これが整数の場合は問題ないのですが,分数,無理数,文字を表示したいときは
\begin{verbatim}
  xlabel=..
  ylabel=..
\end{verbatim}
オプションを用います。

\begin{showEx}(.6,.34){\texttt{[xlabel=..]}}
\unitlength10mm\footnotesize
\begin{zahyou}(-1,3)(-1,3)
  \tenretu{A(1.5,1.414)ne}
  \Put\A[syaei=xy,xlabel=\frac32,
    ylabel=\sqrt2]{}
\end{zahyou}
\end{showEx}

このオプションの右辺値は数式モード内であるという前提です。
また \verb/[...,xlabel=]/ と右辺値を空にすればラベルは打たれません。

また,垂線の線種を変更するには,\verb+syaeisensyu=..+オプションを用います。

\begin{showEx}(.6,.34){\texttt{[syaeisensyu=]}}
\unitlength10mm\footnotesize
\begin{zahyou}(-1,3)(-1,3)
  \tenretu{A(1,2)ne}
  \Put\A[syaei=x,%
    syaeisensyu=\protect\dottedline{.1}]{}
\end{zahyou}
\end{showEx}

垂線の足につける目盛り位置を修正するときは,\verb+[xpos=..]+, \verb+[ypos=..]+
オプションで,右辺値に\cmd{Put}の配置オプションを記述します。

$x$軸に下ろした垂線の足位置につける文字位置は
\verb+[xpos=..]+オプションで修正します。
\index{xpos @ xpos オプション}

\begin{showEx}{\texttt{xpos=..}オプション}
\begin{zahyou}[ul=10mm]%
    (-2.5,2.5)(-1.5,1.5)
  \tenretu{A(2,1)ne;B(-2,-1)sw}
  \Put\A[syaei=xy,xpos={[ne]}]{}
  \Put\B[syaei=xy,%
    xpos={(-2pt,-2pt)[rt]}]{}
\end{zahyou}
\end{showEx}

$y$軸に下ろした垂線の足位置につける文字位置の調整は
\verb+[ypos=..]+オプションです。
\index{ypos @ ypos オプション}

\begin{showEx}{\texttt{ypos=..}オプション}
\begin{zahyou}[ul=10mm]%
    (-2.5,2.5)(-1.5,1.5)
  \tenretu{A(2,1)ne;B(-2,-1)sw}
  \Put\A[syaei=xy,ypos={[ne]}]{}
  \Put\B[syaei=xy,%
    ypos={(-2pt,-2pt)[rt]}]{}
\end{zahyou}
\end{showEx}


%\section{点・直線・円}
\section{$BE@!&D>@~!&1_(B}
\subsection{$BE@$N0LCV$K9u4](B}
$B50@W$r5a$a$kLdBj$J$I$G!$C<E@$,4^$^$l$k$H$+4^$^$l$J$$$H$+$r(B
$B9u4]!&CfH4$-$NGr4]$J$I$GI=<($9$k$3$H$,$"$j$^$9!%(B
$B$=$N$?$a$N%3%^%s%I$,(B \cmd{Kuromaru}, \cmd{Siromaru} $B$G$9!%(B

\showexample[$B9u4]!&Gr4](B](0.6)(0.34){example/zahyou01}

$B$=$N=q<0$O(B\cindex{Kuromaru}\cindex{Siromaru}

\begin{boxnote}
\begin{verbatim}
$B9u4]!$Gr4](B
\Kuromaru[#1]#2
\Siromaru[#1]#2
   #1 : $B4]$NH>7B!JC10L$r$D$1$k!K%G%U%)%k%H$O(B 1pt
   #2 : $B0LCV(B
\end{verbatim}
\end{boxnote}

$B9u4]$NBg$-$5$rJQ99$7$?$$$H$-$O!$(B\cmd{Kuromaru}$B$N(B\verb/[#1]/$B%*%W%7%g%s$r(B
$BMxMQ$9$l$P$h$$$N$G$9$,!$(B
$BJ8=qA4BN$G!$$"$k$$$O9-HO0O$K(B\cmd{Kuromaru}$B$NBg$-$5$rJQ99$7$?$$$H$-$O(B
\cmd{KuromaruHankei}$B%3%^%s%I$N0z?t$K9u4]$NH>7B!JC10L$D$-!K(B $B$r(B
$BM?$($kJ}K!$b$"$j$^$9!#(B\cindex{KuromaruHankei}

\showexample[$B9u4]$NBg$-$5(B](0.6)(0.34){example/zahyou02}

$BJ#?t$NE@$K9u4]!JGr4]!K$r$D$1$k%3%^%s%I$,(B \cmd{kuromaru}(\cmd{siromaru})
$B$G$9!#J#?t$NE@$r(B`;'$B$G6h@Z$j$^$9!#(B\cindex{kuromaru}\cindex{siromaru}

\showexample[$BJ#?t$NE@$K9u4](B](0.6)(0.34){example/zahyou03}

$B4]$N%5%$%:$r;XDj$9$k(B \texttt{[..]}$B%*%W%7%g%s$b(B \cmd{Kuromaru} $B$J$I$H(B
$BF1MM$KM-8z$G$9!#(B

\subsection{2$BE@4V$N5wN%(B}
2$BE@4V$N5wN%$r5a$a$k$K$O!$%3%^%s%I(B \cmd{Kyori} $B$rMQ$$$^$9!%(B\cindex{Kyori}

    \showexample[\cmd{Kyori}](0.52)(0.42){example/kyori01}\bigskip

\cmd{Kyori} $B$N=q<0$G$9!%(B

\begin{boxnote}
\begin{verbatim}
% 2$BE@4V$N5wN%(B
% \Kyori#1#2#3
%   2$BE@(B #1, #2 $B$N5wN%$r(B #3 $B$KBeF~(B
\end{verbatim}
\end{boxnote}\bigskip

$B5wN%$NJ?J}$r5a$a$k%3%^%s%I$,(B \cmd{Kyorii} $B$G$9!%(B

    \showexample[\cmd{Kyorii}](0.52)(0.42){example/kyori02}\bigskip

\cmd{Kyorii} $B$N=q<0$G$9!%(B\cindex{Kyorii}

\begin{boxnote}
\begin{verbatim}
% 2$BE@4V$N5wN%$NJ?J}(B
% \Kyorii#1#2#3
%   2$BE@(B #1, #2 $B$N5wN%$NJ?J}$r(B #3 $B$KBeF~(B
\end{verbatim}
\end{boxnote}\bigskip


\subsection{$BD>@~(B}
\subsubsection{$B#2E@$rDL$kD>@~(B}
2$BE@$rDL$kD>@~$rIA2h$9$k%3%^%s%I$,(B \cmd{Tyokusen} $B$G$9!%(B

\showexample[2$BE@$rDL$kD>@~(B](0.5)(0.45){example/tyoku01}

\cmd{Tyokusen}$B$N=q<0$G$9!%(B\cindex{Tyokusen}

\begin{boxnote}
\begin{verbatim}
2$BE@$rM?$($FD>@~$rIA2h$9$k!%(B
\Tyokusen[#1]#2#3[#4]#5[#6]#7
   #1 : \qbezier $B$N(B [...]
        #1$B$rM?$($J$$$H$-$OD>@~$r(B
        \qbezier $B$G$O$J$/!$(B\drawline $B$GIA2h$9$k!%(B
   #2,3 : $BD>@~>e$N#2E@(B
   #5 : $B:8C<$N(B x$B:BI8(B $B!J(B#4 $B$K(B [y] $B$H$9$l$P!$(By$B:BI8!K(B
   #7 : $B1&C<$N(B x$B:BI8(B $B!J(B#6 $B$K(B [y] $B$H$9$l$P!$(By$B:BI8!K(B
\end{verbatim}
\end{boxnote}

$BD>@~$rIA2hNN0h$$$C$Q$$$K$+$/$H$-$O!$(B\verb/#5, #7/$B$r(B
$B6u$K$9$k$3$H$,$G$-$^$9!#(B

\showexample[2$BE@$rDL$kD>@~(B](0.5)(0.45){example/tyoku04}

\subsubsection{1$BE@$HJ}8~%Y%/%H%k$K$h$kD>@~(B}
1$BE@$HJ}8~%Y%/%H%k$rM?$($FD>@~$rIA2h$9$k%3%^%s%I$,(B
\cmd{mTyokusen} $B$G$9!%(B

\showexample[1$BE@$HJ}8~%Y%/%H%k$r;XDj(B](0.5)(0.45){example/tyoku02}

$B$=$N=q<0$O(B\cindex{mTyokusen}
\begin{boxnote}
\begin{verbatim}
$B#1E@$HJ}8~%Y%/%H%k$rM?$($FD>@~$rIA2h$9$k!%(B
\mTyokusen[#1]#2#3[#4]#5[#6]#7
   #1 : \qbezier $B$N(B [...]
        #1$B$rM?$($J$$$H$-$OD>@~$r(B
        \qbezier $B$G$O$J$/!$(B\drawline $B$GIA2h$9$k!%(B
   #2 : $BD>@~>e$N#1E@(B
   #3 : $BJ}8~%Y%/%H%k(B
   #5 : $B:8C<$N(B x$B:BI8(B $B!J(B#4 $B$K(B [y] $B$H$9$l$P!$(By$B:BI8!K(B
   #7 : $B1&C<$N(B x$B:BI8(B $B!J(B#6 $B$K(B [y] $B$H$9$l$P!$(By$B:BI8!K(B
\end{verbatim}
\end{boxnote}

$BD>@~$rIA2hNN0h$$$C$Q$$$K$+$/$H$-$O!$(B\verb/#5, #7/$B$r(B
$B6u$K$9$k$3$H$,$G$-$^$9!#(B

\showexample[$BC<E@$N;XDj>JN,(B](0.5)(0.45){example/tyoku05}

\subsubsection{1$BE@$HJ}8~3Q$K$h$kD>@~(B}
1$BE@$HJ}8~3Q$rM?$($FD>@~$rIA2h$9$k%3%^%s%I$,(B
\cmd{kTyokusen} $B$G$9!%(B

\showexample[1$BE@$HJ}8~3Q$r;XDj(B](0.5)(0.45){example/tyoku06}

$B$=$N=q<0$O(B\cindex{kTyokusen}
\begin{boxnote}
\begin{verbatim}
$B#1E@$HJ}8~3Q$rM?$($FD>@~$rIA2h$9$k!%(B
\kTyokusen[#1]#2#3[#4]#5[#6]#7
   #1 : \qbezier $B$N(B [...]
        #1$B$rM?$($J$$$H$-$OD>@~$r(B
        \qbezier $B$G$O$J$/!$(B\drawline $B$GIA2h$9$k!%(B
   #2 : $BD>@~>e$N#1E@(B
   #3 : $BJ}8~3Q(B
   #5 : $B:8C<$N(B x$B:BI8(B $B!J(B#4 $B$K(B [y] $B$H$9$l$P!$(By$B:BI8!K(B
   #7 : $B1&C<$N(B x$B:BI8(B $B!J(B#6 $B$K(B [y] $B$H$9$l$P!$(By$B:BI8!K(B
\end{verbatim}
\end{boxnote}

$BD>@~$rIA2hNN0h$$$C$Q$$$K$+$/$H$-$O!$(B\verb/#5, #7/$B$r(B
$B6u$K$9$k$3$H$,$G$-$^$9!#(B

\subsubsection{$B@~<o$NJQ99(B}
$BD>@~$r<B@~$G$O$J$/!$E@@~$H$+GK@~$GIA2h$9$k$K$O(B \cmd{sensyu}$B$r:FDj5A$7$^$9!#(B
\cindex{sensyu}
$B%G%U%)%k%H$O(B
\begin{jquote}(4zw)
\begin{verbatim}
\def\sensyu{\drawline}
\end{verbatim}
\end{jquote}
$B$H$J$C$F$$$^$9!#(B

\showexample[$B@~<o$NJQ99(B](0.5)(0.45){example/tyoku07}

\subsubsection{$BD>@~(B \texorpdfstring{%
      \protect $ax+by+c=0$}{ax+by+c=0}}
$BJ}Dx<0(B $ax+by+c=0$ $B$GM?$($i$l$?D>@~$rIA2h$9$k(B \cmd{tyokusen} $B$G$9!%(B

\showexample[$BJ}Dx<0$N78?t;XDj(B](0.5)(0.45){example/tyoku03}

$B$=$N=q<0$O(B\cindex{tyokusen}
\begin{boxnote}
\begin{verbatim}
$BD>@~(B ax+by+c=0 $B$rIA2h$9$k!%(B
\tyokusen[#1]#2#3#4[#5]#6[#7]#8
   #1 : \qbezier $B$N(B [...]
        #1$B$rM?$($J$$$H$-$OD>@~$r(B
        \qbezier $B$G$O$J$/!$(B\drawline $B$GIA2h$9$k!%(B
   #2 : a
   #3 : b
   #4 : c
   #6 : $B:8C<$N(B x$B:BI8(B $B!J(B#5 $B$K(B [y] $B$H$9$l$P!$(By$B:BI8!K(B
   #8 : $B1&C<$N(B x$B:BI8(B $B!J(B#7 $B$K(B [y] $B$H$9$l$P!$(By$B:BI8!K(B
\end{verbatim}
\end{boxnote}

$B$3$N%3%^%s%I$G$b!$:BI8J?LLA4BN$KIA2h$9$k$H$-$O(B
\verb/#5, #7/$B$O6u$G$+$^$$$^$;$s!#(B

\subsubsection{$BD>@~(B \texorpdfstring{\protect $y=ax+b$}{y=ax+b}}
$B$3$l$O!$<!@a$N0l<!4X?t$N%0%i%UIA2h%3%^%s%I$r$4MxMQ$/$@$5$$!%(B

\subsubsection{\texorpdfstring{%
  $BE@(BP\protect\retu(x_1,y_1)$B$HD>@~(B$ax+by+c=0$$B$N5wN%(B}{%
  $BE@(BP(x1,y1)$B$HD>@~(Bax+by+c=0}}
$BE@(BP$B$HD>@~(B$l$$B$N5wN%$r5a$a$k$3$H$K$h$j!$E@(BP$B$rCf?4$H$7$F(B$l$$B$K@\$9$k1_$r(B
$BIA2h$9$k$3$H$,=PMh$^$9!%(B

    \showexample[\cmd{tentotyokusen}](0.52)(0.42){example/kyori03}\bigskip

$B$=$N=q<0$O(B\cindex{tentotyokusen}

\begin{boxnote}
\begin{verbatim}
% $BE@$HD>@~$N5wN%(B(1)
% $BE@(B P$(x1,y1)$ $B$HD>@~(B $ax+by+c=0$ $B$H$N5wN%$r5a$a$k!%(B
% \tentotyokusen#1#2#3#4#5{%
%   #1 : $BE@(BP
%   #2 : $a$
%   #3 : $b$
%   #4 : $c$
%   #5 : $BE@$HD>@~$N5wN%(B
\end{verbatim}
\end{boxnote}\bigskip

\subsubsection{$BE@(BP(x1,y1)$B$H(B2$BE@(BA,B$B$rDL$kD>@~$N5wN%(B}
$BD>@~$,$=$N>e$N(B2$BE@$G;XDj$5$l$?>l9g$G$9!%(B

    \showexample[\cmd{tentoTyokusen}](0.52)(0.42){example/kyori04}\bigskip

$B$=$N=q<0$O(B\cindex{tentoTyokusen}

\begin{boxnote}
\begin{verbatim}
% $BE@$HD>@~$N5wN%(B(2)
% $BE@(B P(x1,y1) $B$HD>@~(B AB $B$H$N5wN%$r5a$a$k!%(B
% \tentoTyokusen#1#2#3#4{%
%   #1 : $BE@(BP
%   #2 : $BE@(BA
%   #3 : $BE@(BB
%   #4 : $BE@$HD>@~$N5wN%(B
\end{verbatim}
\end{boxnote}\bigskip

\subsection{$B1_$N@\@~(B}
\subsubsection{$B1_<~>e$NE@$K$*$1$k@\@~(B}
$B1_<~>e$NE@$K$*$1$k@\@~$rIA2h$9$k$?$a$K!$(B
$B$=$NJ}8~%Y%/%H%k$r5a$a$k%3%^%s%I$,(B \cmd{ennoSessen} $B$G$9!%(B

\showexample[$B!!1_<~>e$NE@$K$*$1$k@\@~(B](0.5)(0.45){example/sessen01}

$B$=$N=q<0$O(B\cindex{ennoSessen}

\begin{boxnote}
\begin{verbatim}
$B1_<~>e$NE@$K$*$1$k@\@~$NC10LJ}8~%Y%/%H%k$r5a$a$k!%(B
\ennoSessen#1#2#3{%
   #1 : $B1_$NCf?4(B
   #2 : $B1_<~>e$NE@(B
   #3 : $B@\@~$NJ}8~%Y%/%H%k!JC10L%Y%/%H%k!K(B
\end{verbatim}
\end{boxnote}

\subsubsection{$B1_30$NE@$+$i$N@\@~(B}
$B1_$N30It$NE@$+$i1_$K0z$$$?@\@~$N(B2$B8D$N@\E@$r5a$a$k%3%^%s%I$,(B
\cmd{enniSessen}$B$G$9!%(B

\showexample[$B1_30$NE@$+$i$N@\@~(B](0.5)(0.45){example/sessen02}

$B=q<0$O(B\cindex{enniSessen}
\begin{boxnote}
\begin{verbatim}
$B1_$N30It$NE@$+$i1_$K0z$$$?@\@~$N@\E@$r5a$a$k!%(B
\enniSessen#1#2#3#4#5{%
   #1 : $B1_$NCf?4(B
   #2 : $B1_$NH>7B(B
   #3 : $B1_$N30It$NE@(B
   #4 : $B@\E@(B(1)
   #5 : $B@\E@(B(2)
      2$B$D$N@\E@$N$&$A(B
         x$B:BI8$N>.$5$$J}(B
            $BEy$7$$$H$-$O(B y$B:BI8$N>.$5$$J}(B
         $B$,(B #4 $B$KF~$j$^$9!%(B
\end{verbatim}
\end{boxnote}

\subsubsection{$BFs$D$N1_$N6&DL@\@~(B}
2$B$D$N1_$N6&DL30!JFb!K@\@~$rIA2h$9$k$?$a$N%3%^%s%I$,(B
\cmd{KTGAISessen}, \cmd{KTNAISessen} $B$G$9!#(B
$B=q<0$O(B \cindex{KTGAISessen}\cindex{KTNAISessen}
\begin{boxnote}
\begin{verbatim}
\KTGAISessen#1#2#3#4#5#6#7#8
\KTNAISessen#1#2#3#4#5#6#7#8
  #1 : $B1_(B1$B$NCf?4(B
  #2 : $B1_(B1$B$NH>7B(B
  #3 : $B1_(B2$B$NCf?4(B
  #4 : $B1_(B2$B$NH>7B(B
  #5 : $B6&DL@\@~(B1$B$H1_(B1$B$N@\E@(B
  #6 : $B6&DL@\@~(B1$B$H1_(B2$B$N@\E@(B
  #7 : $B6&DL@\@~(B2$B$H1_(B1$B$N@\E@(B
  #8 : $B6&DL@\@~(B2$B$H1_(B2$B$N@\E@(B
\end{verbatim}
\end{boxnote}
\noindent
$B$G!$6&DL@\@~$H1_$H$N@\E@$r5a$a!$(B
\cmd{Tyokusen} $B$J$I$G6&DL@\@~$r0z$-$^$9!#$=$N0lNc$G$9!#(B

\showexample[$B6&DL@\@~(B](0.9)(0.9){example/sessen03}

\subsection{$BH>D>@~(B}
$BC<E@$HJ}8~3Q$rO;==J,K!$G;XDj$7$FH>D>@~$rIA2h$9$k(B
$B%3%^%s%I$,(B \cmd{kHantyokusen} $B$G$9!#(B

\showexample[$BH>D>@~!JC<E@$HJ}8~3Q;XDj(B)](0.5)(0.45){example/tyoku08}

$BC<E@$HJ}8~%Y%/%H%k$r;XDj$9$k(B \cmd{mHantyokusen} $B$G$9!#(B

\showexample[$BH>D>@~!JC<E@$HJ}8~%Y%/%H%k;XDj(B)](0.5)(0.45){example/tyoku09}

$BC<E@$HDL2a$9$k(B1$BE@$r;XDj$9$k(B \cmd{Hantyokusen} $B$G$9!#(B

\showexample[$BH>D>@~!JC<E@$HDL2a$9$k(B1$BE@;XDj(B)](0.5)(0.45){example/tyoku10}

$B$3$l$i$NH>D>@~$N@~<o$rJQ99$9$k$K$O!$(B
\begin{jquote}
\begin{verbatim}
<sensyu=....>
\end{verbatim}
\end{jquote}
$B%*%W%7%g%s$rMxMQ$7$^$9!#0lNc$G$9!#(B

\showexample[$B:?@~$GH>D>@~(B](0.5)(0.45){example/tyoku11}

$BH>D>@~$NIA2hNN0h$K$*$1$kC<E@$rJV$9%3%^%s%I$,(B\cmd{HtyokuT}$B$G$9!#(B
$BH>D>@~$K%i%Y%k$r$D$1$?$j$9$k$N$KJXMx$G$7$g$&!#(B
\cindex{HtyokuT}

\begin{showEx}{\cmd{HtyokuT}}
\begin{zahyou}[ul=10mm]%
(-1.5,1.5)(-1.5,1.5)
  \En\O{1}
  \kHantyokusen\O{60}
  \Put\HtyokuT[se]{$\ell$}
\end{zahyou}
\end{showEx}


\section{関数のグラフ(0) \textsf{Perl}との連携}
関数のグラフを描画するのに,\textsf{Perl}を用いる方法があります。
関数式の記述が簡単でお勧めですが,機種に依存する面もありますので
別ファイルにしてあります。
\textsf{samplePp.tex}をご覧ください。

%\section{関数のグラフ(1) 整関数}
\section{関数のグラフ(1) 整関数}
この節のマクロは,\textsf{emathPg.sty}で定義されています。

\subsection{一次関数のグラフ}
整関数は係数を降べき順にコンマで切って並べることにより表現します.
例えば,$y=2x-1$ は \verb+{2,-1}+ と表されます.
そのグラフを描かせてみます.\cindex{Gurafu}

\showexample[一次関数のグラフ](0.5)(0.45){example/func11}
\vspace{3\baselineskip}

\begin{boxnote}
\begin{verbatim}
\Gurafu<#1>[#2]#3[#4]#5[#6]#7
  #1 : xの刻み値(3次以上の場合)
  #2 : 曲線上の点の個数
  #3 : 整関数(コンマ区切りの係数並び)
  #4 : y または Y
  #5 : グラフを描画する x の下限
  #6 : y または Y
  #7 : グラフを描画する x の上限

  #4 で y を指定したときは y=#5 となる x が下限となる.
  #6 で y を指定したときは y=#7 となる x が上限となる.
  二次関数では,y=y1 となる x は二つあるが,
      y で小さい方を, Y で大きい方
  を表す.
\end{verbatim}
\end{boxnote}

これも引数の多いコマンドですが,基本的な使用法は
\begin{jquote}
\cmd{Gurafu}\verb+{整関数}{x の下限}{x の上限}+
\end{jquote}
です.

なお,\textsf{zahyou}環境内では,$x$の下限,上限を \cmd{xmin},
\cmd{xmax} で引用することができます.$y$ についても
\cmd{ymin}, \cmd{ymax} が使えます.

さて,上のグラフで左下が描画範囲から飛び出しています.

この場合,\verb+/\xmin\leqq x \leqq xmax+ と指定するのではなく,
左端は \texttt{y} が \verb+\ymin+ となるときの \texttt{x} の値,
と指定したいものです.これを実現するのが \texttt{[y]} オプションです.

\showexample(0.5)(0.45){example/func12}

係数が分数のときは,それを小数で近似します。

\showexample(0.5)(0.45){example/func13}

分数のままで描画させる \cmd{GurafuB} もあります。
分数は \verb+-1/2+ のように表現します。

\showexample(0.5)(0.45){example/func14}

\subsection{二次関数のグラフ}
つぎは二次関数のグラフです.

\showexample[二次関数のグラフ](0.5)(0.45){example/func21}

これは,ベジェ曲線を描くコマンド \cmd{qbezier} を用いていますが,
頂点の辺りと他の部分とで点の稠密度が異なるようです.
頂点近辺と他の部分を分けて描く方が良さそうです.

\showexample[二次関数のグラフ(2)](0.5)(0.45){example/func22}

一部を点線で描画したいこともあります.
その場合は,\cmd{qbezier} のオプション引数を与えます.

\showexample[一部を点線で](0.7)(0.65){example/func23}

\subsection{二直線の交点}
2つの整関数のグラフの交点を求めるコマンドが \cmd{GKouten} です.
\cindex{GKouten}

\showexample[二直線の交点](0.75)(0.45){example/kouten11}

\cmd{GKouten} の書式です.

\begin{boxnote}
\begin{verbatim}
\GKouten#1#2#3#4
  #1 : 整関数1
  #2 : 整関数2

#1, #2 がともに1次関数のときは
  #3 : 交点のx座標
  #4 : 交点のy座標
#1, #2 のいずれかが2次関数のときは
  #3 : 2つの交点のうち,x座標の小さい方
  #4 : 2つの交点のうち,x座標の大きい方
\end{verbatim}
\end{boxnote}

\subsection{直線と放物線の交点}

\showexample[直線と放物線の交点](0.8)(0.45){example/kouten12}

\subsection{三次関数のグラフ}
三次以上の関数のグラフは折れ線で近似されます.
xの刻み値のデフォルトは 0.05 です.この値は \verb+<#1>+ オプション
で変更することができます.

\showexample[三次関数のグラフ](0.65)(0.45){example/func31}

範囲外は自動的にクリッピングされます.

グラフ(の一部)を点線で描画するには,\cmd{Gurafu} ではなく,
一般関数のグラフ描画コマンド \cmd{yGurafu} をご利用ください.


%\section{関数のグラフ(2) 一般}
\section{$B4X?t$N%0%i%U(B(2) $B0lHL(B}
$B$3$N@a$N%^%/%m$O(B,\textsf{emathPg.sty}$B$GDj5A$5$l$F$$$^$9!#(B

\subsection{$B4X?t7W;;<0$N5-=R(B}
$B$=$NB>$N4X?t$N%0%i%U$rIA2h$9$k$K$O!$4X?t$N7W;;<0$rM?$($kI,MW$,$"$j$^$9!%(B
$BNc$($P!$(B$y=\bunsuu{1}{x}$ $B$N7W;;<0$O<!$N$h$&$K5-=R$7$^$9!%(B
\begin{screen}
\begin{verbatim}
\def\Fx#1#2{\Div{1}{#1}\y\edef#2{\y}}
\end{verbatim}
\end{screen}

$B$9$J$o$A!$4X?tDj5A$OI,$:(B2$B$D$N0z?t$r$H$j$^$9!%(B
\begin{jquote}
\begin{verbatim}
#1 $B$OFHN)JQ?t$NCM(B
#2 $B$O7W;;7k2L$r<u$1<h$k%3%s%H%m!<%k%7!<%1%s%9(B
\end{verbatim}
\end{jquote}
$B$G$9!%$3$NDj5A$N85$G(B
\begin{jquote}
\begin{verbatim}
\Fx{4}\kekka
\end{verbatim}
\end{jquote}
$B$H5-=R$9$k$H!$(B\cmd{kekka}$B$K(B $\bunsuu14=0.25$ $B$,%;%C%H$5$l$^$9!%(B

$B$^$?!$;MB'1i;;$O(B \textsf{eclarith.sty} $B$GDj5A$5$l$F$$$k(B
\begin{boxnote}
\begin{verbatim}
$B2CK!$,(B \Add#1#2#3 
  $B$G(B #1 + #2 $B$,(B #3 $B$N%3%s%H%m!<%k%7!<%1%s%9$KM?$($i$l!$(B
$B8:K!$O(B \Sub#1#2#3
$B>hK!$O(B \Mul#1#2#3
$B=|K!$O(B \Div#1#2#3
\end{verbatim}
\end{boxnote}
\cindex{Add}\cindex{Sub}\cindex{Mul}\cindex{Div}
\noindent
$B$rMQ$$$^$9!%(B

$B;O$a$N4X?tDj5A$O(B
\begin{screen}
\begin{verbatim}
\def\Fx#1#2{\Div{1}{#1}#2}
\end{verbatim}
\end{screen}
$B$H4JN,2=$9$k$3$H$b$G$-$^$9!#(B

$BAH9~$_4X?t$K$D$$$F$O(B
\begin{boxnote}
\begin{verbatim}
$B@589$,(B \Sin#1#2 $B$G(B sin(#1) $B$,(B #2 $B$K%;%C%H$5$l!$(B
$BM>89$,(B \Cos#1#2
\end{verbatim}
\end{boxnote}
\cindex{Sin}\cindex{Cos}
\noindent
$B$O!$(B\textsf{eclarith.sty} $B$GDj5A$5$l$F$$$^$9!%(B
$BJ?J}:,$O(B \cmd{Sqroot} $B$,Dj5A$5$l$F$$$^$9$,!$(B\cindex{Sqroot}
\cmd{Sqroot}\verb+{0}+ $B$,(B 1 $B$H$J$k%P%0!J;EMM$+$b!K$,$"$j$^$9$N$G(B
\cmd{Heihoukon} $B$r$+$V$;$^$9!%;X?t!&BP?t$b$[$7$$$N$G(B\cindex{Heihoukon}
\begin{boxnote}
\begin{verbatim}
$BJ?J}:,$,(B   \Heihoukon#1#2
$B;X?t4X?t$,(B \Exp#1#2
$BBP?t4X?t$,(B \Log#1#2

$B;03Q4X?t$NDI2C(B \Tan#1#2
               \Sec#1#2
\end{verbatim}
\end{boxnote}
\cindex{Exp}\cindex{Heihoukon}\cindex{Tan}\cindex{Sec}
\noindent
$B$r(B \textsf{emathP.sty} $B$GDI2C$7$F$$$^$9!%$?$@$7!$(B
$B%0%i%U$,IT<+A3$K8+$($J$$DxEY$N@:EY$G$9!%(B

$B$^$?!$1_<~N((B $\pi$, $B<+A3BP?t$NDl(B $e$ $B$J$I$NDj?t$O<!$NJQ?tL>$GMxMQ$G$-$^$9!%(B

\cindex{Pie}\cindex{Pii}\cindex{Pih}\cindex{Piq}
\begin{boxnote}
\begin{verbatim}
% \Pi is a greek letter.
\def\Piq{0.78539816}%     $B&P(B/4
\def\Pih{1.57079633}%     $B&P(B/2
\def\Pie{3.14159265}%     $B&P(B
\def\Pii{6.28318531}%     2$B&P(B
\end{verbatim}
\end{boxnote}
\noindent
$B$,(B \textsf{eclarith.sty} $B$GDj5A$5$l$F$$$^$9!%(B
\textsf{emathP.sty}$B$G$O<!$N$b$N$rDI2C$7$F$$$^$9!#(B
\cindex{Pis}\cindex{Pit}
\begin{boxnote}
\begin{verbatim}
\def\Pis{0.52359878}%     $B&P(B/6
\def\Pit{1.04719755}%     $B&P(B/3
\def\Pihiii{4.71238898}%  3$B&P(B/2
\def\Pitii{2.094395}%     2$B&P(B/3
\def\Pitiv{4.188790}%     4$B&P(B/3
\def\Pitv{5.235988}%      5$B&P(B/3
\def\Piqiii{2.356194}%    3$B&P(B/4
\def\Piqv{3.926991}%      5$B&P(B/4
\def\Piqvii{5.497787}%    7$B&P(B/4
\def\Pisv{2.617994}%      5$B&P(B/6
\def\Pisvii{3.665191}%    7$B&P(B/6
\def\Pisxi{5.759587}%    11$B&P(B/6
\end{verbatim}
\end{boxnote}
\noindent
$B$5$i$K$O!$(B75\Deg $B$r%i%8%"%s$KJQ49$9$k$K$O(B \verb/\DegRad{75}\kaku/ $B$G(B 
\cmd{kaku} $B$KJQ49CM$,5a$a$i$l$^$9!#(B

$B<+A3BP?t$NDl$NJ}$O(B \textsf{emathP.sty} $B$G<!$N$h$&$KDj5A$5$l$F$$$^$9!%(B

\cindex{Napier}\cindex{Napierii}\cindex{Napiermi}
\begin{boxnote}
\begin{verbatim}
\def\Napier{2.718281828}% $B<+A3BP?t$NDl(B e
\def\Napierii{7.389056}%  e^2
\def\Napiermi{0.367879441}% 1/e
\end{verbatim}
\end{boxnote}

\subsection{\texorpdfstring{$y=f(x)$}{y=f(x)}$B$N%0%i%U(B}
\subsubsection{$BJ,?t4X?t(B}
$BA0@a$G(B $y=\bunsuu{1}{x}$ $B$N4X?tDj5A$r>R2p$7$^$7$?$,!$(B
$B$=$l$rMQ$$$F$3$N4X?t$N%0%i%U$rIA2h$7$^$9!%(B

\showexample[$y=\bunsuu{1}{x}$$B$N%0%i%U(B](0.51)(0.44){example/func41}
\cindex{yGurafu}

$B$9$J$o$A0lHL$N4X?t%0%i%U$rIA2h$9$k%3%^%s%I(B \cmd{yGurafu} $B$N=q<0$O(B
\begin{boxnote}
\begin{verbatim}
\yGurafu(#1)(#2)#3#4#5
    #1 : x$B$N9o$_CM(B $B%G%U%)%k%H(B=0.05
    #2 : $BE@@~$GIA2h$9$k$H$-$NIA2h$9$kItJ,$N(B x $B$N%l%s%8(B
    #3 : $B4X?t<0(B
    #4 : x$B$N;O$aCM(B
    #5 : x$B$N=*$jCM(B
\end{verbatim}
\end{boxnote}

\subsubsection{$BL5M}4X?t(B}
\showexample[$y=\sqrt{x}$$B$N%0%i%U(B](0.51)(0.44){example/func42}

\subsubsection{$B;03Q4X?t(B}
\showexample[$y=\sin 2x$$B$N%0%i%U(B](0.7)(0.7){example/func43}

\subsubsection{$B;X?t4X?t(B}
\showexample[$y=e^{-x^2}$$B$N%0%i%U(B](0.9)(0.6){example/func44}

\subsubsection{$BBP?t4X?t(B}
\showexample[$y=\bunsuu{\log x}{x}$$B$N%0%i%U(B](0.9)(0.6){example/func45}

\subsection{\texorpdfstring{$x=g(y)$}{x=g(y)} $B$GM?$($i$l$?6J@~(B}
$x=g(y)$ $B$N7A$GM?$($i$l$?6J@~$rIA2h$9$k$K$O(B \cmd{xGurafu} $B$rMQ$$$^$9!%(B
\cindex{xGurafu}

\showexample[$x=\sin y$ $B$N%0%i%U(B](0.55)(0.35){example/func50}


\subsection{$BG^2pJQ?tI=<((B}
$x=f(t)$, $y=g(t)$ $B$H!$(B2$B$D$N4X?t$rM?$($kI,MW$,$"$j$^$9!%$9$J$o$A(B
\cindex{bGurafu}

\begin{jquote}
\begin{verbatim}
\bGurafu(#1)(#2)#3#4#5#6
   #1 : t $B$N9o$_CM!J%G%U%)%k%HCM$O(B 0.05 $B!K(B
   #2 : $BE@@~$GIA2h$9$k$H$-$NIA2h$9$kItJ,$N(B t $B$N%l%s%8(B
   #3 : x=f(t)
   #4 : y=g(t)
   #5 : t $B$N;O$aCM(B
   #6 :     $B=*$jCM(B
\end{verbatim}
\end{jquote}
$B$H$$$&=q<0$G$9!%$G$O!$%5%$%/%m%$%I$rIA$$$F$_$^$9!%(B

\showexample[$B%5%$%/%m%$%I(B](1)(0.8){example/func46}

$BFs<!6J@~$OG^2pJQ?tI=<($rMxMQ$9$k$N$,NI$$$G$7$g$&!%(B

\showexample[$BAP6J@~(B](1)(0.75){example/func48}

\subsection{$B6KJ}Dx<0(B}
$r=f(\theta)$ $B$rM?$($^$9!%(B
\cindex{rGurafu}

\showexample[$B%l%`%K%9%1!<%H(B](0.97)(0.75){example/func47}

\begin{boxnote}
\begin{verbatim}
\rGurafu(#1)(#2)#3#4#5
    #1 : $B&H$N9o$_CM!J%G%U%)%k%H$O(B 0.05$B!K(B
    #2 : $BE@@~$GIA2h$9$k$H$-$NIA2h$9$kItJ,$N(B $B&H(B $B$N%l%s%8(B
    #3 : $B4X?t(B r=f($B&H(B)
    #4 : $B&H$N=i4|CM(B
    #5 : $B&H$N=*CM(B
\end{verbatim}
\end{boxnote}

\subsection{$B%0%i%U$NE@@~IA2h(B}
$B%0%i%U$N0lIt$rE@@~$GIA2h$7$?$$$H$-$O!$(B

\centerline{\cmd{yGurafu}, \cmd{bGurafu}, \cmd{rGurafu}}
$B$NBh(B2$B$N(B \texttt{(...)} $B%*%W%7%g%s$rMxMQ$7$^$9!%(B

\showexample[$B%0%i%U$NE@@~IA2h(B](1)(0.85){example/func49}


\subsection{$B%9%W%i%$%s6J@~(B}
\bgroup
$B$3$N@a$O!$(Bdvi-ware $B0MB8$G$9!#(B
\texttt{Windows}$B>e$N(B\texttt{dviout.exe}$B$rA[Dj$7$F$$$^$9!#(B
\textsf{tpic}$B$K$O!$(B\textsf{spline}$B6J@~$rIA2h$9$k%3%^%s%I(B
\begin{jquote}
\begin{verbatim}
sp
\end{verbatim}
\end{jquote}
$B$,$"$j$^$9!#(B

\special{Bz 1}%
\begin{showEx}(.54,.4){\cmd{Drawtpic}}
\begin{zahyou}[ul=8mm](-1,5)(-1,4.5)
  \zahyouMemori[g]
  \tenretu{A(1,3)nw;B(2,4)n;%
      C(3,3)ne;D(4,0)ne}
  \Drawtpic{\O\A\B\C\D}
  \kuromaru{\O;\A;\B;\C;\D}
\end{zahyou}
\end{showEx}

$B?7@_%3%^%s%I(B\cmd{Drawtpic}$B$N0z?t$K$O!$E@Ns$rM?$($^$9!#(B
$B%9%W%i%$%s6J@~$H$$$C$F$$$^$9$,!$$4MwD:$1$P$o$+$k$h$&$K!$(B
$BN>C<$O;XDj$5$l$?E@$K$J$j$^$9$,!$ESCfE@(BA, B, C$B$O(B
$B@)8fE@$H$7$F;H$o$l!$DL2a$O$7$^$;$s!#(B
$B$=$N0UL#$G$O%Y%8%'6J@~$H$$$&$Y$-$G$7$g$&!#(B

\textsf{tpic}$B$N3HD%5!G=$K$O!$ESCfE@(BA, B, C$B$bDL2a$9$k%9%W%i%$%s6J@~$H$9$k(B
$B5!G=$,$"$j$^$9$,!$$3$l$r%5%]!<%H$7$F$$$k(B dvi $B%&%'%"$O>/$J$$$h$&$G$9!#(B

dviout $B$O$3$N3HD%5!G=$r%5%]!<%H$7$F$$$^$9$+$i!$(B
Windows $B$G(B dviout $B$rMQ$$$?>l9g$O!$<!$N%j%9%H$,M-8z$G$9!#(B

\begin{showEx}(.54,.4){\cmd{Drawtpic}}
\begin{zahyou}[ul=8mm](-1,5)(-1,4.5)
  \zahyouMemori[g]
  \tenretu{A(1,3)nw;B(2,4)n;%
      C(3,3)ne;D(4,0)ne}
  \Drawtpic[tpicBz=0]{\O\A\B\C\D}
  \kuromaru{\O;\A;\B;\C;\D}
\end{zahyou}
\end{showEx}

$B;DG0$J$,$i!$(Bdvips(k), dvipdfm(x) $B$O!$$3$N3HD%5!G=$r%5%]!<%H$7$F$$$^$;$s$+$i!$(B
pdf $B$r:n$k$3$H$O$G$-$^$;$s!#(B
\egroup


%\section{塗りつぶし(2)}
\section{塗りつぶし(2)}
この節のマクロは,\textsf{emathPg.sty}で定義されています。

\subsection{整関数のグラフで囲まれる図形の塗りつぶし}
放物線 $y=x^2$ の上にある2点 A$(-1,~1)$, B(2, 4) を結ぶ線分と
放物線とで囲まれる図形を塗りつぶします.

\showexample[放物線と弦で囲まれる図形](.5)(0.4){example/nuri01}

\cmd{Nuri}の書式です.\cindex{Nuri}
\begin{boxnote}
\begin{verbatim}
\Nuri[#1](#2)#3#4#5
    #1 : 濃さ
    #2 : x の刻み値
    #3 : 整関数
    #4 : x1
    #5 : x2
#3 で与えられる整関数のグラフとその上の
2点 (x1,y1), (x2,y2) を結ぶ弦との間を
指定された濃さで塗りつぶします.
\end{verbatim}
\end{boxnote}

次は2つの整関数で挟まれた部分を塗りつぶします.

\showexample[2つの放物線で囲まれる図形](.5)(0.4){example/nuri02}

\cmd{Nurii}の書式です.\cindex{Nurii}
\begin{boxnote}
\begin{verbatim}
\Nurii[#1](#2)#3#4#5#6
    #1 : 濃さ
    #2 : x の刻み値
    #3 : 整関数1
    #4 : 整関数2
    #5 : x1
    #6 : x2
上下は #3 と #4 で与えられる整関数のグラフ
左右は直線 $x=x_1$, $x=x_2$ で挟まれる部分を
指定された濃さで塗りつぶします.
\end{verbatim}
\end{boxnote}

\subsection{一般の関数のグラフで囲まれる図形の塗りつぶし}
整関数用の \cmd{Nuri}, \cmd{Nurii} は,一般関数ではそれぞれ
\cmd{yNuri}, \cmd{yNurii} に対応します.\cindex{yNuri}\cindex{yNurii}

\showexample[一般関数と弦で囲まれる図形](.5)(0.4){example/nuri03}

\showexample[2つの一般関数で囲まれる図形](.5)(0.4){example/nuri04}

\subsection{$x=g(y)$ で与えられる曲線の塗りつぶし}
曲線が $x=g(y)$ で与えられる場合は、\cmd{yNurii} に代えて \cmd{xNurii}
などを用います.\cindex{xNurii}

\showexample[$x=g(y)$で囲まれる図形](.6)(0.5){example/nuri07}

\subsection{媒介変数表示の曲線で囲まれる図形の塗りつぶし}
整関数の \cmd{Nuri} に対応するものが \cmd{bNuri} です.
\cindex{bNuri}

\showexample[媒介変数表示曲線で囲まれる図形](1)(0.7){example/nuri05}

\subsection{極方程式であらわされた曲線で囲まれる図形の塗りつぶし}
整関数の \cmd{Nuri} に対応するものが \cmd{rNuri} です.
\cindex{rNuri}

\showexample[極方程式で表される曲線の塗りつぶし](1)(0.7){example/nuri06}



%\section{斜線塗り(2)}
\section{斜線塗り(2)}
この節のマクロは,\textsf{emathPg.sty}で定義されています。

前節の,曲線で囲まれた図形の塗りつぶしコマンドの末尾に \verb+*+ を
付加すると,斜線による塗りつぶしを行います.

代表として,\cmd{yNurii*} の使用例を見ていただきましょう.

$y=\sin x$, $y=\cos x$ と2直線 $x=0$, $x=2\pi$ で囲まれる図形を
斜線で塗りつぶしています.斜線の方向角を60度としてあります.

\showexample[2曲線で囲まれる図形の斜線塗り](.6)(0.6){example/hatch07}

\cmd{yNurii*}の書式です.\cindex{yNurii*}
\begin{boxnote}
\begin{verbatim}
\yNurii*[#1]<#2>(#3)#4#5#6#7
    #1 : 斜線の方向角(デフォルト値 = 45)単位は度
    #2 : 斜線の間隔(デフォルト値 = 0.125)
    #2 : 折れ線近似をする時の x の刻み値(デフォルト 0.1)
    #3 : 関数1
    #4 : 関数2
    #5 : x1
    #6 : x2
上下は #3 と #4 で与えられる関数のグラフ
左右は直線 $x=x_1$, $x=x_2$ で挟まれる部分を
斜線で塗りつぶします.
\end{verbatim}
\end{boxnote}

このほか \cmd{Nuri*}, \cmd{Nurii*}, \cmd{yNuri*}, \cmd{xNurii*},\cmd{bNuri*},
\cmd{rNuri*} も同様です.
\cindex{Nuri*}\cindex{Nurii*}\cindex{yNuri*}\cindex{xNurii*}
\cindex{bNuri*}\cindex{rNuri*}

斜線を点線・破線で描画する方法については,
§\ref{S-hasen} (\pageref{S-hasen} ページ)をご覧ください.

\showexample[点線による斜線塗り](.6)(0.6){example/hatch10}


%\section{空間座標}
\section{空間座標}
\subsection{\textsf{Zahyou} 環境}
\textsf{zahyou}環境が平面座標を扱ったのに対して,
\textsf{Zahyou}環境は空間座標を扱います。

単純な例です。

\showexample[空間座標](0.58)(0.36){example/zahyou31}

簡単に解説します。

まず,\textsf{Zahyou}環境は
\index{Zahyou@Zahyou 環境}
\begin{jquote}
$x$の範囲,$y$の範囲,$z$の範囲
\end{jquote}
を\vspace{-\baselineskip}
\begin{jquote}
\begin{verbatim}
(xの下限,xの上限),(yの下限,yの上限),(zの下限,zの上限)
\end{verbatim}
\end{jquote}
の形で与えます。これらの値は,\textsf{Zahyou}環境内では,順に
\begin{jquote}
\begin{verbatim}
\Xmin, \Xmax, \Ymin, \Ymax, \Zmin, \Zmax
\end{verbatim}
\end{jquote}
で引用することができます。

点は \verb/\def\P{(1,2,3)}/のように3次元ベクトルで与えます。

次いで,\verb/\iiiKuromaru\P/で点Pに黒丸を打ちます。\cindex{iiiKuromaru}
座標平面で定義された多くのコマンドの先頭に\texttt{`iii'}をつけた
空間座標用のコマンドが定義されています。現時点で定義されているコマンドは
次の通りです。\vspace{-.5\baselineskip}
\begin{jquote}
\begin{verbatim}
\iiiPut, \iiiPutStr
\iiiBunten
\iiiDrawline, \iiiDashline, \iiiArrowLine
\iiiTyokkaku, \iiiHen_ko
\iiiKuromaru, \iiiSiromaru
\iiiKyori, \iiiKyorii
\iiiAddvec, \iiiSubvec, \iiiMulvec
\iiiNuritubusi
\iiibGurafu
\end{verbatim}
\end{jquote}
\cindex{iiiPut}\cindex{iiiPutStr}
\cindex{iiiKyori}\cindex{iiiKyorii}
\cindex{iiiBunten}
\cindex{iiiDrawline}\cindex{iiiDashline}\cindex{iiiArrowLine}
\cindex{iiiTyokkaku}\cindex{iiiHen\_ko}
\cindex{iiiKuromaru}\cindex{iiiSiromaru}
\cindex{iiiKyori}\cindex{iiiKyorii}\cindex{iiiNuritubusi}
\cindex{iiiAddvec}\cindex{iiiSubvec}\cindex{iiiMulvec}

空間特有のコマンドについては,次節以降でとりあげます。

さて,座標軸の向き,単位長の変更についてのオプション機能です。

\textsf{Zahyou}環境の\texttt{[...]}オプションで$x$, $y$, $z$軸の
単位ベクトルを指定できるようにしてあります。
ただし,それらは描画する平面を座標平面と見立てての成分表示です。
デフォルトでは
\begin{jquote}
\begin{verbatim}
z軸:(0, 1)
y軸:(1, 0)
x軸は \kyokuTyoku(.667,-138)
\end{verbatim}
\end{jquote}
となっています。これを変更してみます。

\showexample[基本単位ベクトルの変更](0.58)(0.36){example/zahyou32}

\subsection{角錐}
角錐を描画するのにコマンド\cmd{Kakusui}を用意しました。
O(0, 0, 0), A(1, 0, 0), B(0, 1, 0)を頂点とする三角形OABを底面とし,
C(0, 0, 1)を頂点とする角錐---四面体を描画してみます。

\showexample[角錐](0.58)(0.36){example/kakusui3}

ここでは,座標軸を描画しないように指定するのに,
\cmd{Drawaxisfalse}としています。

\cmd{Kakusui}の書式です。\cindex{Kakusui}
\begin{boxnote}
\begin{verbatim}
\Kakusui[#1]#2#3#4
   #2 : 見える頂点列
   #3 : 見えない頂点列
   #4 : 錐の頂点
\end{verbatim}
\end{boxnote}

図の見えない部分の点線のスタイルは,デフォルトでは

\begin{jquote}
\begin{verbatim}
\def\iiiTensen{\iiiDashline[40]{.1}}
\end{verbatim}
\end{jquote}
となっていますが,これを再定義してみます。
\cindex{iiiTensen}

\showexample[\cmd{iiiTensen}の再定義](0.58)(0.36){example/kakusui4}

\subsection{角柱}
立方体,直方体,平行六面体を描画するために\cmd{Kakutyuu}を用意
してあります。平行六面体を描画してみます。

\showexample[角柱](0.58)(0.36){example/kakutyu3}

直方体ではないことを示すため,$z$軸方向の単位ベクトルを
変更した斜交座標系を用いています。

\cmd{Kakutyuu}の書式です。\cindex{Kakutyuu}
\begin{boxnote}
\begin{verbatim}
\Kakutyuu#1#2#3
   #1 : 見える頂点列
   #2 : 見えない頂点列
   #3 : 高さベクトル
\end{verbatim}
\end{boxnote}



\subsection{直線と平面の交点}
直線と平面の交点を求めるコマンドは,次の4つを用意してあります。
\begin{jquote}
\cmd{PandL}, \cmd{Pandl}, \cmd{pandL}, \cmd{pandl}
\end{jquote}
ここで,
\begin{jquote}
\texttt P は3点を指定した平面\\
\texttt p は点と法線ベクトルを指定した平面\\
\texttt L は2点を指定した直線\\
\texttt l は点と方向ベクトルを指定した直線
\end{jquote}
を意味します。

一例として直方体OABC-DEFGの対角線OFと3点A, C, Dを含む平面との
交点Pを描画してみます。


\showexample[直線と平面](0.52)(0.425){example/PandL01}

書式です。
\cindex{iiiKuromaru}\cindex{iiiSiromaru}
\begin{boxnote}
\begin{verbatim}
\PandL#1#2#3#4#5#6
    3点#1, #2, #3を通る平面と,
    2点#4, #5を通る直線との交点を#6に
\Pandl#1#2#3#4#5#6
    3点#1, #2, #3を通る平面と,
    点#4を通り方向ベクトルが#5の直線との交点を#6に
\pandL#1#2#3#4#5
    点#1を通り法線ベクトルが#2の平面と,
    2点#3, #4を通る直線との交点を#5に
\pandl#1#2#3#4#5
    点#1を通り法線ベクトルが#2の平面と,
    点#3を通り方向ベクトルが#4の直線との交点を#5に
\end{verbatim}
\end{boxnote}


\subsection{垂線}
座標平面で,点から直線に下した垂線の足を求めるコマンド\cmd{Suisen}の
3次元版の話しです。

書式です。
\cindex{LSuisen}\cindex{lSuisen}\cindex{PSuisen}\cindex{pSuisen}
\begin{boxnote}
\begin{verbatim}
\LSuisen#1#2#3#4
    点 #1 から直線 #2#3 へ下ろした垂線の足を #4 にセット
\lSuisen#1#2#3#4
    点#1から,点#2を通り,方向ベクトルが#3の直線への垂線
\pSuisen#1#2#3#4
    点#1から,#2を通り法線ベクトルが#3である平面
    に下ろした垂線の足を#4に与える。
\PSuisen#1#2#3#4#5
    点#1から,三点#2,#3,#4を通る平面
    に下ろした垂線の足を#5に与える。
\end{verbatim}
\end{boxnote}

例として,四面体OABCの頂点Oから底面ABCに下した垂線OPと,
Oから線分ABに下した垂線OHを作図します。

\showexample[垂線](0.58)(0.36){example/PSuisen1}

\subsection{空間曲線}
座標平面で媒介変数表示された曲線を描画するコマンド\cmd{bGurafu}
の3次元版が\cmd{iiibGurafu}です。書式は\cindex{iiibGurafu}

\begin{boxnote}
\begin{verbatim}
\iiibGurafu(#1)(#2)#3#4#5#6#7
   #1 : t の刻み値(デフォルト値は 0.05 )
   #2 : 点線で描画するときの描画する部分の t のレンジ
   #3 : x=f(t)
   #4 : y=g(t)
   #5 : z=h(t)
   #6 : t の始め値
   #7 :     終り値
\end{verbatim}
\end{boxnote}

具体例として,円柱螺旋を描画してみましょう。

\showexample[円柱螺旋](1)(0.9){example/rasen31}

なお,空間曲線を近似する折れ線を得るコマンド
\cmd{iiiBKinziOresen}もありますが,これは \textsf{perl} との
連携機能を必要とします。
\cindex{iiiBKinziOresen}


%\section{空間図形}
%\section{空間図形}
\subsection{円柱}

\showexample[円柱](0.9)(0.4){example/entyu01}

\subsection{円錐}

\showexample[円錐](0.9)(0.45){example/ensui01}

\subsection{円錐台}

\showexample[円錐台](0.9)(0.4){example/ensui02}

\subsection{四面体}

\showexample[四面体](0.9)(0.6){example/simentai}

\subsection{四角錐}

\showexample[四角錐](0.9)(0.6){example/kakusui1}

\subsection{平行六面体}

\showexample[平行六面体](0.9)(0.6){example/tamenta1}

基本ベクトルの数値を変更すると,直方体・立方体も作れます.
頂点の記号はエディタの一括置換機能を用いることを想定しています.

\showexample[平行六面体](0.9)(0.6){example/tamenta2}


%\section{作表}
\section{作表}
この節の環境,コマンド類は \textsf{emathT.sty} で定義されています。
したがって,プリアンブルで
\begin{jquote}
  \verb/\usepackage{emathT}/
\end{jquote}
を宣言しておく必要があります.

\subsection{\textsf{Hyou} 環境}
\subsubsection{列幅指定}
表を作成するには,\textsf{tabular}環境,\textsf{array}環境があります.
これらの環境で作成される表の列幅は,列の中に置かれる内容によって
自動的に定まります.これは便利ですが,ときには列幅を指定したいことも
あります.
そのための環境として \textsf{Hyou}環境を作りました.
その一例です.

\showexample[列幅一定の表](1)(.9){example/hyou01}

\textsf{Hyou}環境は,実質 \textsf{tabular}環境と同じです.
\index{Hyou @ Hyou 環境}
ただ,欄指定子として \texttt{l, c, r} に加え,\texttt{L, C, R} が
追加され,これらは欄の横幅を引数にとります.上の例では,
\verb/*{4}{C{4\zw}|}/として,4欄すべての横幅を\texttt{4\zw}の
中央揃えと指定しています.

\subsubsection{カラムに斜線}
また,この環境内では,\cmd{sya} コマンドで欄に斜線を引くことができます.
\cmd{sya}コマンドの書式です.\cindex{sya}

\begin{boxnote}
\begin{verbatim}
   \sya(#1)[#2][#3]<#4>
       #1 : 横幅
       #2 : 斜線の向き
           n 右上から左下(デフォルト)
           r 左上から右下
           x クロス
       #3=高さと深さ
       #4 picture 環境内の記述
\end{verbatim}
\end{boxnote}

各種オプションの説明です。

\verb/[#2]/オプションは,斜線の向きを指定します。
前ページの例では\verb/[r]/オプションで
左上から右下方向の斜線を指定しています。

つぎに,\verb/<#4>/オプションについて簡単に述べます.

\textsf{Hyou}環境で \cmd{sya}コマンドを用いた欄には \textsf{picture}環境が
定義されています.
これを用いて斜線の左下,右上に文字列を配置することが出来ます.
そのためのコマンドが \cmd{Hyoumidasi} で,書式は
\begin{boxnote}
\begin{verbatim}
    \Hyoumidasi#1#2
        #1 : 左下の文字列
        #2 : 右上の文字列
\end{verbatim}
\end{boxnote}
\cindex{Hyoumidasi}

斜線の縦・横幅の調整をするオプションが \verb/(#1)/, \verb/[#3]/ 
オプションです。

まずは \verb/(#1)/ オプションの使用例です。

\showexample[\cmd{sya}の横幅指定オプション](1)(.9){example/hyou02}

\cmd{sya}の横幅についてデフォルトでは,
\textsf{Hyou}環境の引数で L, C, R でサイズ指定した最後の値に
設定されます。
上の例では,\texttt{3\zw}です。
それ以外の横幅のところに斜線をひくにはその横幅を\verb/(#1)/オプションで
指定する必要があります。

次に高さ・深さを変更するオプション\verb/[#3]/オプションの使用例です。

\showexample[\cmd{sya}の高さ指定オプション](1)(.9){example/hyou03}

この例では,1行目が高くなっていますので,その分を\cmd{vphantom}で
\cmd{sya}に伝えています。

\paragraph{\cmd{Lmidasi}}
表の左上欄に斜線を引き見出しをつける際,
その行の文字位置を調整する手段を提供します。

デフォルトの確認です。

\begin{showEx}(1,.9){\textsf{Hyou}環境のデフォルト}
\begin{Hyou}{|L{8\zw}|*2{C{2\zw}|}} \hline
  \sya(8\zw)[r]<\Hyoumidasi{行見出し}{列見出し}> & A & B \\\hline
  あいうえお & 1 & 2 \\\hline
  かきくけこ & 3 & 4 \\\hline
\end{Hyou}
\end{showEx}
\bigskip

行見出し,列見出しとも窮屈ですから,1行目の行高を増やすことにします。

\begin{showEx}(1,.9){1行目の行高を増やす}
\begin{Hyou}{|L{8\zw}|*2{C{2\zw}|}} \hline
  \sya(8\zw)[r][\stackrel{ }{ }\strut]<\Hyoumidasi{行見出し}{列見出し}> &A&B
    \\\hline
  あいうえお & 1 & 2 \\\hline
  かきくけこ & 3 & 4 \\\hline
\end{Hyou}
\end{showEx}

\bigskip

「行見出し」という文字の位置を動かす変数が \verb+\Lmidasiiti+です。
デフォルトは
\begin{jquote}
\begin{verbatim}
\def\Lmidasiiti{(3pt,1pt)[lb]}%
\def\Rmidasiiti{(-3pt,-1pt)[rt]}%
\end{verbatim}
\end{jquote}
となっています。
(\verb+Rmidasiiti+は「列見出し」の文字位置指定変数です。)
この書式は,\verb+\emathPut+の位置指定オプションの形式です。
配置基準点は,
\begin{jquote}
「行見出し」の方が,当該欄の左下コーナー\\
「列見出し」の方が,当該欄の右上コーナー
\end{jquote}
です。
%\clearpage

下の例では,
\begin{jquote}
\begin{verbatim}
\def\Lmidasiiti{(6pt,2pt)[lb]}%
\def\Rmidasiiti{(-6pt,-2pt)[rt]}%
\end{verbatim}
\end{jquote}
と変更しています。

\begin{showEx}(1,.9){\cmd{Lmidasiiti}}
\begin{Hyou}{|L{8\zw}|*2{C{2\zw}|}} \hline
  \def\Lmidasiiti{(6pt,2pt)[lb]}%
  \def\Rmidasiiti{(-6pt,-2pt)[rt]}%
  \sya(8\zw)[r][\stackrel{ }{ }\strut]<\Hyoumidasi{行見出し}{列見出し}> &A&B
    \\\hline
  あいうえお & 1 & 2 \\\hline
  かきくけこ & 3 & 4 \\\hline
\end{Hyou}
\end{showEx}

\paragraph{\cmd{agezoko}}
列見出し -- 上の例での A B -- の位置を修正する変数が
\verb+\agezoko+です。下の例では,
\begin{jquote}
\begin{verbatim}
\def\agezoko{2}
\end{verbatim}
\end{jquote}
とすることにより,ベースラインを上に\verb+2pt+持ち上げています。

\begin{showEx}(1,.9){\cmd{agezoko}}
\begin{Hyou}{|L{8\zw}|*2{C{2\zw}|}} \hline
  \def\Lmidasiiti{(6pt,2pt)[lb]}%
  \def\Rmidasiiti{(-6pt,-2pt)[rt]}%
  \def\agezoko{2}%
  \sya(8\zw)[r][\stackrel{ }{ }\strut]<\Hyoumidasi{行見出し}{列見出し}> &A&B
    \\\hline
  あいうえお & 1 & 2 \\\hline
  かきくけこ & 3 & 4 \\\hline
\end{Hyou}
\end{showEx}

\paragraph{逆向きの斜線に対する\cmd{Hyoumidasi}}
斜線の向きが逆,すなわち左下と右上を結ぶときも\\
\cmd{Hyoumidasi}が使えるようにしたい,というのが今回の改定です。

\begin{showEx}(.5,.44){\cmd{Hyoumidasi}}
\def\arraystretch{1.25}
\begin{Hyou}{|*3{C{4\zw}|}}\hline
  X & 1 & 2 \\\hline
  Y & 10 & 20 \\\hline
  \sya[n]<\Hyoumidasi{左上}{右下}>
    & A & B \\\hline
\end{Hyou}
\end{showEx}

\paragraph{斜線位置の微調整}
\textsf{Hyou}環境で斜線を引く際

\begin{showEx}(.56,.38){\textsf{Hyou}環境の斜線}
\begin{Hyou}{|l|r|r|}\hline
  \sya(8\zw)[r][\bsityuu] & A & B \\\hline
  x & 1 & 2 \\\hline
\end{Hyou}
\end{showEx}

\noindent
とした場合,左上の位置に不満があるというご意見をいただくことがあります。
斜線の左上を少し下げたい,ということのようです。

\begin{showEx}(.56,.38){斜線位置の微調整}
\begin{Hyou}{|l|r|r|}\hline
  \sya(8\zw)[o][\bsityuu]%
   <{\ArrowLine<Henvi={(0,-.3pt)},%
       arrowheadsize=0>\LT\RB}>%
   & A & B \\\hline
  x & 1 & 2 \\\hline
\end{Hyou}
\end{showEx}

下げすぎですって ? \verb+<Henvi=..>+の右辺値をお好みで調整してください。
斜線の右下も調整したければ,\verb+Henvii=..+も指定することとなります。

\begin{showEx}(.56,.38){斜線右下も微調整}
\begin{Hyou}{|l|r|r|}\hline
  \sya(8\zw)[o][\bsityuu]%
   <%
    {%
     \ArrowLine%
      <%
       Henvi={(0,-.3pt)},%
       Henvii={(0,.15pt)},%
       arrowheadsize=0%
      >%
     \LT\RB
    }%
    \Put\RT(-3pt,-2pt)[rt]{行見出し}%
    \Put\LB(3pt,2pt)[lb]{列見出し}%
   >
    & A & B \\\hline
  x & 1 & 2 \\\hline
\end{Hyou}
\end{showEx}

\subsection{\textsf{hyou} 環境}
\index{hyou @ hyou 環境}
\textsf{Hyou}環境が,横幅を指定した \textsf{tabular}環境であるのに対し,
\textsf{hyou}環境は,横幅を指定した \textsf{array}環境です.

使用例として,増減表をご覧いただきましょう.

\showexample[増減表](1)(1){example/zougen1}

次に,微分不能の点には斜線を引いてみましょう.

\showexample[増減表](1)(1){example/zougen2}

$y=x\sqrt{4-x^2}$ の増減表です.ついでにグラフも添えておきます.

\begin{center}
\unitlength1cm
\begin{zahyou}<><><(2pt,-2pt)[lt]>(-3,3)(-3,3)%
\def\Fx#1#2{\Mul{#1}{#1}\y
    \Sub{4}\y\y\Heihoukon\y\y
    \Mul{#1}\y\y
    \edef#2{\y}}%
\Put{(2,0)}(0,-2pt)[t]{2}%
\Put{(1.41,0)}(0,-2pt)[t]{$\sqrt2$}%
\Put{(-2,0)}(0,-2pt)[rt]{$-2$}%
\Put{(-1.41,0)}(0,2pt)[b]{$-\sqrt2$}%
\Put{(0,2)}(0,0)[r]{2 }%
\Put{(0,-2)}(0,0)[l]{ $-2$}%
\dashline[40]{.1}(1.41,0)(1.41,2)(0,2)%
\dashline[40]{.1}(-1.41,0)(-1.41,-2)(0,-2)%
{\thicklines
\yGurafu\Fx{-1.414}{0}%
\yGurafu\Fx{0}{1.414}%
\yGurafu\Fx{-2}{-1.414}%
\yGurafu\Fx{1.414}{2}%
}%
\end{zahyou}
\end{center}

増減表で,増加・減少を表現する矢印に次のものを用意しました.

\begin{boxnote}
\begin{Hyou}{*{4}{L{4\zw} }L{6\zw}}
コマンド & 方位 & 記号 & 内容 \\
\cmd{NE} & 北東 & $\NE$ & 増加 \\
\cmd{SE} & 南東 & $\SE$ & 減少 \\
\cmd{NEN}& 北北東 & $\NEN$ & 下に凸で増加\\
\cmd{NEE}& 東北東 & $\NEE$ & 上に凸で増加\\
\cmd{SES}& 南南東 & $\SES$ & 上に凸で減少\\
\cmd{SEE}& 東南東 & $\SEE$ & 下に凸で減少
\end{Hyou}
\end{boxnote}
\cindex{NE}
\cindex{NEE}
\cindex{NEN}
\cindex{SE}
\cindex{SES}
\cindex{SEE}

\subsection{表の罫線を太く}
\subsubsection{\cmd{arrayrulewidth}}
表の罫線全部を太くするのは,\LaTeX で用意されている
\cmd{arrayrulewidth}の値を変更することで実現できます。

\begin{showEx}{\cmd{arrayrulewidth}の変更}
\arrayrulewidth1pt\relax
\begin{tabular}{|c|c|c|c|}\hline
  A & B & C & D \\\hline
  1 & 2 & 3 & 4 \\\hline
  a & b & c & d \\\hline
\end{tabular}
\end{showEx}

\subsubsection{外枠のみを太く}
外枠だけを太くしたい,など一部の罫線を太くするには,
面倒な手順を踏まねばなりませんので,マクロ化することにしました。

横罫線を太くするために
\begin{jquote}
\begin{verbatim}
\hline, \cline
\end{verbatim}
\end{jquote}
にかえて,それぞれ
\begin{jquote}
\begin{verbatim}
\hlineb, \clineb
\end{verbatim}
\end{jquote}
を新設しました。これらの罫線の太さは\cmd{arrarulewidthb}で指定します。
デフォルトは\verb+1pt+としてあります。
縦罫線を太くする位置には
\begin{jquote}
\begin{verbatim}
|
\end{verbatim}
\end{jquote}
にかえて
\begin{jquote}
\begin{verbatim}
I
\end{verbatim}
\end{jquote}
を用います。

では,これらを用いて外枠だけを太くしてみましょう。

\begin{showEx}(.5,.44){外枠を太く}
\begin{tabular}{Ic|c|c|cI}\hlineb
  A & B & C & D \\\hline
  1 & 2 & 3 & 4 \\\hline
  a & b & c & d \\\hlineb
\end{tabular}
\end{showEx}

\subsubsection{二重罫線との併用}
さらに,見出し行・列と表の内容との境界を二重線にすることもできます。
この部分は\textsf{hhline.sty}の一部を修正しています。
なお,\textsf{emathT.sty}はその中で,\textsf{array.sty}と
\textsf{hhline.sty}を読み込んでいます。

\begin{showEx}(.5,.44){二重罫線も}
\begin{tabular}{Ic||c|c|cI}\hlineb
  A & B & C & D \\\hhline{I=#=|=|=I}
  1 & 2 & 3 & 4 \\\hline
  a & b & c & d \\\hlineb
\end{tabular}
\end{showEx}

\subsubsection{太罫線の太さ}
太罫線の太さは\cmd{arrayrulewidthb}で決まりますから,もっと太くしたければ

\begin{showEx}(.5,.44){\cmd{arrayrulewidthb}}
\arrayrulewidthb=2pt\relax
\begin{tabular}{Ic||c|c|cI}\hlineb
  A & B & C & D \\\hhline{I=#=|=|=I}
  1 & 2 & 3 & 4 \\\hline
  a & b & c & d \\\hlineb
\end{tabular}
\end{showEx}

\subsubsection{特定のブロック枠を太く}

\begin{showEx}(.5,.44){一部分を太罫線で}
\begin{tabular}{|c|c|c|c|}\hlineb
  A & B & C & D \\\hline
  \noalign{\vskip-\arrayrulewidth}
  \clineb{2-3}
  \multicolumn{1}{|cI}{1}
    & \multicolumn{1}{c|}{2}
    & \multicolumn{1}{cI}{3}
    & 4 \\
  \noalign{\vskip-\arrayrulewidthb
  \vskip\arrayrulewidth}
  \clineb{2-3}
  \noalign{\vskip\arrayrulewidthb
  \vskip-\arrayrulewidth}\hline
  a & b & c & d \\\hlineb
\end{tabular}
\end{showEx}

\subsection{罫線を点線で}
罫線を点線で引くには,\textsf{arydshln.sty}を用いることが出来ます。
しかし,このスタイルファイルは\textsf{hhline.sty}と相性が悪いので,
\textsf{emathT.sty}との併用については,別文書(emathT-arydshln.tex)
で述べることとします。

%\clearpage

\subsection{\textsf{hyou}環境・縦罫線に波線}
表の縦罫線に二重の波線を描くには,\textsf{emathPp.sty}の
機能を必要とします。\\
\textsf{samplePp.tex}をご覧ください。

\subsection{\textsf{hyou}環境の行高指定}
\textsf{hyou}環境,すなわち\textsf{array}環境の行の高さ・深さは
表の中に入るものによって変わります。

\begin{showEx}(.64,.3){\textsf{hyou}環境の行高}
\[
\begin{hyou}{|*5{C{1\zw}|}} \hline
  x & 0 &\cdots & 1 & \cdots \\\hline
  y & \sya & \NE   & \bunsuu12 & \SE \\\hline
\end{hyou}
\]
\end{showEx}
では,2行目に分数が入る分だけ行の高さ・深さが変化します。
その結果,斜線についても問題が発生しています。

\subsubsection{\cmd{gyoudaka}}
そこで,コマンド\cmd{gyoudaka}を新設して,
表の行高を揃えてしまうことを試みました。

\begin{showEx}(.64,.3){\cmd{gyoudaka}}
\[
\gyoudaka[10pt]{15pt}
\begin{hyou}{|>{\gyousityuu}*5{C{1\zw}|}} \hline
  x & 0 &\cdots & 1 & \cdots \\\hline
  y & \sya & \NE   & \bunsuu12 & \SE \\\hline
\end{hyou}
\]
\end{showEx}

すなわち,
\begin{jquote}
\begin{verbatim}
\gyoudaka[深さ]{高さ}
\end{verbatim}
\end{jquote}
で,行の高さ・深さを指定し,\textsf{hyou}環境の欄指定子に
\begin{jquote}
\begin{verbatim}
>{\gyousityuu}
\end{verbatim}
\end{jquote}
と,各行に支柱を立てます。
斜線の高さも追随します。

\cmd{gyoudaka}の引数は単位を伴う数値です。
%\clearpage

\subsubsection{\cmd{Gyoudaka}}
高さ・深さを指定するのに,\cmd{vphantom}などを用いるためのコマンドが
\cmd{Gyoudaka}です。

\begin{showpEx}(.64,.3){\cmd{Gyoudaka}}
\[
\Gyoudaka{\EMvphantom[2pt]{$\bunsuu12$}}
% 以下,hyou環境の記述は同じ
!\begin{hyou}{|>{\gyousityuu}*5{C{1\zw}|}} \hline
!  x & 0 &\cdots & 1 & \cdots \\\hline
!  y & \sya & \NE   & \bunsuu12 & \SE \\\hline
!\end{hyou}
\]
\end{showpEx}
分数$\bunsuu{1}{2}$の上下に\texttt{2pt}ずつの余白をつけた支柱を
すべての行に立てています。

\subsection{増減表}
増減表については,前にも触れましたが,細かいことも論じておきましょう。

\subsubsection{増減表(1) --- \textsf{array}環境}
まずは基本からということで,\textsf{array}環境を使用したものから見ていきます。

\begin{showEx}(1,1){\textsf{array}環境}
\[
\begin{array}{c|*7{|c}}\hline
    x & \cdots & 0 & \cdots & 1 & \cdots & 2 & \cdots \\\hline
    y'&   +    & 0 &   -   & &   -    & 0 &   +       \\\hline
    y &  \nearrow & 極大 &  \searrow  &   &  \searrow & 極小 & \nearrow
        \\\hline
\end{array}
\]
\end{showEx}

\subsubsection{増減表(2) --- \textsf{hyou}環境}
前節の\textsf{array}環境による表は,列の横幅が均一ではありません。
さらに,不連続点には斜線を入れたい,ということで\textsf{emathT.sty}で
定義されている\textsf{hyou}環境を用いてみます。

\begin{showEx}(1,1){\textsf{hyou}環境}
\[
\begin{hyou}{c|*7{|C{2\zw}}}\hline
   x & \cdots & 0 & \cdots & 1 & \cdots & 2 & \cdots \\\hline
   y'&   +    & 0 &   -   & \sya[x] &   -    & 0 &   +    \\\hline
   y &  \NE & 極大 &  \SE & \sya[x] &  \SE & 極小 & \NE \\\hline
\end{hyou}
\]
\end{showEx}

\cmd{sya}が斜線を引くコマンドで,オプション引数で斜線の向きを指定します。
\begin{jquote}
\begin{verbatim}
\sya    のみの場合は,右上から左下へ
\sya[r] とすれば,逆に左上から右下へ
\sya[x] でクロスの斜線が引かれます。
\end{verbatim}
\end{jquote}

\subsubsection{増減表(3) --- 2行ぶち抜きで斜線}
不連続点$x=1$において,$y$と$y'$の行別々に斜線を引いていますが,
これを2行ぶち抜きで引いてみましょう。

\begin{showEx}(1,1){2行ぶち抜きの斜線}
\[
\begin{hyou}{c|*7{|C{2\zw}}}\hline
   x & \cdots & 0 & \cdots & 1 & \cdots & 2 & \cdots \\\hline
   y'&   +    & 0 &   -   
     & \smash{\sya(D=\ht\strutbox+2\dp\strutbox)[x]}
     &   -    & 0 &   +    \\\cline{1-4}\cline{6-8}
   y &  \NE & 極大 &  \SE &  &  \SE & 極小 & \NE \\\hline
\end{hyou}
\]
\end{showEx}

この場合,斜線の高さ・深さを\cmd{sya}に伝えるために
\begin{jquote}
\begin{verbatim}
\sya(D=\ht\strutbox+2\dp\strutbox)[x]
\end{verbatim}
\end{jquote}
としています。\cmd{strutbox}に1行分の高さ・深さがありますので,斜線の深さを
\begin{jquote}
\begin{verbatim}
\ht\strutbox+2\dp\strutbox
\end{verbatim}
\end{jquote}
と指定しているのです。
(斜線の高さはデフォルトのままでよいから,指定しません。)\\
さらに,\cmd{smash}で2行目の高さ・深さに影響を与えないようにしています。
%\clearpage

\subsubsection{増減表(4) --- 不連続点$y$の欄には左右極限を}
ぶち抜きではなく,$y$の欄には左右極限を入れておく,という表も便利でしょうか。
これは列の個数を増やしてやりましょう。

\begin{showEx}(1,1){左右極限}
\[
\begin{hyou}{c|*8{|C{2\zw}}}\hline
   x & \cdots & 0 & \cdots & \multicolumn{2}{c|}{1} & \cdots & 2 & \cdots 
      \\\hline
   y'&   +    & 0 &   -   
     & \multicolumn{2}{c|}{\sya(4\zw+\arrayrulewidth+2\arraycolsep)[x]}
     &   -    & 0 &   +    \\\hline
   y &  \NE & 極大 &  \SE & -\infty & +\infty &  \SE & 極小 & \NE \\\hline
\end{hyou}
\]
\end{showEx}
ここでは,斜線の横幅が今までと異なります。
2列ぶち抜きですから\verb+2*2\zw+すなわち\verb+4\zw+必要ですが,
それでは不十分です。列の間の罫線幅\verb+\arrayrulewidth+,
さらにその左右の余白分\verb+2\arraycolsep+も加えたものを斜線の横幅
と指定します。指定法は
\begin{jquote}
\begin{verbatim}
正しくは
\sya(W=4\zw+\arrayrulewidth+2\arraycolsep)
ですが,W=... 指定のみの場合はW=は省略可能としてあります。
\sya(4\zw+\arrayrulewidth+2\arraycolsep)
\end{verbatim}
\end{jquote}

%\clearpage

\subsubsection{増減表(5) --- 凹凸も}
増減表に凹凸も含めたい,となると$y''$の行を挿入することになります。
\begin{showEx}(1,1){凹凸も}
\[
\begin{hyou}{c|*7{|C{2\zw}}}\hline
   x & \cdots & 0 & \cdots & 1 & \cdots & 2 & \cdots \\\hline
   y'&   +    & 0 &   -   
     & \smash{\sya(D=2\ht\strutbox+3\dp\strutbox)[x]}
     &   -    & 0 &   +    \\\cline{1-4}\cline{6-8}
   y'' & \multicolumn{3}{c|}{-} & & \multicolumn{3}{c}{+}
     \\\cline{1-4}\cline{6-8}
   y &  \NEE & 極大 &  \SSE &  &  \SEE & 極小 & \NNE \\\hline
\end{hyou}
\]
\end{showEx}

上のリストに
\begin{jquote}
\begin{verbatim}
\def\arraystretch{1.25}
\end{verbatim}
\end{jquote}
を附加すると

\begin{shadebox}
\def\arraystretch{1.25}
\[
\begin{hyou}{c|*7{|C{2\zw}}}\hline
   x & \cdots & 0 & \cdots & 1 & \cdots & 2 & \cdots \\\hline
   y'&   +    & 0 &   -   
     & \smash{\sya(D=2\ht\strutbox+3\dp\strutbox)[x]}
     &   -    & 0 &   +    \\\cline{1-4}\cline{6-8}
   y'' & \multicolumn{3}{c|}{-} & & \multicolumn{3}{c}{+}
     \\\cline{1-4}\cline{6-8}
   y &  \NEE & 極大 &  \SSE &  &  \SEE & 極小 & \NNE \\\hline
\end{hyou}
\]
\end{shadebox}
行間が少しゆったりします。斜線も追随していることがおわかりいただけるでしょう。

\subsection{表セルの修飾}
表の1つのセルを修飾するコマンドが\cmd{EMcell}です。
\subsubsection{網掛け}
厳密には網掛けとはいえませんが,セルをグレーで塗りつぶします。

\begin{showEx}{\cmd{EMcell*}}
\begin{Hyou}{|*3{L{2\zw}|}}\hline
  甲&乙\\\hline
  \EMcell*{l}{あ}&い\\\hline
\end{Hyou}
\end{showEx}

\cmd{EMcell*}の引数\verb+{l}+は,
この欄が左詰であることを指定するものです。
\textsf{Hyou}環境で,既に指定されているのですが,
\cmd{EMcell*}でもl, c, rのいずれかを指定する必要があります。

グレーの濃度は\verb+[数値]+オプションで指定します。
数値は0と1の間の数値で\cmd{Nuritubusi}と同じです。

\begin{showEx}(.7,.24){\cmd{EMcell*}}
\makeatletter
\begin{Hyou}{|*3{R{2\zw}|}}\hline
  甲& 乙\\\hline
  \EMcell*[.1]{r}{あ}&い\\\hline
\end{Hyou}
\end{showEx}

\subsubsection{カラー塗り}
セルに色をつけるには\texttt{[C=..]}オプションを用います。

\begin{showEx}(.7,.24){\texttt{[C=..]}オプション}
\begin{Hyou}{|*3{C{2\zw}|}}\hline
  甲&乙\\\hline
  \EMcell*[C=yellow]{c}{あ}&い\\\hline
\end{Hyou}
\end{showEx}

(注)Windows において dviout.exe でご覧の場合,dviout の設定によっては,
文字「あ」が黄色で塗りつぶされて見えなくなってしまうかもしれません。

\subsubsection{斜線塗り}
セルを斜線塗りします。

\begin{showEx}(.7,.24){\cmd{EMcell**}}
\begin{Hyou}{|*3{C{2\zw}|}}\hline
  甲&乙\\\hline
  \EMcell**{c}{あ}&い\\\hline
\end{Hyou}
\end{showEx}

斜線の方向を変えるには\verb+[角度]+オプションを使用します。
角度は六十分法で指定します。

\begin{showEx}(.7,.24){斜線の角度}
\begin{Hyou}{|*3{C{2\zw}|}}\hline
  甲&乙\\\hline
  \EMcell**[-60]{c}{あ}&い\\\hline
\end{Hyou}
\end{showEx}

斜線の間隔は\verb+<間隔>+オプションで指定します。デフォルトは\verb+<3>+です。

\begin{showEx}(.7,.24){斜線の間隔}
\begin{Hyou}{|*3{C{2\zw}|}}\hline
  甲&乙\\\hline
  \EMcell**[-60]<5>{c}{あ}&い\\\hline
\end{Hyou}
\end{showEx}

\subsubsection{枠線}
罫線を引かない表で,特定のセルに枠線をつけるには,
\verb+[wakusen=..]+オプションをつけます。

\begin{showEx}(.7,.24){\texttt{[wakusen=..]}オプション}
\begin{Hyou}{*3{C{2\zw}}}
  甲&乙\\
  \EMcell[wakusen=\protect\hasen]{c}{あ}&い
\end{Hyou}
\end{showEx}

右辺値には\verb+\protect+を附加しておく方が無難です。
次の例は,枠線を実線としセル内を網掛けしています。

\begin{showEx}(.7,.24){\texttt{[wakusen=..]}オプション+網掛け}
\begin{Hyou}{*3{C{2\zw}}}
  甲&乙\\
  \EMcell*[wakusen=\protect\drawline]{c}{あ}&い
\end{Hyou}
\end{showEx}

この場合,グレーの濃度も指定するには,\verb+[G=数値]+オプションを用います。
\texttt{G}はgrayの頭文字のつもりです。

\begin{showEx}(.7,.24){\texttt{[wakusen=..]}オプション+網掛け濃度指定}
\begin{Hyou}{*3{C{2\zw}}}
  甲&乙\\
  \EMcell*[G=.8,wakusen=\protect\drawline]{c}{あ}&い
\end{Hyou}
\end{showEx}

枠線と斜線塗りも可能です。

\begin{showEx}(.7,.24){\texttt{[wakusen=..]}オプション+斜線塗り}
\begin{Hyou}{*3{C{2\zw}}}
  甲&乙\\
  \EMcell**[wakusen=\protect\drawline]{c}{あ}&い
\end{Hyou}
\end{showEx}

斜線の方向・間隔を変更するには,それぞれ
\verb+[syasenkaku=角度,syasenkankaku=間隔]+オプションを用います。

\begin{showEx}(.7,.24){\texttt{[wakusen=..]}オプション+斜線方向・角度指定}
\begin{Hyou}{*3{C{2\zw}}}
  甲&乙\\
  \EMcell**[wakusen=\protect\drawline,%
    syasenkaku=-45,syasenkankaku=2]{c}{あ}&い
\end{Hyou}
\end{showEx}

\subsubsection{横幅指定}
\hyouretuhaba=0pt%

\begin{showEx}(.6,.34){横幅指定が必要な例}
\begin{tabular}{|r|r|}\hline
  あいうえお&甲乙\\\hline
  \EMcell*{r}{1}&2\\\hline
\end{tabular}
\end{showEx}

前節までのように,\textsf{Hyou, hyou}環境で列幅を指定している場合は
よいのですが,上の例のように\textsf{tabular, array}環境で列幅が表に入るものに
合わせて決まるような場合には,\cmd{EMcell}はそれに追随していません。

このような場面で使用するためには,当該列の最長の文字列の長さを
\cmd{EMcell}の\verb+[W=長さ]+オプションで指定する必要があります。

\begin{showEx}(.6,.34){横幅指定オプション}
\settowidth\templa{あいうえお}
\begin{tabular}{|r|r|}\hline
  あいうえお&甲乙\\\hline
  \EMcell*[W=\templa]{r}{1}&2\\\hline
\end{tabular}
\end{showEx}

なお,当該セルがその列の中で最長の場合は\verb+[W=..]+オプションは不要です。

\begin{showEx}(.6,.34){最長のカラムの場合}
\begin{tabular}{|r|r|}\hline
  \EMcell*{r}{あいうえお}&甲乙\\\hline
  1&2\\\hline
\end{tabular}
\end{showEx}

\subsubsection{高さ指定}

\begin{showEx}(.6,.34){高さ指定が必要な例}
\[ 
  \begin{array}{|r|r|}\hline
    あいうえお&甲乙\\\hline
    \EMcell*{r}{1}&\bunsuu12\\\hline
  \end{array}
\]
\end{showEx}

上の場合は,第2行に分数が登場して行高が変化しています。
\cmd{EMcell*}はその変化に追随できません。
対策は\cmd{vphantom}を用いるのが簡単です。
幅指定オプションと合わせて

\begin{showEx}(.6,.34){高さ指定(1)}
\[ 
\settowidth\templa{あいうえお}
\begin{array}{|r|r|}\hline
  あいうえお&甲乙\\\hline
  \EMcell*[W=\templa]{r}%
    {1\vphantom{\bunsuu12}}&\bunsuu12
  \\ \hline
\end{array}
\]
\end{showEx}

なお,第2行に支柱を立てる場合は,当該セルで支柱を立てるのがよいでしょう。

\begin{showEx}(.6,.34){高さ指定(2)}
\[ 
\settowidth\templa{あいうえお}
\begin{array}{|r|r|}\hline
  あいうえお&甲乙\\\hline
  \EMcell*[W=\templa]{r}%
    {1\bsityuu}&\bunsuu12\\\hline
\end{array}
\]
\end{showEx}

\subsubsection{セルに斜線など}
\cmd{EMcell(*)}を使用したセルには,\textsf{picture}環境が定義されています。
\verb+<...>+オプションで,その\textsf{picture}環境内に配置するものを
記述可能です。次の例は,その\textsf{picture}環境の原点に$\bullet$を打ちます。

\begin{showEx}(.6,.34){\textsf{picture}環境の原点}
\[ 
  \begin{array}{|r|r|}\hline
    \EMcell<\Put\O(0,0)[c]{$\bullet$}>%
      {r}{あいうえお}&甲乙\\\hline
    1&2\\\hline
  \end{array}
\]
\end{showEx}

すなわち,原点の縦位置はベースライン上,横位置はセルの中央であり,
\verb+\unitlength=1pt+となっています。


次の例は,当該セルに×印をつけます。

\begin{showEx}(.6,.34){\textsf{picture}環境内の記述}
\[ 
  \begin{array}{|r|r|}\hline
    \EMcell<\Drawlines{\LT\RB;\LB\RT}>%
      {r}{あいうえお}&甲乙\\\hline
    1&2\\\hline
  \end{array}
\]
\end{showEx}
\clearpage

\subsection{\textsf{tabular}, \textsf{array}環境でもL, C, R}
\textsf{Hyou}, \textsf{hyou}環境で使用していた,欄指定子L, C, Rを
\textsf{tabular}, \textsf{array}環境でも使用可能です。
また,太罫線の使用も可能です。

まずは,\textsf{tabular}環境です。

\begin{showEx}(.5,.44){\textsf{tabular}環境でL, C, R}
\begin{tabular}%
    {IL{2\zw}|C{2\zw}|R{5\zw}I}
    \hlineb
  A & a & xyz \\ \hline
  B & b & \the\tabcolsep \\ \hlineb
\end{tabular}
\end{showEx}

次は\textsf{array}環境です。

\begin{showEx}(.5,.44){\textsf{array}環境でL, C, R}
$\begin{array}%
    {IL{2\zw}|C{2\zw}|R{5\zw}I}
    \hlineb
  A & a & xyz \\ \hline
  B & b & \the\arraycolsep
    \\ \hlineb
\end{array}$
\end{showEx}

同じ横幅を指定したのに,結果が異なる ? ですか。

それは\LaTeX の仕様で,罫線左右の空きが\textsf{tabular}, \textsf{array}では
別の変数で制御され,その初期値は,上のそれぞれの表の
右下欄のようになっているからです。
この件に関しては,\textsf{emath}は無関係です。

\subsection{\textsf{longtable}環境でもL, C, R}
\textsf{longtable}環境でも可能とする試みです。

\begin{longtable}{IL{2\zw}|C{2\zw}|R{5\zw}I}\hlineb
  A & a & 123 \\ \hline
  B & b & 234 \\ \hline
  C & c & 345 \\ \hline
  D & d & 456 \\ \hline
  E & e & 567 \\ \hline
  F & f & 678 \\ \hline
  G & g & 789 \\ \hline
  H & h & 890 \\ \hline
  I & i & 901 \\ \hline
  J & j & 012 \\ \hline
  K & k & 123 \\ \hline
  L & l & 234 \\ \hline
  M & m & 345 \\ \hline
  N & n & 456 \\ \hline
  O & o & 567 \\ \hline
  Z & z & \the\tabcolsep \\ \hlineb
\end{longtable}

この表は,ページをまたぎますから\textsf{shadebox}環境に入れることができません。
なお,ソースリストは次のとおりです。
\begin{jquote}
\begin{verbatim}
\begin{longtable}{IL{2\zw}|C{2\zw}|R{5\zw}I}\hlineb
  A & a & 123 \\ \hline
  B & b & 234 \\ \hline
  C & c & 345 \\ \hline
  D & d & 456 \\ \hline
  E & e & 567 \\ \hline
  F & f & 678 \\ \hline
  G & g & 789 \\ \hline
  H & h & 890 \\ \hline
  I & i & 901 \\ \hline
  J & j & 012 \\ \hline
  K & k & 123 \\ \hline
  L & l & 234 \\ \hline
  M & m & 345 \\ \hline
  N & n & 456 \\ \hline
  O & o & 567 \\ \hline
  Z & z & \the\tabcolsep \\ \hlineb
\end{longtable}
\end{verbatim}
\end{jquote}
%\clearpage

\subsection{\textsf{tabular}, \textsf{array}環境でも\cmd{sya}}
\textsf{tabular}環境でも,カラムに斜線を引く\cmd{sya}は有効ですが,
横幅を指定する必要があります。
また,斜線の左下\右上に見出しをつけるには,
\cmd{Hyoumidasi}を用います。

\begin{showEx}(.5,.44){\textsf{tabular}環境で\cmd{sya}}
\begin{tabular}%
    {IL{2\zw}|C{2\zw}|R{5\zw}I}
    \hlineb
  \sya(2\zw)[r]<\Hyoumidasi{X}{Y}>%
     & p & q \\ \hline
  A & a & xyz \\ \hline
  B & b & \the\tabcolsep \\ \hlineb
\end{tabular}
\end{showEx}

ついで,\textsf{array}環境です。
こちらでは,斜線の左下\右上に見出しをつけるには,
\cmd{hyoumidasi}を用います。

\begin{showEx}(.5,.44){\textsf{array}環境で\cmd{sya}}
$\begin{array}%
    {IL{2\zw}|C{2\zw}|R{5\zw}I}
    \hlineb
  \sya(2\zw)[r]<\hyoumidasi{X}{Y}>%
     & p & q \\ \hline
  A & a & xyz \\ \hline
  B & b & \the\tabcolsep \\ \hlineb
\end{array}$
\end{showEx}


%\section{囲み}
\section{囲み}
\subsection{\textsf{rectbox}環境}
この節で紹介する\textsf{rectbox}環境は
\textsf{emathPb.sty}で定義されています。

\subsubsection{\textsf{rectbox}とは}
罫線囲みの一種です。
罫線を\textsf{picture}環境で描画します。

ページをまたぐことはできません。また,傍注もつけられません。

\begin{showEx}(.9,1){\textsf{rectbox}環境}
\begin{rectbox}
あああああああああああああああああああ
あああああああああああああああああああ
あああああああああああああああああああ

いいいいいいいいいいいいいいいいいいい
いいいいいいいいいいいいいいいいいいい
いいいいいいいいいいいいいいいいいいい
いいいいいいいいいいいいいいいいいいい
\end{rectbox}
\end{showEx}
%\pagebreak

\subsubsection{罫線と本文との間隔}
\paragraph{\texttt{fboxsep}}
罫線と中のテキストとの間隔は,\cmd{fboxsep}で決まります。
これを変更したいときは,\verb+[fbox=..]+オプションを与えます。

\begin{showEx}(.9,1){\texttt{fboxsep}オプション}
\begin{rectbox}[fboxsep=1\zw]
あああああああああああああああああああ
あああああああああああああああああああ
あああああああああああああああああああ
\end{rectbox}
\end{showEx}

\paragraph{\texttt{hsep vsep}}
\verb+[fboxsep=..]+オプションは,左右・上下すべてを一律に変更しますし,
当該環境内のみではありますが,\cmd{fboxsep}の値が変更されています。

\begin{showEx}(.9,1){\texttt{fboxsep}オプションの副作用}
\begin{rectbox}[fboxsep=1\zw]
あああああああああああああああああああ

\fbox{いいい}
\end{rectbox}

\fbox{ううう}
\end{showEx}

そこで,左右の空きを指定するオプション\verb+hsep=..+と,
上下の空きを指定するオプション\verb+vsep=..+を新設しました。

\begin{showEx}(.9,1){\texttt{hsep, vsep}オプション}
\begin{rectbox}[hsep=3\zw,vsep=1\zw]
あああああああああああああああああああ
あああああああああああああああああああ

\fbox{いいい}
\end{rectbox}

\fbox{ううう}
\end{showEx}

\subsubsection{横幅}
\paragraph{\texttt{rectboxwidth}}
\textsf{rectbox}環境の横幅は\cmd{linewidth}で,横いっぱいに広がります。
これを制限するオプションが\verb+rectboxwidth=..+です。

\begin{showEx}(.9,1){\texttt{rectboxwidth}オプション}
あああ
\begin{rectbox}[rectboxwidth=8\zw]
いいいいいいいいいいいいいいいい
いいいいいいいいいいいいいいいい
\end{rectbox}

ううう
\end{showEx}

横幅を8\zwと指定していますが,実際のボックス幅は,
これに,左右の空きが加わります。

\paragraph{\cmd{Rectbox}}
\textsf{rectbox}の中身が短いときなど,横幅をいちいち指定するのが面倒なときは,
\cmd{Rectbox}コマンドを用います。

\begin{showEx}(.9,1){\cmd{Rectbox}コマンド}
\Rectbox{あいうえお}
\end{showEx}

このコマンドは,複数行に対しても使用可能です。

\begin{showEx}(.9,1){\cmd{Rectbox}コマンド(複数行)}
\Rectbox{%
  $y=ax^2+bx+c$\\
  $x^2+y^2=r^2$}
\end{showEx}

(注)\cmd{Rectbox}コマンドの引数内に,別行立て数式環境を
入れることはできません。
%\clearpage

\subsubsection{見出しをつける}
上部枠線に見出しをつけることができます。

\begin{showEx}(.9,1){\texttt{item}オプション}
\begin{rectbox}[item={~見出し~}]
あああああああああああああああああああ
あああああああああああああああああああ
あああああああああああああああああああ
\end{rectbox}
\end{showEx}

その位置はデフォルトでは左ですが,中央(右)にするには,
\verb+itempos=..+オプションを用います。

\begin{showEx}(.9,1){\texttt{itempos=c}オプション}
\begin{rectbox}[item={~中央見出し~},itempos=c]
あああああああああああああああああああ
あああああああああああああああああああ
あああああああああああああああああああ
\end{rectbox}
\end{showEx}

\begin{showEx}(.9,1){\texttt{itempos=r}オプション}
\begin{rectbox}[item={~右見出し~},itempos=r]
あああああああああああああああああああ
あああああああああああああああああああ
あああああああああああああああああああ
\end{rectbox}
\end{showEx}

下に見出しをつけることもできます.

\begin{showEx}(.9,1){\texttt{bitemp=..}オプション}
\begin{rectbox}[bitem={~下見出し~}]
あああああああああああああああああああ
あああああああああああああああああああ
あああああああああああああああああああ
\end{rectbox}
\end{showEx}

左/中央/右の配置指定は\verb+[bitempos=l/c/r]+で行います.

\begin{showEx}(.9,1){\texttt{bitemp=..}オプション}
\begin{rectbox}[bitem={~下見出し~},bitempos=c]
あああああああああああああああああああ
あああああああああああああああああああ
あああああああああああああああああああ
\end{rectbox}
\end{showEx}

\subsubsection{罫線の種類変更}
デフォルトでは,罫線は\verb+\drawline+で引かれます。
これを変更するオプションです。

\begin{showEx}(.9,1){\texttt{sensyu}オプション}
\begin{rectbox}[sensyu=\hasen]
あああああああああああああああああああ
あああああああああああああああああああ
あああああああああああああああああああ
\end{rectbox}
\end{showEx}

破線に変更しましたが,次は点線にしてみましょう。

\begin{showEx}(.9,1){\texttt{sensyu}オプション}
\begin{rectbox}[sensyu=\dottedline{.2}]
あああああああああああああああああああ
あああああああああああああああああああ
あああああああああああああああああああ
\end{rectbox}
\end{showEx}
%\pagebreak

\subsubsection{罫線の太さ変更}

\begin{showEx}(.9,1){\texttt{allinethickness}オプション}
\begin{rectbox}[allinethickness=1pt]
あああああああああああああああああああ
あああああああああああああああああああ
あああああああああああああああああああ
\end{rectbox}
\end{showEx}

罫線の太さを\texttt{1pt}と太くしてみましたが,
四隅のつながり具合が感心しません。

その点は\textsf{emathPs.sty}で定義されている
\textsf{EMpsrectbox}環境の方がよろしいでしょう。

詳しくは,\texttt{samplePs.tex}をご覧ください。

\subsubsection{枠線に色付}
\textsf{rectbox}環境で作られる囲み枠の枠線に色をつけるには,
\texttt{[framecolor=..]}オプションを用います。

\begin{showEx}{\texttt{[framecolor=..]}オプション}
\begin{rectbox}[framecolor=red]
あああああああああああああああああ
あああああああああああああああああ
あああああああああああああああああ
\end{rectbox}
\end{showEx}

\subsubsection{背景色}
囲み枠の中の背景色をしているするには,
\texttt{[backgroundcolor=..]}オプションを用います。

\begin{showEx}{\texttt{[backgroundcolor=..]}オプション}
\begin{rectbox}[framecolor=red,%
  backgroundcolor=cyan]
あああああああああああああああああ
あああああああああああああああああ
あああああああああああああああああ
\end{rectbox}
\end{showEx}

\subsubsection{見出しと背景色は}
見出しをつけるとき,現時点での仕様は

\begin{showEx}{見出し}
\begin{rectbox}[framecolor=red,%
  backgroundcolor=cyan,item=見出し]
あああああああああああああああああ
あああああああああああああああああ
あああああああああああああああああ
\end{rectbox}
\end{showEx}

\noindent
「見出し」が背景色を侵食する,ことにしていますが,どんなものでしょうか。

\subsubsection{見出しの背景色指定}
見出しの背景色を指定することもできますが,ちょっとしつこいでしょうか。

\begin{showEx}{見出し}
\begin{rectbox}[framecolor=red,%
  backgroundcolor=cyan,%
  item=見出し,itempos=c,%
  midasibackgroundcolor=lightgray]
あああああああああああああああああ
あああああああああああああああああ
あああああああああああああああああ
\end{rectbox}
\end{showEx}

\begin{showEx}{下見出し}
\begin{rectbox}[framecolor=red,%
  backgroundcolor=cyan,%
  bitem=下見出し,bitempos=l,%
  midasibackgroundcolor=lightgray]
あああああああああああああああああ
あああああああああああああああああ
あああああああああああああああああ
\end{rectbox}
\end{showEx}

\subsubsection{\textsf{enumrectbox}環境}
\textsf{rectbox}環境の表題部に,自動的に連番を振っていくのが
\textsf{enumrectbox}環境です。

\begin{showEx}{\textsf{enumrectbox}環境}
\begin{enumrectbox}
ああああああああああ
\end{enumrectbox}

\begin{enumrectbox}
いいいいいいいいいい
\end{enumrectbox}
\end{showEx}

この番号付けは,カウンタ

\begin{boxnote}
\begin{verbatim}
\newcounter{rectboxenum}
\end{verbatim}
\end{boxnote}
\noindent
によって行われます。
したがって
\begin{jquote}
\verb+\resetcounter{rectboxenum}+\\
\verb+\setcounter{rectboxenum}+
\end{jquote}
などで制御することができます。

\begin{showEx}{番号初期値の指定}
\setcounter{rectboxenum}{100}
\begin{enumrectbox}
ああああああああああ
\end{enumrectbox}

\begin{enumrectbox}
いいいいいいいいいい
\end{enumrectbox}
\end{showEx}
%\newpage

\subsubsection{\cmd{labelrectboxenum}}
番号部を,単に
\begin{jquote}
1, 2, 3, \Cdots
\end{jquote}
ではなく
\begin{jquote}
例題1, 例題2, 例題3, \Cdots
\end{jquote}
などと修飾するためにコマンド\verb+\labelrectboxenum+を用意しました。

\begin{showEx}{\cmd{labelrectboxenum}}
\resetcounter{rectboxenum}
\def\labelrectboxenum{%
例題\arabic{rectboxenum}}
\begin{enumrectbox}
ああああああああああ
\end{enumrectbox}

\begin{enumrectbox}
いいいいいいいいいい
\end{enumrectbox}
\end{showEx}

\cmd{labelenumi}などと同様の使い方をします。

\subsubsection{見出し文も付加}
番号に続けて,見出し文もつけたいときは,\textsf{rectbox}環境に対する
\verb+[item=...]+オプションを使用します。

\begin{showEx}{\cmd{labelrectboxenum}}
\resetcounter{rectboxenum}
\def\labelrectboxenum{%
例題\arabic{rectboxenum}.~}
\begin{enumrectbox}[item=アアア]
ああああああああああ
\end{enumrectbox}

\begin{enumrectbox}[item=イイイ]
いいいいいいいいいい
\end{enumrectbox}
\end{showEx}
\newpage

\subsubsection{相互参照}
相互参照をするには,\verb+\label{...}+に代えて,\verb+\begin{enumrectbox}+に
\verb+<...>+オプションを用います。\verb+\ref+は,そのまま有効です。

\begin{showEx}{相互参照}
\resetcounter{rectboxenum}
\def\labelrectboxenum{%
例題\arabic{rectboxenum}}
\begin{enumrectbox}<L1>
ああああああああああ
\end{enumrectbox}

\begin{enumrectbox}
例題\ref{L1}において,
\end{enumrectbox}
\end{showEx}

\subsubsection{\textsf{enumrectbox}の書式}
\verb+\begin{enumrectbox}+の書式です。

\begin{boxnote}
\begin{verbatim}
\begin{enumrectbox}<#1>[#2]
  #1 : ラベル名
  #2 : rectbox環境に引き継がれるオプション
\end{verbatim}
\end{boxnote}

\subsubsection{\textsf{rectbox}環境における罫線との間隔}
\textsf{rectbox}において,罫線とテキスト部との間隔は,
デフォルトでは\verb+\fboxsep+ですが,これを変更するコマンドを用意しました。

\begin{boxnote}
\begin{verbatim}
\HVsep#1#2
  #1 : 左右の空き
  #2 : 上下の空き
いずれも,単位つきの数値
\end{verbatim}
\end{boxnote}

使用例です。

\begin{showEx}{罫線とテキスト部との間隔}
デフォルトです。

\begin{rectbox}
あああああああああああああ
あああああああああああああ
あああああああああああああ
\end{rectbox}

\verb+\HVsep{1\zw}{.5\zh}+
を発行します。

\HVsep{1\zw}{.5\zh}
\begin{rectbox}
あああああああああああああ
あああああああああああああ
あああああああああああああ
\end{rectbox}
\end{showEx}

なお,\textsf{rectbox}環境において,
\verb+[hsep=..,vsep=..]+オプションを発行したときは,
そのオプションが有効となります。

このコマンド\verb+HVsep+は,\textsf{enumrectbox}環境に対しても有効です。

\begin{showEx}{\textsf{enumrectbox}環境の場合}
\resetcounter{rectboxenum}
\HVsep{2\zw}{.5\baselineskip}
\begin{enumrectbox}
ああああああああああ
ああああああああああ
\end{enumrectbox}

\begin{enumrectbox}
いいいいいいいいいい
いいいいいいいいいい
\end{enumrectbox}
\end{showEx}
\newpage

\subsubsection{コーナーを丸く}
\textsf{rectbox}環境のコーナーを丸くするには,
\verb+[rectboxoval=...]+オプションを用います。
右辺値はコーナー四分円の半径を,単位つき数値で与えます。

\begin{showEx}(.9,1){\texttt{rectboxoval=..}オプション}
\begin{rectbox}[rectboxoval=10pt]
あああああああああああああああああああ
あああああああああああああああああああ
あああああああああああああああああああ

いいいいいいいいいいいいいいいいいいい
いいいいいいいいいいいいいいいいいいい
いいいいいいいいいいいいいいいいいいい
いいいいいいいいいいいいいいいいいいい
\end{rectbox}
\end{showEx}

\verb+\rectboxoval+コマンドを用いると,
すべての\textsf{(enum)rectbox}環境のコーナーを丸くすることができます。
引数は半径を,単位つき数値で与えます。

\begin{showEx}{\cmd{rectboxoval}}
\resetcounter{rectboxenum}
\rectboxoval{6pt}
\begin{enumrectbox}
ああああああああああ
ああああああああああ
\end{enumrectbox}

\begin{rectbox}[rectboxoval=0pt]
ナンバリングしない
\textsf{rectbox}環境です。
コーナーを丸くしたくなければ
\verb+[rectboxoval=0pt]+
オプションが必要です。
\end{rectbox}

\begin{enumrectbox}
いいいいいいいいいい
いいいいいいいいいい
\end{enumrectbox}
\end{showEx}

\subsection{丸囲み文字}
\subsubsection{\cmd{maru}}
文字を○で囲むのに,\textsf{emath.sty}では,文字を小さめにした\cmd{maru}
を用意しています。

\begin{showEx}(.6,.34){\cmd{maru}}
\maru{1}, \maru{39}, \maru{あ}
\end{showEx}

\subsubsection{\cmd{emPmaru}}
文字が小さいのが難点です。
そこで,文字サイズはそのままで,○を大きくする方式のコマンドを
新設しました。

\begin{showEx}(.6,.34){\cmd{emPmaru}}
\emPmaru{1}, \emPmaru{39}, \emPmaru{あ}
\end{showEx}

\subsubsection{\cmd{emPmaruKizyun}}
\cmd{emPmaru}は,中の文字に対応して,○のサイズが決まりますが,
これを一定にしたいときは
\begin{jquote}
\begin{verb}
\emPmaruKizyun{あ}
\end{verb}
\end{jquote}
などと,基準の文字を指定します。

\begin{showEx}(.6,.34){\cmd{emPmaru}}
\emPmaruKizyun{あ}
\emPmaru{1}, \emPmaru{39}, \emPmaru{あ}
\end{showEx}
%\clearpage

\subsubsection{\textsf{enumerate}環境での番号付け}
\cmd{emPmaru}を\textsf{enumerate}環境での番号付けに使用してみましょう。

\begin{showEx}(.7,.24){\cmd{emPmaru}}
\emPmaruKizyun{あ}
\begin{enumerate}[\protect\expandafter\emPmaru 1]
  \item あああ
  \item あああ
  \item あああ
  \item あああ
  \item あああ
  \item あああ
  \item あああ
  \item あああ
  \item あああ
  \item あああ
  \item あああ
  \item あああ
  \item あああ
  \item あああ
  \item あああ
  \item あああ
  \item あああ
  \item あああ
  \item あああ
  \item あああ
\end{enumerate}
\end{showEx}
%\clearpage

\subsubsection{数式番号に\cmd{emPmaru}}
\textsf{align}環境など,数式番号に\cmd{emPmaru}を使用してみましょう。

\begin{showEx}{数式番号に\cmd{emPmaru}}
\emPmaruKizyun{あ}
\def\tagform#1{\emPmaru{#1}}
\begin{align}
  y &= x\\
  y &= x\\
  y &= x\\
  y &= x\\
  y &= x\\
  y &= x\\
  y &= x\\
  y &= x\\
  y &= x\\
  y &= x\\
  y &= x\\
  y &= x
\end{align}
\end{showEx}

数式番号が目立ちすぎですか。サイズを落としてみましょうか。

\resetcounter{equation}
\begin{showEx}{数式番号に\cmd{emPmaru}}
\emPmaruKizyun{あ}
\def\tagform#1{%
  \footnotesize\emPmaru{#1}}
\begin{align}
  y &= x\\
  y &= x\\
  y &= x\\
  y &= x\\
  y &= x\\
  y &= x\\
  y &= x\\
  y &= x\\
  y &= x\\
  y &= x\\
  y &= x\label{LL}\\
  y &= x
\end{align}
\eqref{LL}において,...
\end{showEx}

\cmd{eqref}による参照時もサイズが落ちます。

\cmd{footnotesize}では,落としすぎですか。\cmd{small}ではどうでしょう。
お好みで.....

\subsubsection{○の太さ}
\cmd{emPmaru}の○印は,\textsf{emathPh}で定義されている\cmd{En}を用いて
描画されています。したがって,このコマンドを使用する際は,
\textsf{emathPh}をロードしておく必要があります。
ということで,○印の線の太さを変更するには,\cmd{thicklines}等が有効です。
\begin{showEx}{\cmd{emPmaru}の線の太さ}
\emPmaru{ア}\hspace{1\zw}%
\thicklines\emPmaru{ア}
\end{showEx}

\subsection{吹き出し}
漫画などで用いられる「吹き出し」もどきです。
\textsf{emathPb.sty}で定義されています。

\subsubsection{\cmd{hukidasi}}
素朴な形から始めます。

\begin{showEx}{\cmd{hukidasi}}
ああああああああああ%
\hukidasi{%
  この間に\\
  いいいいいいい\\
  を補う
}%
うううううううううううううううううう
うううううううううううううううううう
うううううううううううううううううう
うううううううううううううううううう
\end{showEx}

\subsubsection{罫線と文字の間隔}
吹き出しの台詞とそれを囲む罫線との間隔は\cmd{fboxsep}で決まります。
ただし,他への影響を避けるためには\cmd{hukidasi}への\verb+[fboxsep=..]+
オプションを利用するのが良いでしょう。

\begin{showEx}{\texttt{[fboxsep=..]オプション}}
ああああああああああ%
\hukidasi[fboxsep=10pt]{%
  この間に\\
  いいいいいいい\\
  を補う
}%
うううううううううううううううううう
うううううううううううううううううう
\fbox{ええええ}おおおおおおおおおお
\end{showEx}

上下・左右を別々に指定するには,\verb+[hsep=..,vsep=..]+オプションを用います。

\begin{showEx}{\texttt{[hsep=..,vsep=..]オプション}}
ああああああああああ%
\hukidasi[vsep=10pt,hsep=4pt]{%
  この間に\\
  いいいいいいい\\
  を補う
}%
うううううううううううううううううう
うううううううううううううううううう
\fbox{ええええ}おおおおおおおおおお
\end{showEx}

\subsubsection{丸いコーナー}
4隅を丸くするには,\cmd{hukidasi}に\verb+[hankei=..]+で,4隅の4分円の半径を
指定します。値は単位を伴います。

\begin{showEx}{\texttt{[hankei=..]オプション}}
ああああああああああ%
\hukidasi[hankei=10pt]{%
  この間に\\
  いいいいいいい\\
  を補う
}%
うううううううううううううううううう
うううううううううううううううううう
\fbox{ええええ}おおおおおおおおおお
\end{showEx}

この場合,枠と台詞の間の間隔を左右するものは\cmd{fboxsep}ではなく,
\verb+[hankei=..]+の右辺値です。もっと狭くしたい,などのときは
\verb+hsep, vsep+オプションをご利用ください。

\begin{showEx}{\texttt{[hsep=..,vsep=..]オプション}}
ああああああああああ%
\hukidasi[hankei=10pt,%
hsep=2pt,vsep=5pt]{%
  この間に\\
  いいいいいいい\\
  を補う
}%
うううううううううううううううううう
うううううううううううううううううう
\fbox{ええええ}おおおおおおおおおお
\end{showEx}

\subsubsection{吹き出し口の形状}
吹き出し口は,二等辺三角形ですが,その底辺と高さを
\begin{jquote}
\begin{verbatim}
hukidasihaba=..
hukidasitakasa=..
\end{verbatim}
\end{jquote}
オプションで指定することができます。デフォルト値は
\begin{jquote}
\begin{verbatim}
hukidasihaba=20pt
hukidasitakasa=10pt
\end{verbatim}
\end{jquote}
となっています。

\begin{showEx}{吹き出し口の形状}
ああああああああああ%
\hukidasi[hankei=10pt,%
hukidasihaba=10pt,%
hukidasitakasa=20pt]{%
  この間に\\
  いいいいいいい\\
  を補う
}%
うううううううううううううううううう
うううううううううううううううううう
\fbox{ええええ}おおおおおおおおおお
\end{showEx}

\subsubsection{横幅の変更}
吹き出し枠の横幅は,台詞の自然長に設定されますが,
これを指定するには,\verb+[width=..]+オプションを用います。
これは,台詞の文字部分の長さを指定します。

\begin{showEx}{\texttt{width}オプション}
ああああああああああ%
\hukidasi[width=4\zw]{%
  この間に\\
  いいいいいいい\\
  を補う
}%
うううううううううううううううううう
うううううううううううううううううう
うううううううううううううううううう
うううううううううううううううううう
\end{showEx}

\subsubsection{文字サイズの変更}
台詞部分の文字サイズの変更は,\cmd{hukidasi}の前で行います。

\begin{showEx}{文字サイズ変更}
ああああああああああ%
\begin{footnotesize}
\hukidasi[hankei=10pt,%
vsep=4pt,hsep=8pt,%
hukidasihaba=10pt,%
hukidasitakasa=20pt]{%
  この間に\\
  いいいいいいい\\
  を補う
}%
\end{footnotesize}%
うううううううううううううううううう
うううううううううううううううううう
\fbox{ええええ}おおおおおおおおおお
\end{showEx}

\subsubsection{吹き出し口の位置}
吹き出し口は,デフォルトでは,枠の上部につきます。\\
これを変更するには,\verb+[hukidasiiti=T/B/L/R}+オプションを用います
(デフォルトは\verb+T+)。

(注)現時点では\verb+R+オプションは実装していません。

\paragraph{\texttt{[hukidasiiti=B]}オプション}
枠の下部に吹き出し口をつけるのが,\texttt{[hukidasiiti=B]}オプションです。

\begin{showEx}{\texttt{hukidasiiti=B}}
ああああああああああああああああああ
あああああああ%
\hukidasi[hukidasiiti=B]{%
  この間に\\
  いいいいいいい\\
  を補う
}%
うううううううううううううううううう
うううううううううううううううううう
\end{showEx}

コーナーを丸くする\verb+hankei=..+オプションを附加した場合です。

\begin{showEx}{\texttt{hukidasiiti=B}}
ああああああああああああああああああ
あああああああ%
\hukidasi[%
    hankei=10pt,hukidasiiti=B]{%
  この間に\\
  いいいいいいい\\
  を補う
}%
うううううううううううううううううう
うううううううううううううううううう
\end{showEx}

\paragraph{\texttt{[hukidasiiti=L]}オプション}
枠の左辺に吹き出し口をつける\texttt{[hukidasiiti=L]}オプションは,
傍注コマンド\verb+\marginpar+と併用することを前提とします。

次の例では,\verb+\marginpar+を使用するため,
今までの例のようにタイプセット結果を\verb+\shadebox+に入れることができません。

\begin{itembox}{\texttt{hukidasiiti=L}}
\begin{verbatim}
ああああああああああああああああああああああああああああああああああああ
ああああああああああああああああああああああああああああああああああああ
ああああああああああああああああああああああああああああああああああああ
ああああああああああああああああああああああああああああああああああああ
あああああああ%
\marginpar{%
\hukidasi[hukidasiiti=L]{%
  この間に\\
  いいいいいいい\\
  を補う
}%
}%
うううううううううううううううううううううううううううううううううううう
うううううううううううううううううううううううううううううううううううう
うううううううううううううううううううううううううううううううううううう
うううううううううううううううううううううううううううううううううううう
うううううううううううううううううううううううううううううううううううう
\marginpar{ウウウウウウウウウウウ}
ええええええええええええええええええええええええええええええええええええ
ええええええええええええええええええええええええええええええええええええ
\end{verbatim}
\end{itembox}
をタイプセットした結果は\smallskip
\hrule\smallskip
ああああああああああああああああああああああああああああああああああああ
ああああああああああああああああああああああああああああああああああああ
ああああああああああああああああああああああああああああああああああああ
ああああああああああああああああああああああああああああああああああああ
あああああああ%
\marginpar{%
\hukidasi[hukidasiiti=L]{%
  この間に\\
  いいいいいいい\\
  を補う
}%
}%
うううううううううううううううううううううううううううううううううううう
うううううううううううううううううううううううううううううううううううう
うううううううううううううううううううううううううううううううううううう
うううううううううううううううううううううううううううううううううううう
うううううううううううううううううううううううううううううううううううう
\marginpar{ウウウウウウウウウウウ}
ええええええええええええええええええええええええええええええええええええ
ええええええええええええええええええええええええええええええええええええ
\smallskip\hrule\bigskip

\textsf{emathBk.sty}で定義されている\cmd{MigiRangai}などとの併用も可能です。

\begin{showEx}(1,1){\cmd{MigiRangai}との併用}
\begin{breakRline}[fboxsep=10pt,sensyu=\protect\drawline]
ああああああああああああああああああああああああああああああああああああ
ああああああああああああああああああああああああああああああああああああ
ああああああああああああああああああああああああああああああああああああ
ああああああああああああああああああああああああああああああああああああ
あああああああ%
\MigiRangai{%
\hukidasi[hukidasiiti=L]{%
  この間に\\
  いいいいいいい\\
  を補う
}%
}%
うううううううううううううううううううううううううううううううううううう
うううううううううううううううううううううううううううううううううううう
うううううううううううううううううううううううううううううううううううう
うううううううううううううううううううううううううううううううううううう
うううううううううううううううううううううううううううううううううううう
\MigiRangai{ウウウウウウウウウウウ}
ええええええええええええええええええええええええええええええええええええ
ええええええええええええええええええええええええええええええええええええ
\end{breakRline}
\end{showEx}

\textsf{tyuukai}環境内の\cmd{tyuu}との併用をする例です。

\begin{showEx}(1,1){\cmd{tyuu}との併用}
問題問題問題問題問題問題問題問題問題問題問題問題問題問題問題問題問題問題
問題問題問題問題問題問題問題問題問題問題問題問題問題問題問題問題問題問題

\migityuukeisenfalse
\begin{tyuukai}
\tyuumark{}%
【解答】\verb+tyuukai+環境内に解答部分を記述する場合にも,
\cmd{hukidasi}の\verb+[[hukidasiiti=L]+オプションは有効です。
あああああああ
\tyuu{%
\hukidasi[hukidasiiti=L,hankei=8pt,hsep=4pt,vsep=4pt,hukidasihaba=10pt]{%
  この間に\\
  いいいいいいい\\
  を補う
}%
}%
ここで注釈をつけてみます。
うううううううううううううううううううううううううううううううううううう
\end{tyuukai}
\end{showEx}

\subsubsection{\textsf{mawarikomi}環境との併用}
\begin{showEx}(1,1){\textsf{mawarikomi}環境との併用}
あああああああああああああああああああああ
あああああああああああああああああああああ

\begin{mawarikomi}[4]{}{%
\begin{zahyou*}[ul=1\zw](0,4.5)(0,1)
\put(0,-2.5){\hukidasi[hukidasiiti=L,width=4\zw]{アアアアアアアア}}
\end{zahyou*}
}
おおおおおおおおおおおおおおおおおおおおおおおおおおおおおおおおおおおおおお
おおおおおおおおおおおおおおおおおおおおおお
いいいいいいいいいいいいいいいいいいいいいいいいいいいいいいいいいいいいい
いいいいいいいいいいいいいいいいいいいいいいいいいいいいいいいいいいいいい
いいいいいいいいいいいいいいいいいいいいいいいいいいいいいいいいいいいいい
いいいいいいいいいいいいいいいいいいいいいいいいいいいいいいいいいいいいい
いいいいいいいいいいいいいいいいいいいいいいいいいいいいいいいいいいいいい
\end{mawarikomi}
\end{showEx}


\subsection{数列の上下に\cmd{HenKo}}
\subsubsection{数列の上に\cmd{HenKo}}
等差数列の説明図などで,数列の上に円弧を連ねて,
$d$ずつ増えていくことを示すことがあります。
\textsf{emathPh.sty}には,\cmd{DrawHatC}を用意してありますが,
円弧を連ねる場合は,面倒ですから,\cmd{DrawHatS}を作ってみました。

\begin{showEx}(.6,.34){\cmd{DrawHatS}}
  $\DrawHatS{a_1,a_2,a_3, \cdots}{+d}$
\end{showEx}

円弧の部分が上に飛び出しています。\\
円弧の部分は\cmd{HenKo}を用いていますが,
\verb+\begin{picture}(0,0)+の中で\cmd{HenKo}を発行していますので,
そのサイズは$0\times0$としか,\TeX には認識されません。
\cmd{sityuu}で支柱を立てて対応します。

\begin{showEx}(.6,.34){支柱を立てて}
  \sityuu{20pt}%
  $\DrawHatS{a_1,a_2,a_3, \cdots}{+d}$
\end{showEx}

`$+d$'の位置ですが,\cmd{HenKo}を用いていますから,デフォルトでは円弧を切って
その上に配置されます。これをずらすには,\cmd{HenKo}のオプションを用います。
\cmd{DrawHatS}は\verb+<...>+オプション内に,\cmd{HenKo}に渡すオプションを
記述できるようにしてあります。

\begin{showEx}(.6,.34){`$+d$'の位置}
  \sityuu{28pt}%
  $\DrawHatS<putoption={(0,2pt)[b]}>%
    {a_1,a_2,a_3, \cdots}{+d}$
\end{showEx}

\noindent
この場合,必要に応じて支柱の高さも修正します。

円弧に矢印をつけるには,\cmd{HenKo}に\verb+yazirusi=r/a+オプションをつけます。

\begin{showEx}(.6,.34){円弧に矢印}
  \sityuu{28pt}%
  $\DrawHatS%
    <yazirusi=r,putoption={(0,2pt)[b]}>%
    {a_1,a_2,a_3, \cdots}{+d}$
\end{showEx}

`$+d$'の文字サイズを小さくしたければ,
数式モード内ですから\verb+\scriptstyle+などとします。

\begin{showEx}(.6,.34){`$+d$'の文字サイズ}
  \sityuu{28pt}%
  $\DrawHatS<putoption={(0,2pt)[b]}>%
    {a_1,a_2,a_3, \cdots}{\scriptstyle +d}$
\end{showEx}
%\clearpage

円弧の端点(横位置)は,`$a_1$'の中央ですが,添え字に災いされて,
もっと左に寄せたい,と思われる方も多いでしょう。
そのために,\cmd{DrawHatS}に\verb+[offset=..]+オプションを用意しました。

\begin{showEx}(.6,.34){円弧の横位置}
  \sityuu{28pt}%
  $\DrawHatS[offset=-3pt]%
    <putoption={(0,2pt)[b]}>%
    {a_1,a_2,a_3, \cdots}{+d}$
\end{showEx}

円弧の端点(縦位置)は,デフォルトではベースラインの上方\verb+1\zh+
としてあります。
これを変更するオプションが\verb+[takasa=..]+です。
少し下げてみましょう。

\begin{showEx}(.6,.34){円弧の縦位置}
  \sityuu{28pt}%
  $\DrawHatS[takasa=6pt,offset=-3pt]%
    <putoption={(0,2pt)[b]}>%
    {a_1,a_2,a_3, \cdots}{+d}$
\end{showEx}

円弧自体の高さは,\cmd{HenKo}の\verb+<henkoH=..>+オプションで変更できます。

\begin{showEx}(.6,.34){円弧の高さ}
  \sityuu{28pt}%
  $\DrawHatS[takasa=6pt,offset=-3pt]%
    <henkoH=4pt,putoption={(0,2pt)[b]}>%
    {a_1,a_2,a_3, \cdots}{+d}$
\end{showEx}

数列を\verb+a_1,a_2,a_3, \cdots+と与えても,各項の間が空いています。
これは,区切り子をデフォルトで`\verb+,\kern1em+'と設定しているためです。\\
これを修正するのが\verb+[kugirisi=...]+オプションです。
コンマをつけず,間を広げてみます。

\begin{showEx}(.54,.4){区切りの変更}
  \sityuu{28pt}%
  $\DrawHatS
    [kugirisi=\kern2em]%
    {a_1,a_2,a_3, \cdots}{+d}$
\end{showEx}

\cmd{DrawHatS}をつなげることもできます。

\begin{showEx}(.47,.47){\cmd{DrawHatS}の連結}
  \sityuu{28pt}%
$
\DrawHatS{a_1,a_2,a_3, \cdots}{+d}
 \cdots
\DrawHatS{\cdots,a_{n-1},a_n}{+d}
$
\end{showEx}
%\clearpage

\subsubsection{数列の下に\cmd{HenKo}}
\cmd{DrawHatS}の\verb+[takasa=..]+オプションの右辺値が正の値であれば,
円弧は数列の上方につきます。それに対して0以下の値を指定すれば,
円弧は数列の下方につきます。

\begin{showEx}(.6,.34){円弧を下に}
  $\DrawHatS%
    [takasa=-2pt]%
    {a_1,a_2,a_3, \cdots}{+d}$
\end{showEx}

今度は,支柱の深さを増やしてやらねばなりませんし,
\cmd{HenKo}に与えるオプションも変更する必要があります。

\begin{showEx}(.6,.34){\cmd{HenKo}の調整}
  \sityuu[20pt]{0pt}%
  $\DrawHatS%
    [takasa=-2pt]%
    <yazirusi=a,putoption={(0,-2pt)[t]}>%
    {a_1,a_2,a_3, \cdots}{+d}$
\end{showEx}

\subsubsection{階差数列}
等差数列の場合は,\cmd{HenKo}に与える文字列は一律ですが,
階差数列となると,ばらばらです。

\begin{showEx}(.6,.34){階差数列}
  \sityuu[24pt]{0pt}%
  $\DrawHatS%
    [takasa=-4pt]%
    <putoption={(0,-2pt)[t]}>%
    {a_1,a_2,a_3, \cdots}{b_1,b_2,\cdots}$
\end{showEx}

円弧ではなく,折れ線にしたいですか。

\begin{showEx}(.6,.34){\texttt{henkotype=2}オプション}
  \sityuu[24pt]{0pt}%
  $\DrawHatS%
    [takasa=-4pt]%
    <henkotype=2,putoption={(0,-2pt)[t]}>%
    {a_1,a_2,a_3, \cdots}{b_1,b_2,\cdots}$
\end{showEx}

階差数列については,あとでもう一度取り上げます。
%\clearpage

\subsubsection{\cmd{DrawHatS*}}
数列の上下に円弧列を描画するには,
まず,\verb+*+つきのコマンド\cmd{DrawHatS*}で上の円弧を描画し,
ついで,\cmd{DrawHatS}で下の円弧を描画します。

\begin{showEx}(.95,.9){\cmd{DrawHatS*}}
  \sityuu[25pt]{36pt}%
$
  \DrawHatS*[takasa=1.5\zh]<yazirusi=r,putoption={(0,2pt)[b]}>
    {\bunsuu12,\bunsuu34,\bunsuu58,\bunsuu7{16},\cdots}{+2}
  \DrawHatS[takasa=-1\zh]<yazirusi=a,putoption={(0,-2pt)[t]}>
    {\bunsuu12,\bunsuu34,\bunsuu58,\bunsuu7{16},\cdots}{\times2}
$
\end{showEx}

\begin{showEx}(.95,.9){\texttt{<offset=..>}と併用}
  \sityuu[20pt]{28pt}%
$
  \DrawHatS*[offset=-5pt,takasa=1\zh,kugirisi=+]%
    <yazirusi=r,putoption={(0,2pt)[b]}>
    {1\,x\,,3x^2,5x^3,7x^5,\cdots}{+2}
  \DrawHatS[offset=1pt,takasa=-2pt,kugirisi=+]%
    <yazirusi=a,putoption={(0,-2pt)[t]}>
    {1\,x\,,3x^2,5x^3,7x^5,\cdots}{\times x}
$
\end{showEx}
%\clearpage

\subsubsection{書式}
\cmd{DrawHatS} の書式です。

\begin{boxnote}
\begin{verbatim}
\DrawHatS[#1]<#2>#3#4
  #1 : key=val の形式
          kugirisi=.. 数列の項間を制御
                 デフォルトは `,\kern1em'
          takasa=..   円弧の端末とベースラインとの距離 デフォルトは 1\zh
                      (この値を0以下に指定すると,円弧は数列の下方につく)
          offset=..   円弧の端末横位置の微調整 デフォルトは0pt
  #2 : \HenKo の <...> オプションに引き渡される
          henkoH=..      辺と弧の距離(単位付数値) デフォルト値=1.6ex
          putoption=..   この中央に配置される文字列の配置オプション
          yazirusi=..    弧に矢印をつける(aで正の回転,rで負の回転を表す)
          henkotype=..   2とすると,円弧ではなく折れ線
  #3 : 数列(コンマ区切りの文字列)
  #4 : 円弧の中央に配置される文字
       (一律でないときは,コンマ区切りで与える。)

\DrawHatS*[#1]<#2>#3#4
  円弧を描画するのみで,数列はタイプセットされない。
  参照点も動かない。
  引数は \DrawHatS と同じ。
\end{verbatim}
\end{boxnote}
%\clearpage

\subsubsection{\textsf{manDrawHatS}環境}
\cmd{DrawHatS}コマンドでは,円弧などは一律の条件で描画されます。
これらを場所に応じて修正したい,などという目的のために,
\textsf{manDrawHatS}環境を用意しました。

この環境では,円弧は描画されません。
手動で円弧を描画するための材料を提供することが目的です。

この環境は実質的には,\textsf{picture}環境です。
原点は数列,初項の左下にあり,\verb+\unitlength+は\verb+1pt+となっています。
この環境内では,次の変数が定義されています。
\begin{jquote}
\begin{verbatim}
\DHSTi, \DHSTii, \DHSTiii, ....:数列の上方に描画される円弧の端点
\DHSBi, \DHSBii, \DHSBiii, ....:数列の下方に描画される円弧の端点
\end{verbatim}
\end{jquote}

では,一例を

\begin{showEx}(1,.9){\textsf{manDrawHatS}環境}
  \sityuu[20pt]{28pt}%
\begin{manDrawHatS}{1,3,5,7,\cdots}
  \HenKo<putoption={(0,2pt)[b]}>\DHSTii\DHSTi{$\scriptstyle 3-1=2$}
  \HenKo<putoption={(0,2pt)[b]}>\DHSTiv\DHSTiii{$\scriptstyle 7-5=2$}
  \HenKo<putoption={(0,-2pt)[t]}>\DHSBii\DHSBiii{$\scriptstyle 5-3=2$}
  \HenKo<putoption={(0,-2pt)[t]}>\DHSBiv\DHSBv{}
\end{manDrawHatS}
\end{showEx}

上方円弧端点の高さは\verb+[takasa=..]+オプションで指定できるのは
\cmd{DrawHatS}コマンドと同様です。下方円弧については\verb+[hukasa=..]+
オプションで指定します。その例です。

\begin{showEx}(1,.9){\texttt{hukasa=..}オプション}
  \sityuu[30pt]{36pt}%
\begin{manDrawHatS}[takasa=15pt,hukasa=10pt,kugirisi={,\,}]%
    {1,\bunsuu12,\bunsuu14,\bunsuu18,\cdots}
  \HenKo<putoption={(0,2pt)[b]}>\DHSTii\DHSTi{$\times\frac12$}
  \HenKo<putoption={(0,2pt)[b]}>\DHSTiv\DHSTiii{$\times\frac12$}
  \HenKo<putoption={(0,-2pt)[t]}>\DHSBii\DHSBiii{$\times\frac12$}
  \HenKo<putoption={(0,-2pt)[t]}>\DHSBiv\DHSBv{$\times\frac12$}
\end{manDrawHatS}
\end{showEx}

最後に少し複雑な例をご覧ください。

\begin{showEx}(1,.9){雑例}
  \sityuu[36pt]{1\zh}%
\begin{manDrawHatS}{a_1,a_2,a_3,a_4,\cdots,a_{n-1},a_n}
  \HenKo<henkotype=2,putoption={(2pt,-2pt)[t]}>\DHSBi\DHSBii{$b_1$,}
    \edef\P{\HenKoTyuuten}%
  \HenKo<henkotype=2,putoption={(2pt,-2pt)[t]}>\DHSBii\DHSBiii{$b_2$,}
  \HenKo<henkotype=2,putoption={(2pt,-2pt)[t]}>\DHSBiii\DHSBiv{$b_3$,}
  \HenKo<henkotype=2,putoption={(0,-2pt)[t]}>\DHSBiv\DHSBv{$\cdots$}
  \HenKo<henkotype=2,putoption={(0,-2pt)[t]}>\DHSBv\DHSBvi{$\cdots$}
  \HenKo<henkotype=2,putoption={(8pt,-2pt)[t]}>\DHSBvi\DHSBvii{$b_{n-1}$}
    \edef\Q{\HenKoTyuuten}%
  \rotUbrace[depth=12pt]\P\Q{n-1~個}
\end{manDrawHatS}
\end{showEx}

\subsubsection{ふたたび階差数列}
階差数列はよく使いそうですから,\textsf{manDrawHatS}環境を利用したコマンド
\cmd{Kaisasuuretu}を作っておきます。

\begin{showEx}(1,.9){\cmd{Kaisasuuretu}}
  \sityuu[2.2\zh]{1\zh}%
$
  \Kaisasuuretu[kugirisi=\kern1em]<putoption={(0,-2pt)[t]}>%
    {1,2,4,7,11,\cdots}%
    {1,2,3,4,\cdots}
$
\end{showEx}
%\clearpage

第2階差数列まで必要な場合は,

\begin{showEx}(1,.9){\cmd{iiKaisasuuretu}}
  \sityuu[4.4\zh]{1\zh}%
$
  \iiKaisasuuretu[kugirisi=\kern1em]<putoption={(0,-2pt)[t]}>%
    {6,8,18,42,86,156,258,\cdots}%
    {2,10,24,44,70,102,\cdots}%
    {8,14,20,26,32,\cdots}
$
\end{showEx}

連結も可能です。

\begin{showEx}(1,.9){\cmd{iiKaisasuuretu}の連結}
  \sityuu[4.4\zh]{1\zh}%
$
  \iiKaisasuuretu[kugirisi=\kern1em]<putoption={(0,-2pt)[t]}>%
    {a_1,a_2,a_3,a_4,\cdots}%
    {b_1,b_2,b_3,\cdots\hbox to 0pt{$\cdots\cdots\cdots$}}%
    {c_1,c_2,\cdots\hbox to 0pt{$\cdots\cdots\cdots\cdots\cdots$}}%
  \cdots
  \iiKaisasuuretu[kugirisi=\kern1em]<putoption={(0,-2pt)[t]}>%
    {\cdots,a_{n-2},a_{n-1},a_n,\cdots}%
    {\cdots,b_{n-2},b_{n-1},\cdots}%
    {\cdots,c_{n-2},\cdots}%
$
\end{showEx}

\subsection{下線}
文字列に下線を引くコマンドは\LaTeX に\cmd{underline}が用意されています。

\subsubsection{\cmd{phkasen}}
\cmd{phkasen}は,下線を引く\cmd{underline}に
\begin{jquote}
\begin{enumerate}[(1)]
  \item 下線を二重
  \item 下線の太さを変更
  \item 下線の線種を変更
  \item 下線に色をつける
  \item 下線と上下の間隔調整
  \item 下線の近傍に文字を配置
\end{enumerate}
\end{jquote}
機能を附加したものです。ただし,下線は\verb+tpic-specials+で描画されます。

\subsubsection{基本用法}
\cmd{phkasen}の基本用法は\cmd{underline}と同様です。

\begin{showEx}(.7,.24){\cmd{phkasen}}
あい\underline{うえ}お

あい\phkasen{うえ}お
\end{showEx}

下線部と前後の文との間に空白ができますが,これは\cmd{underline}と同様です。
この空白を除去したければ,下線部を\verb+\mbox+で囲むなどの対策をとります。

\begin{showEx}(.7,.24){前後の空白除去}
あい\phkasen{うえ}お

あい\mbox{\phkasen{うえ}}お

あいうえお
\end{showEx}

\subsubsection{下線の太さ}
下線の太さを変更するには,\verb+<allinethickness=..>+オプションを用います。

\begin{showEx}(.7,.24){下線の太さ変更}
あい\phkasen<allinethickness=2pt>{うえ}お
\end{showEx}

\subsubsection{下線の線種}
下線の線種を変更するには,\verb+<sensyu=..>+オプションを用います。

\begin{showEx}(.7,.24){下線の線種変更}
あい\phkasen<sensyu=\hasen>{うえ}お
\end{showEx}


\subsubsection{二重下線}
下線を二重とするには,\cmd{phkasen}に\verb+[..]+オプションを与えます。
\verb+[ ]+内の数値は二本線の間隔で,単位は\verb+pt+です。

\begin{showEx}(.6,.34){二重下線}
\phkasen[1.5]{あいうえお}
\end{showEx}


\subsubsection{下線の色}
下線に色を付けるには,\verb+<iro=..>+オプションを用います。

\begin{showEx}(.7,.24){下線の色}
あい\phkasen<iro=red>{うえ}お
\end{showEx}

\subsubsection{下線上下の間隔}
下線と下線をつけた文字列との間隔を調整するには,
コマンド\cmd{kasenUehosei}の引数に増減する数値(単位付)を与えます。
下線は,正の値で下方に,負の値で上方に移動します。

\begin{showEx}(.6,.34){\cmd{kasenUehosei}}
\kasenUehosei{-6pt}%
\phkasen<iro=red>[2]{あいうえお}
\end{showEx}

コマンド\cmd{kasenUehosei}による変更は,そのコマンドの有効範囲内にある
すべての\cmd{phkasen}に対して働きますが,
\verb+<kasenUehosei=..>+オプションによる変更は,当該下線のみに働きます。

\begin{showEx}(.6,.34){\cmd{kasenUehosei}}
\phkasen<kasenUehosei=-6pt,iro=red>[2]%
  {あいうえお}
\end{showEx}

下線とその下の行との間隔を調整するコマンドが\cmd{kasenSitahosei}です。

\begin{showEx}(.6,.34){\cmd{kasenSiahosei}}
\phkasen{あいうえお}
おおおおおおおおおおおおおおおおお
おおおおおおおおおおおおおおおおお
おおおおおおおおおおおおおおおおお

\kasenSitahosei{10pt}%
\phkasen{あいうえお}
おおおおおおおおおおおおおおおおお
おおおおおおおおおおおおおおおおお
おおおおおおおおおおおおおおおおお
\end{showEx}

第1段落が標準の間隔です。
第2段落は下線の下を\texttt{10pt}増やすように指示されています。
%\pagebreak

\subsubsection{下線近辺に文字配置}\label{sayuu}
下線の左端など,下線の近辺に文字列を配置したいことがあります。
そのために,\cmd{phkasen}に
\verb+'....'+オプションを用意しました。
\verb+'.....'+内に,下線を引く\textsf{zahyou*}環境内に記述する
コマンドを書くことができます。

\begin{showEx}(.6,.34){\texttt{'.....'オプション}}
\verb+'.....'+内に記述したものは,
下線を描画する\textsf{zahyou}環境に
置かれます。この環境の原点は,下線の左端です。
\phkasen%
  '\Put\O(0,0)[c]{$\bullet$}'%
  {あいうえお}

下線の右端は\cmd{XMAX}で,
その座標は\verb+(\xmax,0)+となっています。
\phkasen%
  '\Put\XMAX(0,0)[c]{$\bullet$}'%
  {あいうえお}
\end{showEx}

下線に番号を振って区別する例です。

\begin{showEx}(.6,.34){下線に番号}
あいう
\phkasen<kasenSitahosei=5pt>%
  '\Put\O(0,-2pt)[r]{\scriptsize (1)}'%
  {かきくけこ}
さしすせそ
\phkasen%
  '\Put\O(0,-2pt)[r]{\scriptsize (2)}'%
  {なにぬねの}
\begin{enumerate}[(1)]
  \item 下線部(1)について...
  \item 下線部(2)について...
\end{enumerate}
\end{showEx}
%\pagebreak

\subsubsection{\cmd{phkasen}の書式}
\cmd{phkasen}の書式です。

\begin{boxnote}
\begin{verbatim}
\phkasen<#1>[#2]'#3'#4
  #1 : key=val の形式
       有効なkeyは
          allinethickness
          sensyu
          iro
          kasenUehosei
            (このオプションによる補正は,当該下線のみです。
              一方,コマンド\kasenUehosei は有効範囲内すべてに効きます)
          kasenSitahosei
       で,いずれも効果は局所的です。
  #2 : 下線を二重にするとき,二重線の間隔(無名数で単位はptがつきます。)
  #3 : 下線を引くzahyou*環境内にそのまま配置されます。
       そのzahyou*環境について
           \unitlength は 1pt
           原点(\O)は,下線の左端
           右端が \XMAX, 座標は (\xmax,0)
  #4 : 下線を引く対象
\end{verbatim}
\end{boxnote}


%\section{メモリ不足への対策}
\section{$B$=$NB>(B}
\subsection{$B@~$NB@$5(B}
$B@~$NB@$5$O(B3$B<oN`MQ0U$5$l$F$$$^$9!#(B
$B!J%G%U%)%k%H$O(B \cmd{thinlines} $B$G$9!#!K(B

\showexample[$B@~$NB@$5(B](0.6)(0.34){example/pen01}

$B$b$C$HB@$/$7$?$$!$$H$$$&$H$-$O(B \cmd{allinethickness} $B$H$$$&%3%^%s%I$b(B
\textsf{eepic.sty}$B$GMQ0U$5$l$F$$$^$9!#0z?t$KB@$5$rM?$($^$9!#(B

\showexample[\cmd{allinethickness}](0.6)(0.34){example/pen02}


\printindex
\end{document}
