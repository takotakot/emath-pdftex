\section{円・直線の交点}
直線と直線,円と直線,円と円の交点を求めるコマンド類です.
コマンド名は \verb+\*and*+ の形で \texttt{*} のところは

\begin{jquote}
\begin{verbatim}
    C : 円
    L : 2点を与えた直線
    l : 1点と方向ベクトルを与えた直線
    k : 1点と方向角を与えた直線
\end{verbatim}
\end{jquote}
のいずれかで,
\begin{jquote}
\begin{verbatim}
\CandC
\CandL
\Candl
\Candk
\LandL
\Landl
\landl
\Landk
\kandk
\end{verbatim}
\end{jquote}
の9種類があります.

\subsection{2直線の交点}
\subsubsection{2直線の交点(1) \texorpdfstring{\cmd{LandL}}{LandL}}

\cmd{LandL} の使用例です.\cindex{LandL}

\showexample[\cmd{LandL}](0.55)(0.4){example/LandL01}

\begin{boxnote}
\begin{verbatim}
\LandL#1#2#3#4#5
    2点 #1, #2 を通る直線と
    2点 #3, #4 を通る直線の交点を #5 に与える.
\end{verbatim}
\end{boxnote}

\subsubsection{2直線の交点(2) \texorpdfstring{\cmd{Landl}}{Landl}}
\sankaku{ABC} の辺AB上に $\kaku{BCD}=30\Deg$ となる点Dを求める例です.
\cindex{Landl}

\showexample[\cmd{Landl}](0.55)(0.4){example/LandL02}

\begin{boxnote}
\begin{verbatim}
\Landl#1#2#3#4#5
    2点 #1, #2 を通る直線と
    点 #3 を通り,方向ベクトルが #4 の直線の交点を #5 に与える.
\end{verbatim}
\end{boxnote}

\subsubsection{2直線の交点(3) \texorpdfstring{\cmd{landl}}{landl}}
直線を1点と方向ベクトルで与える場合の,交点を求めるコマンドが
\cmd{landl} です.\cindex{landl}

\showexample[\cmd{landl}](0.6)(0.34){example/LandL03}

\begin{boxnote}
\begin{verbatim}
\landl#1#2#3#4#5
    点 #1 を通り, 方向ベクトルが #2 の直線と
    点 #3 を通り,方向ベクトルが #4 の直線の交点を #5 に与える.
\end{verbatim}
\end{boxnote}

\subsubsection{2直線の交点(4) \texorpdfstring{\cmd{Landk}}{Landk}}
直線を1点と方向角($x$軸の正の向きとなす角---六十分法)で与えた場合です。
まずは \cmd{kandk} の使用例です。\cindex{Landk}

\showexample[\cmd{kandk}](0.54)(0.4){example/LandL04}

\begin{boxnote}
\begin{verbatim}
\kandk#1#2#3#4#5
    点 #1 を通り, 方向角が #2 の直線と
    点 #3 を通り,方向角が #4 の直線の交点を #5 に与える.
\end{verbatim}
\end{boxnote}

ついで,\cmd{Landk} の使用例です。\cindex{Landk}

\showexample[\cmd{Landk}](0.54)(0.4){example/LandL05}

\begin{boxnote}
\begin{verbatim}
\Landk#1#2#3#4#5
    2点 #1, #2 を通る直線と
    点 #3 を通り,方向角が #4 の直線の交点を #5 に与える.
\end{verbatim}
\end{boxnote}

\subsubsection{垂線の足}
三角形の頂点から対辺(またはその延長)に下した垂線の足を求める
コマンド \cmd{Suisen} です.\cindex{Suisen}

\showexample[\cmd{Suisen}](0.55)(0.35){example/suisen01}

\begin{boxnote}
\begin{verbatim}
\Suisen#1#2#3#4
    点 #1 から直線 #2#3 へ下ろした垂線の足を #4 にセット
\end{verbatim}
\end{boxnote}

関連して,直線が1点と方向ベクトルまたは方向角で与えられている場合に
用いるコマンドがそれぞれ
\cmd{mSuisen}, \cmd{kSuisen} です。使用例を一つずつあげます。
\cindex{mSuisen}
\cindex{kSuisen}

\begin{showEx}(.5,.44){\cmd{mSuisen}}
\unitlength9mm\footnotesize
\begin{zahyou}(-3,3)(-3,3)
\tenretu{A(-2,-1)s;P(-1,2)n}
\def\houkouV{(1,1)}
\mSuisen\P\A\houkouV\Q
\kuromaru{\A;\P;\Q}
\Put\Q[se]{Q}
\Drawline{\P\Q}
\Tyokkakukigou\P\Q\A
\mTyokusen\A\houkouV{}{}
\end{zahyou}
\end{showEx}

\begin{showEx}(.5,.44){\cmd{kSuisen}}
\unitlength9mm\footnotesize
\begin{zahyou}(-3,3)(-3,3)
\tenretu{A(-2,-1)s;P(-1,2)n}
\def\kaku{60}
\kSuisen\P\A\kaku\Q
\kuromaru{\A;\P;\Q}
\Put\Q[s]{Q}
\Drawline{\P\Q}\Tyokkakukigou\P\Q\A
\kTyokusen\A\kaku{}{}
\end{zahyou}
\end{showEx}

\subsubsection{直線に関する対称点}
前節の垂線を発展させて,点の直線に関する対称点を求めるコマンド
\begin{jquote}
\begin{verbatim}
\Taisyouten 2点を通る直線
\mTaisyouten 1点と方向ベクトルを指定した直線
\kTaisyouten 1点と方向角を指定した直線
\end{verbatim}
\end{jquote}
を新設しました。以下,その使用例を並べます。
\cindex{Taisyouten}
\cindex{mTaisyouten}
\cindex{kTaisyouten}

\begin{showEx}(.5,.44){\cmd{Taisyouten}}
\unitlength9mm\footnotesize
\begin{zahyou}(-3,3)(-3,3)
\tenretu{A(-2,-1)s;B(1,1)s;P(-1,2)n}
\Taisyouten\P\A\B\Q
\kuromaru{\A;\B;\P;\Q}
\Put\Q[s]{Q}
\Drawline{\P\Q}
\Tyokusen\A\B{}{}
\end{zahyou}
\end{showEx}

\begin{showEx}(.5,.44){\cmd{mTaisyouten}}
\unitlength9mm\footnotesize
\begin{zahyou}(-3,3)(-3,3)
\tenretu{A(-2,-1)s;P(-1,2)n}
\def\houkouV{(1,1)}
\mTaisyouten\P\A\houkouV\Q
\kuromaru{\A;\P;\Q}
\Put\Q[s]{Q}
\Drawline{\P\Q}
\mTyokusen\A\houkouV{}{}
\end{zahyou}
\end{showEx}

\begin{showEx}(.5,.44){\cmd{kTaisyouten}}
\unitlength9mm\footnotesize
\begin{zahyou}(-3,3)(-3,3)
\tenretu{A(-2,-1)s;P(-1,2)n}
\def\kaku{60}
\kTaisyouten\P\A\kaku\Q
\kuromaru{\A;\P;\Q}
\Put\Q[s]{Q}
\Drawline{\P\Q}
\kTyokusen\A\kaku{}{}
\end{zahyou}
\end{showEx}
\bigskip

対称の中心(垂線の足)が必要なときは,オプション引数 \verb/[#4]/ に
その点を受取る制御綴を与えておきます。

\begin{showEx}(.5,.44){対称の中心}
\unitlength9mm\footnotesize
\begin{zahyou}(-3,3)(-3,3)
\tenretu{A(-2,-1)s;B(1,1)s;P(-1,2)n}
\Taisyouten\P\A\B[\H]\Q
\kuromaru{\A;\B;\P;\Q;\H}
\Put\Q[s]{Q}
\Tyokkakukigou\P\H\A
\touhenkigou<2>{\P\H;\H\Q}
\Drawline{\P\Q}
\Tyokusen\A\B{}{}
\end{zahyou}
\end{showEx}

この機能は,\cmd{mTaisyouten}, \cmd{kTaisyouten} にも使用できます。

\begin{showEx}(.5,.44){\cmd{mTaisyouten}}
\unitlength9mm\footnotesize
\begin{zahyou}(-3,3)(-3,3)
\tenretu{A(-2,-1)s;P(-1,2)n}
\def\houkouV{(1,1)}
\mTaisyouten\P\A\houkouV[\H]\Q
\kuromaru{\A;\P;\Q;\H}
\Put\Q[s]{Q}
\Drawline{\P\Q}
\Tyokkakukigou\P\H\A
\touhenkigou<2>{\P\H;\H\Q}
\mTyokusen\A\houkouV{}{}
\end{zahyou}
\end{showEx}

\begin{showEx}(.5,.44){\cmd{kTaisyouten}}
\unitlength9mm\footnotesize
\begin{zahyou}(-3,3)(-3,3)
\tenretu{A(-2,-1)s;P(-1,2)n}
\def\kaku{60}
\kTaisyouten\P\A\kaku[\H]\Q
\kuromaru{\A;\P;\Q;\H}
\Put\Q[s]{Q}
\Drawline{\P\Q}
\Tyokkakukigou\P\H\A
\touhenkigou<2>{\P\H;\H\Q}
\kTyokusen\A\kaku{}{}
\end{zahyou}
\end{showEx}

\subsection{円と直線の交点}
\subsubsection{円と直線の交点(1) \texorpdfstring{\cmd{CandL}}{CandL}}
\sankaku{ABC}の中線AMの延長が\sankaku{ABC}の外接円と交わる点Dを求めます.
\cindex{CandL}

\showexample[\cmd{CandL}](0.5)(0.45){example/CandL01}

\begin{boxnote}
\begin{verbatim}
\CandL#1#2#3#4#5#6
\CandL*#1#2#3#4#5#6
        点 #1 を中心とし,半径 #2 の円と
        2点 #3, #4 を通る直線との交点を #5 と #6 にセットする.
円と直線の2つの交点のどちらを #5 とするかについては
  \CandL の場合
    2つの交点のうち,x座標の小さい方が #5
    2つの交点のx座標が一致するときは,y座標の小さい方が #5
  \CandL* の場合
    円の中心(#1)と2つの交点(#5, #6)で作られる三角形の周を
        #1 → #5 → #6 → #1
    とたどる回り方が正の回転となるように定める。
    (三角形がつぶれる場合は,\CandL の定め方に従う)
\end{verbatim}
\end{boxnote}

\subsubsection{円と直線の交点(2) \texorpdfstring{\cmd{Candl}}{Candl}}
原点中心,半径1の円と,点(2,0)を通り傾き$-\bunsuu13$の直線との交点を求めます.\cindex{Candl}

\showexample[\cmd{Candl}](0.5)(0.45){example/CandL02}

\begin{boxnote}
\begin{verbatim}
\Candl#1#2#3#4#5#6
        点 #1 を中心とし,半径 #2 の円と
        点 #3 を通り, 方向ベクトルが #4 の直線
        との交点を #5 と #6 にセットする.
    2つの交点のうち,どちらを #5 とするかは \CandL と同じ。
\end{verbatim}
\end{boxnote}

\subsubsection{円と直線の交点(3) \texorpdfstring{\cmd{Candk}}{Candk}}
原点中心,半径1の円と,点(2,0)を通り傾き方向角15\Deg の直線
との交点を求めます.\cindex{Candk}

\showexample[\cmd{Candl}](0.5)(0.45){example/CandL03}

\begin{boxnote}
\begin{verbatim}
\Candk#1#2#3#4#5#6
        点 #1 を中心とし,半径 #2 の円と
        点 #3 を通り, 方向角が #4 の直線
        との交点を #5 と #6 にセットする.
    2つの交点のうち,どちらを #5 とするかは \CandL と同じ。
\end{verbatim}
\end{boxnote}

\subsection{円と円の交点 \texorpdfstring{\cmd{CandC}}{CandC}}
線分BCの長さが7のとき,Bを中心とする半径5の円と,Cを中心とする半径3の
円との交点をAとすれば,3辺の長さが7, 5, 3の三角形がえられます.
\cindex{CandC}

\showexample[\cmd{CandC}](0.5)(0.42){example/CandC01}

\begin{boxnote}
\begin{verbatim}
\CandC#1#2#3#4#5#6
        点 #1 を中心,半径 #2 の円と
        点 #3 を中心,半径 #4 の円との交点を#5と#6にセット
円と円の2つの交点のどちらを #5 とするかについては
  \CandC の場合
    2つの交点のうち,x座標の小さい方が #5
    2つの交点のx座標が一致するときは,y座標の小さい方が #5
  \CandC* の場合
    円の中心(#1)と2つの交点(#5, #6)で作られる三角形の周を
        #1 → #5 → #6 → #1
    とたどる回り方が正の回転となるように定める。
    (三角形がつぶれる場合は,\CandC の定め方に従う)
\end{verbatim}
\end{boxnote}
