\documentclass[a4j]{jarticle}
\usepackage{emathT}
\usepackage{arydshln}
\usepackage{showexample}

\begin{document}
\section{罫線を点線で}
\subsection{\textsf{arydshln.sty}}
\textsf{tabular}, \textsf{array}環境の罫線を点線とするには,
\textsf{arydshln.sty}を用います。

\begin{showEx}(.6,.34){\textsf{arydshln.sty}}
\begin{tabular}{|c|c:c|}\hline
A&B&C\\\hline
1&2&3\\\hdashline
4&5&6\\\hline
\end{tabular}
\end{showEx}

すなわち,縦罫線を点線とするには,欄指定子の`$|$'に替えて`:'を用います。
また,横罫線を点線とするには,\cmd{hline}に替えて,\cmd{hdashline}
を用います。\textsf{arydshln.sty}については
\begin{jquote}
\begin{verbatim}
http://www.para.tutics.tut.ac.jp/~nakasima/latex/
\end{verbatim}
\end{jquote}
をお尋ねください。

\subsection{\textsf{hhline.sty}と\textsf{arydshln.sty}の併用}
\textsf{hhline.sty}と\textsf{arydshln.sty}を併用すると,
縦横の罫線がつながらない現象が発生します。

\begin{showEx}(.6,.34){\textsf{hhline}と\textsf{arydshln}との併用}
\begin{tabular}{|c|c:c|}\hline
A&B&C\\\hhline{|=|=|=|}
1&2&3\\\hdashline
4&5&6\\\hline
\end{tabular}
\end{showEx}

表の最上部横罫線と縦罫線が離れています。
この現象は,使用パッケージを
\begin{jquote}
  \sffamily
  hhline, arydshln
\end{jquote}
の2つだけにしても発生することをお確かめください。
すなわち\textsf{emath}のせいではありません。

従って解決は,上の2つのパッケージ側においてなされるべき事柄です。

\subsection{\textsf{emathT}における当面の臨時措置}
それまでの間の臨時措置として,\textsf{emathT.sty}では,
\cmd{hhcr}を用意しました。\cmd{hhline}の直前の\verb+\\+に代えて用います。

ただし,ロード順が問題で,\textsf{emathT}を先,その後で \textsf{arydshln}
をロードしなければなりません。

\begin{showEx}(.6,.34){\cmd{hhcr}}
\begin{tabular}{|c|c:c|}\hline
A&B&C\hhcr\hhline{|=|=|=|}
1&2&3\\\hdashline
4&5&6\\\hline
\end{tabular}
\end{showEx}

ただし,この措置はごまかしでして,点線の縦罫線の部分を
\verb+\hhline{|=|=:=|}+としても横二重罫線の部分を縦罫線が突っ切ってしまいます。
そのくらいは目をつぶろう,という間に合わせです。

\begin{showEx}(.6,.34){\cmd{hhcr}}
\begin{tabular}{|c|c:c|}\hline
A&B&C\hhcr\hhline{|=|=:=|}
1&2&3\\\hdashline
4&5&6\\\hline
\end{tabular}
\end{showEx}

\subsection{\textsf{emathT}の太罫線との併用}
\textsf{emathT}の太罫線との併用を見ておきます。
\begin{showEx}(.6,.34){太罫線との併用}
\begin{Hyou}{IC{1zw}|C{1zw}:C{1zw}I}\hlineb
A&B&C\hhcr\hhline{I=|=:=I}
1&2&3\\\hdashline
4&5&6\\\hlineb
\end{Hyou}
\end{showEx}

エッ!
また罫線がつながっていませんか。罫線が太くなった分修正量を変更しなければ
ならないようです。

\begin{showEx}(.6,.34){太罫線との併用--修正}
\begin{Hyou}{IC{1zw}|C{1zw}:C{1zw}I}\hlineb
A&B&C\hhcr[4.4pt]\hhline{I=|=:=I}
1&2&3\\\hdashline
4&5&6\\\hlineb
\end{Hyou}
\end{showEx}

すなわち,\cmd{hhcr}のオプション引数に\verb+[4.4pt]+と,
修正量を指定しています。このデフォルト値は\verb+[3.2pt]+です。
普通罫線と太罫線との太さの差$2\times(1.0-0.4)=1.2$(pt)増やしているわけです。

\subsection{\cmd{arraystretch}への対応}
\cmd{arraystretch}の変更は大丈夫でしょうか。確認しておきます。

\begin{showEx}(.6,.34){\cmd{arraystretch}の変更}
\def\arraystretch{2}
\begin{Hyou}{IC{1zw}|C{1zw}:C{1zw}I}\hlineb
A&B&C\hhcr[4.4pt]\hhline{I=|=:=I}
1&2&3\\\hdashline
4&5&6\\\hlineb
\end{Hyou}
\end{showEx}

つぎは,\cmd{extrarowheight}

\begin{showEx}(.6,.34){\cmd{extrarowheight}の指定}
\extrarowheight=20pt\relax
\begin{Hyou}{IC{1zw}|C{1zw}:C{1zw}I}\hlineb
A&B&C\hhcr[4.4pt]\hhline{I=|=:=I}
1&2&3\\\hdashline
4&5&6\\\hlineb
\end{Hyou}
\end{showEx}
いずれも問題ないようです。

\subsection{改行量の変更}
\cmd{extrarowheight}と反対に,改行量を変更してみます。
表の最下行に下余白をつけてみます。

\begin{showEx}(.6,.34){改行量の変更}
\begin{Hyou}{IC{1zw}|C{1zw}:C{1zw}I}\hlineb
A&B&C\hhcr[4.4pt]\hhline{I=|=:=I}
1&2&3\\\hdashline
4&5&6\\[20pt]\hlineb
\end{Hyou}
\end{showEx}

こちらもよさそうですが,後で取り上げるように\cmd{multicolumn}と
併用すると問題が発生します。

\subsection{2箇所に横二重罫線}
横二重罫線を2箇所以上で使用するときは\cmd{hhcr}の修正量を
調整する必要があります。

\begin{showEx}(.6,.34){2箇所に横二重罫線}
\begin{tabular}{|c|c:c|}\hline
A&B&C\hhcr\hhline{|=|=|=|}
1&2&3\\\hdashline
A&B&C\hhcr[2.8pt]\hhline{|=|=|=|}
4&5&6\\\hline
\end{tabular}
\end{showEx}
\clearpage

\subsection{数式を含んだ表}
表のカラムに数式を含む場合,行の高さが変化することがあります。

\begin{showEx}(.64,.3){数式を含む場合}
$\begin{hyou}{IC{1zw}|C{1zw}:C{1zw}I}\hlineb
A&B&C\hhcr[4.4pt]\hhline{I=|=:=I}
1&\bunsuu{\bunsuu1c}{\bunsuu1d}&3\\\hdashline
4&5&6\\\hlineb
\end{hyou}$
\end{showEx}

真中のカラムに繁分数式を放り込んで見ました。
罫線は追随しているようです。
もっともこの場合は2行目の高さをもう少し増やしてやりたいですね。
\textsf{emath}の\cmd{EMvphantom}というコマンドが便利でしょう。

\begin{showEx}(.64,.3){\cmd{EMvphantom}による調整}
$\begin{hyou}{IC{1zw}|C{1zw}:C{1zw}I}\hlineb
A&B&C\hhcr[4.4pt]\hhline{I=|=:=I}
\EMvphantom[5pt]{\bunsuu{\bunsuu1c}{\bunsuu1d}}
1&\bunsuu{\bunsuu1c}{\bunsuu1d}&3\\\hdashline
4&5&6\\\hlineb
\end{hyou}$
\end{showEx}

\subsection{\cmd{multicolumn}は問題です}
出発点に戻って,使用パッケージを
\begin{jquote}
  \sffamily
  hhline, arydshln
\end{jquote}
の2つだけにして状況をみておきます。

\begin{showEx}(.6,.34){\textsf{multicolumn}は問題です。}
\begin{tabular}{|c|c:c|}\hline
A&B&C\\\hhline{|=|=|=|}
1&2&3\\\hdashline
4&\multicolumn{2}{c|}{11}\\[20pt]\hline
\end{tabular}
\end{showEx}

なんとも悲惨な状況です。\cmd{hhcr}を使用しても

\begin{showEx}(.6,.34){\cmd{hhcr}も不十分}
\begin{tabular}{|c|c:c|}\hline
A&B&C\hhcr\hhline{|=|=|=|}
1&2&3\\\hdashline
4&\multicolumn{2}{c|}{11}\\[20pt]\hline
\end{tabular}
\end{showEx}

\cmd{hhcr}の修正量を変更しても
\begin{showEx}(.6,.34){\cmd{hhcr}の修正量変更も}
\begin{tabular}{|c|c:c|}\hline
A&B&C\hhcr[23.2pt]\hhline{|=|=|=|}
1&2&3\\\hdashline
4&\multicolumn{2}{c|}{11}\\[20pt]\hline
\end{tabular}
\end{showEx}
だいぶ改善されましたが,右下の欄に縦の点線が侵入しています。

\cmd{hhcr}のいい加減さがばれてしまいました。

何とかの上塗りで,右端にダミー列を置いてみました。

\begin{showEx}(.6,.34){右端にダミー列}
\begin{Hyou}{|C{1zw}|C{1zw}:C{1zw}C{-8pt}|}
  \hline
A&B&C&\hhcr\hhline{|=|=:==|}
1&2&3&\\\hdashline
4&\multicolumn{2}{c}{11}&\\[20pt]\hline
\end{Hyou}
\end{showEx}

苦し紛れのしのぎです。

でも振り出しに戻り,解決は
\begin{jquote}
  \sffamily
  hhline, arydshln
\end{jquote}
側においてなされるべきことで,それまでの間に合わせです。

これで不完全ではありますが,
\begin{jquote}
  細罫線\\
  太罫線\\
  二重罫線\\
  点線罫線
\end{jquote}
を併用することが可能となったのは,表の表現力を増すことになるでしょう。
\end{document}
