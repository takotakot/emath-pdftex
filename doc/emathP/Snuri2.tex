\section{塗りつぶし(2)}
この節のマクロは,\textsf{emathPg.sty}で定義されています。

\subsection{整関数のグラフで囲まれる図形の塗りつぶし}
放物線 $y=x^2$ の上にある2点 A$(-1,~1)$, B(2, 4) を結ぶ線分と
放物線とで囲まれる図形を塗りつぶします.

\showexample[放物線と弦で囲まれる図形](.5)(0.4){example/nuri01}

\cmd{Nuri}の書式です.\cindex{Nuri}
\begin{boxnote}
\begin{verbatim}
\Nuri[#1](#2)#3#4#5
    #1 : 濃さ
    #2 : x の刻み値
    #3 : 整関数
    #4 : x1
    #5 : x2
#3 で与えられる整関数のグラフとその上の
2点 (x1,y1), (x2,y2) を結ぶ弦との間を
指定された濃さで塗りつぶします.
\end{verbatim}
\end{boxnote}

次は2つの整関数で挟まれた部分を塗りつぶします.

\showexample[2つの放物線で囲まれる図形](.5)(0.4){example/nuri02}

\cmd{Nurii}の書式です.\cindex{Nurii}
\begin{boxnote}
\begin{verbatim}
\Nurii[#1](#2)#3#4#5#6
    #1 : 濃さ
    #2 : x の刻み値
    #3 : 整関数1
    #4 : 整関数2
    #5 : x1
    #6 : x2
上下は #3 と #4 で与えられる整関数のグラフ
左右は直線 $x=x_1$, $x=x_2$ で挟まれる部分を
指定された濃さで塗りつぶします.
\end{verbatim}
\end{boxnote}

\subsection{一般の関数のグラフで囲まれる図形の塗りつぶし}
整関数用の \cmd{Nuri}, \cmd{Nurii} は,一般関数ではそれぞれ
\cmd{yNuri}, \cmd{yNurii} に対応します.\cindex{yNuri}\cindex{yNurii}

\showexample[一般関数と弦で囲まれる図形](.5)(0.4){example/nuri03}

\showexample[2つの一般関数で囲まれる図形](.5)(0.4){example/nuri04}

\subsection{$x=g(y)$ で与えられる曲線の塗りつぶし}
曲線が $x=g(y)$ で与えられる場合は、\cmd{yNurii} に代えて \cmd{xNurii}
などを用います.\cindex{xNurii}

\showexample[$x=g(y)$で囲まれる図形](.6)(0.5){example/nuri07}

\subsection{媒介変数表示の曲線で囲まれる図形の塗りつぶし}
整関数の \cmd{Nuri} に対応するものが \cmd{bNuri} です.
\cindex{bNuri}

\showexample[媒介変数表示曲線で囲まれる図形](1)(0.7){example/nuri05}

\subsection{極方程式であらわされた曲線で囲まれる図形の塗りつぶし}
整関数の \cmd{Nuri} に対応するものが \cmd{rNuri} です.
\cindex{rNuri}

\showexample[極方程式で表される曲線の塗りつぶし](1)(0.7){example/nuri06}

