\section{三角形の決定}
\subsection{三辺}
三辺の長さがわかっているときに,その三角形をどのように描画するか,
という話です.

具体例として $\mathrm{AB=7,~BC=5,~CA=3}$ としましょう.
まず A(0,0), B(7,0) とします.
点Cは,Aを中心とする半径3の円と,Bを中心とする半径5の円との交点
として求めます.

\showexample[三角形の決定(1)三辺](0.55)(0.8){example/sankaku1}

\subsection{二角夾辺}
$\mathrm{BC=5,~\kaku{B}=60\Deg,~\kaku{C}=45\Deg}$
としましょう.
これは,B(0,0), C(5,0) とした上で
\begin{jquote}
    点Bを通り,方向角 $60\Deg$ の直線と\\
    点Cを通り,方向角 $135\Deg$ の直線
\end{jquote}
の交点としてAを求めます.

\showexample[三角形の決定(2) 二角夾辺](0.8)(0.8){example/sankaku2}

\subsection{二辺夾角}
$\mathrm{AB=3,~BC=5,~\kaku{B}=60\Deg}$
としましょう.
これは B(0,0), C(5,0) とした上で A の座標を極座標→直交座標変換で
求めます.すなわち
\begin{jquote}
\begin{verbatim}
    \kyokuTyoku(3,60)\A
\end{verbatim}
\end{jquote}

\showexample[三角形の決定(3) 二辺夾角](0.8)(0.8){example/sankaku3}
