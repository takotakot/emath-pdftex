\section{斜線塗り(2)}
この節のマクロは,\textsf{emathPg.sty}で定義されています。

前節の,曲線で囲まれた図形の塗りつぶしコマンドの末尾に \verb+*+ を
付加すると,斜線による塗りつぶしを行います.

代表として,\cmd{yNurii*} の使用例を見ていただきましょう.

$y=\sin x$, $y=\cos x$ と2直線 $x=0$, $x=2\pi$ で囲まれる図形を
斜線で塗りつぶしています.斜線の方向角を60度としてあります.

\showexample[2曲線で囲まれる図形の斜線塗り](.6)(0.6){example/hatch07}

\cmd{yNurii*}の書式です.\cindex{yNurii*}
\begin{boxnote}
\begin{verbatim}
\yNurii*[#1]<#2>(#3)#4#5#6#7
    #1 : 斜線の方向角(デフォルト値 = 45)単位は度
    #2 : 斜線の間隔(デフォルト値 = 0.125)
    #2 : 折れ線近似をする時の x の刻み値(デフォルト 0.1)
    #3 : 関数1
    #4 : 関数2
    #5 : x1
    #6 : x2
上下は #3 と #4 で与えられる関数のグラフ
左右は直線 $x=x_1$, $x=x_2$ で挟まれる部分を
斜線で塗りつぶします.
\end{verbatim}
\end{boxnote}

このほか \cmd{Nuri*}, \cmd{Nurii*}, \cmd{yNuri*}, \cmd{xNurii*},\cmd{bNuri*},
\cmd{rNuri*} も同様です.
\cindex{Nuri*}\cindex{Nurii*}\cindex{yNuri*}\cindex{xNurii*}
\cindex{bNuri*}\cindex{rNuri*}

斜線を点線・破線で描画する方法については,
§\ref{S-hasen} (\pageref{S-hasen} ページ)をご覧ください.

\showexample[点線による斜線塗り](.6)(0.6){example/hatch10}
