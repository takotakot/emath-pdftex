\section{\textsf{zahyou}環境}
\subsection{なぜ\textsf{zahyou}環境か}
\LaTeX で座標平面を描画するには,\textsf{picture}環境があります。
ただし,負の数を扱うには,\textsf{picture}環境の引数を
計算しなければなりません。
例えば,$-3<x<3$, $-2<y<2$の範囲を描画するには

\begin{showEx}(.5,.44){\textsf{picture}環境}
\unitlength=8mm
\begin{picture}(6,4)(-3,-2)%
\put(-3,0){\vector(1,0){6}}%
\put(0,-2){\vector(0,1){4}}%
\end{picture}%
\end{showEx}

引数の与え方が面倒であることと,
座標軸の描画もまとめて面倒を見てしまおう,
ということで\textsf{zahyou}環境です。

\subsection{\textsf{zahyou}環境}

\begin{showEx}(.5,.44){\textsf{zahyou}環境}
\unitlength=8mm
\begin{zahyou}(-3,3)(-2,2)%
\end{zahyou}%
\end{showEx}

\textsf{picture}環境とは,引数の与え方が異なり,
\begin{jquote}
\begin{verbatim}
(xmin, xmax)(ymin, ymax)
\end{verbatim}
\end{jquote}
と,$x$, $y$の区間を与えます。

なお,この環境内では,次の変数が定義されています。
\begin{jquote}
\begin{verbatim}
\xmin xの区間の最小値
\xmax xの区間の最大値
\ymin yの区間の最小値
\ymax yの区間の最大値

\O    原点
\XMAX x軸の右端の点
\XMIN x軸の左端の点
\YMAX y軸の上端の点
\YMIN y軸の下端の点
\RT   描画領域右上のコーナー
\RB   描画領域右下のコーナー
\LT   描画領域左上のコーナー
\LB   描画領域左下のコーナー
\end{verbatim}
\end{jquote}
\bigskip

座標軸を描画する必要がない場合に対しては
\textsf{zahyou*}環境を用意してあります。

\begin{showEx}(.5,.44){\textsf{zahyou*}環境}
\unitlength=8mm
\begin{zahyou*}(-3,3)(-2,2)%
  \Drawline{\LB\RT\RB\LT\LB}
  \En\O{2}%
\end{zahyou*}%
\end{showEx}

上記リスト中,\cmd{Drawline}は,指定点を折れ線で結びます。
\verb+\En\O{2}+は点\cmd{O}(原点)を中心とする半径2の円を描画しています。
詳細は後述します。

\textsf{zahyou}環境には,細かい微調整をするためのオプションがありますが,
これについては後述します。

\section{直線図形}
\subsection{折れ線}
折れ線を描画するのに,\textsf{epic.sty} で
定義されている \cmd{drawline} を用いることができます.

\showexample[\cmd{drawline}](0.55)(0.35){example/oresen01}

このスタイルファイルでは,点を \cmd{def}\cmd{A\{(3,0)\}}などと
変数で表すことにしています.その場合は,\cmd{drawline} は使えません.
代わりに \cmd{Drawline} を用意しました.このコマンドは内部で
\cmd{drawline} を呼び出していますから,\textsf{epic.sty} を必要とします.

\showexample[\cmd{Drawline}](0.55)(0.35){example/oresen02}
\cindex{Drawline}\bigskip

\cmd{Drawline} の書式は

\begin{boxnote}
\begin{verbatim}
\Drawline#1
    #1 : 点列
\end{verbatim}
\end{boxnote}

\subsection{点線}
点線を描画するには\textsf{eepic.sty}で定義されている\cmd{dottedline}を用います。

\begin{showEx}(.64,.3){\cmd{dottedline}}
\begin{zahyou}[ul=5mm](-3,3)(-3,3)
  \dottedline{.2}(-2,2)(2,-1)
\end{zahyou}
\end{showEx}

\subsubsection{\cmd{emDottedline}}
\textsf{emath}では,点列を一つにまとめた扱いますので,
\cmd{emDottedline}を定義しました。

\begin{showEx}(.64,.3){\cmd{emDottedline}}
\begin{zahyou}[ul=5mm](-3,3)(-3,3)
  \tenretu{A(-2,2)w;B(2,-1)e}
  \emDottedline{\A\B}
\end{zahyou}
\end{showEx}

\cmd{dottedline}では,第1引数に点を打つ間隔を指定しますが,\cmd{unitlength}を単位とする
無名数であることが使い難いので,\cmd{emDottedline}では省略することとしました。
デフォルトは\verb+3pt+としてありますが,これを変更するには\verb+<G=..>+オプションを用います。

\subsubsection{\texttt{<G=..>}オプション}
さて,その\verb+<G=..>+オプションを用いて,もう少し点を稠密にして見ましょう。

\begin{showEx}(.64,.3){\texttt{<G=..>}オプション}
\begin{zahyou}[ul=5mm](-3,3)(-3,3)
  \tenretu{A(-2,2)w;B(2,-1)e}
  \emDottedline<G=2pt>{\A\B}
\end{zahyou}
\end{showEx}
\newpage

\subsubsection{\texttt{<C=..>}オプション}
\cmd{dottedline}には,配置する点の形状を変更するオプション\verb+[...]+が用意されています。

\cmd{emDottedline}では,\verb+<C=..>+オプションを用意しました。

\begin{showEx}(.64,.3){\texttt{<C=..>}オプション}
\begin{zahyou}[ul=5mm](-3,3)(-3,3)
  \tenretu{A(-2,2)w;B(2,-1)e}
  \emDottedline<C=$*$,G=5pt>{\A\B}
\end{zahyou}
\end{showEx}

小さい円(黒塗りの)にしますか。

\begin{showEx}(.64,.3){小さい円を配置}
\begin{zahyou}[ul=5mm](-3,3)(-3,3)
  \ukansan{1pt}\tyokkei
  \tenretu{A(-2,2)w;B(2,-1)e}
  \emDottedline<C=\circle*\tyokkei>{\A\B}
\end{zahyou}
\end{showEx}

ここで登場しているコマンド\verb+\ukansan#1#2+は,
\verb+#1+に与えられた,単位付の長さを,\cmd{unitlength}を単位とする無名数に換算した結果を
\verb+#2+の制御綴に与えるものです。

\subsubsection{応用例}
奥村先生の掲示板 \verb+Q&A+ に「行列の点々」という投稿がありました。
\textsf{emath}を用いる一案です。

\setbox0=\hbox{$b_{n-1}+d_{n-1}$}\setlength{\templa}{\wd0}\edef\wdtbl{\the\templa}%
\setlength{\templa}{2\wd0}\edef\xlen{\mumeisuu\templa}
\setlength{\templa}{\ht\strutbox+\dp\strutbox+20pt}
\edef\ylen{\mumeisuu\templa}
\def\PQ{(0,-3)(\xlen,-\ylen)}%
\[
\setlength{\arraycolsep}{0pt}%
  \left[
    \begin{array}{*8{C{\wdtbl}}}
    a_1 &b_1+d_1  &c_1    &   &   &   &   \\
      &a_2    &b_2+d_2  &c_2    &   &   &   \\
      &   &\EMcell<\emDottedline\PQ>{c}{a_3}
          &\EMcell<\emDottedline\PQ>{c}{b_3+d_3}
          &\EMcell<\emDottedline\PQ>{c}{c_3}    &   &   \\[20pt]
      &   &   &   &   &   & \\
      &   &   &   &a_{n-1}  &b_{n-1}+d_{n-1}&c_{n-1}  \\
      &   &   &   &   &a_n    &b_n+d_n  & c_n
    \end{array}
  \right]
\]

ただし,このコマンドは\textsf{multido.sty}で定義されている\cmd{multido}を用います。
つぎの\cmd{Dottedline}は,\cmd{multido}は使いません。

\subsubsection{旧定義 \cmd{Dottedline}}
旧コマンドは \cmd{Dottedline} も有効です.

\showexample[\cmd{Dottedline}](0.55)(0.35){example/tensen01}

このコマンドは内部で \textsf{eepic.sty} の \cmd{dottedline} を
呼び出しています.書式もほぼ同様で

\begin{boxnote}
\begin{verbatim}
\Dottedline#1#2
    #1 : dotgap
    #2 : 点列
\end{verbatim}
\end{boxnote}

\subsection{破線}
\subsubsection{\texorpdfstring{\cmd{Dashline}}{Dashline}}
破線を引くコマンドは \cmd{Dashline} です.\cindex{Dashline}

\showexample[\cmd{Dashline}](0.55)(0.35){example/hasen01}

このコマンドは内部で \textsf{eepic.sty} の \cmd{dashline} を
呼び出しています.書式もほぼ同様で

\begin{boxnote}
\begin{verbatim}
\Dashline[#1]#2#3
    #1 : stretch
    #2 : dashlength
    #3 : 点列
\end{verbatim}
\end{boxnote}
です.\texttt{[dashdotgap]} はサポートしていません.
\texttt{[stretch]} で代用してください.

\showexample[\cmd{Dashline}\texttt{[stretch]}](0.55)(0.35){example/hasen02}

\subsubsection{\texorpdfstring{\cmd{hasen}}{hasen}}
\cmd{dashline}の第1引数は \cmd{unitlength}の値によって変えなければなりません。

\begin{showEx}{\cmd{Dashline}と\cmd{unitlength}}
\unitlength=10mm
\begin{picture}(3,3)
\dashline{.2}(0,0)(3,3)
\end{picture}\\
\unitlength=1mm
\def\B{(30,30)}
\begin{picture}(30,30)
\dashline{2}(0,0)(30,30)
\end{picture}
\end{showEx}

これは面倒ですから,\cmd{hasen}を作りました。

\begin{showEx}{\cmd{hasen}}
\unitlength=10mm
\begin{picture}(3,3)
\hasen(0,0)(3,3)
\end{picture}\\
\unitlength=1mm
\begin{picture}(30,30)
\hasen(0,0)(30,30)
\end{picture}
\end{showEx}

また,\cmd{dashline}における第1引数は不要とし,
\cmd{hasen}の形状はオプション引数で指定する方式を取りました。
\cmd{hasen}の書式です。

\begin{boxnote}
\begin{verbatim}
\hasen[#1](x1,y1)(x2,y2).....(xN,yN)
    #1 : L=(破線の実線部分の長さ) デフォルト値=1mm
       : G=(破線のギャップの長さ) デフォルト値=0.9mm
    (x1,y1)...(xN,yN) : 折れ線の頂点列
\end{verbatim}
\end{boxnote}\cindex{hasen}

オプション引数を与えて破線の形状を変更する例です。

\begin{showEx}{\cmd{hasen}の形状変更}
\unitlength=10mm
\begin{picture}(3,3)
\hasen[L=2mm,G=2mm](0,0)(3,3)
\end{picture}
\end{showEx}

ただし,ギャップの長さは必ずしも指定した長さにはなりません。
というのは,破線の両端における実線部分が指定した長さとなるように
ギャップを調整しているからです。

細かいことですが,\cmd{drawline}で描画した直線と
\cmd{dashline}で描画した破線の位置がずれることがあります。
これを修正するのも\cmd{hasen}を開発した目的の一つです。

\subsubsection{\texorpdfstring{\cmd{Hasen}}{Hasen}}
\cmd{drawline}, \cmd{dashline} に対して,\cmd{Drawline}, \cmd{Dashline}
があるように,\cmd{hasen} に対しても \cmd{Hasen} を定義しています。
その書式は

\begin{boxnote}
\begin{verbatim}
\Hasen[#1]#2
    #1 : L=(破線の実線部分の長さ) デフォルト値=1mm
       : G=(破線のギャップの長さ) デフォルト値=0.9mm
    #2 : 破線で結ぶ点列
\end{verbatim}
\end{boxnote}\cindex{Hasen}

\begin{showEx}(.58,.36){\cmd{Hasen}}
\begin{zahyou}[ul=10mm](-.5,2.5)(-.5,1.5)
\tenretu{A(0,1)w;B(2,1)e;C(2,0)s}
\Hasen{\A\B\C}
\end{zahyou}
\end{showEx}
複数の点列を定義する\cmd{tenretu}など,
まだ説明していないコマンドが登場してしまいましたが,
ここでは折れ線を破線で引く例としてご覧ください。

\subsubsection{\texorpdfstring{\cmd{Hasen*}}{Hasen*}}
2点を破線で結ぶコマンド\cmd{hasen}, \cmd{Hasen}は,
両端から実線部分が始まりますが,これを両端からは空白部分が
始まるようにしたものが\verb+*+つきコマンドです。

下の例では,実線ABと破線BCの境界がBではなくなってしまいます。

\begin{showEx}(.6,.345){\cmd{Hasen}}
\begin{zahyou}[ul=10mm](-.5,1.5)(-2.5,2.5)
\tenretu{A(1,2)e;B(1,1)e;C(1,-1)e;D(1,-2)e}
\kuromaru{\A;\B;\C;\D}
\Drawlines{\A\B;\C\D}
\Hasen{\B\C}
\end{zahyou}
\end{showEx}

\cmd{Hasen*}と比較してみてください。
\begin{showEx}(.6,.345){\cmd{Hasen*}}
\begin{zahyou}[ul=10mm](-.5,1.5)(-2.5,2.5)
\tenretu{A(1,2)e;B(1,1)e;C(1,-1)e;D(1,-2)e}
\kuromaru{\A;\B;\C;\D}
\Drawlines{\A\B;\C\D}
\Hasen*{\B\C}
\end{zahyou}
\end{showEx}
\cindex{Hasen*}


\subsection{鎖線}
鎖線をひくコマンドは \cmd{Chainline}です。\cindex{Chainline}

{\unitlength8mm
\showexample[\cmd{Chainline}](0.64)(0.3){example/sasen01}}

その書式は
\begin{boxnote}
\begin{verbatim}
\Chainline[#1][#2]#3#4
    #1 : 一つの線分の長さ
    #2 : 線分と線分の間の長さ
    #3 : 始点
    #4 : 終点
\end{verbatim}
\end{boxnote}

\subsection{複数の折れ線}
複数の折れ線を描画するコマンド \cmd{Drawlines} もあります。
\cindex{Drawlines}
複数の折れ線を`;'で区切ります。

\showexample[\cmd{Drawlines}](0.55)(0.35){example/oresen03}

折れ線の線種を変更するには,\verb/<...>/ オプションを用います。

\showexample[\cmd{線種の変更}](0.55)(0.35){example/oresen04}

\cmd{Drawlines}の書式です。

\begin{boxnote}
\begin{verbatim}
\Drawlines<#1>#2
  <#1>  : <sensyu=\dashlines[40]{.1}>などによる
          線種のローカルな変更
   #2   : 複数の点列を`;'で区切る
\end{verbatim}
\end{boxnote}

\subsection{矢線}
矢線を引くコマンドは2種類用意してあります。

\cmd{yasen}は,成分を指定します。

もうひとつの\cmd{ArrowLine}は,始点と終点を指定します。

\subsubsection{\texorpdfstring{\cmd{yasen}}{yasen}}
まずは成分を与えて矢線を描画するコマンド \cmd{yasen} を使用する一例です:

\begin{showEx}(.54,.4){\cmd{yasen}}
  \begin{zahyou*}[ul=10mm](0,4)(0,3)
    \put(0,0){\kousi43}%
    \put(1,0){\yasen(-1,3)}%
    \put(3,0){\yasen(1,3)}%
  \end{zahyou*}
\end{showEx}

このコマンドは,矢線の傍に文字列を配置するオプションをもっています。

\begin{showEx}(.54,.4){\texttt{<..>}オプション}
  \begin{zahyou*}[ul=10mm](0,4)(0,3)
    \put(0,0){\kousi43}%
    \put(1,0){%
      \yasen<[ne]{\beku a}>(-1,3)}%
    \put(3,0){%
      \yasen<[nw]{\beku b}>(1,3)}%
  \end{zahyou*}
\end{showEx}

すなわち\verb+<...>+オプションに,文字列を与えます。その際,\cmd{Put}の
配置オプションを附加することができます。
なお,文字列を配置する基準点は,矢線の中点です。
これを動かすのが\verb+[..]+オプションです。

\begin{showEx}(.54,.4){\texttt{[..]}オプション}
  \begin{zahyou*}[ul=10mm](0,4)(0,4)
    \put(0,0){\kousi43}%
    \put(1,0){%
      \yasen[1]<[n]{\beku a}>(-1,3)}%
    \put(3,0){%
      \yasen[.8]<[nw]{\beku b}>(1,3)}%
  \end{zahyou*}
\end{showEx}

\verb+[..]+オプションのデフォルト値は\verb+0.5+,すなわち中点です。
これを\verb+[1]+とすれば,終点,\verb+[0]+とすれば,始点がそれぞれ
文字列配置の基準点となります。

\cmd{yasen}の書式です。

\begin{boxnote}
\begin{verbatim}
\yasen[#1]<#2>(#3,#4)
\Yasen[#1]<#2>#3
  #1 : ラベルを置く位置(デフォルト=0.5, 中点)
  #2 : ラベル(位置指定を含めて)
  (#3,#4) : 成分(始点は \put (\Put) で指定)
\end{verbatim}
\end{boxnote}
\cindex{yasen}

成分を,まとめてひとつにして扱うものが \cmd{Yasen} です。
\cindex{Yasen}

\begin{showEx}(.54,.4){\cmd{Yasen}}
  \begin{zahyou*}[ul=10mm](0,4)(0,3)
    \def\avec{(1,3)}
    \put(0,0){\kousi43}%
    \put(0,0){\Yasen\avec}%
    \put(3,0){\Yasen\avec}%
  \end{zahyou*}
\end{showEx}

\cmd{Yasen}に対しても,\verb+[...]+, \verb+<...>+ オプションが
\cmd{yasen}と同様に使用できます。


\subsubsection{\texorpdfstring{\cmd{ArrowLine}}{ArrowLine}}
矢線を引くコマンドが \cmd{ArrowLine} です.

\showexample[\cmd{ArrowLine}](0.55)(0.35){example/yasen01}

矢線の傍に文字列を配置することも可能です。

\begin{showEx}(.54,.4){\cmd{ArrowLine}の文字列配置オプション}
  \begin{zahyou*}[ul=10mm](0,4)(-1,4)
    \tenretu{A(0,0)s;B(3,0)s;
      C(4,3)n;D(1,3)n}
    \Drawline{\B\C\D}
    {\thicklines
    \ArrowLine<putstr=[s]{\beku b}>
      \A\B
    \ArrowLine<putstr=[nw]{\beku d}>
      \A\D}%
  \end{zahyou*}
\end{showEx}

すなわち,\verb+<...>+オプションとして,\verb+putstr+の右辺値に,
文字列を配置オプションとともに記述します。配置基準点(デフォルトは中点)
を変更するには,キーワード\verb+putpos+を用います。

\begin{showEx}(.54,.4){\cmd{ArrowLine}の文字列配置基準点の変更}
  \begin{zahyou*}[ul=10mm](0,4)(-1,4)
    \tenretu{A(0,0)s;B(3,0)s;
      C(4,3)n;D(1,3)n}
    \Drawline{\B\C\D\A}
    {\thicklines
    \ArrowLine<putpos=0.9,
      putstr=[n]{\beku b}>
        \A\B}
  \end{zahyou*}
\end{showEx}

\cmd{ArrowLine}の書式です。

\begin{boxnote}
\begin{verbatim}
\ArrowLine<#1>[#2]#3#4
  #1 : key=val
         sensyu=
         putstr=      (矢線の傍に置く配置オプション+文字列)
         putpos=      (文字列の配置基準位置:デフォルトは0.5, 中点)
  #2 : 矢印を置く位置(デフォルト = 1 すなわち終点)
       ただし #2=b のときは,両端に矢印
  #3 : 始点
  #4 : 終点
\end{verbatim}
\end{boxnote}
\cindex{ArrowLine}

\subsubsection{始点・終点位置の微調整}
\cmd{ArrowLine}で引かれる矢線の始点・終点を少しずらしたいときがあります。
例えば

\begin{showEx}(.64,.3){\cmd{ArrowLine}}
\begin{zahyou*}[ul=10mm](-.5,2.5)(-.5,1.5)
  \tenretu{A(0,0)w;B(2,1)e}
  \Kuromaru\B
  \ArrowLine\A\B
\end{zahyou*}
\end{showEx}

\noindent
のような場合,矢印が終点の黒丸にめり込んでいます。
矢印の終点を左下に少しずらしたいときなどのために
\begin{jquote}
\begin{verbatim}
<Henvi=..>    始点に対する修正ベクトル
<Henvii=...>  終点に対する修正ベクトル
\end{verbatim}
\end{jquote}
オプションを新設しました。
右辺値はベクトルで,成分は単位を伴った長さです。
右辺値には`,'が入りますから,右辺値全体を\verb+{...}+でくくる必要があります。

\begin{showEx}(.64,.3){\texttt{<Henvii=..>}オプション}
\begin{zahyou*}[ul=10mm](-.5,2.5)(-.5,1.5)
  \tenretu{A(0,0)w;B(2,1)e}
  \Kuromaru\B
  \ArrowLine<Henvii={(-.8pt,-.4pt)}>\A\B
\end{zahyou*}
\end{showEx}

なお
\begin{jquote}
\begin{verbatim}
<henvi=..>    始点に対する修正ベクトル
<henvii=...>  終点に対する修正ベクトル
\end{verbatim}
\end{jquote}
も同機能ですが,右辺値の成分は\cmd{unitlength}を単位とする無名数で与えます。

\subsubsection{鏃の形状}
鏃のサイズを変更することもできます.例えば
\begin{jquote}
\begin{verbatim}
\changeArrowHeadSize{1.5}
\end{verbatim}
\end{jquote}
とすれば,鏃の大きさは1.5倍になります。\cindex{changeArrowHeadSize}

{\showexample[\cmd{changeArrowHeadSize}](0.65)(0.3){example/yasen02}}

鏃の開き具合も調整できます.デフォルトは\cindex{ArrowHeadAngle}
\begin{jquote}
\begin{verbatim}
\def\ArrowHeadAngle{18}%
\end{verbatim}
\end{jquote}
となっています.これを 30 とすると,鏃は正三角形になります.

{\showexample[\cmd{ArrowHeadAngle}](0.65)(0.3){example/yasen03}}

サイズ・開き具合の両方を同時に変更するときは
\begin{jquote}
\begin{verbatim}
\changeArrowHeadSize[30]{3}
\end{verbatim}
\end{jquote}

と\cmd{changeArrowHeadSize}のオプションで開き角を指定することもできます。

\showexample[\cmd{ArrowHeadAngle}](0.65)(0.3){example/yasen07}

\cmd{ArrowLine}で引かれる矢線において,鏃は二等辺三角形を塗りつぶしていますが,
その底辺に窪みをつけることも可能です。
窪みの深さの「二等辺三角形の高さを1としての比率」
を\cmd{ArrowHeadPit}で指定します。

次の図では,座標軸が2個描画されますが,上が従来の形状,下が窪みをつけた形状です。

\begin{showEx}(.5,.44){\cmd{ArrowHeadPit}}
\begin{zahyou}[ul=8mm](-3,3)(-3,3)
  \ArrowLine\LB\RT
\end{zahyou}

\def\ArrowHeadPit{0.25}
\begin{zahyou}[ul=8mm](-3,3)(-3,3)
  \ArrowLine\LB\RT
\end{zahyou}
\end{showEx}

\cmd{changeArrowHeadSize}に\verb+<...>+オプションを用いて,
窪みを指定することもできます。
\begin{jquote}
\begin{verbatim}
  \def\ArrowHeadPit{.25}
と
  \changeArrowHeadSize<.25>{1}
\end{verbatim}
は同値です。前者の方が簡潔ですが,矢印の長さ・開き角を変更する際には後者の方が
便利でしょうか。

そんな馬鹿な,という極端な例です:

\begin{showEx}(.5,.44){\cmd{changeArrowSize}による指定}
\changeArrowHeadSize[30]<0.3333>{4}
\bekutorukata{fill}

\bekutoru{AB}

\begin{zahyou}[ul=8mm](-3,3)(-3,3)
  \ArrowLine\LB\RT
\end{zahyou}
\end{showEx}

\end{jquote}

なお,これら矢線では鏃が塗りつぶされますが,これを枠線のみ描画させるには
\begin{jquote}
\begin{verbatim}
\renewcommand\ArrowHeadType{l}
\end{verbatim}
\end{jquote}
とします。
デフォルトは \verb/\newcommand\ArrowHeadType{f}/ としてあります.
\cindex{ArrowHeadType}

\showexample[塗りつぶさない鏃](0.55)(0.35){example/yasen06}

\subsubsection{矢印の位置}
矢印を途中に付けるためには,\texttt{[...]} オプションで,
矢印を置く位置を指定します.

{\showexample[\cmd{ArrowLine[0.5]}](0.65)(0.4){example/yasen05}}

また,矢印を両向きにつけたいときは\verb+[b]+オプションをつけます。

\begin{showEx}{両向き矢印}
\unitlength10mm\relax
\begin{picture}(2,1)
\def\A{(0,.5)}
\def\B{(2,.5)}
\ArrowLine[b]\A\B
\end{picture}
\end{showEx}

\subsubsection{矢線を点線・破線}

矢線の線種を変更するには \cmd{ArrowLine} に\verb+<sensyu=...>+ オプションを
付与します。

\begin{showEx}(.54,.4){矢線を点線で}
\unitlength10mm\relax
\begin{picture}(2,1)
\def\A{(0,.5)}
\def\B{(2,.5)}
\ArrowLine%
  <sensyu=\protect\dottedline{.2}>%
  \A\B
\end{picture}
\end{showEx}

\begin{showEx}(.54,.4){矢線を破線で}
\unitlength10mm\relax
\begin{picture}(2,1)
\def\A{(0,.5)}
\def\B{(2,.5)}
\ArrowLine%
  <sensyu=\protect\hasen>%
  \A\B
\end{picture}
\end{showEx}

\subsubsection{複数の矢線}
複数の矢線を引くコマンドが \cmd{ArrowLines} です。
始点・終点を`;'で区切って並べます。

\begin{showEx}{\cmd{ArrowLines}}
\unitlength=10mm
\begin{picture}(4,2)
\def\A{(0,0)}
\def\B{(3,0)}
\def\C{(4,2)}
\def\D{(1,2)}
\ArrowLines[.5]{%
  \A\B;\B\C;\C\D;\D\A}
\end{picture}
\end{showEx}

\subsection{多角形}
多角形を描画するコマンドが\cmd{Takakkei}です。\cindex{Takakkei}
\bigskip

\showexample[\cmd{多角形}](0.55)(0.35){example/polygon2}

\cmd{Takakkei}の始点Oと終点Bを結ぶ線分が強制的に付加されます。
\bigskip

\subsection{正多角形,極座標→直交座標}
円周上に点をとるときは,極座標形式の方が便利です.
これを直交座標に変換するコマンド \cmd{kyokuTyoku} です.
使用例として,正六角形を描画してみました.\cindex{kyokuTyoku}

\showexample[\cmd{kyokuTyoku}](0.6)(0.35){example/polygon0}

\begin{jquote}
\cmd{kyokuTyoku(1,90)}\cmd{A}
\end{jquote}
で,極座標 (1,90) を直交座標に変換した (0,1) が \cmd{A} に
セットされます.\cmd{kyokuTyoku} の書式です.

\begin{boxnote}
\begin{verbatim}
\kyokuTyoku(#1,#2)#3
    (#1,#2) : 極座標
    #3      : 変換した直交座標を代入するコントロールシーケンス
\end{verbatim}
\end{boxnote}

\begin{enumerate}[(1)]
  \item このスタイルファイルでは,原則として,角の単位は六十分法で表します.
  \item \cmd{kyokuTyoku}は点を一つずつ定義しますが,
    複数の点列を定義するコマンド
\begin{jquote}
\begin{verbatim}
\rtenretu, \rtenretu*
\end{verbatim}
\end{jquote}
    もあります。
  \item 正多角形を描画するための\textsf{emPoly.sty}もあります。
\end{enumerate}

\subsection{角の丸い多角形}
多角形を描画するコマンド\cmd{Takakkei}の
バリエーション\cmd{ovalTakakkei}コマンドは
多角形のコーナーを丸くするコマンドです。
まずは,\cmd{Takakkei}の使用例からご覧ください。

\begin{showEx}(.54,.4){\cmd{Takakkei}}
  \begin{zahyou*}%
      [ul=10mm,Ueyohaku=1em,
        Hidariyohaku=1em,%
        Sitayohaku=1em]%
      (0,4)(0,4)
    \tenretu{A(1,0)s;B(4,1)e;%
      C(2,4)n;D(0,3)w}
    \Takakkei{\A\B\C\D}
  \end{zahyou*}
\end{showEx}

コーナーを切り取って,円弧で結びます。
切り取る線分の長さを\cmd{ovalTakakkei}の第1引数に与えます。
単位を伴った数値を指定します。

\begin{showEx}(.54,.4){\cmd{ovalTakakkei}}
  \begin{zahyou*}%
      [ul=10mm,Ueyohaku=1em,
        Hidariyohaku=1em,%
        Sitayohaku=1em]%
      (0,4)(0,4)
    \tenretu*{A(1,0)s;B(4,1)e;%
      C(2,4)n;D(0,3)w}
    \ovalTakakkei{5mm}{\A\B\C\D}
  \end{zahyou*}
\end{showEx}


\subsection{分点}
線分を内分(外分)する点を求めるコマンド \cmd{Bunten} です.
\sankaku{ABC} の辺BCの中点MとAを結ぶ線分を描画する例です.

\showexample[\cmd{Bunten}](0.55)(0.35){example/bunten01}

\cmd{Bunten} の書式は
\begin{boxnote}
\begin{verbatim}
\Bunten#1#2#3#4#5
    #1 : 端点1
    #2 : 端点2
    #3#4 : 内(外)分比
    #5 : 分点
\end{verbatim}
\end{boxnote}
すなわち \verb+#1+ と \verb+#2+ を結ぶ線分を \verb+#3:#4+ に
分ける点の座標を \verb+#5+ にセットします.

特に$1:1$の内分点,すなわち中点を求めるコマンド\cmd{Tyuuten}も便利です。

\showexample[\cmd{Bunten}](0.55)(0.35){example/bunten02}

\subsection{格子}
碁盤の目状の街路図などは良く登場します.
格子を描画するコマンド \cmd{kousi} です.\cindex{kousi}

\showexample[\cmd{kousi}](0.4)(0.4){example/kousi01}

縦横のサイズを変更したいときは,
\texttt{(横サイズ,縦サイズ)} オプションを指定します.
サイズの単位は \cmd{unitlength} です.

\showexample[\cmd{kousi(..,..)オプション}](0.45)(0.5){example/kousi02}

\cmd{kousi} の書式です.

\begin{boxnote}
\begin{verbatim}
\kousi(#1,#2)#3#4
    #1 : 横方向1区画の長さ(デフォルト=1)
    #2 : 縦方向1区画の長さ(デフォルト=1)
    #3 : 横方向のブロック数
    #4 : 縦方向のブロック数
置く位置は \put で指定する.
\end{verbatim}
\end{boxnote}

格子を点線・破線で引いてみます。

\showexample[格子線を破線で引く](0.6)(0.34){example/kousi03}

%外枠だけは実線でとなると,

%\showexample[外枠は実線で](0.6)(0.34){example/kousi04}

\subsection{さいころ}
さいころの目を表すコマンド \cmd{saikoro} を用意しました.
\cindex{saikoro}\index{さいころ}

\showexample[さいころ](0.45)(0.45){example/saikoro}

ただし,このコマンドは\textsf{ascmac.sty}で定義されている
\cmd{keytop}コマンドを使用していますから,このスタイルファイルを
使用することが前提です.
さらに,このスタイルファイルは \textsf{emathP} から読み込まれる
\textsf{eepic.sty}と相性が悪いことがあります.その対策を
\textsf{itembbox.sty}で施していますので,\textsf{ascmac.sty}に
引き続き \textsf{itembbox.sty}も読み込んでおく必要があります.
