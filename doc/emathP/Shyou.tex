\section{作表}
この節の環境,コマンド類は \textsf{emathT.sty} で定義されています。
したがって,プリアンブルで
\begin{jquote}
  \verb/\usepackage{emathT}/
\end{jquote}
を宣言しておく必要があります.

\subsection{\textsf{Hyou} 環境}
\subsubsection{列幅指定}
表を作成するには,\textsf{tabular}環境,\textsf{array}環境があります.
これらの環境で作成される表の列幅は,列の中に置かれる内容によって
自動的に定まります.これは便利ですが,ときには列幅を指定したいことも
あります.
そのための環境として \textsf{Hyou}環境を作りました.
その一例です.

\showexample[列幅一定の表](1)(.9){example/hyou01}

\textsf{Hyou}環境は,実質 \textsf{tabular}環境と同じです.
\index{Hyou @ Hyou 環境}
ただ,欄指定子として \texttt{l, c, r} に加え,\texttt{L, C, R} が
追加され,これらは欄の横幅を引数にとります.上の例では,
\verb/*{4}{C{4\zw}|}/として,4欄すべての横幅を\texttt{4\zw}の
中央揃えと指定しています.

\subsubsection{カラムに斜線}
また,この環境内では,\cmd{sya} コマンドで欄に斜線を引くことができます.
\cmd{sya}コマンドの書式です.\cindex{sya}

\begin{boxnote}
\begin{verbatim}
   \sya(#1)[#2][#3]<#4>
       #1 : 横幅
       #2 : 斜線の向き
           n 右上から左下(デフォルト)
           r 左上から右下
           x クロス
       #3=高さと深さ
       #4 picture 環境内の記述
\end{verbatim}
\end{boxnote}

各種オプションの説明です。

\verb/[#2]/オプションは,斜線の向きを指定します。
前ページの例では\verb/[r]/オプションで
左上から右下方向の斜線を指定しています。

つぎに,\verb/<#4>/オプションについて簡単に述べます.

\textsf{Hyou}環境で \cmd{sya}コマンドを用いた欄には \textsf{picture}環境が
定義されています.
これを用いて斜線の左下,右上に文字列を配置することが出来ます.
そのためのコマンドが \cmd{Hyoumidasi} で,書式は
\begin{boxnote}
\begin{verbatim}
    \Hyoumidasi#1#2
        #1 : 左下の文字列
        #2 : 右上の文字列
\end{verbatim}
\end{boxnote}
\cindex{Hyoumidasi}

斜線の縦・横幅の調整をするオプションが \verb/(#1)/, \verb/[#3]/ 
オプションです。

まずは \verb/(#1)/ オプションの使用例です。

\showexample[\cmd{sya}の横幅指定オプション](1)(.9){example/hyou02}

\cmd{sya}の横幅についてデフォルトでは,
\textsf{Hyou}環境の引数で L, C, R でサイズ指定した最後の値に
設定されます。
上の例では,\texttt{3\zw}です。
それ以外の横幅のところに斜線をひくにはその横幅を\verb/(#1)/オプションで
指定する必要があります。

次に高さ・深さを変更するオプション\verb/[#3]/オプションの使用例です。

\showexample[\cmd{sya}の高さ指定オプション](1)(.9){example/hyou03}

この例では,1行目が高くなっていますので,その分を\cmd{vphantom}で
\cmd{sya}に伝えています。

\paragraph{\cmd{Lmidasi}}
表の左上欄に斜線を引き見出しをつける際,
その行の文字位置を調整する手段を提供します。

デフォルトの確認です。

\begin{showEx}(1,.9){\textsf{Hyou}環境のデフォルト}
\begin{Hyou}{|L{8\zw}|*2{C{2\zw}|}} \hline
  \sya(8\zw)[r]<\Hyoumidasi{行見出し}{列見出し}> & A & B \\\hline
  あいうえお & 1 & 2 \\\hline
  かきくけこ & 3 & 4 \\\hline
\end{Hyou}
\end{showEx}
\bigskip

行見出し,列見出しとも窮屈ですから,1行目の行高を増やすことにします。

\begin{showEx}(1,.9){1行目の行高を増やす}
\begin{Hyou}{|L{8\zw}|*2{C{2\zw}|}} \hline
  \sya(8\zw)[r][\stackrel{ }{ }\strut]<\Hyoumidasi{行見出し}{列見出し}> &A&B
    \\\hline
  あいうえお & 1 & 2 \\\hline
  かきくけこ & 3 & 4 \\\hline
\end{Hyou}
\end{showEx}

\bigskip

「行見出し」という文字の位置を動かす変数が \verb+\Lmidasiiti+です。
デフォルトは
\begin{jquote}
\begin{verbatim}
\def\Lmidasiiti{(3pt,1pt)[lb]}%
\def\Rmidasiiti{(-3pt,-1pt)[rt]}%
\end{verbatim}
\end{jquote}
となっています。
(\verb+Rmidasiiti+は「列見出し」の文字位置指定変数です。)
この書式は,\verb+\emathPut+の位置指定オプションの形式です。
配置基準点は,
\begin{jquote}
「行見出し」の方が,当該欄の左下コーナー\\
「列見出し」の方が,当該欄の右上コーナー
\end{jquote}
です。
%\clearpage

下の例では,
\begin{jquote}
\begin{verbatim}
\def\Lmidasiiti{(6pt,2pt)[lb]}%
\def\Rmidasiiti{(-6pt,-2pt)[rt]}%
\end{verbatim}
\end{jquote}
と変更しています。

\begin{showEx}(1,.9){\cmd{Lmidasiiti}}
\begin{Hyou}{|L{8\zw}|*2{C{2\zw}|}} \hline
  \def\Lmidasiiti{(6pt,2pt)[lb]}%
  \def\Rmidasiiti{(-6pt,-2pt)[rt]}%
  \sya(8\zw)[r][\stackrel{ }{ }\strut]<\Hyoumidasi{行見出し}{列見出し}> &A&B
    \\\hline
  あいうえお & 1 & 2 \\\hline
  かきくけこ & 3 & 4 \\\hline
\end{Hyou}
\end{showEx}

\paragraph{\cmd{agezoko}}
列見出し -- 上の例での A B -- の位置を修正する変数が
\verb+\agezoko+です。下の例では,
\begin{jquote}
\begin{verbatim}
\def\agezoko{2}
\end{verbatim}
\end{jquote}
とすることにより,ベースラインを上に\verb+2pt+持ち上げています。

\begin{showEx}(1,.9){\cmd{agezoko}}
\begin{Hyou}{|L{8\zw}|*2{C{2\zw}|}} \hline
  \def\Lmidasiiti{(6pt,2pt)[lb]}%
  \def\Rmidasiiti{(-6pt,-2pt)[rt]}%
  \def\agezoko{2}%
  \sya(8\zw)[r][\stackrel{ }{ }\strut]<\Hyoumidasi{行見出し}{列見出し}> &A&B
    \\\hline
  あいうえお & 1 & 2 \\\hline
  かきくけこ & 3 & 4 \\\hline
\end{Hyou}
\end{showEx}

\paragraph{逆向きの斜線に対する\cmd{Hyoumidasi}}
斜線の向きが逆,すなわち左下と右上を結ぶときも\\
\cmd{Hyoumidasi}が使えるようにしたい,というのが今回の改定です。

\begin{showEx}(.5,.44){\cmd{Hyoumidasi}}
\def\arraystretch{1.25}
\begin{Hyou}{|*3{C{4\zw}|}}\hline
  X & 1 & 2 \\\hline
  Y & 10 & 20 \\\hline
  \sya[n]<\Hyoumidasi{左上}{右下}>
    & A & B \\\hline
\end{Hyou}
\end{showEx}

\paragraph{斜線位置の微調整}
\textsf{Hyou}環境で斜線を引く際

\begin{showEx}(.56,.38){\textsf{Hyou}環境の斜線}
\begin{Hyou}{|l|r|r|}\hline
  \sya(8\zw)[r][\bsityuu] & A & B \\\hline
  x & 1 & 2 \\\hline
\end{Hyou}
\end{showEx}

\noindent
とした場合,左上の位置に不満があるというご意見をいただくことがあります。
斜線の左上を少し下げたい,ということのようです。

\begin{showEx}(.56,.38){斜線位置の微調整}
\begin{Hyou}{|l|r|r|}\hline
  \sya(8\zw)[o][\bsityuu]%
   <{\ArrowLine<Henvi={(0,-.3pt)},%
       arrowheadsize=0>\LT\RB}>%
   & A & B \\\hline
  x & 1 & 2 \\\hline
\end{Hyou}
\end{showEx}

下げすぎですって ? \verb+<Henvi=..>+の右辺値をお好みで調整してください。
斜線の右下も調整したければ,\verb+Henvii=..+も指定することとなります。

\begin{showEx}(.56,.38){斜線右下も微調整}
\begin{Hyou}{|l|r|r|}\hline
  \sya(8\zw)[o][\bsityuu]%
   <%
    {%
     \ArrowLine%
      <%
       Henvi={(0,-.3pt)},%
       Henvii={(0,.15pt)},%
       arrowheadsize=0%
      >%
     \LT\RB
    }%
    \Put\RT(-3pt,-2pt)[rt]{行見出し}%
    \Put\LB(3pt,2pt)[lb]{列見出し}%
   >
    & A & B \\\hline
  x & 1 & 2 \\\hline
\end{Hyou}
\end{showEx}

\subsection{\textsf{hyou} 環境}
\index{hyou @ hyou 環境}
\textsf{Hyou}環境が,横幅を指定した \textsf{tabular}環境であるのに対し,
\textsf{hyou}環境は,横幅を指定した \textsf{array}環境です.

使用例として,増減表をご覧いただきましょう.

\showexample[増減表](1)(1){example/zougen1}

次に,微分不能の点には斜線を引いてみましょう.

\showexample[増減表](1)(1){example/zougen2}

$y=x\sqrt{4-x^2}$ の増減表です.ついでにグラフも添えておきます.

\begin{center}
\unitlength1cm
\begin{zahyou}<><><(2pt,-2pt)[lt]>(-3,3)(-3,3)%
\def\Fx#1#2{\Mul{#1}{#1}\y
    \Sub{4}\y\y\Heihoukon\y\y
    \Mul{#1}\y\y
    \edef#2{\y}}%
\Put{(2,0)}(0,-2pt)[t]{2}%
\Put{(1.41,0)}(0,-2pt)[t]{$\sqrt2$}%
\Put{(-2,0)}(0,-2pt)[rt]{$-2$}%
\Put{(-1.41,0)}(0,2pt)[b]{$-\sqrt2$}%
\Put{(0,2)}(0,0)[r]{2 }%
\Put{(0,-2)}(0,0)[l]{ $-2$}%
\dashline[40]{.1}(1.41,0)(1.41,2)(0,2)%
\dashline[40]{.1}(-1.41,0)(-1.41,-2)(0,-2)%
{\thicklines
\yGurafu\Fx{-1.414}{0}%
\yGurafu\Fx{0}{1.414}%
\yGurafu\Fx{-2}{-1.414}%
\yGurafu\Fx{1.414}{2}%
}%
\end{zahyou}
\end{center}

増減表で,増加・減少を表現する矢印に次のものを用意しました.

\begin{boxnote}
\begin{Hyou}{*{4}{L{4\zw} }L{6\zw}}
コマンド & 方位 & 記号 & 内容 \\
\cmd{NE} & 北東 & $\NE$ & 増加 \\
\cmd{SE} & 南東 & $\SE$ & 減少 \\
\cmd{NEN}& 北北東 & $\NEN$ & 下に凸で増加\\
\cmd{NEE}& 東北東 & $\NEE$ & 上に凸で増加\\
\cmd{SES}& 南南東 & $\SES$ & 上に凸で減少\\
\cmd{SEE}& 東南東 & $\SEE$ & 下に凸で減少
\end{Hyou}
\end{boxnote}
\cindex{NE}
\cindex{NEE}
\cindex{NEN}
\cindex{SE}
\cindex{SES}
\cindex{SEE}

\subsection{表の罫線を太く}
\subsubsection{\cmd{arrayrulewidth}}
表の罫線全部を太くするのは,\LaTeX で用意されている
\cmd{arrayrulewidth}の値を変更することで実現できます。

\begin{showEx}{\cmd{arrayrulewidth}の変更}
\arrayrulewidth1pt\relax
\begin{tabular}{|c|c|c|c|}\hline
  A & B & C & D \\\hline
  1 & 2 & 3 & 4 \\\hline
  a & b & c & d \\\hline
\end{tabular}
\end{showEx}

\subsubsection{外枠のみを太く}
外枠だけを太くしたい,など一部の罫線を太くするには,
面倒な手順を踏まねばなりませんので,マクロ化することにしました。

横罫線を太くするために
\begin{jquote}
\begin{verbatim}
\hline, \cline
\end{verbatim}
\end{jquote}
にかえて,それぞれ
\begin{jquote}
\begin{verbatim}
\hlineb, \clineb
\end{verbatim}
\end{jquote}
を新設しました。これらの罫線の太さは\cmd{arrarulewidthb}で指定します。
デフォルトは\verb+1pt+としてあります。
縦罫線を太くする位置には
\begin{jquote}
\begin{verbatim}
|
\end{verbatim}
\end{jquote}
にかえて
\begin{jquote}
\begin{verbatim}
I
\end{verbatim}
\end{jquote}
を用います。

では,これらを用いて外枠だけを太くしてみましょう。

\begin{showEx}(.5,.44){外枠を太く}
\begin{tabular}{Ic|c|c|cI}\hlineb
  A & B & C & D \\\hline
  1 & 2 & 3 & 4 \\\hline
  a & b & c & d \\\hlineb
\end{tabular}
\end{showEx}

\subsubsection{二重罫線との併用}
さらに,見出し行・列と表の内容との境界を二重線にすることもできます。
この部分は\textsf{hhline.sty}の一部を修正しています。
なお,\textsf{emathT.sty}はその中で,\textsf{array.sty}と
\textsf{hhline.sty}を読み込んでいます。

\begin{showEx}(.5,.44){二重罫線も}
\begin{tabular}{Ic||c|c|cI}\hlineb
  A & B & C & D \\\hhline{I=#=|=|=I}
  1 & 2 & 3 & 4 \\\hline
  a & b & c & d \\\hlineb
\end{tabular}
\end{showEx}

\subsubsection{太罫線の太さ}
太罫線の太さは\cmd{arrayrulewidthb}で決まりますから,もっと太くしたければ

\begin{showEx}(.5,.44){\cmd{arrayrulewidthb}}
\arrayrulewidthb=2pt\relax
\begin{tabular}{Ic||c|c|cI}\hlineb
  A & B & C & D \\\hhline{I=#=|=|=I}
  1 & 2 & 3 & 4 \\\hline
  a & b & c & d \\\hlineb
\end{tabular}
\end{showEx}

\subsubsection{特定のブロック枠を太く}

\begin{showEx}(.5,.44){一部分を太罫線で}
\begin{tabular}{|c|c|c|c|}\hlineb
  A & B & C & D \\\hline
  \noalign{\vskip-\arrayrulewidth}
  \clineb{2-3}
  \multicolumn{1}{|cI}{1}
    & \multicolumn{1}{c|}{2}
    & \multicolumn{1}{cI}{3}
    & 4 \\
  \noalign{\vskip-\arrayrulewidthb
  \vskip\arrayrulewidth}
  \clineb{2-3}
  \noalign{\vskip\arrayrulewidthb
  \vskip-\arrayrulewidth}\hline
  a & b & c & d \\\hlineb
\end{tabular}
\end{showEx}

\subsection{罫線を点線で}
罫線を点線で引くには,\textsf{arydshln.sty}を用いることが出来ます。
しかし,このスタイルファイルは\textsf{hhline.sty}と相性が悪いので,
\textsf{emathT.sty}との併用については,別文書(emathT-arydshln.tex)
で述べることとします。

%\clearpage

\subsection{\textsf{hyou}環境・縦罫線に波線}
表の縦罫線に二重の波線を描くには,\textsf{emathPp.sty}の
機能を必要とします。\\
\textsf{samplePp.tex}をご覧ください。

\subsection{\textsf{hyou}環境の行高指定}
\textsf{hyou}環境,すなわち\textsf{array}環境の行の高さ・深さは
表の中に入るものによって変わります。

\begin{showEx}(.64,.3){\textsf{hyou}環境の行高}
\[
\begin{hyou}{|*5{C{1\zw}|}} \hline
  x & 0 &\cdots & 1 & \cdots \\\hline
  y & \sya & \NE   & \bunsuu12 & \SE \\\hline
\end{hyou}
\]
\end{showEx}
では,2行目に分数が入る分だけ行の高さ・深さが変化します。
その結果,斜線についても問題が発生しています。

\subsubsection{\cmd{gyoudaka}}
そこで,コマンド\cmd{gyoudaka}を新設して,
表の行高を揃えてしまうことを試みました。

\begin{showEx}(.64,.3){\cmd{gyoudaka}}
\[
\gyoudaka[10pt]{15pt}
\begin{hyou}{|>{\gyousityuu}*5{C{1\zw}|}} \hline
  x & 0 &\cdots & 1 & \cdots \\\hline
  y & \sya & \NE   & \bunsuu12 & \SE \\\hline
\end{hyou}
\]
\end{showEx}

すなわち,
\begin{jquote}
\begin{verbatim}
\gyoudaka[深さ]{高さ}
\end{verbatim}
\end{jquote}
で,行の高さ・深さを指定し,\textsf{hyou}環境の欄指定子に
\begin{jquote}
\begin{verbatim}
>{\gyousityuu}
\end{verbatim}
\end{jquote}
と,各行に支柱を立てます。
斜線の高さも追随します。

\cmd{gyoudaka}の引数は単位を伴う数値です。
%\clearpage

\subsubsection{\cmd{Gyoudaka}}
高さ・深さを指定するのに,\cmd{vphantom}などを用いるためのコマンドが
\cmd{Gyoudaka}です。

\begin{showpEx}(.64,.3){\cmd{Gyoudaka}}
\[
\Gyoudaka{\EMvphantom[2pt]{$\bunsuu12$}}
% 以下,hyou環境の記述は同じ
!\begin{hyou}{|>{\gyousityuu}*5{C{1\zw}|}} \hline
!  x & 0 &\cdots & 1 & \cdots \\\hline
!  y & \sya & \NE   & \bunsuu12 & \SE \\\hline
!\end{hyou}
\]
\end{showpEx}
分数$\bunsuu{1}{2}$の上下に\texttt{2pt}ずつの余白をつけた支柱を
すべての行に立てています。

\subsection{増減表}
増減表については,前にも触れましたが,細かいことも論じておきましょう。

\subsubsection{増減表(1) --- \textsf{array}環境}
まずは基本からということで,\textsf{array}環境を使用したものから見ていきます。

\begin{showEx}(1,1){\textsf{array}環境}
\[
\begin{array}{c|*7{|c}}\hline
    x & \cdots & 0 & \cdots & 1 & \cdots & 2 & \cdots \\\hline
    y'&   +    & 0 &   -   & &   -    & 0 &   +       \\\hline
    y &  \nearrow & 極大 &  \searrow  &   &  \searrow & 極小 & \nearrow
        \\\hline
\end{array}
\]
\end{showEx}

\subsubsection{増減表(2) --- \textsf{hyou}環境}
前節の\textsf{array}環境による表は,列の横幅が均一ではありません。
さらに,不連続点には斜線を入れたい,ということで\textsf{emathT.sty}で
定義されている\textsf{hyou}環境を用いてみます。

\begin{showEx}(1,1){\textsf{hyou}環境}
\[
\begin{hyou}{c|*7{|C{2\zw}}}\hline
   x & \cdots & 0 & \cdots & 1 & \cdots & 2 & \cdots \\\hline
   y'&   +    & 0 &   -   & \sya[x] &   -    & 0 &   +    \\\hline
   y &  \NE & 極大 &  \SE & \sya[x] &  \SE & 極小 & \NE \\\hline
\end{hyou}
\]
\end{showEx}

\cmd{sya}が斜線を引くコマンドで,オプション引数で斜線の向きを指定します。
\begin{jquote}
\begin{verbatim}
\sya    のみの場合は,右上から左下へ
\sya[r] とすれば,逆に左上から右下へ
\sya[x] でクロスの斜線が引かれます。
\end{verbatim}
\end{jquote}

\subsubsection{増減表(3) --- 2行ぶち抜きで斜線}
不連続点$x=1$において,$y$と$y'$の行別々に斜線を引いていますが,
これを2行ぶち抜きで引いてみましょう。

\begin{showEx}(1,1){2行ぶち抜きの斜線}
\[
\begin{hyou}{c|*7{|C{2\zw}}}\hline
   x & \cdots & 0 & \cdots & 1 & \cdots & 2 & \cdots \\\hline
   y'&   +    & 0 &   -   
     & \smash{\sya(D=\ht\strutbox+2\dp\strutbox)[x]}
     &   -    & 0 &   +    \\\cline{1-4}\cline{6-8}
   y &  \NE & 極大 &  \SE &  &  \SE & 極小 & \NE \\\hline
\end{hyou}
\]
\end{showEx}

この場合,斜線の高さ・深さを\cmd{sya}に伝えるために
\begin{jquote}
\begin{verbatim}
\sya(D=\ht\strutbox+2\dp\strutbox)[x]
\end{verbatim}
\end{jquote}
としています。\cmd{strutbox}に1行分の高さ・深さがありますので,斜線の深さを
\begin{jquote}
\begin{verbatim}
\ht\strutbox+2\dp\strutbox
\end{verbatim}
\end{jquote}
と指定しているのです。
(斜線の高さはデフォルトのままでよいから,指定しません。)\\
さらに,\cmd{smash}で2行目の高さ・深さに影響を与えないようにしています。
%\clearpage

\subsubsection{増減表(4) --- 不連続点$y$の欄には左右極限を}
ぶち抜きではなく,$y$の欄には左右極限を入れておく,という表も便利でしょうか。
これは列の個数を増やしてやりましょう。

\begin{showEx}(1,1){左右極限}
\[
\begin{hyou}{c|*8{|C{2\zw}}}\hline
   x & \cdots & 0 & \cdots & \multicolumn{2}{c|}{1} & \cdots & 2 & \cdots 
      \\\hline
   y'&   +    & 0 &   -   
     & \multicolumn{2}{c|}{\sya(4\zw+\arrayrulewidth+2\arraycolsep)[x]}
     &   -    & 0 &   +    \\\hline
   y &  \NE & 極大 &  \SE & -\infty & +\infty &  \SE & 極小 & \NE \\\hline
\end{hyou}
\]
\end{showEx}
ここでは,斜線の横幅が今までと異なります。
2列ぶち抜きですから\verb+2*2\zw+すなわち\verb+4\zw+必要ですが,
それでは不十分です。列の間の罫線幅\verb+\arrayrulewidth+,
さらにその左右の余白分\verb+2\arraycolsep+も加えたものを斜線の横幅
と指定します。指定法は
\begin{jquote}
\begin{verbatim}
正しくは
\sya(W=4\zw+\arrayrulewidth+2\arraycolsep)
ですが,W=... 指定のみの場合はW=は省略可能としてあります。
\sya(4\zw+\arrayrulewidth+2\arraycolsep)
\end{verbatim}
\end{jquote}

%\clearpage

\subsubsection{増減表(5) --- 凹凸も}
増減表に凹凸も含めたい,となると$y''$の行を挿入することになります。
\begin{showEx}(1,1){凹凸も}
\[
\begin{hyou}{c|*7{|C{2\zw}}}\hline
   x & \cdots & 0 & \cdots & 1 & \cdots & 2 & \cdots \\\hline
   y'&   +    & 0 &   -   
     & \smash{\sya(D=2\ht\strutbox+3\dp\strutbox)[x]}
     &   -    & 0 &   +    \\\cline{1-4}\cline{6-8}
   y'' & \multicolumn{3}{c|}{-} & & \multicolumn{3}{c}{+}
     \\\cline{1-4}\cline{6-8}
   y &  \NEE & 極大 &  \SSE &  &  \SEE & 極小 & \NNE \\\hline
\end{hyou}
\]
\end{showEx}

上のリストに
\begin{jquote}
\begin{verbatim}
\def\arraystretch{1.25}
\end{verbatim}
\end{jquote}
を附加すると

\begin{shadebox}
\def\arraystretch{1.25}
\[
\begin{hyou}{c|*7{|C{2\zw}}}\hline
   x & \cdots & 0 & \cdots & 1 & \cdots & 2 & \cdots \\\hline
   y'&   +    & 0 &   -   
     & \smash{\sya(D=2\ht\strutbox+3\dp\strutbox)[x]}
     &   -    & 0 &   +    \\\cline{1-4}\cline{6-8}
   y'' & \multicolumn{3}{c|}{-} & & \multicolumn{3}{c}{+}
     \\\cline{1-4}\cline{6-8}
   y &  \NEE & 極大 &  \SSE &  &  \SEE & 極小 & \NNE \\\hline
\end{hyou}
\]
\end{shadebox}
行間が少しゆったりします。斜線も追随していることがおわかりいただけるでしょう。

\subsection{表セルの修飾}
表の1つのセルを修飾するコマンドが\cmd{EMcell}です。
\subsubsection{網掛け}
厳密には網掛けとはいえませんが,セルをグレーで塗りつぶします。

\begin{showEx}{\cmd{EMcell*}}
\begin{Hyou}{|*3{L{2\zw}|}}\hline
  甲&乙\\\hline
  \EMcell*{l}{あ}&い\\\hline
\end{Hyou}
\end{showEx}

\cmd{EMcell*}の引数\verb+{l}+は,
この欄が左詰であることを指定するものです。
\textsf{Hyou}環境で,既に指定されているのですが,
\cmd{EMcell*}でもl, c, rのいずれかを指定する必要があります。

グレーの濃度は\verb+[数値]+オプションで指定します。
数値は0と1の間の数値で\cmd{Nuritubusi}と同じです。

\begin{showEx}(.7,.24){\cmd{EMcell*}}
\makeatletter
\begin{Hyou}{|*3{R{2\zw}|}}\hline
  甲& 乙\\\hline
  \EMcell*[.1]{r}{あ}&い\\\hline
\end{Hyou}
\end{showEx}

\subsubsection{カラー塗り}
セルに色をつけるには\texttt{[C=..]}オプションを用います。

\begin{showEx}(.7,.24){\texttt{[C=..]}オプション}
\begin{Hyou}{|*3{C{2\zw}|}}\hline
  甲&乙\\\hline
  \EMcell*[C=yellow]{c}{あ}&い\\\hline
\end{Hyou}
\end{showEx}

(注)Windows において dviout.exe でご覧の場合,dviout の設定によっては,
文字「あ」が黄色で塗りつぶされて見えなくなってしまうかもしれません。

\subsubsection{斜線塗り}
セルを斜線塗りします。

\begin{showEx}(.7,.24){\cmd{EMcell**}}
\begin{Hyou}{|*3{C{2\zw}|}}\hline
  甲&乙\\\hline
  \EMcell**{c}{あ}&い\\\hline
\end{Hyou}
\end{showEx}

斜線の方向を変えるには\verb+[角度]+オプションを使用します。
角度は六十分法で指定します。

\begin{showEx}(.7,.24){斜線の角度}
\begin{Hyou}{|*3{C{2\zw}|}}\hline
  甲&乙\\\hline
  \EMcell**[-60]{c}{あ}&い\\\hline
\end{Hyou}
\end{showEx}

斜線の間隔は\verb+<間隔>+オプションで指定します。デフォルトは\verb+<3>+です。

\begin{showEx}(.7,.24){斜線の間隔}
\begin{Hyou}{|*3{C{2\zw}|}}\hline
  甲&乙\\\hline
  \EMcell**[-60]<5>{c}{あ}&い\\\hline
\end{Hyou}
\end{showEx}

\subsubsection{枠線}
罫線を引かない表で,特定のセルに枠線をつけるには,
\verb+[wakusen=..]+オプションをつけます。

\begin{showEx}(.7,.24){\texttt{[wakusen=..]}オプション}
\begin{Hyou}{*3{C{2\zw}}}
  甲&乙\\
  \EMcell[wakusen=\protect\hasen]{c}{あ}&い
\end{Hyou}
\end{showEx}

右辺値には\verb+\protect+を附加しておく方が無難です。
次の例は,枠線を実線としセル内を網掛けしています。

\begin{showEx}(.7,.24){\texttt{[wakusen=..]}オプション+網掛け}
\begin{Hyou}{*3{C{2\zw}}}
  甲&乙\\
  \EMcell*[wakusen=\protect\drawline]{c}{あ}&い
\end{Hyou}
\end{showEx}

この場合,グレーの濃度も指定するには,\verb+[G=数値]+オプションを用います。
\texttt{G}はgrayの頭文字のつもりです。

\begin{showEx}(.7,.24){\texttt{[wakusen=..]}オプション+網掛け濃度指定}
\begin{Hyou}{*3{C{2\zw}}}
  甲&乙\\
  \EMcell*[G=.8,wakusen=\protect\drawline]{c}{あ}&い
\end{Hyou}
\end{showEx}

枠線と斜線塗りも可能です。

\begin{showEx}(.7,.24){\texttt{[wakusen=..]}オプション+斜線塗り}
\begin{Hyou}{*3{C{2\zw}}}
  甲&乙\\
  \EMcell**[wakusen=\protect\drawline]{c}{あ}&い
\end{Hyou}
\end{showEx}

斜線の方向・間隔を変更するには,それぞれ
\verb+[syasenkaku=角度,syasenkankaku=間隔]+オプションを用います。

\begin{showEx}(.7,.24){\texttt{[wakusen=..]}オプション+斜線方向・角度指定}
\begin{Hyou}{*3{C{2\zw}}}
  甲&乙\\
  \EMcell**[wakusen=\protect\drawline,%
    syasenkaku=-45,syasenkankaku=2]{c}{あ}&い
\end{Hyou}
\end{showEx}

\subsubsection{横幅指定}
\hyouretuhaba=0pt%

\begin{showEx}(.6,.34){横幅指定が必要な例}
\begin{tabular}{|r|r|}\hline
  あいうえお&甲乙\\\hline
  \EMcell*{r}{1}&2\\\hline
\end{tabular}
\end{showEx}

前節までのように,\textsf{Hyou, hyou}環境で列幅を指定している場合は
よいのですが,上の例のように\textsf{tabular, array}環境で列幅が表に入るものに
合わせて決まるような場合には,\cmd{EMcell}はそれに追随していません。

このような場面で使用するためには,当該列の最長の文字列の長さを
\cmd{EMcell}の\verb+[W=長さ]+オプションで指定する必要があります。

\begin{showEx}(.6,.34){横幅指定オプション}
\settowidth\templa{あいうえお}
\begin{tabular}{|r|r|}\hline
  あいうえお&甲乙\\\hline
  \EMcell*[W=\templa]{r}{1}&2\\\hline
\end{tabular}
\end{showEx}

なお,当該セルがその列の中で最長の場合は\verb+[W=..]+オプションは不要です。

\begin{showEx}(.6,.34){最長のカラムの場合}
\begin{tabular}{|r|r|}\hline
  \EMcell*{r}{あいうえお}&甲乙\\\hline
  1&2\\\hline
\end{tabular}
\end{showEx}

\subsubsection{高さ指定}

\begin{showEx}(.6,.34){高さ指定が必要な例}
\[ 
  \begin{array}{|r|r|}\hline
    あいうえお&甲乙\\\hline
    \EMcell*{r}{1}&\bunsuu12\\\hline
  \end{array}
\]
\end{showEx}

上の場合は,第2行に分数が登場して行高が変化しています。
\cmd{EMcell*}はその変化に追随できません。
対策は\cmd{vphantom}を用いるのが簡単です。
幅指定オプションと合わせて

\begin{showEx}(.6,.34){高さ指定(1)}
\[ 
\settowidth\templa{あいうえお}
\begin{array}{|r|r|}\hline
  あいうえお&甲乙\\\hline
  \EMcell*[W=\templa]{r}%
    {1\vphantom{\bunsuu12}}&\bunsuu12
  \\ \hline
\end{array}
\]
\end{showEx}

なお,第2行に支柱を立てる場合は,当該セルで支柱を立てるのがよいでしょう。

\begin{showEx}(.6,.34){高さ指定(2)}
\[ 
\settowidth\templa{あいうえお}
\begin{array}{|r|r|}\hline
  あいうえお&甲乙\\\hline
  \EMcell*[W=\templa]{r}%
    {1\bsityuu}&\bunsuu12\\\hline
\end{array}
\]
\end{showEx}

\subsubsection{セルに斜線など}
\cmd{EMcell(*)}を使用したセルには,\textsf{picture}環境が定義されています。
\verb+<...>+オプションで,その\textsf{picture}環境内に配置するものを
記述可能です。次の例は,その\textsf{picture}環境の原点に$\bullet$を打ちます。

\begin{showEx}(.6,.34){\textsf{picture}環境の原点}
\[ 
  \begin{array}{|r|r|}\hline
    \EMcell<\Put\O(0,0)[c]{$\bullet$}>%
      {r}{あいうえお}&甲乙\\\hline
    1&2\\\hline
  \end{array}
\]
\end{showEx}

すなわち,原点の縦位置はベースライン上,横位置はセルの中央であり,
\verb+\unitlength=1pt+となっています。


次の例は,当該セルに×印をつけます。

\begin{showEx}(.6,.34){\textsf{picture}環境内の記述}
\[ 
  \begin{array}{|r|r|}\hline
    \EMcell<\Drawlines{\LT\RB;\LB\RT}>%
      {r}{あいうえお}&甲乙\\\hline
    1&2\\\hline
  \end{array}
\]
\end{showEx}
\clearpage

\subsection{\textsf{tabular}, \textsf{array}環境でもL, C, R}
\textsf{Hyou}, \textsf{hyou}環境で使用していた,欄指定子L, C, Rを
\textsf{tabular}, \textsf{array}環境でも使用可能です。
また,太罫線の使用も可能です。

まずは,\textsf{tabular}環境です。

\begin{showEx}(.5,.44){\textsf{tabular}環境でL, C, R}
\begin{tabular}%
    {IL{2\zw}|C{2\zw}|R{5\zw}I}
    \hlineb
  A & a & xyz \\ \hline
  B & b & \the\tabcolsep \\ \hlineb
\end{tabular}
\end{showEx}

次は\textsf{array}環境です。

\begin{showEx}(.5,.44){\textsf{array}環境でL, C, R}
$\begin{array}%
    {IL{2\zw}|C{2\zw}|R{5\zw}I}
    \hlineb
  A & a & xyz \\ \hline
  B & b & \the\arraycolsep
    \\ \hlineb
\end{array}$
\end{showEx}

同じ横幅を指定したのに,結果が異なる ? ですか。

それは\LaTeX の仕様で,罫線左右の空きが\textsf{tabular}, \textsf{array}では
別の変数で制御され,その初期値は,上のそれぞれの表の
右下欄のようになっているからです。
この件に関しては,\textsf{emath}は無関係です。

\subsection{\textsf{longtable}環境でもL, C, R}
\textsf{longtable}環境でも可能とする試みです。

\begin{longtable}{IL{2\zw}|C{2\zw}|R{5\zw}I}\hlineb
  A & a & 123 \\ \hline
  B & b & 234 \\ \hline
  C & c & 345 \\ \hline
  D & d & 456 \\ \hline
  E & e & 567 \\ \hline
  F & f & 678 \\ \hline
  G & g & 789 \\ \hline
  H & h & 890 \\ \hline
  I & i & 901 \\ \hline
  J & j & 012 \\ \hline
  K & k & 123 \\ \hline
  L & l & 234 \\ \hline
  M & m & 345 \\ \hline
  N & n & 456 \\ \hline
  O & o & 567 \\ \hline
  Z & z & \the\tabcolsep \\ \hlineb
\end{longtable}

この表は,ページをまたぎますから\textsf{shadebox}環境に入れることができません。
なお,ソースリストは次のとおりです。
\begin{jquote}
\begin{verbatim}
\begin{longtable}{IL{2\zw}|C{2\zw}|R{5\zw}I}\hlineb
  A & a & 123 \\ \hline
  B & b & 234 \\ \hline
  C & c & 345 \\ \hline
  D & d & 456 \\ \hline
  E & e & 567 \\ \hline
  F & f & 678 \\ \hline
  G & g & 789 \\ \hline
  H & h & 890 \\ \hline
  I & i & 901 \\ \hline
  J & j & 012 \\ \hline
  K & k & 123 \\ \hline
  L & l & 234 \\ \hline
  M & m & 345 \\ \hline
  N & n & 456 \\ \hline
  O & o & 567 \\ \hline
  Z & z & \the\tabcolsep \\ \hlineb
\end{longtable}
\end{verbatim}
\end{jquote}
%\clearpage

\subsection{\textsf{tabular}, \textsf{array}環境でも\cmd{sya}}
\textsf{tabular}環境でも,カラムに斜線を引く\cmd{sya}は有効ですが,
横幅を指定する必要があります。
また,斜線の左下\右上に見出しをつけるには,
\cmd{Hyoumidasi}を用います。

\begin{showEx}(.5,.44){\textsf{tabular}環境で\cmd{sya}}
\begin{tabular}%
    {IL{2\zw}|C{2\zw}|R{5\zw}I}
    \hlineb
  \sya(2\zw)[r]<\Hyoumidasi{X}{Y}>%
     & p & q \\ \hline
  A & a & xyz \\ \hline
  B & b & \the\tabcolsep \\ \hlineb
\end{tabular}
\end{showEx}

ついで,\textsf{array}環境です。
こちらでは,斜線の左下\右上に見出しをつけるには,
\cmd{hyoumidasi}を用います。

\begin{showEx}(.5,.44){\textsf{array}環境で\cmd{sya}}
$\begin{array}%
    {IL{2\zw}|C{2\zw}|R{5\zw}I}
    \hlineb
  \sya(2\zw)[r]<\hyoumidasi{X}{Y}>%
     & p & q \\ \hline
  A & a & xyz \\ \hline
  B & b & \the\tabcolsep \\ \hlineb
\end{array}$
\end{showEx}
