\section{正弦定理・余弦定理}
\subsection{正弦定理}
正弦定理を用いて三角形の辺・角,外接円の半径を求めるコマンドを
解説します.
一例として,
\begin{jquote}
$A=120\Deg$,~B=15\Deg,~c=10 
\end{jquote}
である三角形を BC を底辺として描画することを考えてみます.
そのために $\mathrm{BC}=a$ を正弦定理で求めます.

\resetcounter{equation}
まずは $C=45\Deg$ と $c=10$ から
\[ \bunsuu{c}{\sin C}=2R \]
を用いて外接円の半径を求めます.そのためのコマンドが
\cmd{seigenR}です.書式は\cindex{seigenR}\cindex{lR}\cindex{lRR}
\begin{boxnote}
\begin{verbatim}
\seigenR#1#2
    #1 : 向かい合った辺・角のうち辺
    #2 :             角
  外接円の直径が \lRR, 半径が \lR にセットされる.
\end{verbatim}
\end{boxnote}
ここでは,\verb/\seigenR{10}{45}/ とします.

次いで,
\[ \bunsuu{a}{\sin A}=2R \]
を用いて $a$ を求めます.そのためのコマンドが \cmd{seigen} です.
\cindex{seigen}
\begin{boxnote}
\begin{verbatim}
\seigen#1#2
   #1 : 角
   #2 : 角と向かい合った辺の長さを受取るコントロールシーケンス
\end{verbatim}
\end{boxnote}
ここでは \verb/\seigen{120}\la/ として \verb/\la/ に $a$ が
セットされます.

関連して,逆に辺を与えて角を求めるコマンドが \cmd{Seigen} です.
\cindex{Seigen}
\begin{boxnote}
\begin{verbatim}
\Seigen[#1]#2#3
  #2 : a
  #3 : A を受取るコントロールシーケンス
  #1 = e のときは,角A(鋭角)を求める.(デフォルト)
  #1 = d のときは,角A(鈍角)を求める.
\end{verbatim}
\end{boxnote}

では,この節のはじめに取り上げた三角形を実際に描画してみましょう.

\showexample[正弦定理](0.9)(0.6){example/seigen01}

\subsection{余弦定理}
2辺夾角がわかっているときに,第3辺を求めるコマンドが
\cmd{yogen} です.\cindex{yogen}
\begin{boxnote}
\begin{verbatim}
2辺夾角から第3辺の長さを求める.
\yogen#1#2#3#4
  #1=b, #2=c, #3=A → a を#4にセットする.
\end{verbatim}
\end{boxnote}
では,このコマンドを用いて
\[ b=2,~c=3,~A=60\Deg \]
である三角形を BC を底辺として描画してみます.

\showexample[余弦定理](0.9)(0.4){example/yogen01}

関連して,3辺の長さがわかっているとき,
角を求めるコマンドが \cmd{Yogen} です.\cindex{Yogen}
\begin{boxnote}
\begin{verbatim}
3辺の長さから角の余弦を求める.
\Yogen[#1]#2#3#4#5
   cos(A)=(b^2+c^2-a^2)/(2bc)
   #2=a, #3=b, #4=c 結果は #5 にセット
   #1=a のときは角を求める.(単位は度)
\end{verbatim}
\end{boxnote}
