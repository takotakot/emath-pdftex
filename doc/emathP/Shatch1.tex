\section{斜線塗り(1)}
前節のコマンドの末尾に更に \texttt{*} を付加すると,斜線塗りとなります.

\subsection{多角形}
まずは,多角形の内部に斜線を引いてみます.

\showexample[多角形の内部に斜線](0.5)(0.45){example/hatch01}

\cmd{Nuritubusi*} の書式は\cindex{Nuritubusi*}

\begin{boxnote}
\begin{verbatim}
\Nuritubusi*[#1]<#2>#3
  #1 : 斜線の方向角
      ( -90 と 90 の間の数(単位は度).デフォルトは 45)
  #2 : 斜線の間隔 (デフォルトは 0.125 )
  #3 : 多角形の点列(閉じていなければなりません.)
\end{verbatim}
\end{boxnote}\medskip

\subsection{円}
円の内部を斜線塗りする例です.
\verb+[-45]+ オプションをつけて,斜線の方向角を変更してみました.

\showexample[円の内部に斜線](0.45)(0.45){example/hatch02}

\cmd{En**} の書式です.\cindex{En**}

\begin{boxnote}
\begin{verbatim}
\En**[#1]<#2>#3#4
    #1 : 斜線の方向角
      ( -90 と 90 の間の数(単位は度).デフォルトは 45)
    #2 : 斜線の間隔 (デフォルトは 0.125 )
    #3 : 中心の座標
    #4 : 半径
\end{verbatim}
\end{boxnote}

\subsection{扇形など}
\cmd{Daen**}, \cmd{yumigata**}, \cmd{ougigata**}, \cmd{Daenko**},
\cmd{Nuritubusi*} も同様です.
\cindex{Daen**}\cindex{yumigata**}\cindex{ougigata**}
\cindex{Daenko**}\cindex{Nuritubusi*}

代表的に\verb/\yumigata**/ の書式を記します。

\begin{boxnote}
\begin{verbatim}
\yumigata**[#1]<#2>#3#4#5
  #1 : 斜線塗りの傾斜角(デフォルトは45)
  #2 : オプション
        <border=1> で境界線を描画する。
        <syanurisiteiten=xx> で斜線群が指定点xxを通るようにする。
        <syanurikankaku=xx> 斜線の間隔を指定する無名数(デフォルトは 0.125)
          #2 に単に無名数を記述すれば,斜線の間隔を変更する指示と解釈される。
  #3 〜 #5 : \yumigata の #1〜#3 と同じ

(注)始め角と終り角の与え方の順序について
    2つの角を指定すると,円は2つの弓形に分割されます。
    そのどちらが描画されるのかという話です。
    始め角で指定された動径を正の向きに(時計の針と反対向き)
    回転して終り角で指定した動径と重なるまでを描画します。
\end{verbatim}
\end{boxnote}

\showexample[弓形の斜線塗り](0.5)(0.45){example/hatch04}

\showexample[扇形の斜線塗り](0.5)(0.45){example/hatch03}

ただし,扇形の中心角は180度以下でなければなりません.
優角の扇形の斜線塗りは,円を斜線塗りした後,余分の扇形を白塗りします.

\showexample[扇形の斜線塗り(優角の場合)](0.5)(0.45){example/hatch05}

\showexample[弓形(楕円弧)の斜線塗り](0.5)(0.45){example/hatch06}

\subsection{格子セルのぬりつぶし}
格子を描画する \cmd{kousi} に
\begin{jquote}
\begin{verbatim}
セルを塗る
セルの座標を指定しての \put, 
黒丸配置
\end{verbatim}
\end{jquote}
などのを実現する\texttt{<nuri=..>}オプションの説明です。
\verb+<nuri=..>+の右辺に,塗りつぶしを行うセルの左下コーナーの
格子座標を与えます。

\begin{showEx}(.64,.3){塗りつぶし}
\begin{zahyou*}[ul=4mm](0,8)(0,8)
\kousi<nuri={(1,2)}>(2,1.5){4}{5}
\end{zahyou*}
\end{showEx}

\verb+<nuri=..>+の右辺は,コンマを含みますから,\verb+{...}+で
くくっておく必要があります。濃度指定は\verb+[..]+オプションというのは
\cmd{Nuritubusi}と同様です。

\begin{showEx}(.64,.3){濃度指定}
\begin{zahyou*}[ul=4mm](0,8)(0,8)
\kousi<nuri={[1](3,1)}>(2,1.5){4}{5}
\end{zahyou*}
\end{showEx}

\verb+<nuri=..(x,y)>+における\verb+..+の部分は\cmd{Nuritubusi}の
オプションとして引き渡されます。赤で塗りつぶしてみましょう。

\begin{showEx}(.64,.3){カラー塗り}
\begin{zahyou*}[ul=4mm](0,8)(0,8)
\kousi<nuri={[nuriiro=red](3,0)}>(2,1.5){4}{5}
\end{zahyou*}
\end{showEx}

\noindent
ということなれば,斜線塗りは

\begin{showEx}(.64,.3){斜線塗り}
\begin{zahyou*}[ul=4mm](0,8)(0,8)
\kousi<nuri={*(1,4)}>(2,1.5){4}{5}
\end{zahyou*}
\end{showEx}

斜線の角度,間隔指定も

\begin{showEx}(.64,.3){斜線角度,間隔}
\begin{zahyou*}[ul=4mm](0,8)(0,8)
\kousi<nuri={*[-45]<.5>(0,2)}>(2,1.5){4}{5}
\end{zahyou*}
\end{showEx}

\cmd{Nuritubusi}のオプションと同様であることが納得いただけたでしょうか。

複数のセルに対して塗りを指定するには,\verb+<nuri=..>+の右辺に,
上記の指定を`;'で区切って並べれば実現できます。

\begin{showEx}(.64,.3){複数指定}
\begin{zahyou*}[ul=4mm](0,8)(0,8)
\kousi<nuri={%
  (1,2);%
  [1](3,1);%
  [nuriiro=red](3,0);%
  *(1,4);%
  *[-45]<.5>(0,2)%
}>(2,1.5){4}{5}
\end{zahyou*}
\end{showEx}


\subsection{境界線と斜線の間を空ける}
領域の境界を含まないことを示すのに,
斜線と境界をくっつけず,少し空ける流儀があります。

これを実現するにはいくつかの方法がありますが,
田中 徹 さんが\verb+BBS #354+に投稿された方法
\begin{jquote}\relax
(1) ひとまず塗りつぶしを行う\\
(2) 境界線を太めの白い線で上書き\\
(3) 改めて境界線を引く
\end{jquote}
は有力な手法です。

そのアイデアを頂いてマクロ化してみました。
\textsf{emathPh.sty v 1.07}で
\begin{jquote}
\begin{verbatim}
syasentanmatu=...(単位付の長さ)
\end{verbatim}
\end{jquote}
オプションにより白塗りの幅を指定することにしてみました。
境界線を中心として,その左右に指定した幅を有する白い曲線を描画します。

(斜線の端点と境界線の距離が指定した長さとなり,
白塗りの曲線の太さは指定した幅の2倍となります。)

\begin{showEx}(.6,.34){\texttt{syasentanmatu=..}オプション}
\begin{zahyou*}[ul=6mm](0,5)(0,4)
  \tenretu*{A(0,0);B(5,0);C(4,4);D(1,3)}
  \Nuritubusi*<syasentanmatu=1mm>%
    {\A\B\C\D\A}
  \Drawline{\A\B\C\D\A}
\end{zahyou*}
\end{showEx}

\subsection{2円の共通部分に斜線}
2円の共通部分を斜線塗りするのは,結構面倒です。

まずは斜線塗りではなく,ベタ塗りの場合を検討します。
これは,共通弦で2分割したそれぞれの弓形を塗りつぶします。

\begin{showEx}(.6,.34){2円の共通部分のベタ塗り}
\unitlength12mm\drawaxisfalse
\begin{zahyou}(-1.5,1.5)(-1.5,2.5)
  \tenretu*{A(0,0);B(0,1)}
  \edef\hankeiA{1}
  \edef\hankeiB{1}
  \En\A\hankeiA
  \En\B\hankeiB
  \CandC\A\hankeiA\B\hankeiB\P\Q
  \Put\A{\yumigata*\hankeiA{%
    hazimeten=\Q}{owariten=\P}}
  \Put\B{\yumigata*\hankeiB{%
    hazimeten=\P}{owariten=\Q}}
\end{zahyou}
\end{showEx}

これが斜線塗りとなると

\begin{showEx}(.6,.34){2円の共通部分の斜線塗り(1)}
\unitlength12mm\drawaxisfalse
\begin{zahyou}(-1.5,1.5)(-1.5,2.5)
  \tenretu*{A(0,0);B(0,1)}
  \edef\hankeiA{1}
  \edef\hankeiB{1}
  \En\A\hankeiA
  \En\B\hankeiB
  \CandC\A\hankeiA\B\hankeiB\P\Q
  \Put\A{\yumigata**\hankeiA{%
    hazimeten=\Q}{owariten=\P}}
  \Put\B{\yumigata**\hankeiB{%
    hazimeten=\P}{owariten=\Q}}
\end{zahyou}
\end{showEx}
2つの弓形の斜線がつながりません。

そこで,斜線塗りに
\begin{verbatim}
  syanurisiteiten=
\end{verbatim}
という書式で,
斜線群が通過すべき1点を指定するためのオプションを付加しました。
その効果は

\begin{showEx}(.6,.34){2円の共通部分の斜線塗り(1)}
\unitlength12mm\drawaxisfalse
\begin{zahyou}(-1.5,1.5)(-1.5,2.5)
  \tenretu*{A(0,0);B(0,1)}
  \edef\hankeiA{1}
  \edef\hankeiB{1}
  \En\A\hankeiA
  \En\B\hankeiB
  \CandC\A\hankeiA\B\hankeiB\P\Q
  \Put\A{\yumigata**<%
    syanurisiteiten=\Q>\hankeiA{%
    hazimeten=\Q}{owariten=\P}}
  \Put\B{\yumigata**<%
    syanurisiteiten=\Q>\hankeiB{%
    hazimeten=\P}{owariten=\Q}}
\end{zahyou}
\end{showEx}

一見つながったように見えますが,よく目を凝らすとちょっとずれています。
これが気になれば,斜線塗りする領域の境界線を多角形近似して,
多角形を斜線塗りするコマンド \cmd{Nuritubusi} を使うことになりましょう。
円弧を折れ線近似するには,
\begin{verbatim}
  $n$次関数のグラフを折れ線近似する \KinziOresen
  一般の関数のグラフを折れ線近似する \yKinziOresen
  媒介変数表示の曲線を折れ線近似する \bKinziOresen
\end{verbatim}
などを用いれば可能ですが,この際,円弧を折れ線近似するコマンド
を作ってみました。名付けて \cmd{KinziEnko}
書式はあとで記述することにして,とりあえず使用効果を見てみましょう。

\begin{showEx}(.65,.29){2円の共通部分の斜線塗り(1)}
\unitlength12mm\drawaxisfalse
\begin{zahyou}(-1.1,1.1)(-1.5,2.5)
  \tenretu*{A(0,0);B(0,1)}
  \edef\hankeiA{1}\edef\hankeiB{1}
  \En\A\hankeiA\En\B\hankeiB
  \CandC\A\hankeiA\B\hankeiB\P\Q
  \Put\P[w]{P}\Put\Q[e]{Q}
  \KinziEnko\A\hankeiA{hazimeten=\Q}{%
    owariten=\P}\oresenA
  \KinziEnko\B\hankeiB{hazimeten=\P}{%
    owariten=\Q}\oresenB
  \edef\oresen{\Q\oresenA\P\oresenB\Q}
  \Nuritubusi*\oresen
\end{zahyou}
\end{showEx}
きれいに斜線塗りができたようです。

\cmd{KinziEnko} の書式です。

\begin{boxnote}
\begin{verbatim}
円弧を近似する折れ線
\KinziEnko<#1>#2#3#4#5#6
  #1 : 刻み値
  #2 : 中心
  #3 : 半径を直接与えるか
       tuukaten=xx として,円弧の周上の一点を与える
  #4 : 始め角を直接与えるか
       hazimeten=xx として,中心を始点,xx を終点とするベクトルの
       方向角を 始め角とするように指定する。
  #5 : 終り角を直接与えるか
       owariten=xx として,中心を始点,xx を終点とするベクトルの
       方向角を 終り角とするように指定する。
  #6 : 近似折れ線を受け取る制御綴
\end{verbatim}
\end{boxnote}

\subsection{破線による斜線塗り}\label{S-hasen}
斜線を破線で引くには
\begin{jquote}
\verb/\def\sensyu{\hasen}%/
\end{jquote}
などとします.

\showexample[破線による斜線塗り](0.5)(0.45){example/hatch08}

なお,斜線を \verb/\color{lightgray}/ などとするのも有力な手法です.
ただし,gray の濃度はプリンタ環境によって調整する必要があります.

\showexample[グレースケールによる斜線塗り](0.5)(0.45){example/hatch09}

\subsection{斜線塗りの制約条項}
%\begin{jquote}
\begin{description}
\item[\textgt{斜線塗りの制約条項}]
        斜線の方向角を$\theta$として,直線群 $y\cos\theta-x\sin\theta=k$ 
        が領域によってきり取られる線分は単一の連結な線分であること.

      駄目な場合の典型は,ドーナッツです.これはどのような方向で切っても
        中心を通る直線がドーナッツによって切り取られる部分は2個の線分に
        なってしまします.このような領域の斜線塗りはいっぺんにはできません.
        ドーナッツの場合は,大きい円の内部全部を一旦斜線塗りしたうえで,
        小さい円の内部を白塗りすることで実現できます.
\end{description}
%\end{jquote}
