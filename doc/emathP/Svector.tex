\section{ベクトル}
\subsection{ベクトル演算}
点の座標を位置ベクトルの成分と見て,ベクトル演算をするコマンドを
用意しました.

\begin{boxnote}
\begin{list}{}{\leftmargin7\zw\labelwidth7\zw}
\item[和:]\cmd{Addvec}\verb+#1#2#3+:
    2つのベクトル\verb+#1, #2+の和ベクトルを\verb+#3+に与えます.

\item[差:]\cmd{Subvec}\verb+#1#2#3+:
    2つのベクトル\verb+#1, #2+の差ベクトルを\verb+#3+に与えます.

\item[スカラー倍:]\cmd{Mulvec}\verb+#1#2#3+:
    ベクトル\verb+#2+ のスカラー\verb+#1+倍を\verb+#3+に与えます.

\item[大きさ:] \cmd{Absvec}\verb+#1#2+:
    ベクトル\verb+#1+ の大きさを \verb+#2+ に与えます.

\item[方向角:] \cmd{Argvec}\verb+#1#2+:
    ベクトルが x軸の正の向きとなす角を \verb+#2+ に与えます.

\item[法線ベクトル:]\cmd{Nvec}\verb+#1#2+:
    ベクトル \verb+#1+ に垂直な単位ベクトルを \verb+#2+ に与えます.

\item[回転:] \cmd{Rotvec}\verb+[#1]<#2>#3#4#5+:
    ベクトル \verb+#3+ を 角 \verb+#4+ だけ回転したベクトルを \verb+#5+に
    与えます.\verb+[#1]+ を指定した場合は,長さを指定した値にします.
    また,\verb+<#2>+ を指定した場合は,長さを元のベクトルの \verb+#2+ 倍
    にします.

\item[成分:] \cmd{vecXY}\verb+#1#2#3+:
    ベクトル \verb+#1+ のx成分を \verb+#2+ へ,y成分を \verb+#3+ へ抽出します.
\end{list}
\end{boxnote}
\cindex{Addvec}\cindex{Subvec}\cindex{Mulvec}\cindex{Absvec}
\cindex{Argvec}\cindex{Nvec}\cindex{Rotvec}\cindex{vecXY}

\subsection{平行四辺形}
A(2,3), B(1,1), C(4,1) を頂点とする平行四辺形 ABCD を作図するには
3点の座標 \cmd{A}, \cmd{B}, \cmd{C} を位置ベクトルと見て
\begin{quote}
    $\mbox{\cmd{A}}+\mbox{\cmd{C}}-\mbox{\cmd{B}}$
\end{quote}
として \cmd{D} を求めます.

\showexample[平行四辺形](0.6)(0.55){example/sikakkei}

\subsection{回転}
ベクトルを回転させるコマンド \cmd{Rotvec} を用いて
回転を行うことができます.ここではさらに一般化した
\cmd{Kaiten} を用いて,
指定した2点 A, B を結ぶ線分を一辺とする正三角形を作図します.
\cindex{Kaiten}

\showexample[回転](0.55)(0.35){example/kaiten01}

\cmd{Kaiten} コマンドの書式です.
\begin{boxnote}
\begin{verbatim}
   \Kaiten[#1]<#2>#3#4#5#6
      #1 : 長さ指定
      #2 : 長さの倍率指定
      #3 : 回転の中心
      #4 : 回転させる点
      #5 : 回転角
      #6 : 結果の座標を受け取るコントロールシーケンス
\end{verbatim}
\end{boxnote}
