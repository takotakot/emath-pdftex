\section{三角形の五心}
\subsection{重心}

\showexample[重心](0.55)(0.4){example/zyusin01}
\cindex{Zyuusin}

\begin{boxnote}
\begin{verbatim}
\Zyuusin#1#2#3#4
    #1,#2,#3 を頂点とする三角形の重心を #4 にセットします.
\end{verbatim}
\end{boxnote}

\subsection{外心}
三角形の外心を求めるコマンド \cmd{Gaisin} です.

\showexample[外心](0.55)(0.4){example/gaisin01}
\cindex{Gaisin}

\begin{boxnote}
\begin{verbatim}
\Gaisin#1#2#3#4
    #1,#2,#3 を頂点とする三角形の外心を #4 にセットします.
\end{verbatim}
\end{boxnote}

三角形の外接円を描くコマンド \cmd{Gaisetuen} もあります.
\cindex{Gaisetuen}

\showexample[外接円](0.55)(0.4){example/gaisin02}

\begin{boxnote}
\begin{verbatim}
\Gaisetuen#1#2#3
    #1,#2,#3 を頂点とする三角形の外接円を描画します.
        外心は \vGaisin にセット,半径は \lR
\end{verbatim}
\end{boxnote}
\cindex{vGaisin}\cindex{lR}

\subsection{内心}
三角形の内心を求めるコマンド \cmd{Naisin} です.
\cindex{Naisin}

\showexample[内心](0.55)(0.4){example/naisin01}

\begin{boxnote}
\begin{verbatim}
\Naisin#1#2#3#4
    #1,#2,#3 を頂点とする三角形の内心を #4 にセットします.
\end{verbatim}
\end{boxnote}


三角形の内接円を描くコマンド \cmd{Naisetuen} もあります.
\cindex{Naisetuen}

\showexample[内接円](0.55)(0.4){example/naisin02}

\begin{boxnote}
\begin{verbatim}
\Naisetuen#1#2#3
    #1,#2,#3 を頂点とする三角形の内接円を描画します.
        内心は \vNaisin にセット,半径は \lr
\end{verbatim}
\end{boxnote}
\cindex{vNaisin}\cindex{lr}

\subsection{傍心}
三角形の傍心を求めるコマンド \cmd{Bousin} です.
\cindex{Bousin}

\showexample[傍心](0.55)(0.4){example/bousin01}

\begin{boxnote}
\begin{verbatim}
\Bousin#1#2#3#4
    #1,#2,#3 を頂点とする三角形の,
      ∠A内にある傍心を #4 にセットします.
\end{verbatim}
\end{boxnote}


三角形の傍接円を描くコマンド \cmd{Bousetuen} もあります.
\cindex{Bousetuen}

\showexample[傍接円](0.55)(0.4){example/bousin02}

\begin{boxnote}
\begin{verbatim}
\Bousetuen#1#2#3
    #1,#2,#3 を頂点とする三角形の傍接円を描画します.
        傍心は \vBousin にセット,半径は \BousetuenHankei
\end{verbatim}
\end{boxnote}
\cindex{vBousin}\cindex{BousetuenHankei}

三角形の内接円,3つの傍接円を描画する例です。

\showexample[内接円と傍接円](0.9)(0.9){example/bousin03}

\subsection{垂心}
三角形の垂心を求めるコマンドは用意してありません.
\cmd{Suisen} で垂線を引き,\cmd{LandL} で交点を求めることで
垂心が得られます.

\showexample[垂心](0.55)(0.4){example/suisin01}

\subsection{角の二等分線}
三角形の二等分線と対辺の交点を求めるには,
コマンド\cmd{Nitoubunsen}を用います。
その書式です。

\begin{boxnote}
\begin{verbatim}
三角形の二等分線と対辺の交点を求める。
\Nitoubunsen[#1]#2#3#4#5
  角#2#3#4(#3が角の頂点)を2等分する直線が辺#2#4と交わる点を#5に与える。
  #1を与えたときは,外角の2等分線を#1に与える。
\end{verbatim}
\end{boxnote}
%\clearpage

まずは,内角の二等分線です。

\begin{showEx}(1,1){\cmd{Nitoubunsen}}
\begin{zahyou*}[ul=10mm](-1,5)(-1,3)\footnotesize
  \tenretu{A(3,2)n;B(0,0)s;C(4,0)s}
  \Nitoubunsen\B\A\C\D
  \Put\D[s]{D}
  \Drawline{\A\B\C\A\D}
  \Kakukigou<0>\B\A\D(0,0)[c]{$\bullet$}
  \Kakukigou<0>\D\A\C(0,0)[c]{$\bullet$}
\end{zahyou*}
\end{showEx}
%\clearpage

外角の二等分線と対辺(の延長)との交点を得るには,
オプション引数\verb+[#1]+に交点を受け取る制御綴を与えます。

\begin{showEx}(1,1){外角の二等分線}
\begin{zahyou*}[ul=10mm](-1,12)(-1,4)\footnotesize
  \tenretu{A(3,2)n;B(0,0)s;C(4,0)s}
  \Nitoubunsen[\E]\B\A\C\D
  \Put\D[s]{D}
  \Put\E[s]{E}
  \Hantyokusen\B\C
  \Hantyokusen\B\A
  \Drawlines{\A\B\C\A\D;\A\E}
  \Kakukigou<0>\B\A\D(0,0)[c]{$\bullet$}
  \Kakukigou<0>\D\A\C(0,0)[c]{$\bullet$}
  \Bunten\A\B{-1}{2}\T
  \Kakukigou<0>\C\A\E(0,0)[c]{$\times$}
  \Kakukigou<0>\E\A\T(0,0)[c]{$\times$}
\end{zahyou*}
\end{showEx}
