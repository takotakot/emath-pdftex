\section{関数のグラフ(2) 一般}
この節のマクロは,\textsf{emathPg.sty}で定義されています。

\subsection{関数計算式の記述}
その他の関数のグラフを描画するには,関数の計算式を与える必要があります.
例えば,$y=\bunsuu{1}{x}$ の計算式は次のように記述します.
\begin{screen}
\begin{verbatim}
\def\Fx#1#2{\Div{1}{#1}\y\edef#2{\y}}
\end{verbatim}
\end{screen}

すなわち,関数定義は必ず2つの引数をとります.
\begin{jquote}
\begin{verbatim}
#1 は独立変数の値
#2 は計算結果を受け取るコントロールシーケンス
\end{verbatim}
\end{jquote}
です.この定義の元で
\begin{jquote}
\begin{verbatim}
\Fx{4}\kekka
\end{verbatim}
\end{jquote}
と記述すると,\cmd{kekka}に $\bunsuu14=0.25$ がセットされます.

また,四則演算は \textsf{eclarith.sty} で定義されている
\begin{boxnote}
\begin{verbatim}
加法が \Add#1#2#3 
  で #1 + #2 が #3 のコントロールシーケンスに与えられ,
減法は \Sub#1#2#3
乗法は \Mul#1#2#3
除法は \Div#1#2#3
\end{verbatim}
\end{boxnote}
\cindex{Add}\cindex{Sub}\cindex{Mul}\cindex{Div}
\noindent
を用います.

始めの関数定義は
\begin{screen}
\begin{verbatim}
\def\Fx#1#2{\Div{1}{#1}#2}
\end{verbatim}
\end{screen}
と簡略化することもできます。

組込み関数については
\begin{boxnote}
\begin{verbatim}
正弦が \Sin#1#2 で sin(#1) が #2 にセットされ,
余弦が \Cos#1#2
\end{verbatim}
\end{boxnote}
\cindex{Sin}\cindex{Cos}
\noindent
は,\textsf{eclarith.sty} で定義されています.
平方根は \cmd{Sqroot} が定義されていますが,\cindex{Sqroot}
\cmd{Sqroot}\verb+{0}+ が 1 となるバグ(仕様かも)がありますので
\cmd{Heihoukon} をかぶせます.指数・対数もほしいので\cindex{Heihoukon}
\begin{boxnote}
\begin{verbatim}
平方根が   \Heihoukon#1#2
指数関数が \Exp#1#2
対数関数が \Log#1#2

三角関数の追加 \Tan#1#2
               \Sec#1#2
\end{verbatim}
\end{boxnote}
\cindex{Exp}\cindex{Heihoukon}\cindex{Tan}\cindex{Sec}
\noindent
を \textsf{emathP.sty} で追加しています.ただし,
グラフが不自然に見えない程度の精度です.

また,円周率 $\pi$, 自然対数の底 $e$ などの定数は次の変数名で利用できます.

\cindex{Pie}\cindex{Pii}\cindex{Pih}\cindex{Piq}
\begin{boxnote}
\begin{verbatim}
% \Pi is a greek letter.
\def\Piq{0.78539816}%     π/4
\def\Pih{1.57079633}%     π/2
\def\Pie{3.14159265}%     π
\def\Pii{6.28318531}%     2π
\end{verbatim}
\end{boxnote}
\noindent
が \textsf{eclarith.sty} で定義されています.
\textsf{emathP.sty}では次のものを追加しています。
\cindex{Pis}\cindex{Pit}
\begin{boxnote}
\begin{verbatim}
\def\Pis{0.52359878}%     π/6
\def\Pit{1.04719755}%     π/3
\def\Pihiii{4.71238898}%  3π/2
\def\Pitii{2.094395}%     2π/3
\def\Pitiv{4.188790}%     4π/3
\def\Pitv{5.235988}%      5π/3
\def\Piqiii{2.356194}%    3π/4
\def\Piqv{3.926991}%      5π/4
\def\Piqvii{5.497787}%    7π/4
\def\Pisv{2.617994}%      5π/6
\def\Pisvii{3.665191}%    7π/6
\def\Pisxi{5.759587}%    11π/6
\end{verbatim}
\end{boxnote}
\noindent
さらには,75\Deg をラジアンに変換するには \verb/\DegRad{75}\kaku/ で 
\cmd{kaku} に変換値が求められます。

自然対数の底の方は \textsf{emathP.sty} で次のように定義されています.

\cindex{Napier}\cindex{Napierii}\cindex{Napiermi}
\begin{boxnote}
\begin{verbatim}
\def\Napier{2.718281828}% 自然対数の底 e
\def\Napierii{7.389056}%  e^2
\def\Napiermi{0.367879441}% 1/e
\end{verbatim}
\end{boxnote}

\subsection{\texorpdfstring{$y=f(x)$}{y=f(x)}のグラフ}
\subsubsection{分数関数}
前節で $y=\bunsuu{1}{x}$ の関数定義を紹介しましたが,
それを用いてこの関数のグラフを描画します.

\showexample[$y=\bunsuu{1}{x}$のグラフ](0.51)(0.44){example/func41}
\cindex{yGurafu}

すなわち一般の関数グラフを描画するコマンド \cmd{yGurafu} の書式は
\begin{boxnote}
\begin{verbatim}
\yGurafu(#1)(#2)#3#4#5
    #1 : xの刻み値 デフォルト=0.05
    #2 : 点線で描画するときの描画する部分の x のレンジ
    #3 : 関数式
    #4 : xの始め値
    #5 : xの終り値
\end{verbatim}
\end{boxnote}

\subsubsection{無理関数}
\showexample[$y=\sqrt{x}$のグラフ](0.51)(0.44){example/func42}

\subsubsection{三角関数}
\showexample[$y=\sin 2x$のグラフ](0.7)(0.7){example/func43}

\subsubsection{指数関数}
\showexample[$y=e^{-x^2}$のグラフ](0.9)(0.6){example/func44}

\subsubsection{対数関数}
\showexample[$y=\bunsuu{\log x}{x}$のグラフ](0.9)(0.6){example/func45}

\subsection{\texorpdfstring{$x=g(y)$}{x=g(y)} で与えられた曲線}
$x=g(y)$ の形で与えられた曲線を描画するには \cmd{xGurafu} を用います.
\cindex{xGurafu}

\showexample[$x=\sin y$ のグラフ](0.55)(0.35){example/func50}


\subsection{媒介変数表示}
$x=f(t)$, $y=g(t)$ と,2つの関数を与える必要があります.すなわち
\cindex{bGurafu}

\begin{jquote}
\begin{verbatim}
\bGurafu(#1)(#2)#3#4#5#6
   #1 : t の刻み値(デフォルト値は 0.05 )
   #2 : 点線で描画するときの描画する部分の t のレンジ
   #3 : x=f(t)
   #4 : y=g(t)
   #5 : t の始め値
   #6 :     終り値
\end{verbatim}
\end{jquote}
という書式です.では,サイクロイドを描いてみます.

\showexample[サイクロイド](1)(0.8){example/func46}

二次曲線は媒介変数表示を利用するのが良いでしょう.

\showexample[双曲線](1)(0.75){example/func48}

\subsection{極方程式}
$r=f(\theta)$ を与えます.
\cindex{rGurafu}

\showexample[レムニスケート](0.97)(0.75){example/func47}

\begin{boxnote}
\begin{verbatim}
\rGurafu(#1)(#2)#3#4#5
    #1 : θの刻み値(デフォルトは 0.05)
    #2 : 点線で描画するときの描画する部分の θ のレンジ
    #3 : 関数 r=f(θ)
    #4 : θの初期値
    #5 : θの終値
\end{verbatim}
\end{boxnote}

\subsection{グラフの点線描画}
グラフの一部を点線で描画したいときは,

\centerline{\cmd{yGurafu}, \cmd{bGurafu}, \cmd{rGurafu}}
の第2の \texttt{(...)} オプションを利用します.

\showexample[グラフの点線描画](1)(0.85){example/func49}


\subsection{スプライン曲線}
\bgroup
この節は,dvi-ware 依存です。
\texttt{Windows}上の\texttt{dviout.exe}を想定しています。
\textsf{tpic}には,\textsf{spline}曲線を描画するコマンド
\begin{jquote}
\begin{verbatim}
sp
\end{verbatim}
\end{jquote}
があります。

\special{Bz 1}%
\begin{showEx}(.54,.4){\cmd{Drawtpic}}
\begin{zahyou}[ul=8mm](-1,5)(-1,4.5)
  \zahyouMemori[g]
  \tenretu{A(1,3)nw;B(2,4)n;%
      C(3,3)ne;D(4,0)ne}
  \Drawtpic{\O\A\B\C\D}
  \kuromaru{\O;\A;\B;\C;\D}
\end{zahyou}
\end{showEx}

新設コマンド\cmd{Drawtpic}の引数には,点列を与えます。
スプライン曲線といっていますが,ご覧頂けばわかるように,
両端は指定された点になりますが,途中点A, B, Cは
制御点として使われ,通過はしません。
その意味ではベジェ曲線というべきでしょう。

\textsf{tpic}の拡張機能には,途中点A, B, Cも通過するスプライン曲線とする
機能がありますが,これをサポートしている dvi ウェアは少ないようです。

dviout はこの拡張機能をサポートしていますから,
Windows で dviout を用いた場合は,次のリストが有効です。

\begin{showEx}(.54,.4){\cmd{Drawtpic}}
\begin{zahyou}[ul=8mm](-1,5)(-1,4.5)
  \zahyouMemori[g]
  \tenretu{A(1,3)nw;B(2,4)n;%
      C(3,3)ne;D(4,0)ne}
  \Drawtpic[tpicBz=0]{\O\A\B\C\D}
  \kuromaru{\O;\A;\B;\C;\D}
\end{zahyou}
\end{showEx}

残念ながら,dvips(k), dvipdfm(x) は,この拡張機能をサポートしていませんから,
pdf を作ることはできません。
\egroup
