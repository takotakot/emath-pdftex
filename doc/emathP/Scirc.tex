\section{円・楕円}
\subsection{円}
\subsubsection{円}
\paragraph{中心と半径を指定}
円を描画するには,\LaTeX で \cmd{circle} がありますが,
直径を指定することになっています.半径の方が使いやすいので,
\cmd{En} を用意しました.次のように使います.\cindex{En}

\showexample[\cmd{En}](0.55)(0.35){example/en01}
\bigskip

すなわち,\cmd{En} は,中心と半径の2つの引数をとります.

\paragraph{半径の間接指定}
半径を指定する代わりに,円周上の1点を指定して描画する方法もあります。

\showexample[円周上の1点を指定](0.55)(0.35){example/en22}
\bigskip

\paragraph{直径を指定}
2点を指定して,それを結ぶ線分を直径とする円を描画するコマンド\cmd{EnT}
もあります。

\showexample[直径を指定](0.55)(0.35){example/en23}
\bigskip

\subsubsection{円の破線描画}
円を破線で描画するには,

\showexample[円周を破線で](0.55)(0.35){example/en15}

と\verb/<hasen=[破線の長さ][破線の間隔]>/オプションを与えます。

\begin{boxnote}
\begin{verbatim}
\En<#1>#2#3
  <#1>: hasen=xx で円弧を破線で描画するときの
                  [破線の長さ][破線の間隔]
   #2 : 中心の座標
   #3 : 半径
\end{verbatim}
\end{boxnote}


\subsubsection{円弧}
円弧を描画するために,\cmd{Enko} を用意しました.\cindex{Enko}

\showexample[\cmd{Enko}](0.55)(0.4){example/en04}

すなわち,\cmd{Enko} は,
\begin{jquote}
\begin{verbatim}
\Enko{中心}{半径}{開始角}{終了角}
\end{verbatim}
\end{jquote}
中心,半径,開始角,終了角と4つの引数を取ります.
\bigskip

円弧を破線で描画するには,\verb/<#1>/ に
\begin{verbatim}
   <hasen=[破線の長さ][破線の間隔]>
\end{verbatim}
オプションを与えます。

\showexample[円弧を破線で](0.55)(0.4){example/en16}

点線で描画するには,\verb/<#1>/に
\begin{verbatim}
   <ten=周上に置く点の個数>
\end{verbatim}
オプションを与えます。

\showexample[円弧を点線で](0.55)(0.4){example/en21}

\cmd{Enko}の書式です。

\begin{boxnote}
\begin{verbatim}
% 円弧
\Enko<#1>#2#3#4#5
  #1 : オプション
         hasen=xx で円弧を破線で描画するときの
                  [破線の長さ][破線の間隔]
                オプションを与える。
         ten=xx 円周を点線で描画するときの周上の点の個数
         yazirusi=a : 正方向に矢印をつける
                 =r : 負方向に矢印をつける
                 =b : 両方向に矢印をつける
                 =n : 矢印をつけない
  #2 : 中心
  #3 : 半径を直接与えるか
         tuukaten=xx として,円弧の周上の一点を与える
  #4 : 始め角を直接与えるか
        hazimeten=xx として,中心を始点,xx を終点とするベクトルの
         方向角を 始め角とするように指定する。
  #5 : 終り角を直接与えるか
         owariten=xx として,中心を始点,xx を終点とするベクトルの
         方向角を 終り角とするように指定する。
\end{verbatim}
\end{boxnote}

始め角,終り角を六十分法の角度で与える代わりに,
点を指定して,中心から指定した点に向かうベクトルの
方向角で与えるオプションの利用例です。

\showexample[両端の間接的指定](0.55)(0.4){example/en17}

半径も円周上の一点を指定することにより,間接的に与えることもできます。

\showexample[半径の間接的指定](0.55)(0.4){example/en18}

\subsubsection{矢印付きの円弧(1) 偏角指定}
円弧に矢印をつけるには,\cmd{Enko}に\verb+<yazirusi=a>+オプションを
つけます。

\showexample[\texttt{<yazirusi=a>}](0.55)(0.4){example/en10}

矢印を逆向きにしたいときは,オプションの右辺値を`r'とします。

\showexample[\texttt{<yazirusi=r>}](0.55)(0.4){example/en11}

また,オプションの右辺値を`b'とすれば,両向きの矢印がつきます。

\showexample[\texttt{<yazirusi=b>}](0.55)(0.4){example/en19}

さらには,\verb+<hasen=..>+オプションと併用も可能です。

\showexample[\texttt{<yazirusi=b>}](0.55)(0.4){example/en20}

\subsubsection{矢印付きの円弧(2) 端点指定}
矢印付きの円弧を,半径と端点を指定して描画するのが
\cmd{ArrowArc} コマンドです.\cindex{ArrowArc}

\showexample[\cmd{ArrowArc}](0.55)(0.4){example/en12}

%\verb+[r]+ オプションをつけると,
%回転が負の方向となるように描画されます.

半径に負の値を与えると,
回転が負の方向となるように描画されます.

\showexample[回転の向き](0.55)(0.4){example/en13}


\begin{boxnote}
\begin{verbatim}
\ArrowArc[#1]#2#3#4
      #1 : 
      #2 : 半径(負のときは回転が負となる)
      #3 : 始点
      #4 : 終点
\end{verbatim}
\end{boxnote}

このコマンドを造った理由の一つは,
図が込み入って来たりして,文字列を該当個所に置けないとき,
ゆったりしたところに文字列を置き,そこから矢線を引こう,
というねらいがあります.少し曲げた方が見やすいのではないか,
と思います.

\showexample[\cmd{ArrowArc}の一応用](0.55)(0.4){example/en14}

\subsubsection{等弧記号}
二つの円弧の長さが等しいことを表すのに,
円弧の中央部分に短い縦線を入れるコマンドが
\cmd{Toukokigou}です。

\begin{showEx}(.6,.345){\cmd{Toukokigou}}
\begin{zahyou}[ul=15mm](-0.2,2.5)(-1.2,1.5)
\small
\def\O{(1,0)}
\rtenretu[\O]{A(1,0)ne;B(1,60)ne;
C(1,90)nw;D(1,150)nw}
\Enko\O{1}{0}{60}
\Enko\O{1}{90}{150}
\Toukokigou<2>\O\A\B
\Toukokigou<2>\O\C\D
\Drawlines{\A\O\B;\C\O\D}
\end{zahyou}
\end{showEx}

書式は
\begin{boxnote}
\begin{verbatim}
\Toukokigou<#1>#2#3#4
  #1 : 中央に配置する短い縦棒の個数(デフォルト値=1)
  #2 : 円弧の中心
  #3 : 弧の端点1
  #4 : 弧の端点2
\end{verbatim}
\end{boxnote}
\cindex{Toukokigou}

\subsubsection{扇形}
扇形を描画するために,\cmd{ougigata} を用意しました.\cindex{ougigata}

\showexample[\cmd{ougigata}](0.55)(0.4){example/en02}

すなわち,\cmd{ougigata} は,半径,開始角,終了角と3つの引数を取ります.

\begin{boxnote}
\begin{verbatim}
\ougigata#1#2#3
  #1 : 半径を直接与えるか
       tuukaten=xx として,円弧の周上の一点を与える
  #2 : 始め角を直接与えるか
       hazimeten=xx として,中心を始点,xx を終点とするベクトルの
       方向角を 始め角とするように指定する。
  #3 : 終り角を直接与えるか
       owariten=xx として,中心を始点,xx を終点とするベクトルの
       方向角を 終り角とするように指定する。
 (中心は \put (\emathPut) で指定する.)
\end{verbatim}
\end{boxnote}

\subsubsection{弓形}
弓形を描画するために,\cmd{yumigata} を用意しました.\cindex{yumigata}

\showexample[\cmd{yumigata}](0.55)(0.4){example/en03}

すなわち,\cmd{yumigata} は,半径,開始角,終了角と3つの引数を取ります.

\begin{boxnote}
\begin{verbatim}
\yumigata#1#2#3
  #1 : 半径を直接与えるか
       tuukaten=xx として,円弧の周上の一点を与える
  #2 : 始め角を直接与えるか
       hazimeten=xx として,中心を始点,xx を終点とするベクトルの
       方向角を 始め角とするように指定する。
  #3 : 終り角を直接与えるか
       owariten=xx として,中心を始点,xx を終点とするベクトルの
       方向角を 終り角とするように指定する。
 (中心は \put (\emathPut) で指定する.)
\end{verbatim}
\end{boxnote}

\subsection{楕円}
\subsubsection{楕円}
軸が水平,垂直な楕円を描画するコマンドは \cmd{Daen} です.\cindex{Daen}

\showexample[\cmd{Daen}](0.45)(0.45){example/en05}

中心,横方向半径,縦方向半径 と三つの引数をとります.

\begin{boxnote}
\begin{verbatim}
\Daen#2#3#4
        #1 : 中心の座標
        #2 : 横軸方向の半径
        #3 : 縦軸方向の半径
\end{verbatim}
\end{boxnote}

\subsubsection{楕円弧}
楕円の一部を描画する \cmd{Daenko} です.\cindex{Daenko}

\showexample[\cmd{Daenko}](0.45)(0.45){example/en06}

\begin{boxnote}
\begin{verbatim}
\Daenko<#1>#2#3#4#5
        #1 : key=val
             hasen=[破線の長さ][破線の間隔]
             yazirusi=a : 正方向に矢印をつける
                     =r : 負方向に矢印をつける
                     =b : 両方向に矢印をつける
                     =n : 矢印をつけない
        #2 : 横軸方向の半径
        #3 : 縦軸方向の半径
        #4 : 始め角
        #5 : 終り角
             (中心は \put (\emathPut) で指定する.)
\end{verbatim}
\end{boxnote}

\subsubsection{破線}
(楕)円(弧)を破線で描画するには \cmd{Daenko} に
\verb+<hasen=..>+オプションを与えます.

\showexample[楕円弧の破線](0.5)(0.45){example/en07}

破線の長さを変更するには,オプションの値を変更します.
上の図を標準として比率を指定します.

\showexample[破線の長さ](0.5)(0.45){example/en08}

破線の間隔を調整するには,第2の \texttt{[..]}オプションです.
やはり,標準に対する比率です.

\showexample[破線の間隔](0.5)(0.45){example/en09}

なお,横軸方向の半径と縦軸方向の半径を同じ値にすれば,円(弧)を
破線で描画することもできます.

\subsubsection{矢印}
(楕)円(弧)に矢印をつけるには \cmd{Daenko} に
\verb+<yazirusi=..>+オプションを与えます.

\begin{showEx}{楕円弧に矢印}
\begin{picture}(4,2)%
\Put{(2,1)}{%
\Daenko<yazirusi=b>%
  {2}{1}{0}{90}}%
\end{picture}
\end{showEx}

\subsubsection{回転記号}
楕円弧に矢印を付ける機能を利用して,
回転軸の周りに回転を表す記号を付けることができます。

\begin{showEx}(.5,.44){\cmd{kaitenkigou}}
\begin{zahyou}[ul=10mm](-2,2)(-1,1)
\Put{(1.5,0)}{\kaitenkigou}
\end{zahyou}
\end{showEx}

$y$軸の周りの回転を表すには,\verb+[90]+オプションを付けます。

\begin{showEx}(.5,.44){$y$軸の周りの回転}
\begin{zahyou}[ul=10mm](-2,2)(-2,2)
\Put{(0,1.5)}{\kaitenkigou[90]}
\end{zahyou}
\end{showEx}

ここで,矢印は正の回転を表すようにつきますが,これを負の回転を表すように
つけるためには
\begin{jquote}
\begin{verbatim}
<muki=r>
\end{verbatim}
\end{jquote}
オプションを用います。

\begin{showEx}(.54,.4){矢印を逆向き}
\begin{zahyou}[ul=10mm](-1,3.5)(-.5,.5)
  \Put{(3,0)}{\kaitenkigou<muki=r>}
\end{zahyou}
\end{showEx}

\verb+<..>+オプションは,本来は倍率指定オプションです。

\begin{showEx}(.54,.4){倍率指定}
\begin{zahyou}[ul=10mm](-1,3.5)(-.5,.5)
  \Put{(3,0)}{\kaitenkigou<1.5>}
\end{zahyou}
\end{showEx}

倍率指定と矢印向きなど他のオプションと併用したいときは
\verb+<bairitu=..>+オプションを用います。

\begin{showEx}(.54,.4){倍率指定}
\begin{zahyou}[ul=10mm](-1,3.5)(-.5,.5)
  \Put{(3,0)}{%
    \kaitenkigou<bairitu=1.5,muki=r>}
\end{zahyou}
\end{showEx}

楕円のサイズは,デフォルトでは
\begin{jquote}
\begin{verbatim}
tyouhankei=3mm
tanhankei=1.5mm
\end{verbatim}
\end{jquote}
としてあります。これを変更するオプションです:

\begin{showEx}(.54,.4){サイズ変更}
\begin{zahyou}[ul=10mm](-1,3.5)(-.5,.5)
  \Put{(3,0)}{%
    \kaitenkigou<tanhankei=3mm>}
\end{zahyou}
\end{showEx}

楕円の右端を一部切って矢印をつけていますが,
切る場所を変えるオプションを紹介します。

\begin{showEx}(.54,.4){矢印の位置変更}
\begin{zahyou}[ul=10mm](-1,3.5)(-.5,.5)
  \Put{(3,0)}{\kaitenkigou%
    <hazimekaku=-165,owarikaku=165>}
\end{zahyou}
\end{showEx}

デフォルト値は
\begin{jquote}
\begin{verbatim}
hazimekaku=15, owarikaku=345
\end{verbatim}
\end{jquote}
となっています。

回転オプションと併用するときは,上記の指定角は回転する前の状況での値です。
上の図を$90\Deg$回転してみましょう。

\begin{showEx}(.54,.4){回転との併用}
\begin{zahyou}[ul=10mm]%
    (-1,3.5)(-.5,1.5)
  \Put{(0,1)}{\kaitenkigou%
    <hazimekaku=-165,%
    owarikaku=165>[90]}
\end{zahyou}
\end{showEx}

最後に,\cmd{kaitenkigou}の書式です。

\begin{boxnote}
\begin{verbatim}
直線のまわりに回転させることを表す記号
\kaitenkigou<#1>[#2]
   #1 : 倍率
     または key=val
            bairitu= (倍率)デフォルト値 : 1
            muki = r/n                      n
                r で,負の向き
                n で,正の向き
            hazimekaku=                     15
            owarikaku=                     345
            tyouhankei=                      3mm
            tanhankei=                     1.5mm
   #2 : 回転角
   位置は \emathPut で指定
\end{verbatim}
\end{boxnote}

直線$y=x$の周りに回転させる記号を,2倍のサイズで描いてみます。

\begin{showEx}(.5,.44){$y$軸の周りの回転}
\begin{zahyou}[ul=10mm](-2,2)(-2,2)
\Put{(1.8,1.8)}{\kaitenkigou<2>[45]}
\kTyokusen\O{45}{}{}
\end{zahyou}
\end{showEx}

