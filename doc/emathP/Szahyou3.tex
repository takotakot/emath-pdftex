\section{空間座標}
\subsection{\textsf{Zahyou} 環境}
\textsf{zahyou}環境が平面座標を扱ったのに対して,
\textsf{Zahyou}環境は空間座標を扱います。

単純な例です。

\showexample[空間座標](0.58)(0.36){example/zahyou31}

簡単に解説します。

まず,\textsf{Zahyou}環境は
\index{Zahyou@Zahyou 環境}
\begin{jquote}
$x$の範囲,$y$の範囲,$z$の範囲
\end{jquote}
を\vspace{-\baselineskip}
\begin{jquote}
\begin{verbatim}
(xの下限,xの上限),(yの下限,yの上限),(zの下限,zの上限)
\end{verbatim}
\end{jquote}
の形で与えます。これらの値は,\textsf{Zahyou}環境内では,順に
\begin{jquote}
\begin{verbatim}
\Xmin, \Xmax, \Ymin, \Ymax, \Zmin, \Zmax
\end{verbatim}
\end{jquote}
で引用することができます。

点は \verb/\def\P{(1,2,3)}/のように3次元ベクトルで与えます。

次いで,\verb/\iiiKuromaru\P/で点Pに黒丸を打ちます。\cindex{iiiKuromaru}
座標平面で定義された多くのコマンドの先頭に\texttt{`iii'}をつけた
空間座標用のコマンドが定義されています。現時点で定義されているコマンドは
次の通りです。\vspace{-.5\baselineskip}
\begin{jquote}
\begin{verbatim}
\iiiPut, \iiiPutStr
\iiiBunten
\iiiDrawline, \iiiDashline, \iiiArrowLine
\iiiTyokkaku, \iiiHen_ko
\iiiKuromaru, \iiiSiromaru
\iiiKyori, \iiiKyorii
\iiiAddvec, \iiiSubvec, \iiiMulvec
\iiiNuritubusi
\iiibGurafu
\end{verbatim}
\end{jquote}
\cindex{iiiPut}\cindex{iiiPutStr}
\cindex{iiiKyori}\cindex{iiiKyorii}
\cindex{iiiBunten}
\cindex{iiiDrawline}\cindex{iiiDashline}\cindex{iiiArrowLine}
\cindex{iiiTyokkaku}\cindex{iiiHen\_ko}
\cindex{iiiKuromaru}\cindex{iiiSiromaru}
\cindex{iiiKyori}\cindex{iiiKyorii}\cindex{iiiNuritubusi}
\cindex{iiiAddvec}\cindex{iiiSubvec}\cindex{iiiMulvec}

空間特有のコマンドについては,次節以降でとりあげます。

さて,座標軸の向き,単位長の変更についてのオプション機能です。

\textsf{Zahyou}環境の\texttt{[...]}オプションで$x$, $y$, $z$軸の
単位ベクトルを指定できるようにしてあります。
ただし,それらは描画する平面を座標平面と見立てての成分表示です。
デフォルトでは
\begin{jquote}
\begin{verbatim}
z軸:(0, 1)
y軸:(1, 0)
x軸は \kyokuTyoku(.667,-138)
\end{verbatim}
\end{jquote}
となっています。これを変更してみます。

\showexample[基本単位ベクトルの変更](0.58)(0.36){example/zahyou32}

\subsection{角錐}
角錐を描画するのにコマンド\cmd{Kakusui}を用意しました。
O(0, 0, 0), A(1, 0, 0), B(0, 1, 0)を頂点とする三角形OABを底面とし,
C(0, 0, 1)を頂点とする角錐---四面体を描画してみます。

\showexample[角錐](0.58)(0.36){example/kakusui3}

ここでは,座標軸を描画しないように指定するのに,
\cmd{Drawaxisfalse}としています。

\cmd{Kakusui}の書式です。\cindex{Kakusui}
\begin{boxnote}
\begin{verbatim}
\Kakusui[#1]#2#3#4
   #2 : 見える頂点列
   #3 : 見えない頂点列
   #4 : 錐の頂点
\end{verbatim}
\end{boxnote}

図の見えない部分の点線のスタイルは,デフォルトでは

\begin{jquote}
\begin{verbatim}
\def\iiiTensen{\iiiDashline[40]{.1}}
\end{verbatim}
\end{jquote}
となっていますが,これを再定義してみます。
\cindex{iiiTensen}

\showexample[\cmd{iiiTensen}の再定義](0.58)(0.36){example/kakusui4}

\subsection{角柱}
立方体,直方体,平行六面体を描画するために\cmd{Kakutyuu}を用意
してあります。平行六面体を描画してみます。

\showexample[角柱](0.58)(0.36){example/kakutyu3}

直方体ではないことを示すため,$z$軸方向の単位ベクトルを
変更した斜交座標系を用いています。

\cmd{Kakutyuu}の書式です。\cindex{Kakutyuu}
\begin{boxnote}
\begin{verbatim}
\Kakutyuu#1#2#3
   #1 : 見える頂点列
   #2 : 見えない頂点列
   #3 : 高さベクトル
\end{verbatim}
\end{boxnote}



\subsection{直線と平面の交点}
直線と平面の交点を求めるコマンドは,次の4つを用意してあります。
\begin{jquote}
\cmd{PandL}, \cmd{Pandl}, \cmd{pandL}, \cmd{pandl}
\end{jquote}
ここで,
\begin{jquote}
\texttt P は3点を指定した平面\\
\texttt p は点と法線ベクトルを指定した平面\\
\texttt L は2点を指定した直線\\
\texttt l は点と方向ベクトルを指定した直線
\end{jquote}
を意味します。

一例として直方体OABC-DEFGの対角線OFと3点A, C, Dを含む平面との
交点Pを描画してみます。


\showexample[直線と平面](0.52)(0.425){example/PandL01}

書式です。
\cindex{iiiKuromaru}\cindex{iiiSiromaru}
\begin{boxnote}
\begin{verbatim}
\PandL#1#2#3#4#5#6
    3点#1, #2, #3を通る平面と,
    2点#4, #5を通る直線との交点を#6に
\Pandl#1#2#3#4#5#6
    3点#1, #2, #3を通る平面と,
    点#4を通り方向ベクトルが#5の直線との交点を#6に
\pandL#1#2#3#4#5
    点#1を通り法線ベクトルが#2の平面と,
    2点#3, #4を通る直線との交点を#5に
\pandl#1#2#3#4#5
    点#1を通り法線ベクトルが#2の平面と,
    点#3を通り方向ベクトルが#4の直線との交点を#5に
\end{verbatim}
\end{boxnote}


\subsection{垂線}
座標平面で,点から直線に下した垂線の足を求めるコマンド\cmd{Suisen}の
3次元版の話しです。

書式です。
\cindex{LSuisen}\cindex{lSuisen}\cindex{PSuisen}\cindex{pSuisen}
\begin{boxnote}
\begin{verbatim}
\LSuisen#1#2#3#4
    点 #1 から直線 #2#3 へ下ろした垂線の足を #4 にセット
\lSuisen#1#2#3#4
    点#1から,点#2を通り,方向ベクトルが#3の直線への垂線
\pSuisen#1#2#3#4
    点#1から,#2を通り法線ベクトルが#3である平面
    に下ろした垂線の足を#4に与える。
\PSuisen#1#2#3#4#5
    点#1から,三点#2,#3,#4を通る平面
    に下ろした垂線の足を#5に与える。
\end{verbatim}
\end{boxnote}

例として,四面体OABCの頂点Oから底面ABCに下した垂線OPと,
Oから線分ABに下した垂線OHを作図します。

\showexample[垂線](0.58)(0.36){example/PSuisen1}

\subsection{空間曲線}
座標平面で媒介変数表示された曲線を描画するコマンド\cmd{bGurafu}
の3次元版が\cmd{iiibGurafu}です。書式は\cindex{iiibGurafu}

\begin{boxnote}
\begin{verbatim}
\iiibGurafu(#1)(#2)#3#4#5#6#7
   #1 : t の刻み値(デフォルト値は 0.05 )
   #2 : 点線で描画するときの描画する部分の t のレンジ
   #3 : x=f(t)
   #4 : y=g(t)
   #5 : z=h(t)
   #6 : t の始め値
   #7 :     終り値
\end{verbatim}
\end{boxnote}

具体例として,円柱螺旋を描画してみましょう。

\showexample[円柱螺旋](1)(0.9){example/rasen31}

なお,空間曲線を近似する折れ線を得るコマンド
\cmd{iiiBKinziOresen}もありますが,これは \textsf{perl} との
連携機能を必要とします。
\cindex{iiiBKinziOresen}
