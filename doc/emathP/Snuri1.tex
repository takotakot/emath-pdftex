\section{塗りつぶし(1)}
\subsection{多角形内部の塗りつぶし}
多角形の内部を塗りつぶすコマンド \cmd{Nuritubusi} です.
\cindex{Nuritubusi}

\showexample[多角形の塗りつぶし](0.51)(0.44){example/shade01}

\cmd{Nuritubusi} の書式は

\begin{boxnote}
\begin{verbatim}
\Nuritubusi[#1]#2
  #1 : 塗る濃さ 
      ( 0 と 1 の間の数. 0 は真っ白,1は真っ黒.
        デフォルトは 0.5)
  #2 : 多角形の点列(閉じていなければなりません.)
\end{verbatim}
\end{boxnote}

\subsection{円内部の塗りつぶし}
円の内部を塗りつぶすのが \cmd{En*} コマンドです.\cindex{En*}

\showexample[円の塗りつぶし](0.5)(0.45){example/shade03}

\begin{boxnote}
\begin{verbatim}
\En*[#1]#2#3
    #1 : 塗る濃さ 
      ( 0 と 1 の間の数. 0 は真っ白,1は真っ黒.
        デフォルトは 0.5)
        #2 : 中心の座標
        #3 : 半径
\end{verbatim}
\end{boxnote}

オプション引数\texttt{[..]}で濃さを指定することができます.
デフォルトは0.5です.特に0を指定すると白抜きとなります.

\showexample[白抜き](0.5)(0.45){example/shade04}

\subsection{扇形の塗りつぶし}
\showexample[扇形の塗りつぶし](0.5)(0.45){example/shade05}
\cindex{ougigata*}

\begin{boxnote}
\begin{verbatim}
\ougigata*[#1]<#2>#3#4#5
  #1 : 塗りつぶしの濃さ
  #2 : 境界線描画オプション
        <border=1> で境界線を描画する。
  #3 〜 #5 : \ougigata の #1〜#3 と同じ
  (中心は \put (\Put) で指定する.)
\end{verbatim}
\end{boxnote}

\subsection{弓形の塗りつぶし}
\showexample[弓形の塗りつぶし](0.5)(0.45){example/shade06}
\cindex{yumigata*}

\begin{boxnote}
\begin{verbatim}
\yumigata*[#1]<#2>#3#4#5
  #1 : 塗りつぶしの濃さ
  #2 : 境界線描画オプション
        <border=1> で境界線を描画する。
  #3 〜 #5 : \yumigata の #1〜#3 と同じ
  (中心は \put (\Put) で指定する.)
\end{verbatim}
\end{boxnote}


\subsection{楕円の塗りつぶし}
\showexample[楕円の塗りつぶし](0.5)(0.45){example/shade07}
\cindex{Daen*}

\begin{boxnote}
\begin{verbatim}
\Daen*[#1]#2#3#4
    #1 : 塗る濃さ 
      ( 0 と 1 の間の数. 0 は真っ白,1は真っ黒.
        デフォルトは 0.5)
        #2 : 中心の座標
        #3 : 横軸方向の半径
        #4 : 縦軸方向の半径
\end{verbatim}
\end{boxnote}


\subsection{弓形(楕円弧)の塗りつぶし}
\showexample[弓形(楕円弧)の塗りつぶし](0.5)(0.45){example/shade08}
\cindex{Daenko*}

\begin{boxnote}
\begin{verbatim}
\Daenko*[#1]#2#3#4#5
    #1 : 塗る濃さ 
         ( 0 と 1 の間の数. 0 は真っ白,1は真っ黒.
         デフォルトは 0.5)
    #2 : 横軸方向の半径
    #3 : 縦軸方向の半径
    #4 : 始め角
    #5 : 終り角
        (中心は \put (\Put) で指定する.)
\end{verbatim}
\end{boxnote}

\subsection{カラー指定}
\edef\saveUL{\the\unitlength}%
\unitlength=1pt\relax
\verb+\En*+, \verb+\Nuritubusi+において,色をつけるときは少々厄介なことがあります。
\subsubsection{\cmd{color}による塗り色指定}
次の例では,
\begin{showEx}(.64,.3){\cmd{color}による塗り色指定}
\begin{picture}(50,50)
  \color{red}\put(25,25){\circle*{40}}
\end{picture}
\end{showEx}

さて,あなたの環境では円は何色で塗りつぶされていますか。
実は,それは \verb+dvi-ware+に依存するのです。\textsf{emath}では
\begin{jquote}
\begin{verbatim}
 .tex ----> .dvi by platex
 .dvi ----> .ps  by dvipsk
(.ps  ----> .pdf by Distiller)
\end{verbatim}
\end{jquote}
を標準としていますが,この方式で作成される .ps(, .pdf) は真っ黒に塗りつぶされています。
赤はかけらも見えません。

\begin{description}
  \item[注] このあたり,さらに分析をしておきます。
  
    \verb+\color{red}\circle*{..}+は分析すると
    \begin{jquote}
\begin{verbatim}
\color{red}\special{bk}\circle{..}
\end{verbatim}
    \end{jquote}
    となります。この順序を一部入れ替えて
    \begin{jquote}
\begin{verbatim}
\special{bk}\color{red}\circle{..}
\end{verbatim}
    \end{jquote}
    とすると,\texttt{dvipsk}で作られる .ps ファイルも円の内部が赤くなります。

\begin{showEx}(.7,.24){\cmd{color}コマンド発行のタイミング}
\begin{picture}(50,50)
  \put(25,25){\color{red}\special{bk}\circle{40}}
\end{picture}
\begin{picture}(50,50)
  \put(25,25){\special{bk}\color{red}\circle{40}}
\end{picture}
\end{showEx}

2つの円が描画されますが,\texttt{dvipsk}で変換した .ps ファイルでは,円の内部は
\begin{jquote}
上は黒\\
下は赤
\end{jquote}
で塗りつぶされます。

\end{description}
\subsubsection{\cmd{En*}の\texttt{[nuriiro=..]}オプション}
そこで,\textsf{emath}では,\verb+En*+などに\verb+[nuriiro=...]+オプションをつけることで
内部に色をつけることにしています。


\begin{showEx}(.64,.3){\cmd{En*}の\texttt{[nuriiro=..]}オプション}
\begin{picture}(50,50)
  \En*[nuriiro=red]{(25,25)}{20}
\end{picture}
\end{showEx}

\begin{description}
  \item[注] 現時点では,\verb+[nuriiro=..]+オプションではなく,\verb+\color+コマンドを用いても,
    同様の結果を得ることができます。

\begin{showEx}(.64,.3){暫定仕様}
\begin{picture}(50,50)
  \color{red}\En*{(25,25)}{20}
\end{picture}
\end{showEx}

しかし,将来的にはこの方式は破産する予感がありますので,\verb+[nuriiro=..]+オプションの使用を
推奨します。
\end{description}

\subsubsection{円周の色は?}
円の内部は色付けができましたが,円周は黒のままですねぇ〜。
これは\verb+tpic specials+の仕様かも知れません。いや,そもそも\verb+tpic specials+は
カラー対応しているんでしょうか。浅学にしてわかりません。\verb+(^^ゞ+

そこで何とかの上塗りをすることとします。

\begin{showEx}(.64,.3){\cmd{En*}の\texttt{[border=..]}オプション}
\begin{picture}(50,50)
  \En*[nuriiro=red,border=red]{(25,25)}{20}
\end{picture}
\end{showEx}

\verb+[border=red]+オプションで円周を赤で上塗りしました。
円周も赤になったかに見えますが,よく見ると元の黒が一部はみ出していますね。
赤の上塗りをもう少し線を太くすれば良いかもしれませんが.......

しかし,労多くして何とやらという気もします。
カラーはやはり\verb+PostScript+で扱うべきものでしょうか。

\begin{description}
  \item[注] \verb+tpic specials+でも
    \begin{jquote}
\begin{verbatim}
円・楕円
 と
多角形
\end{verbatim}
    \end{jquote}
    では,結果が異なり,前者は境界が黒で描画されますが,
    後者は境界は描画されません。
    どうも,\verb+tpic specials+でカラーを扱うのは気が進みませんね。

\begin{showEx}(.64,.3){\cmd{Nuritubusi}の場合}
\begin{picture}(50,50)
  \Nuritubusi[nuriiro=red]{%
    (0,0)(50,0)(50,50)(10,40)}
\end{picture}
\end{showEx}
    
\end{description}

\subsubsection{\texttt{PostScript}では}
以上の考察から,カラー塗りつぶしは\texttt{PostScript --- pszahyou環境}を
使用するのがよさそうです。
\unitlength=\saveUL\relax

%\begin{showEx}(.64,.3){\textsf{pszahyou}環境}
%\begin{pszahyou*}(0,50)(0,50)
%  \En*[nuriiro=red]{(25,25)}{20}
%\end{pszahyou*}
%\end{showEx}
