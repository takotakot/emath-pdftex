\documentclass{jarticle}
\usepackage{emathPh}
%\usepackage{emathPb}

\begin{document}
\unitlength=1pt\relax
\begin{sikipicture}{%
    x^2+,
    \protect\emPmaru<iro=red>[\protect\mathstrut]{6},
    x+,
    \protect\emPmaru<iro=red>[\protect\mathstrut]{3},
    ^2=1+,
    \protect\emPmaru<iro=red>[\protect\mathstrut]{3},
    ^2,
  }
  \Tyuuten\sikiBii\sikiBiv\P
  \Addvec\P{(0,-8)}\P
  {\color{red}%
    \ArrowLine\sikiBii\P
    \ArrowLine\P\sikiBiv
    \ArrowLine\P\sikiBvi
  }%
  \Put\P(0,-1pt)[t]{\phovalbox{\small 半分の2乗}}
\end{sikipicture}
%
\begin{picture}(20,10)
\color{red}%
\ArrowLine{(5,3)}{(15,3)}
\end{picture}
%
\begin{sikipicture}{(x+3)^2+10}
  \HenKo<henkoH=3pt,henkotype=bracket,henkocolor=red,putoption={[s]}>%
    \LB\RB{\maru2 の形}
\end{sikipicture}
\end{document}
