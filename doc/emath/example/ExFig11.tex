\begin{enumerate}
\item \begin{Tmawarikomi}(0,5pt){9\zw}{%
    \caption{表の例}\label{T1}\hfil
    \begin{tabular}{|c|c|c|}\hline
      aaa & bbb & ccc \\\hline
      1 & 2 & 3 \\\hline
    \end{tabular}}
  第1問 右の表\ref{T1}において
  ああああああああああああああああああああああああああああああああああ
  ああああああああああああああああああああああああああああああああああ
  ああああああああああああああああああああああああああああああああああ
  ああああああああああああああああ
 \end{Tmawarikomi}
\item \begin{Fmawarikomi}{60pt}{%
      \unitlength1pt%
      \begin{picture}(60,30)%
        \put(0,0){\line(2,1){60}}%
      \end{picture}%
      \caption{図の例}\label{Fig1}}
    第2問 右の図\ref{Fig1}において,
  いいいいいいいいいいいいいいいいいいいいいいいいいいいいいいいいいい
  いいいいいいいいいいいいいいいいいいいいいいいいいいいいいいいいいい
  いいいいいいいいいいいいいいいいいいいいいいいいいいいいいいいいいい
  いいいいいいいいいいいいいいいいいいいいいいいいいいいいいいいいいい
  いいいいいいいいいいいいいいいいいいいいいいいいいいいいいいいいいい
 \end{Fmawarikomi}
\end{enumerate}
